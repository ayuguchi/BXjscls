% ^^A -*- japanese-latex -*-
% \iffalse meta-comment
%
%  BXJS ドキュメントクラス
% (BXJS Document Classes)
%
%  本ソフトウェアは奥村晴彦氏および日本語TeX開発コミュニティによる
%  「pLaTeX2ε新ドキュメントクラス」を
%  「八登 崇之(別名 ZR)」が改変したものである。
% (This software is a modified version of
%  "New Document CLasses for pLaTeX2e"
%  authored by Haruhiko Okumura and the Japanese TeX Development Community,
%  and the modification is by Takayuki YATO (aka."ZR").)
%
%  本ソフトウェアは修正 BSD ライセンスの下で配布される。
% (This software is distributed under the BSD 2-Clause License.)
%
%  以下に原版についての原版著者による情報を示す:
% (Below is shown the information on the original version,
%  provided by the original authors.)
%---------------------------------------
%
%  pLaTeX2ε新ドキュメントクラス
%
%  これは LaTeX3 Project の classes.dtx と株式会社アスキーの jclasses.dtx
%  に基づいて,もともと奥村晴彦 <okumura@okumuralab.org> により
%  作成されたものです。
%  現在は日本語\TeX 開発コミュニティにより管理されています。
%
%  [2002-12-19] いろいろなものに収録していただく際にライセンスを明確にする
%  必要が生じてきました。アスキーのものが最近はmodified BSDライセンスになっ
%  ていますので,私のものもそれに準じてmodified BSDとすることにします。
%
%  [2016-07-11] abenoriさんによる\texttt{nomag}オプションをマージしました。
%  なお,ソースコードの管理はGitHubで行うことにしました。
%
%  [2016-07-13] 日本語\TeX 開発コミュニティによる管理に移行しました。
%
%---------------------------------------
%  Copyright 1993 1994 1995 1996 1997 1998 1999
%  The LaTeX3 Project and any individual authors listed elsewhere
%  in this file.
%
%  Copyright 1995-1999 ASCII Corporation.
%  Copyright 1999-2016 Haruhiko Okumura
%  Copyright 2016 Japanese TeX Development Community
%
%  Copyright 2013-2016 Takayuki YATO (aka."ZR")
%
% \fi
% \CheckSum{0}
%
% \iffalse
%
%    \begin{macrocode}
%<article|report|book|slide>\NeedsTeXFormat{LaTeX2e}
%<article>\ProvidesClass{bxjsarticle}
%<report>\ProvidesClass{bxjsreport}
%<book>\ProvidesClass{bxjsbook}
%<slide>\ProvidesClass{bxjsslide}
%<minimal>\ProvidesFile{bxjsja-minimal.def}
%<standard>\ProvidesFile{bxjsja-standard.def}
%<modern>\ProvidesFile{bxjsja-modern.def}
%<pandoc>\ProvidesFile{bxjsja-pandoc.def}
%<compat>\ProvidesPackage{bxjscompat}
%<cjkcat>\ProvidesPackage{bxjscjkcat}
%<*driver>
\ProvidesFile{bxjscls.dtx}
%</driver>
  [2017/01/28 v1.3a BXJS document classes]
% based on jsclasses [2017/01/13]
% [2016/11/01 v1.3 BXJS document classes]
% based on jsclasses [2016/10/08]
% [2016/08/16 v1.2a BXJS document classes]
% [2016/08/01 v1.2 BXJS document classes]
% based on jsclasses [2016/07/18]
% [2016/07/16 v1.1f BXJS document classes]
% based on jsclasses [2016/07/15]
% [2016/05/28 v1.1e BXJS document classes]
% [2016/05/21 v1.1d BXJS document classes]
% [2016/05/01 v1.1c BXJS document classes]
% [2016/03/27 v1.1b BXJS document classes]
% [2016/02/20 v1.1a BXJS document classes]
% [2016/02/14 v1.1 BXJS document classes]
% [2015/11/21 v1.0d BXJS document classes]
% [2015/10/18 v1.0c BXJS document classes]
% [2015/09/07 v1.0b BXJS document classes]
% [2015/08/23 v1.0a BXJS document classes]
% [2015/08/05 v1.0 BXJS document classes]
% based on jsclasses [2014/02/07]
% [2013/10/03 v0.9pre BXJS document classes]
% based on jsclasses [2013/05/14]
%<*driver>
\documentclass[a4paper]{ltjsarticle}
\usepackage[ipaex]{luatexja-preset}
\usepackage{metalogo}
\usepackage{doc}
\newcommand{\Pkg}[1]{\textsf{#1}}
\newcommand{\Meta}[1]{$\langle$\mbox{}#1\mbox{}$\rangle$}
\newcommand{\Note}{\par\noindent ※}
\newcommand{\TODO}{\par\noindent
  {\usefont{OT1}{cmss}{sbc}{n}\color{red}TODO:}\ }
\newcommand{\Means}{~:\quad}
\newcommand{\OR}{$\;|\;$}
\newcommand{\ZRX}{☃}
\newenvironment{ZRnote}
  {\StartZRnote}{\EndZRnote}
\newcommand*{\ZRnoteline}[1]{
  \par\noindent\makebox[0pt][l]{\rule[.5ex]{\linewidth}{.4pt}}%
  \makebox{\hspace{.05\linewidth}\rule[#1ex]{.9\linewidth}{.2pt}}\par}
\newcommand*{\StartZRnote}{\ZRnoteline{.1}}
\newcommand*{\EndZRnote}{\ZRnoteline{.9}}
\providecommand*{\eTeX}{$\varepsilon$-\TeX}
\providecommand*{\upTeX}{u\pTeX}
\providecommand*{\XeTeX}{XeTeX}
\newcommand*{\zrWDash}{\symbol{"301C}}% WAVE DASH
\newcommand*{\zrNote}[1]{[#1]}
\makeatletter
\def\meta@font@select{\rmfamily\itshape}
\makeatother
\addtolength{\textwidth}{-1in}
\addtolength{\evensidemargin}{1in}
\addtolength{\oddsidemargin}{1in}
\addtolength{\marginparwidth}{1in}
\setlength\marginparpush{0pt}
% \OnlyDescription
\DisableCrossrefs
\CodelineNumbered
\setcounter{StandardModuleDepth}{1}
\GetFileInfo{bxjscls.dtx}
\begin{document}
  \DocInput{bxjscls.dtx}
\end{document}
%</driver>
%    \end{macrocode}
%
% \fi
%
%^^A========================================================
% \title{\Pkg{BXjscls} パッケージ\\
%  (BXJS文書クラス集)\\
%  ソースコード説明書}
% \author{八登崇之\ (Takayuki YATO; aka.~``ZR''}
% \date{\fileversion\quad[\filedate]}
% \maketitle
%
% \MakeShortVerb{\|}
%
% \begin{ZRnote}
% この文書はソースコード説明書です。
% 一般の文書作成者向けの解説については、
% ユーザマニュアル |bxjscls-manual.pdf| を参照してください。
% \end{ZRnote}
%
% \tableofcontents
%
% \section{はじめに}
%
% \begin{ZRnote}
% この文書は「BXJSドキュメントクラス」のDocStrip形式のソースである。
% インストール時のモジュール指定は以下のようである。
% \begin{quote}
%   \begin{tabular}{lll}
%     $\langle$\textsf{article}$\rangle$ & \texttt{bxjsarticle.cls}
%      & 短いレポート(章なし) \\
%     $\langle$\textsf{report}$\rangle$ & \texttt{bxjsreport.cls}
%      & 長いレポート(章あり) \\
%     $\langle$\textsf{book}$\rangle$    & \texttt{bxjsbook.cls}
%      & 書籍用 \\
%     $\langle$\textsf{slide}$\rangle$   & \texttt{bxjsslide.cls}
%      & スライド用 \\
%   \end{tabular}
% \end{quote}
%
% 本ドキュメントクラスは奥村晴彦氏および日本語TeX開発コミュニティ
% による「p\LaTeXe 新ドキュメントクラス」に改変を加えたものである。
% 本ドキュメントクラスに関する説明は全てこの形式の枠の中に記す。
% 枠の外にあるものは原版著者による原版に対する解説である。
% \end{ZRnote}
%
% これは\LaTeX3 Projectの \texttt{classes.dtx} と
% 株式会社アスキーの \texttt{jclasses.dtx} に基づいて
% 奥村が改変したものです。
% 権利については両者のものに従います。
% 奥村は何の権利も主張しません。
%
% [2009-02-22] 田中琢爾氏によるup\LaTeX 対応パッチを取り込みました。
%
% \iffalse
% ここでは次のドキュメントクラス(スタイルファイル)を作ります。
% \begin{quote}
%   \begin{tabular}{lll}
%     $\langle$\textsf{article}$\rangle$ & \texttt{jsarticle.cls}  & 論文・レポート用 \\
%     $\langle$\textsf{book}$\rangle$    & \texttt{jsbook.cls}     & 書籍用 \\
%     $\langle$\textsf{jspf}$\rangle$    & \texttt{jspf.cls}       & 某学会誌用 \\
%     $\langle$\textsf{kiyou}$\rangle$   & \texttt{kiyou.cls}      & 某紀要用
%   \end{tabular}
% \end{quote}
%
% \LaTeXe あるいは\pLaTeXe 標準のドキュメントクラスとの違いを説明してお
% きます。
%
% \paragraph{JISフォントメトリックの使用}
%
% ここでは和文TFM(\TeX フォントメトリック)として東京書籍印刷の小林肇さ
% んの作られたJISフォントメトリック \texttt{jis.tfm},\texttt{jisg.tfm}
% を標準で使います。従来のフォントメトリック \texttt{min10.tfm},
% \texttt{goth10.tfm} の類を使うには
% \begin{quote}
%   |\documentclass[mingoth]{jsarticle}|
% \end{quote}
% のように \texttt{mingoth} オプションを付けます。
%
% \paragraph{サイズオプションの扱いが違う}
%
% 標準のドキュメントクラスでは本文のポイント数を指定するオプションがあり
% ましたが,ポイント数は10,11,12しかなく,それぞれ別のクラスオプション
% ファイルを読み込むようになっていました。しかも,標準の10ポイント以外で
% は多少フォントのバランスが崩れることがあり,あまり便利ではありませんで
% した。ここでは文字サイズを増すとページを小さくし,\TeX の |\mag| プリ
% ミティブで全体的に拡大するという手を使って,9ポイントや21,25,30,36,
% 43ポイント,12Q,14Qの指定を可能にしています。
% \fi
%
% \StopEventually{}
%
% 以下では実際のコードに即して説明します。
%
% \paragraph{BXJSクラス特有の設定 \ZRX}
%
% \mbox{}
% \begin{ZRnote}
%    \begin{macrocode}
%<*cls>
%% このファイルは日本語文字を含みます.
%    \end{macrocode}
% 長さ値の指定で式を利用可能にするため |calc| を読み込む。
%    \begin{macrocode}
\RequirePackage{calc}
%    \end{macrocode}
% クラスオプションでkey-value形式を使用するため |keyval| を読み込む。
%    \begin{macrocode}
\RequirePackage{keyval}
%    \end{macrocode}
% クラスの本体ではこの他に |geometry| パッケージが読み込まれる。
%
% 互換性のための補助パッケージを読み込む。
%    \begin{macrocode}
\IfFileExists{bxjscompat.sty}{%
  \let\jsAtEndOfClass\@gobble
  \RequirePackage{bxjscompat}%
}{}
%    \end{macrocode}
%
% \begin{macro}{\jsDocClass}
% 〔トークン〕
% 文書クラスの種別。
% 以下の定値トークンの何れかと同等:
% |\jsArticle|=bxjsarticle、
% |\jsBook|=bxjsbook、
% |\jsReport|=bxjsreport、
% |\jsSlide|=bxjsslide。
%    \begin{macrocode}
\let\jsArticle=a
\let\jsBook=b
\let\jsReport=r
\let\jsSlide=s
%<article>\let\jsDocClass\jsArticle
%<article>\def\bxjs@clsname{bxjsarticle}
%<book>\let\jsDocClass\jsBook
%<book>\def\bxjs@clsname{bxjsbook}
%<report>\let\jsDocClass\jsReport
%<report>\def\bxjs@clsname{bxjsreport}
%<slide>\let\jsDocClass\jsSlide
%<slide>\def\bxjs@clsname{bxjsslide}
%    \end{macrocode}
% \end{macro}
%
% \begin{macro}{\jsEngine}
% 〔暗黙文字トークン〕
% エンジン({\TeX}の種類)の種別:
% |j|={\pTeX}系、
% |x|={\XeTeX}、
% |p|=pdf{\TeX}(含DVIモード)、
% |l|=Lua{\TeX}、
% |J|=NTT j{\TeX}、
% |O|=Omega系、
% |n|=以上の何れでもない。
%
%    \begin{macrocode}
\let\jsEngine=n
\def\bxjs@test@engine#1#2{%
  \edef\bxjs@tmpa{\string#1}%
  \edef\bxjs@tmpb{\meaning#1}%
  \ifx\bxjs@tmpa\bxjs@tmpb #2\fi}
\bxjs@test@engine\kanjiskip{\let\jsEngine=j}
\bxjs@test@engine\jintercharskip{\let\jsEngine=J}
\bxjs@test@engine\Omegaversion{\let\jsEngine=O}
\bxjs@test@engine\XeTeXversion{\let\jsEngine=x}
\bxjs@test@engine\pdftexversion{\let\jsEngine=p}
\bxjs@test@engine\luatexversion{\let\jsEngine=l}
%    \end{macrocode}
% \end{macro}
%
% \begin{macro}{\ifjsWithupTeX}
% 〔スイッチ〕
% エンジンが(内部漢字コードがUnicodeの){\upTeX}であるか。
%    \begin{macrocode}
\newif\ifjsWithupTeX
\ifx\ucs\@undefined\else \ifnum\ucs"3000="3000
  \jsWithupTeXtrue
\fi\fi
\let\if@jsc@uplatex\ifjsWithupTeX
%    \end{macrocode}
% \end{macro}
%
% \begin{macro}{\ifjsWithpTeXng}
% 〔スイッチ〕
% エンジンが{\pTeX-ng}であるか。
%    \begin{macrocode}
\newif\ifjsWithpTeXng
\bxjs@test@engine\ngbanner{\jsWithpTeXngtrue}
%    \end{macrocode}
% \end{macro}
%
% \begin{macro}{\ifjsWitheTeX}
% 〔スイッチ〕
% エンジンが{\eTeX}拡張をもつか。
%    \begin{macrocode}
\newif\ifjsWitheTeX
\bxjs@test@engine\eTeXversion{\jsWitheTeXtrue}
%    \end{macrocode}
% \end{macro}
%
% 非サポートのエンジンの場合は強制終了させる。
% \Note NTT j{\TeX}とOmega系。
%    \begin{macrocode}
\let\bxjs@tmpa\relax
\ifx J\jsEngine \def\bxjs@tmpa{NTT-jTeX}\fi
\ifx O\jsEngine \def\bxjs@tmpa{Omega}\fi
\ifx\bxjs@tmpa\relax \expandafter\@gobble
\else
  \ClassError\bxjs@clsname
   {The engine in use (\bxjs@tmpa) is not supported}
   {It's a fatal error. I'll quit right now.}
  \expandafter\@firstofone
\fi{\endinput\@@end}
%    \end{macrocode}
%
% \begin{macro}{\bxjs@protected}
% {\eTeX}拡張が有効な場合にのみ |\protected|
% の効果をもつ。
%    \begin{macrocode}
\ifjsWitheTeX \let\bxjs@protected\protected
\else \let\bxjs@protected\@empty
\fi
%    \end{macrocode}
% \end{macro}
%
% \begin{macro}{\bxjs@robust@def}
% 無引数の頑強な命令を定義する。
%    \begin{macrocode}
\ifjsWitheTeX
  \def\bxjs@robust@def{\protected\def}
\else
  \def\bxjs@robust@def{\DeclareRobustCommand*}
\fi
%    \end{macrocode}
% \end{macro}
%
% \begin{macro}{\ifjsInPdfMode}
% 〔スイッチ〕
% pdf{\TeX}/Lua{\TeX}がPDFモードで動作しているか。
% \Note Lua{\TeX} 0.8x版でのプリミティブ名変更に対応。
%    \begin{macrocode}
\newif\ifjsInPdfMode
\@nameuse{ImposeOldLuaTeXBehavior}
\let\bxjs@tmpa\PackageWarningNoLine
\let\PackageWarningNoLine\PackageInfo % suppress warning
\RequirePackage{ifpdf}
\let\PackageWarningNoLine\bxjs@tmpa
\@nameuse{RevokeOldLuaTeXBehavior}
\let\ifjsInPdfMode\ifpdf
%    \end{macrocode}
% \end{macro}
%
% \begin{macro}{\bxjs@cond}
% |\bxjs@cond\ifXXX|……|\fi{|\Meta{真}|}{|\Meta{偽}|}|\par
% {\TeX}のif-文(|\ifXXX|……\Meta{真}|\else|\Meta{偽}|\fi|)を
% 末尾呼出形式に変換するためのマクロ。
%    \begin{macrocode}
\@gobbletwo\if\if \def\bxjs@cond#1\fi{%
  #1\expandafter\@firstoftwo
  \else\expandafter\@secondoftwo
  \fi}
%    \end{macrocode}
% \end{macro}
%
% \begin{macro}{\jsAtEndOfClass}
% このクラスの読込終了時に対するフック。
% (補助パッケージ中で用いられる。)
%    \begin{macrocode}
\def\jsAtEndOfClass{%
  \expandafter\g@addto@macro\csname\bxjs@clsname.cls-h@@k\endcsname}
%    \end{macrocode}
% \end{macro}
%
% Lua\TeX の場合、原版のコード中のコントロールワード中に現れる
% 日本語文字のカテゴリコードを一時的に11に変更する。
% クラス読込終了時点で元に戻される。
% \Note 現在の{Lua\LaTeX}では、漢字のカテゴリコードは最初から11に
% なっているので、この処理は特段の意味を持たない。
% しかし、昔は12になっていて、この場合、日本語文字の
% コントロールワードの命令を使用するには、カテゴリコードを11に
% 変更する必要がある。
%    \begin{macrocode}
\@onlypreamble\bxjs@restore@jltrcc
\let\bxjs@restore@jltrcc\@empty
\if l\jsEngine
\def\bxjs@change@jltrcc#1{%
  \xdef\bxjs@restore@jltrcc{%
    \bxjs@restore@jltrcc
    \catcode`#1=\the\catcode`#1\relax}%
  \catcode`#1=11\relax}
\@tfor\bxjs@x:=西暦\do
  {\expandafter\bxjs@change@jltrcc\bxjs@x}
\fi
%    \end{macrocode}
%
% |\jsInhibitGlue| は |\inhibitglue| が定義されていればそれを
% 実行し、未定義ならば何もしない。
%    \begin{macrocode}
\bxjs@robust@def\jsInhibitGlue{%
  \ifx\inhibitglue\@undefined\else \inhibitglue \fi}
%    \end{macrocode}
%
% 万が一「2.09互換モード」になっていた場合は、
% これ以上進むと危険なので強制終了させる。
%    \begin{macrocode}
\if@compatibility
  \ClassError\bxjs@clsname
   {Something went chaotic!\MessageBreak
    (How come '\string\documentstyle' is there?)\MessageBreak
    I cannot go a single step further...}
   {If the chant of '\string\documentstyle' was just a blunder of yours,\MessageBreak
    then there'll still be hope....}
  \expandafter\@firstofone
\else \expandafter\@gobble
\fi{\typeout{Farewell!}\endinput\@@end}
%    \end{macrocode}
%
% \end{ZRnote}
%
% \section{オプション}
%
% これらのクラスは |\documentclass{jsarticle}|
% あるいは |\documentclass[オプション]{jsarticle}|
% のように呼び出します。
%
% まず,オプションに関連するいくつかのコマンドやスイッチ(論理変数)を定
% 義します。
%
% \begin{macro}{\if@restonecol}
%
% 段組のときに真になる論理変数です。
%
%    \begin{macrocode}
\newif\if@restonecol
%    \end{macrocode}
% \end{macro}
%
% \begin{macro}{\if@titlepage}
%
% これを真にすると表題,概要を独立したページに出力します。
%
%    \begin{macrocode}
\newif\if@titlepage
%    \end{macrocode}
% \end{macro}
%
% \begin{macro}{\if@openright}
%
% |\chapter|,|\part| を奇数ページ起こしにするかどうかです。
% 書籍では真が標準です。
%
%    \begin{macrocode}
%<book|report>\newif\if@openright
%    \end{macrocode}
% \end{macro}
%
% \begin{macro}{\if@mainmatter}
%
% 真なら本文,偽なら前付け・後付けです。
% 偽なら |\chapter| で章番号が出ません。
%
%    \begin{macrocode}
%<book|report>\newif\if@mainmatter \@mainmattertrue
%    \end{macrocode}
% \end{macro}
%
% \begin{macro}{\if@enablejfam}
%
% 和文フォントを数式フォントとして登録するかどうかを示すスイッチです。
%
%    \begin{macrocode}
\newif\if@enablejfam \@enablejfamtrue
%    \end{macrocode}
% \end{macro}
%
% 以下で各オプションを宣言します。
%
% \paragraph{用紙サイズ}
%
% JISやISOのA0判は面積 $1\,\mathrm{m}^2$,縦横比 $1:\sqrt{2}$
% の長方形の辺の長さを mm 単位に切り捨てたものです。
% これを基準として順に半截しては mm 単位に切り捨てたものがA1,A2,…です。
%
% B判はJISとISOで定義が異なります。
% JISではB0判の面積が $1.5\,\mathrm{m}^2$ ですが,
% ISOではB1判の辺の長さがA0判とA1判の辺の長さの幾何平均です。
% したがってISOのB0判は $1000\,\mathrm{mm} \times 1414\,\mathrm{mm}$ です。
% このため,\LaTeXe の \texttt{b5paper}
% は $250\,\mathrm{mm} \times 176\,\mathrm{mm}$ ですが,
% \pLaTeXe の \texttt{b5paper}
% は $257\,\mathrm{mm} \times 182\,\mathrm{mm}$ になっています。
% ここでは\pLaTeXe にならってJISに従いました。
%
% デフォルトは \texttt{a4paper} です。
%
% \texttt{b5var}(B5変形,182mm×230mm),
% \texttt{a4var}(A4変形,210mm×283mm)を追加しました。
%
% \begin{ZRnote}
% BXJSクラスではページレイアウト設定に |geometry| パッケージを用いる。
% 用紙サイズ設定は |geometry| に渡すオプションの指定と扱われる。
%    \begin{macrocode}
\def\bxjs@setpaper#1{\def\bxjs@param@paper{#1}}
\DeclareOption{a3paper}{\bxjs@setpaper{a3paper}}
\DeclareOption{a4paper}{\bxjs@setpaper{a4paper}}
\DeclareOption{a5paper}{\bxjs@setpaper{a5paper}}
\DeclareOption{a6paper}{\bxjs@setpaper{a6paper}}
\DeclareOption{b4paper}{\bxjs@setpaper{{257truemm}{364truemm}}}
\DeclareOption{b5paper}{\bxjs@setpaper{{182truemm}{257truemm}}}
\DeclareOption{b6paper}{\bxjs@setpaper{{128truemm}{182truemm}}}
\DeclareOption{a4j}{\bxjs@setpaper{a4paper}}
\DeclareOption{a5j}{\bxjs@setpaper{a5paper}}
\DeclareOption{b4j}{\bxjs@setpaper{{257truemm}{364truemm}}}
\DeclareOption{b5j}{\bxjs@setpaper{{182truemm}{257truemm}}}
\DeclareOption{a4var}{\bxjs@setpaper{{210truemm}{283truemm}}}
\DeclareOption{b5var}{\bxjs@setpaper{{182truemm}{230truemm}}}
\DeclareOption{letterpaper}{\bxjs@setpaper{letterpaper}}
\DeclareOption{legalpaper}{\bxjs@setpaper{legalpaper}}
\DeclareOption{executivepaper}{\bxjs@setpaper{executivepaper}}
%    \end{macrocode}
% \end{ZRnote}
%
% \paragraph{横置き}
%
% 用紙の縦と横の長さを入れ換えます。
%
%    \begin{macrocode}
\newif\if@landscape
\@landscapefalse
\DeclareOption{landscape}{\@landscapetrue}
%    \end{macrocode}
%
% \paragraph{slide}
%
% オプション \texttt{slide} を新設しました。
%
% [2016-10-08] \texttt{slide} オプションは article 以外では使い物にならなかったので,
% 簡単のため article のみで使えるオプションとしました。
%
%    \begin{macrocode}
\newif\if@slide
%    \end{macrocode}
%
% \begin{ZRnote}
% BXJSではスライド用のクラス |bxjsslide| を用意しているので、
% 本来はこのスイッチは不要なはずである。
% しかし、JSクラスの一部のコードをそのまま使うために保持している。
% \Note この |\if@slide| という制御綴は、ユニークでないにも関わらず、
% 衝突した場合に正常動作が保たれない、という問題を抱えている。
%    \begin{macrocode}
%<!slide>\@slidefalse
%<slide>\@slidetrue
%    \end{macrocode}
% \end{ZRnote}
%
% \paragraph{サイズオプション}
%
% 10pt,11pt,12pt のほかに,8pt,9pt,14pt,17pt,21pt,25pt,30pt,36pt,43pt を追加しました。
% これは等比数列になるように選んだものです(従来の 20pt も残しました)。
% |\@ptsize| の定義が変だったのでご迷惑をおかけしましたが,
% 標準的なドキュメントクラスと同様にポイント数から10を引いたものに直しました。
%
% [2003-03-22] 14Qオプションを追加しました。
%
% [2003-04-18] 12Qオプションを追加しました。
%
% [2016-07-08] |\mag| を使わずに各種寸法をスケールさせるためのオプション \texttt{nomag} を新設しました。
% \texttt{usemag} オプションの指定で従来通りの動作となります。デフォルトは \texttt{usemag} です。
%
% [2016-07-24] オプティカルサイズを調整するためにNFSSへパッチを当てるオプション \texttt{nomag*} を新設しました。
%
% \begin{ZRnote}
% |\@ptsize| は |10pt|, |11pt|, |12pt| が指定された時のみ従来と同じ値とし、
% それ以外は |\jsUnusualPtSize|(= $-20$)にする。
%
%    \begin{macrocode}
\newcommand{\@ptsize}{0}
\def\bxjs@param@basefontsize{10pt}
\def\jsUnusualPtSize{-20}
%    \end{macrocode}
%
% \begin{macro}{\bxjs@setbasefontsize}
% 基底フォントサイズを実際に変更する。
%    \begin{macrocode}
\def\bxjs@setbasefontsize#1{%
  \bxjs@setbasefontlength\@tempdima{#1}%
  \edef\bxjs@param@basefontsize{\the\@tempdima}%
  \ifdim\@tempdima=10pt         \long\def\@ptsize{0}%
  \else\ifdim\@tempdima=10.95pt \long\def\@ptsize{1}%
  \else\ifdim\@tempdima=12pt    \long\def\@ptsize{2}%
  \else \long\edef\@ptsize{\jsUnusualPtSize}\fi\fi\fi}
%    \end{macrocode}
% \end{macro}
% \begin{macro}{\bxjs@setbasefontlength}
% |base|、|jbase| で指定される長さ(式)のための
% 特別な |\setlength|。
% 与えられた式が“\Meta{実数}|Q|”の形の場合、Q単位の長さを
% 代入する(この場合“式”は使えない)。
% \Note クラスオプションのトークン列の中に展開可能なトークンが
% ある場合、{\LaTeX}はファイルの読込の前にそれを展開しようとする。
% このため、この位置で |\jQ| をサポートすることは
% 原理的に不可能である。
%    \begin{macrocode}
\def\bxjs@setbasefontlength#1#2{%
%    \end{macrocode}
% ここで |true| の長さが使われるのは不合理なので、
% 式が“|true|”を含む場合には警告を出す。
%    \begin{macrocode}
  \bxjs@setbasefontlength@a#2true\@nnil
%    \end{macrocode}
% 式の末尾が“|Q|”である時は特別に扱い、
% それ以外は |\setlength| に移譲する。
%    \begin{macrocode}
  \ifx j\jsEngine \setlength#1{#2}%
  \else
    \bxjs@setbasefontlength@b#2\@nil Q\@nil\@nnil
    \ifx\bxjs@tmpa\relax \setlength#1{#2}%
    \else \@tempdimc0.25mm #1=\bxjs@tmpa\@tempdimc
    \fi
  \fi}
\def\bxjs@setbasefontlength@b#1Q\@nil#2\@nnil{%
  \ifx\@nnil#2\@nnil \let\bxjs@tmpa\relax
  \else \def\bxjs@tmpa{#1}%
  \fi}
\def\bxjs@setbasefontlength@a#1true#2\@nnil{%
  \ifx\@nnil#2\@nnil\else
    \ClassWarningNoLine\bxjs@clsname
    {You should not use 'true' lengths here}%
  \fi}
%    \end{macrocode}
% \end{macro}
%
% |\ifjsc@mag| は「|\mag| を使うか」を表すスイッチ。
%
% |\ifjsc@mag@xreal| は「NFSSにパッチを当てるか」を表すスイッチ。
%    \begin{macrocode}
\newif\ifjsc@mag
\newif\ifjsc@mag@xreal
%\let\jsc@magscale\@undefined
%    \end{macrocode}
%
%    \begin{macrocode}
\DeclareOption{8pt}{\bxjs@setbasefontsize{8pt}}
\DeclareOption{9pt}{\bxjs@setbasefontsize{9pt}}
\DeclareOption{10pt}{\bxjs@setbasefontsize{10pt}}
\DeclareOption{11pt}{\bxjs@setbasefontsize{10.95pt}}
\DeclareOption{12pt}{\bxjs@setbasefontsize{12pt}}
\DeclareOption{14pt}{\bxjs@setbasefontsize{14.4pt}}
\DeclareOption{17pt}{\bxjs@setbasefontsize{17.28pt}}
\DeclareOption{20pt}{\bxjs@setbasefontsize{20pt}}
\DeclareOption{21pt}{\bxjs@setbasefontsize{20.74pt}}
\DeclareOption{25pt}{\bxjs@setbasefontsize{24.88pt}}
\DeclareOption{30pt}{\bxjs@setbasefontsize{29.86pt}}
\DeclareOption{36pt}{\bxjs@setbasefontsize{35.83pt}}
\DeclareOption{43pt}{\bxjs@setbasefontsize{43pt}}
\DeclareOption{12Q}{\bxjs@setjbasefontsize{3mm}}
\DeclareOption{14Q}{\bxjs@setjbasefontsize{3.5mm}}
\DeclareOption{10ptj}{\bxjs@setjbasefontsize{10pt}}
\DeclareOption{10.5ptj}{\bxjs@setjbasefontsize{10.5pt}}
\DeclareOption{11ptj}{\bxjs@setjbasefontsize{11pt}}
\DeclareOption{12ptj}{\bxjs@setjbasefontsize{12pt}}
%    \end{macrocode}
%
% JSクラス互換のmagstyle設定オプション。
%    \begin{macrocode}
\DeclareOption{usemag}{\let\bxjs@magstyle\bxjs@magstyle@usemag}
\DeclareOption{nomag}{\let\bxjs@magstyle\bxjs@magstyle@nomag}
\DeclareOption{nomag*}{\let\bxjs@magstyle\bxjs@magstyle@xreal}
%    \end{macrocode}
% 
% \end{ZRnote}
%
% \paragraph{トンボオプション}
% \mbox{}
% \begin{ZRnote}
% 欧文 {\LaTeX} のカーネルではサポートされないため削除。
%
% ただしJSクラスとの互換のため以下の変数を用意する。
%
% \Note JSクラスでは、2017-01-11の改修で、「トンボオプション指定時
% のみ |\stockwidth|/|height| を定義する」という仕様に変更された。
% BXJSでの対応は要検討。
%
% \end{ZRnote}
%    \begin{macrocode}
\newdimen\stockwidth \newdimen\stockheight
%    \end{macrocode}
%
% \paragraph{面付け}
% \mbox{}
% \begin{ZRnote}
% 欧文 {\LaTeX} のカーネルではサポートされないため削除。
% \end{ZRnote}
%
% \paragraph{両面,片面オプション}
%
% \texttt{twoside} で奇数ページ・偶数ページのレイアウトが変わります。
%
% [2003-04-29] \texttt{vartwoside} でどちらのページも傍注が右側になります。
%
%    \begin{macrocode}
\DeclareOption{oneside}{\@twosidefalse \@mparswitchfalse}
\DeclareOption{twoside}{\@twosidetrue \@mparswitchtrue}
\DeclareOption{vartwoside}{\@twosidetrue \@mparswitchfalse}
%    \end{macrocode}
%
% \paragraph{二段組}
%
% \texttt{twocolumn} で二段組になります。
%
%    \begin{macrocode}
\DeclareOption{onecolumn}{\@twocolumnfalse}
\DeclareOption{twocolumn}{\@twocolumntrue}
%    \end{macrocode}
%
% \paragraph{表題ページ}
%
% \texttt{titlepage} で表題・概要を独立したページに出力します。
%
%    \begin{macrocode}
\DeclareOption{titlepage}{\@titlepagetrue}
\DeclareOption{notitlepage}{\@titlepagefalse}
%    \end{macrocode}
%
% \paragraph{右左起こし}
%
% 書籍では章は通常は奇数ページ起こしになりますが,
% \texttt{openany} で偶数ページからでも始まるようになります。
%
%    \begin{macrocode}
%<book|report>\DeclareOption{openright}{\@openrighttrue}
%<book|report>\DeclareOption{openany}{\@openrightfalse}
%    \end{macrocode}
%
% \paragraph{eqnarray環境と数式の位置}
%
% 森本さんのご教示にしたがって前に移動しました。
%
% \begin{environment}{eqnarray}
%
%    \LaTeX の |eqnarray| 環境では |&| でできるアキが大きすぎる
%    ようですので,少し小さくします。
%    また,中央の要素も |\displaystyle| にします。
%
%    \begin{macrocode}
\def\eqnarray{%
   \stepcounter{equation}%
   \def\@currentlabel{\p@equation\theequation}%
   \global\@eqnswtrue
   \m@th
   \global\@eqcnt\z@
   \tabskip\@centering
   \let\\\@eqncr
   $$\everycr{}\halign to\displaywidth\bgroup
       \hskip\@centering$\displaystyle\tabskip\z@skip{##}$\@eqnsel
      &\global\@eqcnt\@ne \hfil$\displaystyle{{}##{}}$\hfil
      &\global\@eqcnt\tw@ $\displaystyle{##}$\hfil\tabskip\@centering
      &\global\@eqcnt\thr@@ \hb@xt@\z@\bgroup\hss##\egroup
         \tabskip\z@skip
      \cr}
%    \end{macrocode}
% \end{environment}
%
% \texttt{leqno} で数式番号が左側になります。
% \texttt{fleqn} で数式が本文左端から一定距離のところに出力されます。
% 森本さんにしたがって訂正しました。
%
%    \begin{macrocode}
\DeclareOption{leqno}{\input{leqno.clo}}
\DeclareOption{fleqn}{\input{fleqn.clo}%
% fleqn用のeqnarray環境の再定義
  \def\eqnarray{%
    \stepcounter{equation}%
    \def\@currentlabel{\p@equation\theequation}%
    \global\@eqnswtrue\m@th
    \global\@eqcnt\z@
    \tabskip\mathindent
    \let\\=\@eqncr
    \setlength\abovedisplayskip{\topsep}%
    \ifvmode
      \addtolength\abovedisplayskip{\partopsep}%
    \fi
    \addtolength\abovedisplayskip{\parskip}%
    \setlength\belowdisplayskip{\abovedisplayskip}%
    \setlength\belowdisplayshortskip{\abovedisplayskip}%
    \setlength\abovedisplayshortskip{\abovedisplayskip}%
    $$\everycr{}\halign to\linewidth% $$
    \bgroup
      \hskip\@centering$\displaystyle\tabskip\z@skip{##}$\@eqnsel
      &\global\@eqcnt\@ne \hfil$\displaystyle{{}##{}}$\hfil
      &\global\@eqcnt\tw@
        $\displaystyle{##}$\hfil \tabskip\@centering
      &\global\@eqcnt\thr@@ \hb@xt@\z@\bgroup\hss##\egroup
    \tabskip\z@skip\cr
    }}
%    \end{macrocode}
%
% \paragraph{文献リスト}
%
% 文献リストをopen形式(著者名や書名の後に改行が入る)で出力します。
% これは使われることはないのでコメントアウトしてあります。
%
%    \begin{macrocode}
% \DeclareOption{openbib}{%
%   \AtEndOfPackage{%
%    \renewcommand\@openbib@code{%
%       \advance\leftmargin\bibindent
%       \itemindent -\bibindent
%       \listparindent \itemindent
%       \parsep \z@}%
%    \renewcommand\newblock{\par}}}
%    \end{macrocode}
%
% \paragraph{数式フォントとして和文フォントを登録しないオプション}
% \mbox{}
% \begin{ZRnote}
% ここは和文ドライバの管轄。
% \end{ZRnote}
%
% \paragraph{ドラフト}
%
% \texttt{draft} でoverfull boxの起きた行末に5ptの罫線を引きます。
%
% \begin{ZRnote}
% \begin{macro}{\ifjsDraft}
% JSクラスは |\ifdraft| という公開名のスイッチを用いているが、
% これは |ifdraft| パッケージと衝突するので、代わりに |\ifjsDraft|
% の名前を用い、本文開始時に |\ifdraft| が未定義の場合に限り、
% |\ifjsDraft| を |\ifdraft| にコピーする処理にする。
% \Note JSクラスの |\ifdraft| は2016/07/13版で廃止された。
% よって |\ifdraft| は2.0版で廃止を予定する。
%    \begin{macrocode}
\let\ifjsDraft\iffalse
\@onlypreamble\bxjs@draft
\def\bxjs@draft#1{%
  \expandafter\let\expandafter\ifjsDraft\csname if#1\endcsname}
\DeclareOption{draft}{\bxjs@draft{true}\setlength\overfullrule{5pt}}
\DeclareOption{final}{\bxjs@draft{false}\setlength\overfullrule{0pt}}
\AtBeginDocument{%
  \expandafter\ifx\csname ifdraft\endcsname\relax
    \expandafter\let\csname ifdraft\expandafter\endcsname
     \csname ifjsDraft\endcsname
  \fi}
%    \end{macrocode}
% \end{macro}
% \end{ZRnote}
%
% \paragraph{和文フォントメトリックの選択}
% \mbox{}
% \begin{ZRnote}
% ここは和文ドライバの管轄。
% \end{ZRnote}
%
% \paragraph{papersizeスペシャルの利用}
% \mbox{}
% \begin{ZRnote}
% |geometry| パッケージが行う。
%
% \begin{macro}{\ifbxjs@papersize}
% 〔スイッチ〕 papersizeスペシャルを出力するか。
% 既定で有効であるが、|nopapersize| オプションで無効にできる。
% \Note JSクラスでは |\ifpapersize| という制御綴だが、これは採用しない。
%    \begin{macrocode}
\newif\ifbxjs@papersize
\bxjs@papersizetrue
\DeclareOption{nopapersize}{\bxjs@papersizefalse}
\DeclareOption{papersize}{\bxjs@papersizetrue}
%    \end{macrocode}
% \end{macro}
% \end{ZRnote}
%
% \paragraph{英語化}
%
% オプション \texttt{english} を新設しました。
%
%    \begin{macrocode}
\newif\if@english
\@englishfalse
\DeclareOption{english}{\@englishtrue}
%    \end{macrocode}
%
% \paragraph{jsreport相当}
%
% オプション \texttt{report} を新設しました。
%
% \begin{ZRnote}
% BXJS では `|report|' 相当のものは別に |bxjsreport| クラスとして用意する。
% \end{ZRnote}
%
% \paragraph{jslogoパッケージの読み込み}
%
% \LaTeX 関連のロゴを再定義する\texttt{jslogo}パッケージを
% 読み込まないオプション\texttt{nojslogo}を新設しました。
% \texttt{jslogo}オプションの指定で従来どおりの動作となります。
% デフォルトは\texttt{jslogo}で,すなわちパッケージを読み込みます。
%
% \begin{ZRnote}
% BXJSクラスでは、|nojslogo| を既定とする。
% \end{ZRnote}
%    \begin{macrocode}
\newif\if@jslogo \@jslogofalse
\DeclareOption{jslogo}{\@jslogotrue}
\DeclareOption{nojslogo}{\@jslogofalse}
%    \end{macrocode}
%
% \paragraph{BXJS特有のオプションの一覧 \ZRX}
% \mbox{}
% \begin{ZRnote}
% \begin{itemize}
% \item エンジンオプション: |xelatex| 等。
% \item ドライバオプション: |dvipdfmx| 等。
% \item 複合設定オプション: |pandoc| 等。
% \item |nopapersize| : |papersize|(既定で有効)の否定。
% \item |zw|/|nozw| : |\jsZw| と等価な命令として |\zw| を
%   定義する/しない。
% \item |js|/|nojs| : JSクラスを読込済として扱う/扱わない。
% \item |precisetext|/|noprecisetext| : {\XeTeX}の``generateactualtext''を
%    有効/無効にする。
% \item |simplejasetup|/|nosimplejasetup| : {\XeTeX}の``linebreaklocale''を
%    有効/無効にする。
% \item |bigcode|/|nobigcode| : {\upTeX}でCMapとして |UTF8-UCS2| の
%    代わりに |UTF8-UTF16| を使う/使わない。
% \item |oldfontcommands|/|nooldfontcommands| : 古い“二文字フォント命令”
%    に対する警告を抑止する/しない。
% \item |base=|\meta{dimen} : 基底フォントサイズを直接指定する。
%   (|xxpt| オプションの代用なので、既定値は10\,ptである。)
% \item |jbase=|\meta{dimen} : 基底フォントサイズを“和文規準で”
%   直接指定する。
% \item |scale=|\meta{real} : 和文フォントのスケールを表すマクロ |\jsScale|
%   の値を設定する。
%   もちろんこの値を何らかの方法で和文処理モジュールに渡さないと意味を成さない。
%   既定値は 0.924715(= 13\,Q/10\,pt)。
% \item |noscale| : |scale=1| と等価。
% \item |mag=|\meta{int} : |\mag| 値の直接設定。既定は |base| から算出する。
% \item |paper={|\meta{dimen:width}|}{|\meta{dimen:height}|}| : 用紙サイズ設定。
%   用紙サイズオプションの代用で、既定値は |a4paper| 相当。
% \item |ja=|\meta{name} : 使用する和文ドライバの指定。
% \item |jafont=|\meta{name} : 和文フォントプリセットの指定。
% \item |japaram=|\meta{name} : 和文フォントパラメタの指定。
% \item |magstyle=|\meta{name} : “版面拡大”の実現方法の選択。
% \item |dvi=|\meta{name} : DVIモードの時のみに参照されるドライバ指定。
% \item |geometry=|\{|class|\OR|user|\} : |geometry| パッケージの読込を自動的に
%   行うかユーザに任せるか。
% \item |fancyhdr=|\meta{bool} : |fancyhdr| パッケージ用の調整を行うか。
% \item |layout=|\meta{name} : レイアウト変種の指定。
% \item |textwidth-limit=|\meta{number} : |bxjsbook| における、
%   |\textwidth| の上限の全角単位での値。
% \item |paragraph-mark=|\meta{char} : パラグラフのマーク。
% \item |whole-zw-lines| : (予定)
% \item |dummy-microtype| : (予定)
% \item |hyperref-enc| : (予定)
% \end{itemize}
%
% \begin{macro}{\bxjs@invscale}
% |\bxjs@invscale| は{\TeX}における「長さのスケール」の逆関数を求めるもの。
% 例えば |\bxjs@invscale\dimX{1.3}| は |\dimX=1.3\dimX| の
% 逆の演算を行う。
% \Note 局所化の |\begingroup|~|\endgroup| について、
% 以前は |\group|~|\egroup| を使っていたが、これだと数詞モード中では
% 空のサブ数式を生み出してしまうため修正した。
%    \begin{macrocode}
\mathchardef\bxjs@csta=259
\def\bxjs@invscale#1#2{%
  \begingroup \@tempdima=#1\relax \@tempdimb#2\p@\relax
    \@tempcnta\@tempdima \multiply\@tempcnta\@cclvi
    \divide\@tempcnta\@tempdimb \multiply\@tempcnta\@cclvi
    \@tempcntb\p@ \divide\@tempcntb\@tempdimb
    \advance\@tempcnta-\@tempcntb \advance\@tempcnta-\tw@
    \@tempdimb\@tempcnta\@ne
    \advance\@tempcnta\@tempcntb \advance\@tempcnta\@tempcntb
    \advance\@tempcnta\bxjs@csta \@tempdimc\@tempcnta\@ne
    \@whiledim\@tempdimb<\@tempdimc\do{%
      \@tempcntb\@tempdimb \advance\@tempcntb\@tempdimc
      \advance\@tempcntb\@ne \divide\@tempcntb\tw@
      \ifdim #2\@tempcntb>\@tempdima
        \advance\@tempcntb\m@ne \@tempdimc=\@tempcntb\@ne
      \else \@tempdimb=\@tempcntb\@ne \fi}%
    \xdef\bxjs@gtmpa{\the\@tempdimb}%
  \endgroup #1=\bxjs@gtmpa\relax}
%    \end{macrocode}
% \end{macro}
%
% \end{ZRnote}
%
% \paragraph{複合設定オプション \ZRX}
% \mbox{}
% \begin{ZRnote}
% 複合設定オプションとは、
% 「エンジンやドライバや和文ドライバの設定を含む、
% 複数の設定を一度に行うオプション」
% のことである。
% ある特定の設定を短く書く必要性が高いと判断される場合に
% 用意される。
%
% \begin{macro}{\bxjs@composite@proc}
% 複合設定オプションのための遅延処理マクロ。
%    \begin{macrocode}
\let\bxjs@composite@proc\relax
%    \end{macrocode}
% \end{macro}
%
% |pandoc| オプションは、Pandocで{\LaTeX}用の既定テンプレートを
% 用いて他形式から{\LaTeX}(およびPDF)形式に変換する用途に
% 最適化した設定を与える。
%    \begin{macrocode}
\DeclareOption{pandoc}{%
%    \end{macrocode}
% 和文ドライバを |pandoc| に、エンジン指定を |autodetect-engine| に
% 変更する。
% \Note 実際の和文ドライバ・エンジン設定より優先される。
%    \begin{macrocode}
  \def\bxjs@composite@proc{%
    \bxjs@oldfontcommandstrue
    \setkeys{bxjs}{ja=pandoc}%
    \let\bxjs@engine@given=*}%
%    \end{macrocode}
% ドライバオプションを |dvi=dvipdfmx| 相当に変更する。
% \Note これは実際のドライバ設定で上書きできる
% (オプション宣言順に注意)。
% \TODO できない気がする…。
%    \begin{macrocode}
  \def\bxjs@driver@opt{dvipdfmx}%
  \bxjs@dvi@opttrue}
%    \end{macrocode}
%
% \end{ZRnote}
%
% \paragraph{エンジン・ドライバオプション \ZRX}
% \mbox{}
% \begin{ZRnote}
% \begin{macro}{\bxjs@engine@given}
% オプションで明示されたエンジンの種別。
%    \begin{macrocode}
%\let\bxjs@engine@given\@undefined
%    \end{macrocode}
% \end{macro}
%
% \begin{macro}{\bxjs@engine@opt}
% 明示されたエンジンのオプション名。
%    \begin{macrocode}
%\let\bxjs@engine@opt\@undefined
%    \end{macrocode}
% \end{macro}
%
% エンジン明示指定のオプションの処理。
% \Note 0.9pre版の暫定仕様と異なり、エンジン名は |...latex|
% に限定する。
% |xetex| や |pdftex| は一般的な{\LaTeX}の慣習に従って
% 「ドライバの指定」とみなすべきだから。
%
%    \begin{macrocode}
\DeclareOption{autodetect-engine}{%
  \let\bxjs@engine@given=*}
\DeclareOption{latex}{%
  \def\bxjs@engine@opt{latex}%
  \let\bxjs@engine@given=n}
\DeclareOption{platex}{%
  \def\bxjs@engine@opt{platex}%
  \let\bxjs@engine@given=j}
\DeclareOption{uplatex}{%
  \def\bxjs@engine@opt{uplatex}%
  \let\bxjs@engine@given=u}
\DeclareOption{xelatex}{%
  \def\bxjs@engine@opt{xelatex}%
  \let\bxjs@engine@given=x}
\DeclareOption{pdflatex}{%
  \def\bxjs@engine@opt{pdflatex}%
  \let\bxjs@engine@given=p}
\DeclareOption{lualatex}{%
  \def\bxjs@engine@opt{lualatex}%
  \let\bxjs@engine@given=l}
\DeclareOption{platex-ng}{%
  \def\bxjs@engine@opt{platex-ng}%
  \let\bxjs@engine@given=g}
\DeclareOption{platex-ng*}{%
  \def\bxjs@engine@opt{platex-ng*}%
  \let\bxjs@platexng@nodrv=t%
  \let\bxjs@engine@given=g}
%    \end{macrocode}
%
% \begin{macro}{\bxjs@driver@given}
% オプションで明示されたドライバの種別。
%    \begin{macrocode}
%\let\bxjs@driver@given\@undefined
\let\bxjs@driver@@dvimode=0
\let\bxjs@driver@@dvipdfmx=1
\let\bxjs@driver@@pdfmode=2
\let\bxjs@driver@@xetex=3
%    \end{macrocode}
% \end{macro}
%
% \begin{macro}{\bxjs@driver@opt}
% 明示された「ドライバ指定」のオプション名。
%    \begin{macrocode}
%\let\bxjs@driver@opt\@undefined
%    \end{macrocode}
% \end{macro}
%
%    \begin{macrocode}
\DeclareOption{dvips}{%
  \def\bxjs@driver@opt{dvips}%
  \let\bxjs@driver@given\bxjs@driver@@dvimode}
\DeclareOption{dviout}{%
  \def\bxjs@driver@opt{dviout}%
  \let\bxjs@driver@given\bxjs@driver@@dvimode}
\DeclareOption{xdvi}{%
  \def\bxjs@driver@opt{xdvi}%
  \let\bxjs@driver@given\bxjs@driver@@dvimode}
\DeclareOption{dvipdfmx}{%
  \def\bxjs@driver@opt{dvipdfmx}%
  \let\bxjs@driver@given\bxjs@driver@@dvipdfmx}
\DeclareOption{pdftex}{%
  \def\bxjs@driver@opt{pdftex}%
  \let\bxjs@driver@given\bxjs@driver@@pdfmode}
\DeclareOption{luatex}{%
  \def\bxjs@driver@opt{luatex}%
  \let\bxjs@driver@given\bxjs@driver@@pdfmode}
\DeclareOption{xetex}{%
  \def\bxjs@driver@opt{xetex}%
  \let\bxjs@driver@given\bxjs@driver@@xetex}
%    \end{macrocode}
%
% 「もしDVIモードであればドライバを |dvipdfmx| にする」
% というオプション。
% \Note 1.2版で |dvi| オプションが新設されたが、互換性のため
% このオプションも残す。
%    \begin{macrocode}
\DeclareOption{dvipdfmx-if-dvi}{%
  \setkeys{bxjs}{dvi=dvipdfmx}}
%    \end{macrocode}
%
% \end{ZRnote}
%
% \paragraph{その他のBXJS独自オプション \ZRX}
% \mbox{}
% \begin{ZRnote}
% \begin{macro}{\ifbxjs@usezw}
% |\jsZw| の同義語として |\zw| を使えるようにするか。
% 既定は真。
%    \begin{macrocode}
\newif\ifbxjs@usezw \bxjs@usezwtrue
%    \end{macrocode}
% \end{macro}
%
% |zw|、|nozw| オプションの定義。
%    \begin{macrocode}
\DeclareOption{nozw}{%
  \bxjs@usezwfalse}
\DeclareOption{zw}{%
  \bxjs@usezwtrue}
%    \end{macrocode}
%
% \begin{macro}{\ifbxjs@disguise@js}
% JSクラスの派生クラスのふりをするか。
% 既定は真。
%    \begin{macrocode}
\newif\ifbxjs@disguise@js \bxjs@disguise@jstrue
%    \end{macrocode}
% \end{macro}
%
% |nojs|、|js| オプションの定義。
%    \begin{macrocode}
\DeclareOption{nojs}{%
  \bxjs@disguise@jsfalse}
\DeclareOption{js}{%
  \bxjs@disguise@jstrue}
%    \end{macrocode}
%
% \begin{macro}{\ifbxjs@precisetext}
% {\XeTeX}の``generateactualtext''を有効にするか。
% 既定は偽。
%    \begin{macrocode}
\newif\ifbxjs@precisetext
%    \end{macrocode}
% \end{macro}
%
% |noprecisetext|/|precisetext| オプションの定義。
%    \begin{macrocode}
\DeclareOption{noprecisetext}{%
  \bxjs@precisetextfalse}
\DeclareOption{precisetext}{%
  \bxjs@precisetexttrue}
%    \end{macrocode}
%
% \begin{macro}{\ifbxjs@simplejasetup}
% {\XeTeX}の``linebreaklocale''を有効にするか。
% 既定は真(であるが多くの場合は後に無効化される)。
%    \begin{macrocode}
\newif\ifbxjs@simplejasetup \bxjs@simplejasetuptrue
%    \end{macrocode}
% \end{macro}
%
% |nosimplejasetup|/|simplejasetup| オプションの定義。
%    \begin{macrocode}
\DeclareOption{nosimplejasetup}{%
  \bxjs@simplejasetupfalse}
\DeclareOption{simplejasetup}{%
  \bxjs@simplejasetuptrue}
%    \end{macrocode}
%
% \begin{macro}{\ifbxjs@bigcode}
% {\upTeX}で有効化するToUnicode CMapとして
% 「|UTF8-UCS2|」の代わりに「|UTF8-UTF16|」を使うか。
% BMP外の文字に対応できる「|UTF8-UTF16|」の方が望ましいのであるが、
% このファイルが利用可能かの確実な判定が困難であるため、
% オプションで指定することとする。
%    \begin{macrocode}
\newif\ifbxjs@bigcode \bxjs@bigcodefalse
%    \end{macrocode}
% その上で、「{\TeX}環境がある程度新しければ利用可能であろう」
% と判断し |bxjs@bigcode| の既定値を真とする。
% 具体的な判断基準として、
% 「{\TeX}のバージョンが3.14159265(2014年1月)以上であるか」
% を採用する。
%    \begin{macrocode}
\edef\bxjs@tmpa{\expandafter\noexpand\csname\endcsname}
\def\bxjs@tmpb#1 #2#3\@nil{%
  \ifx1#2\bxjs@bigcodetrue \fi}
\expandafter\bxjs@tmpb\meaning\bxjs@tmpa1 0\@nil
%    \end{macrocode}
% \end{macro}
%
% |nobigcode|/|bigcode| オプションの定義。
%    \begin{macrocode}
\DeclareOption{nobigcode}{%
  \bxjs@bigcodefalse}
\DeclareOption{bigcode}{%
  \bxjs@bigcodetrue}
%    \end{macrocode}
%
% \begin{macro}{\ifbxjs@oldfontcommands}
% |\allowoldfontcommands| を既定で有効にするか。
%    \begin{macrocode}
\newif\ifbxjs@oldfontcommands
%    \end{macrocode}
% \end{macro}
%
% |nooldfontcommands|、|oldfontcommands| オプションの定義。
% \Note |oldfontcommands| オプションの名前は\Pkg{memoir}クラスに倣った。
% ちなみに\Pkg{KOMA-Script}では |enabledeprecatedfontcommands| であるが
% これはチョットアレなので避けた。
%    \begin{macrocode}
\DeclareOption{nooldfontcommands}{%
  \bxjs@oldfontcommandsfalse}
\DeclareOption{oldfontcommands}{%
  \bxjs@oldfontcommandstrue}
%    \end{macrocode}
%
% \end{ZRnote}
%
% \paragraph{keyval型のオプション \ZRX}
% \mbox{}
% \begin{ZRnote}
%    \begin{macrocode}
\def\bxjs@setkey{%
  \expandafter\bxjs@setkey@a\expandafter{\CurrentOption}}
\def\bxjs@setkey@a{\bxjs@safe@setkeys{bxjs}}
\DeclareOption*{\bxjs@setkey}
%    \end{macrocode}
%
% \begin{macro}{\bxjs@safe@setkeys}
% 未知のキーに対してエラー無しで無視する |\setkeys|。
%    \begin{macrocode}
\def\bxjs@safe@setkeys#1#2{%
  \let\bxjs@KV@errx\KV@errx
  \let\KV@errx\bxjs@safe@setkeys@a
  \setkeys{#1}{#2}%
  \let\KV@errx\bxjs@KV@errx}
\def\bxjs@safe@setkeys@a#1{}
%    \end{macrocode}
% \end{macro}
%
% \begin{macro}{\bxjs@set@keyval}
% |\bxjs@set@keyval{|\meta{key}|}{|\meta{value}|}{|\meta{error}|}|\par
% |\bxjs@kv@|\meta{key}|@|\meta{value} が定義済ならそれを実行し、
% 未定義ならエラーを出す。
%    \begin{macrocode}
\def\bxjs@set@keyval#1#2#3{%
  \expandafter\let\expandafter\bxjs@next\csname bxjs@kv@#1@#2\endcsname
  \ifx\bxjs@next\relax
    \bxjs@error@keyval{#1}{#2}%
    #3%
  \else \bxjs@next
  \fi}
\@onlypreamble\bxjs@error@keyval
\def\bxjs@error@keyval#1#2{%
  \ClassError\bxjs@clsname
   {Invalid value '#2' for option #1}\@ehc}
%    \end{macrocode}
% \end{macro}
%
% \begin{macro}{\ifbxjs@scaleset}
% 和文スケール値が指定されたか。
%    \begin{macrocode}
\newif\ifbxjs@scaleset
%    \end{macrocode}
% \end{macro}
%
% \begin{macro}{\jsScale}
% 〔実数値マクロ〕
% 和文スケール値。
%    \begin{macrocode}
\def\jsScale{0.924715}
%    \end{macrocode}
% \end{macro}
%
% |base| オプションの処理。
%    \begin{macrocode}
\define@key{bxjs}{base}{\bxjs@setbasefontsize{#1}}
%    \end{macrocode}
%
% |jbase| オプションの処理。
% ここでは |\jsScale| の値を使用する。
% |scale| の処理との順序依存を消すため、
% |jbase| の処理の実行を遅延させている。
%     \begin{macrocode}
\@onlypreamble\bxjs@do@opt@jbase
\let\bxjs@do@opt@jbase\relax
\define@key{bxjs}{jbase}{\bxjs@setjbasefontsize{#1}}
\def\bxjs@setjbasefontsize#1{%
  \def\bxjs@do@opt@jbase{%
    \bxjs@setbasefontlength\@tempdima{#1}%
    \bxjs@invscale\@tempdima\jsScale
    \bxjs@setbasefontsize{\@tempdima}}}
%    \end{macrocode}
%
% |scale| オプションの処理。
%    \begin{macrocode}
\define@key{bxjs}{scale}{%
  \bxjs@scalesettrue
  \edef\jsScale{#1}}
%    \end{macrocode}
%
% |noscale| オプションの処理。
%    \begin{macrocode}
\DeclareOption{noscale}{%
  \bxjs@scalesettrue
  \def\jsScale{1}}
%    \end{macrocode}
%
% \begin{macro}{\bxjs@param@mag}
% |mag| オプションの値。
%    \begin{macrocode}
\let\bxjs@param@mag\relax
%    \end{macrocode}
% \end{macro}
%
% |mag| オプションの処理。
%    \begin{macrocode}
\define@key{bxjs}{mag}{\edef\bxjs@param@mag{#1}}
%    \end{macrocode}
%
% |paper| オプションの処理。
%    \begin{macrocode}
\define@key{bxjs}{paper}{\edef\bxjs@param@paper{#1}}
%    \end{macrocode}
%
% \begin{macro}{\bxjs@jadriver}
% 和文ドライバの名前。
%    \begin{macrocode}
\let\bxjs@jadriver\relax
%\let\bxjs@jadriver@given\@undefined
%    \end{macrocode}
% \end{macro}
%
% |ja| オプションの処理。
% \Note |jadriver| は0.9版で用いられた旧称。
% \Note 単なる |ja| という指定は無視される(Pandoc 対策)。
%    \begin{macrocode}
\define@key{bxjs}{jadriver}{\edef\bxjs@jadriver{#1}}
\define@key{bxjs}{ja}[\relax]{%
  \ifx\relax#1\else\edef\bxjs@jadriver{#1}\fi}
%    \end{macrocode}
%
% \begin{macro}{\jsJaFont}
% 和文フォント設定の名前。
%    \begin{macrocode}
\let\jsJaFont\@empty
%    \end{macrocode}
% \end{macro}
%
% |jafont| オプションの処理。
%    \begin{macrocode}
\define@key{bxjs}{jafont}{\edef\jsJaFont{#1}}
%    \end{macrocode}
%
% \begin{macro}{\jsJaParam}
% 和文ドライバパラメタの文字列。
%    \begin{macrocode}
\let\jsJaParam\@empty
%    \end{macrocode}
% \end{macro}
%
% |japaram| オプションの処理。
%    \begin{macrocode}
\define@key{bxjs}{japaram}{\edef\jsJaParam{#1}}
%    \end{macrocode}
%
% \begin{macro}{\bxjs@magstyle}
% magstyle設定値。(古いイマイチな名前。)
%    \begin{macrocode}
\let\bxjs@magstyle@mag=m
\let\bxjs@magstyle@real=r
\let\bxjs@magstyle@xreal=x
%    \end{macrocode}
% (新しい素敵な名前。)
% \Note ただし制御綴としては、|*|付の名前は扱い難いので、|\bxjs@magstyle@xreal|
% の方を優先させる。
%    \begin{macrocode}
\let\bxjs@magstyle@usemag\bxjs@magstyle@mag
\let\bxjs@magstyle@nomag\bxjs@magstyle@real
\expandafter\let\csname bxjs@magstyle@nomag*\endcsname\bxjs@magstyle@xreal
%    \end{macrocode}
% |\bxjs@magstyle@default| は既定の値を表す。
%    \begin{macrocode}
\let\bxjs@magstyle@default\bxjs@magstyle@usemag
\ifx l\jsEngine \ifnum\luatexversion>86
  \let\bxjs@magstyle@default\bxjs@magstyle@xreal
\fi\fi
\ifjsWithpTeXng
  \let\bxjs@magstyle@default\bxjs@magstyle@xreal
\fi
\let\bxjs@magstyle\bxjs@magstyle@default
%    \end{macrocode}
% \end{macro}
%
% |magstyle| オプションの処理。
%    \begin{macrocode}
\define@key{bxjs}{magstyle}{%
  \expandafter\let\expandafter\bxjs@magstyle\csname
   bxjs@magstyle@#1\endcsname
  \ifx\bxjs@magstyle\relax
    \ClassError\bxjs@clsname
     {Invalid value '#1' for option magstyle}\@ehc
    \let\bxjs@magstyle\bxjs@magstyle@default
  \fi}
%    \end{macrocode}
%
% \begin{macro}{\bxjs@geometry}
% |geometry| オプションの値。
%    \begin{macrocode}
\let\bxjs@geometry@class=c
\let\bxjs@geometry@user=u
\let\bxjs@geometry\bxjs@geometry@class
%    \end{macrocode}
% \end{macro}
%
% |geometry| オプションの処理。
%    \begin{macrocode}
\define@key{bxjs}{geometry}{%
  \expandafter\let\expandafter\bxjs@geometry\csname
   bxjs@geometry@#1\endcsname
  \ifx\bxjs@geometry\relax
    \ClassError\bxjs@clsname
     {Invalid value '#1' for option geometry}\@ehc
    \let\bxjs@geometry\bxjs@geometry@class
  \fi}
%    \end{macrocode}
%
% \begin{macro}{\ifbxjs@fancyhdr}
% 〔スイッチ〕
% |fancyhdr| パッケージに対する調整を行うか。
%    \begin{macrocode}
\newif\ifbxjs@fancyhdr \bxjs@fancyhdrtrue
%    \end{macrocode}
% \end{macro}
%
% |fancyhdr| オプションの処理。
%    \begin{macrocode}
\let\bxjs@kv@fancyhdr@true\bxjs@fancyhdrtrue
\let\bxjs@kv@fancyhdr@false\bxjs@fancyhdrfalse
\define@key{bxjs}{fancyhdr}{%
  \bxjs@set@keyval{fancyhdr}{#1}{}}
%    \end{macrocode}
%
% \begin{macro}{\ifbxjs@dvi@opt}
% |dvi| オプションが指定されたか。
%    \begin{macrocode}
\newif\ifbxjs@dvi@opt
%    \end{macrocode}
% \end{macro}
%
% DVIモードのドライバとドライバ種別との対応。
%    \begin{macrocode}
\let\bxjs@dvidriver@@dvipdfmx=\bxjs@driver@@dvipdfmx
\let\bxjs@dvidriver@@dvips=\bxjs@driver@@dvimode
\let\bxjs@dvidriver@@dviout=\bxjs@driver@@dvimode
\let\bxjs@dvidriver@@xdvi=\bxjs@driver@@dvimode
%    \end{macrocode}
%
% |dvi| オプションの処理。
%    \begin{macrocode}
\define@key{bxjs}{dvi}{%
  \expandafter\let\expandafter\bxjs@tmpa\csname
   bxjs@dvidriver@@#1\endcsname
  \ifx\bxjs@tmpa\relax
    \ClassError\bxjs@clsname
     {Invalid value '#1' for option dvi}\@ehc
  \else
%    \end{macrocode}
% |\bxjs@driver@given| を未定義にしていることに注意。
%    \begin{macrocode}
    \def\bxjs@driver@opt{#1}%
    \let\bxjs@driver@given\@undefined
    \bxjs@dvi@opttrue
  \fi}
%    \end{macrocode}
%
% \begin{macro}{\ifbxjs@layout@buggyhmargin}
% 〔スイッチ〕
% |bxjsbook| の左右マージンがアレか。
% \Note 既定はアレだが1.3版で非アレになる予定。
%    \begin{macrocode}
\newif\ifbxjs@layout@buggyhmargin \bxjs@layout@buggyhmarginfalse
%    \end{macrocode}
% \end{macro}
%
% |layout| オプションの処理。
%    \begin{macrocode}
\@namedef{bxjs@kv@layout@v1}{%
  \bxjs@layout@buggyhmargintrue}
\@namedef{bxjs@kv@layout@v2}{%
  \bxjs@layout@buggyhmarginfalse}
\define@key{bxjs}{layout}{%
  \bxjs@set@keyval{layout}{#1}{}}
%    \end{macrocode}
%
% \begin{macro}{\bxjs@textwidth@limit}
% |textwidth-limit| の指定値。
% |\textwidth| の上限。
%    \begin{macrocode}
%\let\bxjs@textwidth@limit\@undefined
\define@key{bxjs}{textwidth-limit}{%
  \edef\bxjs@textwidth@limit{#1}}
%    \end{macrocode}
% \end{macro}
%
% \begin{macro}{\bxjs@paragraph@mark}
% |paragraph-mark| の指定値。
% パラグラフのマーク。
%    \begin{macrocode}
%\let\bxjs@paragraph@mark\@undefined
\define@key{bxjs}{paragraph-mark}{%
  \edef\bxjs@paragraph@mark{#1}}
%    \end{macrocode}
% \end{macro}
%
% \begin{macro}{\ifbxjs@whole@zw@lines}
% 〔スイッチ〕 |whole-zw-lines| の指定値。
%    \begin{macrocode}
\newif\ifbxjs@whole@zw@lines \bxjs@whole@zw@linestrue
\let\bxjs@kv@wholezwlines@true\bxjs@whole@zw@linestrue
\let\bxjs@kv@wholezwlines@false\bxjs@whole@zw@linesfalse
\define@key{bxjs}{whole-zw-lines}{\bxjs@set@keyval{wholezwlines}{#1}{}}
%    \end{macrocode}
% \end{macro}
%
% \begin{macro}{\ifbxjs@dummy@microtype}
% 〔スイッチ〕 |dummy-microtype| の指定値。
%    \begin{macrocode}
\newif\ifbxjs@dummy@microtype \bxjs@dummy@microtypetrue
\let\bxjs@kv@dummymicrotype@true\bxjs@dummy@microtypetrue
\let\bxjs@kv@dummymicrotype@false\bxjs@dummy@microtypefalse
\define@key{bxjs}{dummy-microtype}{\bxjs@set@keyval{dummymicrotype}{#1}{}}
%    \end{macrocode}
% \end{macro}
%
% \begin{macro}{\ifbxjs@hyperref@enc}
% 〔スイッチ〕 |hyperref-enc| の指定値。
%    \begin{macrocode}
\newif\ifbxjs@hyperref@enc \bxjs@hyperref@enctrue
\let\bxjs@kv@hyperrefenc@true\bxjs@hyperref@enctrue
\let\bxjs@kv@hyperrefenc@false\bxjs@hyperref@encfalse
\define@key{bxjs}{hyperref-enc}{\bxjs@set@keyval{hyperrefenc}{#1}{}}
%    \end{macrocode}
% \end{macro}
%
% \end{ZRnote}
%
% \paragraph{オプションの実行}
% \mbox{}
% \begin{ZRnote}
% {\LaTeX}の実装では、クラスやパッケージのオプションのトークン列の
% 中に |{ }| が含まれると正常に処理ができない。
% これに対処する為 |\@removeelement| の実装に少し手を加える
% (仕様は変わらない)。
% \Note クラスに |\DeclareOption*| がある場合は |\@unusedoptions|
% は常に空のままであることを利用している。
%    \begin{macrocode}
\let\bxjs@ltx@removeelement\@removeelement
\def\@removeelement#1#2#3{%
  \def\reserved@a{#2}%
  \ifx\reserved@a\@empty \let#3\@empty
  \else \bxjs@ltx@removeelement{#1}{#2}{#3}%
  \fi}
%    \end{macrocode}
%
% \end{ZRnote}
%
% デフォルトのオプションを実行し,|dvi| ファイルの先頭にdvipsのpapersize
% specialを書き込みます。このspecialはdvipsや最近のdvioutが対応しています。
% |multicols| や |url| を |\RequirePackage| するのはやめました。
%
%    \begin{macrocode}
%<article>\ExecuteOptions{a4paper,oneside,onecolumn,notitlepage,final}
%<report>\ExecuteOptions{a4paper,oneside,onecolumn,titlepage,openany,final}
%<book>\ExecuteOptions{a4paper,twoside,onecolumn,titlepage,openright,final}
%<slide>\ExecuteOptions{36pt,a4paper,landscape,oneside,onecolumn,titlepage,final}
\ProcessOptions\relax
\bxjs@composite@proc
%    \end{macrocode}
%
% \begin{ZRnote}
%
% グローバルオプションのトークン列に |{ }| が含まれていると、
% やはり後のパッケージの読込処理で不具合を起こすようである
% (|\ProcessOptions*| がエラーになる)。
% 従って、このようなオプションは除外することにする。
%    \begin{macrocode}
\@onlypreamble\bxjs@purge@brace@elts
\def\bxjs@purge@brace@elts{%
  \def\bxjs@tmpa{\@gobble}%
  \expandafter\bxjs@purge@be@a\@classoptionslist,\@nil,%
  \let\@classoptionslist\bxjs@tmpa}
\@onlypreamble\bxjs@purge@be@a
\def\bxjs@purge@be@a#1,{%
  \ifx\@nil#1\relax\else
    \bxjs@purge@be@b#1{}\@nil
    \if@tempswa \edef\bxjs@tmpa{\bxjs@tmpa,#1}\fi
    \expandafter\bxjs@purge@be@a
  \fi}
\@onlypreamble\bxjs@purge@be@b
\def\bxjs@purge@be@b#1#{\bxjs@purge@be@c}
\@onlypreamble\bxjs@purge@be@c
\def\bxjs@purge@be@c#1\@nil{%
  \ifx\@nil#1\@nil \@tempswatrue \else \@tempswafalse \fi}
\bxjs@purge@brace@elts
%    \end{macrocode}
%
% |papersize|、|10pt|、|noscale| の各オプションは他のパッケージと衝突を
% 起こす可能性があるため、グローバルオプションから外す。
%
%    \begin{macrocode}
\@expandtwoargs\@removeelement
  {papersize}\@classoptionslist\@classoptionslist
\@expandtwoargs\@removeelement
  {10pt}\@classoptionslist\@classoptionslist
\@expandtwoargs\@removeelement
  {noscale}\@classoptionslist\@classoptionslist
%    \end{macrocode}
%
% 現在の(正規化前の)和文ドライバの値を\
% |\bxjs@jadriver@given| に保存する。
%    \begin{macrocode}
\ifx\bxjs@jadriver\relax\else
  \let\bxjs@jadriver@given\bxjs@jadriver
\fi
%    \end{macrocode}
%
% エンジン明示指定のオプションが与えられた場合は、
% それが実際のエンジンと一致するかを検査する。
%    \begin{macrocode}
\let\bxjs@tmpb\jsEngine
\ifx j\bxjs@tmpb\ifjsWithpTeXng
  \let\bxjs@tmpb=g
\fi\fi
\ifx j\bxjs@tmpb\ifjsWithupTeX
  \let\bxjs@tmpb=u
\fi\fi
\ifx p\bxjs@tmpb\ifjsInPdfMode\else
  \let\bxjs@tmpb=n
\fi\fi
%    \end{macrocode}
% (この時点で |\bxjs@tmpb| は |\bxjs@engine@given| と
% 同じ規則で分類したコードをもっている。)
%    \begin{macrocode}
\ifx *\bxjs@engine@given
  \let\bxjs@engine@given\bxjs@tmpb
%    \end{macrocode}
% エンジン指定が |autodetect-engine| であり、かつ実際のエンジンが
% {(u)\pLaTeX}だった場合は、本来のエンジンオプションを
% グローバルオプションに加える。
%    \begin{macrocode}
  \ifx j\bxjs@engine@given
    \g@addto@macro\@classoptionslist{,platex}
  \else\ifx u\bxjs@engine@given
    \g@addto@macro\@classoptionslist{,uplatex}
  \fi\fi
\fi
\ifx\bxjs@engine@given\@undefined\else
  \ifx\bxjs@engine@given\bxjs@tmpb\else
    \ClassError\bxjs@clsname
     {Option '\bxjs@engine@opt' used on wrong engine}\@ehc
  \fi
\fi
%    \end{macrocode}
%
% エンジンが{\pTeX-ng}の場合、グローバルオプションに |uplatex| を
% 追加する。
%    \begin{macrocode}
\ifjsWithpTeXng
  \g@addto@macro\@classoptionslist{,uplatex}
\fi
%    \end{macrocode}
%
% ドライバ指定のオプションが与えられた場合は、
% それがエンジンと整合するかを検査する。
%    \begin{macrocode}
\@tempswatrue
\ifx \bxjs@driver@given\@undefined\else
  \ifjsInPdfMode
    \ifx\bxjs@driver@given\bxjs@driver@@pdfmode\else
      \@tempswafalse
    \fi
  \else\ifx x\jsEngine
    \ifx\bxjs@driver@given\bxjs@driver@@xetex\else
      \@tempswafalse
    \fi
  \else
    \ifx\bxjs@driver@given\bxjs@driver@@pdfmode
      \@tempswafalse
    \else\ifx\bxjs@driver@given\bxjs@driver@@xetex
      \@tempswafalse
    \fi\fi
    \ifjsWithpTeXng\ifx\bxjs@driver@given\bxjs@driver@@dvipdfmx\else
      \@tempswafalse
    \fi\fi
  \fi\fi
\fi
\if@tempswa\else
  \ClassError\bxjs@clsname
   {Option '\bxjs@driver@opt' used on wrong engine}\@ehc
\fi
%    \end{macrocode}
%
% DVI出力のエンジンである場合の追加処理。
%    \begin{macrocode}
\ifjsInPdfMode \@tempswafalse
\else\ifx x\jsEngine \@tempswafalse
\else\ifjsWithpTeXng \@tempswafalse
\else \@tempswatrue
\fi\fi\fi
\if@tempswa
%    \end{macrocode}
% ドライバオプションがない場合は警告を出す。
% \Note ただし |ja| 非指定の場合はスキップする
% (0.3 版との互換性のため)。
%    \begin{macrocode}
  \ifx\bxjs@driver@opt\@undefined \ifx\bxjs@jadriver@given\@undefined\else
    \ClassWarningNoLine\bxjs@clsname
    {No driver option is given}
  \fi\fi
%    \end{macrocode}
% |dvi=XXX| が指定されていた場合は、
% |XXX| が指定された時と同じ動作にする。
% (グローバルオプションに |XXX| を追加する。)
%    \begin{macrocode}
  \ifbxjs@dvi@opt
    \edef\bxjs@nxt{%
      \let\noexpand\bxjs@driver@given
       \csname bxjs@dvidriver@@\bxjs@driver@opt\endcsname
      \noexpand\g@addto@macro\noexpand\@classoptionslist
       {,\bxjs@driver@opt}%
    }\bxjs@nxt
  \fi
\fi
%    \end{macrocode}
%
% エンジンが{\pTeX-ng}の場合、グローバルオプションに |dvipdfmx|
% を追加する。
% ただし、エンジンオプションが |platex-ng*|(|*|付)の場合、および
% 既に |dvipdfmx| が指定されている場合を除く。
%    \begin{macrocode}
\ifjsWithpTeXng
  \ifx\bxjs@driver@given\bxjs@driver@@dvipdfmx
    \let\bxjs@platexng@nodrv\@undefined
  \else\ifx t\bxjs@platexng@nodrv\else
    \g@addto@macro\@classoptionslist{,dvipdfmx}
  \fi\fi
\fi
%    \end{macrocode}
%
% |\bxjs@jadriver| の正規化。
% 値が未指定の場合は |minimal| に変える。
% ただしエンジンが{(u)\pTeX}である場合は |standard|
% に変える。
%    \begin{macrocode}
\def\bxjs@@minimal{minimal}
\ifx\bxjs@jadriver\relax
  \ifx j\jsEngine
    \def\bxjs@jadriver{standard}
  \else
    \let\bxjs@jadriver\bxjs@@minimal
  \fi
\fi
%    \end{macrocode}
%
% エンジンオプションがない場合はエラーを出す。
% \Note ただし |ja| 非指定の場合はスキップする。
%    \begin{macrocode}
\ifx\bxjs@jadriver@given\@undefined\else
  \ifx\bxjs@engine@given\@undefined
    \ClassError\bxjs@clsname
     {An engine option must be explicitly given}%
     {When you use a Japanese-driver you must specify a correct\MessageBreak
      engine option.\MessageBreak\@ehc}
\fi\fi
%    \end{macrocode}
%
% 新しいLua{\TeX}(0.87版以降)ではmagがアレなので、
% |magstyle=usemag| が指定されていた場合はエラーを出す。
% (この場合の既定値は |nomag*| であり、
% エラーの場合は既定値に置き換えられる。)
%    \begin{macrocode}
\ifx\bxjs@magstyle@default\bxjs@magstyle@mag\else
  \ifx\bxjs@magstyle\bxjs@magstyle@mag
    \let\bxjs@magstyle\bxjs@magstyle@default
    \ClassError\bxjs@clsname
     {The engine does not support 'magstyle=usemag'}%
     {LuaTeX v0.87 or later no longer supports the "mag" feature of TeX.\MessageBreak
      The default value 'nomag*' is used instead.\MessageBreak \@ehc}
  \fi
\fi
%    \end{macrocode}
%
% オプション処理時に遅延させていた |jbase| の処理をここで実行する。
%    \begin{macrocode}
\bxjs@do@opt@jbase
%    \end{macrocode}
%
% \begin{macro}{\Cjascale}
% 和文クラス共通仕様(※ただしZR氏提唱)における、
% 和文スケール値の変数。
%    \begin{macrocode}
\let\Cjascale\jsScale
%    \end{macrocode}
% \end{macro}
%
% \end{ZRnote}
%
% 後処理
%
%    \begin{macrocode}
\if@slide
  \def\maybeblue{\@ifundefined{ver@color.sty}{}{\color{blue}}}
\fi
\if@landscape
  \setlength\@tempdima  {\paperheight}
  \setlength\paperheight{\paperwidth}
  \setlength\paperwidth {\@tempdima}
\fi
%    \end{macrocode}
%
% \begin{ZRnote}
%
% 8bit欧文{\TeX}の場合は、高位バイトをアクティブ化しておく。
% (和文を含むマクロ定義を通用させるため。)
%    \begin{macrocode}
\if \if p\jsEngine T\else\if n\jsEngine T\else F\fi\fi T
  \@tempcnta="80 \loop \ifnum\@tempcnta<"100
    \catcode\@tempcnta\active
    \advance\@tempcnta\@ne
  \repeat
\fi
%    \end{macrocode}
%
% |js| オプション指定時は、jsarticle(または jsbook)クラスを
% 読込済のように振舞う。
% \Note 「2つのクラスを読み込んだ状態」は |\LoadClass| を使用した
% 場合に出現するので、別に異常ではない。
%    \begin{macrocode}
\ifbxjs@disguise@js
%<book|report>\def\bxjs@js@clsname{jsbook}
%<!book&!report>\def\bxjs@js@clsname{jsarticle}
  \@namedef{ver@\bxjs@js@clsname.cls}{2001/01/01 (bxjs)}
\fi
%    \end{macrocode}
%
% |color|/|graphics| パッケージが持つ出力用紙サイズ設定の機能は、
% BXJSクラスでは余計なので無効にしておく。
% このため、グローバルで |nosetpagesize| を設定しておく。
%    \begin{macrocode}
\g@addto@macro\@classoptionslist{,nosetpagesize}
%    \end{macrocode}
%
% |oldfontcommands| オプション指定時は |\allowoldfontcommands|
% 命令を実行する。
%    \begin{macrocode}
\ifbxjs@oldfontcommands
  \AtEndOfClass{\allowoldfontcommands}
\fi
%    \end{macrocode}
% \end{ZRnote}
%
% \paragraph{papersizeスペシャルの出力}
% \mbox{}
% \begin{ZRnote}
% |geometry| パッケージが行う。
% \end{ZRnote}
%
% \paragraph{基準となる行送り}
%
% \begin{macro}{\n@baseline}
%
% 基準となる行送りをポイント単位で表したものです。
%
%    \begin{macrocode}
%<slide>\def\n@baseline{13}%
%<!slide>\ifdim\bxjs@param@basefontsize<10pt \def\n@baseline{15}%
%<!slide>\else \def\n@baseline{16}\fi
%    \end{macrocode}
% \end{macro}
%
% \paragraph{拡大率の設定}
% \mbox{}
% \begin{ZRnote}
% |\bxjs@magstyle| の値に応じてスイッチ |jsc@mag| と |jsc@mag@xreal| を
% 設定する。
%    \begin{macrocode}
\ifx\bxjs@magstyle\bxjs@magstyle@mag
  \jsc@magtrue
\else\ifx\bxjs@magstyle\bxjs@magstyle@xreal
  \jsc@mag@xrealtrue
\fi\fi
%    \end{macrocode}
% \end{ZRnote}
%
% サイズの変更は\TeX のプリミティブ |\mag| を使って行います。
% 9ポイントについては行送りも若干縮めました。
% サイズについては全面的に見直しました。
%
% [2008-12-26] 1000 / |\mag| に相当する |\inv@mag| を定義しました。
% |truein| を使っていたところを |\inv@mag in| に直しましたので,
% |geometry| パッケージと共存できると思います。
% なお,新ドキュメントクラス側で |10pt| 以外にする場合の注意:
% \begin{itemize}
% \item |geometry| 側でオプション |truedimen| を指定してください。
% \item |geometry| 側でオプション |mag| は使えません。
% \end{itemize}
%
% \begin{ZRnote}
% 設定すべき |\mag| 値を (基底サイズ)/(10\,pt) $\times$ 1000 と算出。
% BXJSクラスでは、|\mag| を直接指定したい場合は、|geometry| 側では
% なくクラスのオプションで行うものとする。
%    \begin{macrocode}
\ifx\bxjs@param@mag\relax
  \@tempdima=\bxjs@param@basefontsize
  \advance\@tempdima.001pt \multiply\@tempdima25
  \divide\@tempdima16384\relax \@tempcnta\@tempdima\relax
  \edef\bxjs@param@mag{\the\@tempcnta}
\else
% mag値が直接指定された場合
  \let\c@bxjs@cnta\@tempcnta
  \setcounter{bxjs@cnta}{\bxjs@param@mag}
  \ifnum\@tempcnta<\z@ \@tempcnta=\z@ \fi
% 有効なmag値の範囲は1--32768
  \edef\bxjs@param@mag{\the\@tempcnta}
  \advance\@tempcnta100000
  \def\bxjs@tmpa#1#2#3#4#5\@nil{\@tempdima=#2#3#4.#5\p@}
  \expandafter\bxjs@tmpa\the\@tempcnta\@nil
  \edef\bxjs@param@basefontsize{\the\@tempdima}
\fi
\@tempcnta\bxjs@param@mag \advance\@tempcnta100000
\def\bxjs@tmpa#1#2#3#4\@nil{\@tempdima=#2#3.#4\p@}
\expandafter\bxjs@tmpa\the\@tempcnta\@nil
\edef\jsc@magscale{\strip@pt\@tempdima}
\let\jsBaseFontSize\bxjs@param@basefontsize
%\typeout{\string\jsDocClass: \meaning\jsDocClass}
%\typeout{\string\jsEngine: \meaning\jsEngine}
%\typeout{\string\jsBaseFontSize: \jsBaseFontSize}
%\typeout{\string\bxjs@param@mag: \bxjs@param@mag}
%\typeout{\string\jsc@magscale: \jsc@magscale}
%\typeout{\string\ifjsc@mag: \meaning\ifjsc@mag}
%\typeout{\string\ifjsc@mag@xreal: \meaning\ifjsc@mag@xreal}
%    \end{macrocode}
% \end{ZRnote}
%
% [2016-07-08] |\jsc@mpt| および |\jsc@mmm| に,それぞれ1ptおよび1mmを拡大させた値を格納します。
% 以降のレイアウト指定ではこちらを使います。
%
% \begin{ZRnote}
% |\mag| する場合(現状はこれが既定)にコードの変更を低減するために、
% 以下では必要に応じて、|\jsc@mpt| を |\p@?| と書く。
% その上で、|\mag| する場合は |?| を無視して |\p@| と解釈させ、
% |\mag| しない場合は |?| を英字扱いにして |\p@?| という制御綴を
% |\jsc@mpt| と同値にする。
% \Note (多分2.0版あたりで)JSクラスに合わせるため |\p@?| 表記を
% 止める予定。
%    \begin{macrocode}
\ifjsc@mag
  \let\jsc@mpt\p@
  \newdimen\jsc@mmm \jsc@mmm=1mm
  \catcode`\?=9 % \p@? read as \p@
\else
  \newdimen\jsc@mpt \jsc@mpt=\jsc@magscale \p@
  \newdimen\jsc@mmm \jsc@mmm=\jsc@magscale mm
  \catcode`\?=11 \let\p@?\jsc@mpt
\fi
\chardef\bxjs@qmcc=\catcode`\?\relax
%    \end{macrocode}
%
% ここで{p\TeX}のzwに相当する単位として用いる長さ変数 |\jsZw| を作成する。
% 約束により、これは |\jsScale| $\times$ (指定フォントサイズ) に等しい。
%
% |nozw| 非指定時は |\zw| を |\jsZw| と同義にする。
%    \begin{macrocode}
\newdimen\jsZw
\jsZw=10\jsc@mpt \jsZw=\jsScale\jsZw
\ifbxjs@usezw
  \providecommand*\zw{\jsZw}
\fi
%    \end{macrocode}
%
% そして、magstyle が |nomag*| の場合は、NFSSにパッチを当てる。
%    \begin{macrocode}
\ifjsc@mag@xreal
  \RequirePackage{type1cm}
  \let\jsc@invscale\bxjs@invscale
%    \end{macrocode}
% ムニャムニャムニャ……。
% \end{ZRnote}
%    \begin{macrocode}
  \expandafter\let\csname OT1/cmr/m/n/10\endcsname\relax
  \expandafter\let\csname OMX/cmex/m/n/10\endcsname\relax
  \let\jsc@get@external@font\get@external@font
  \def\get@external@font{%
    \jsc@preadjust@extract@font
    \jsc@get@external@font}
  \def\jsc@fstrunc#1{%
    \edef\jsc@tmpa{\strip@pt#1}%
    \expandafter\jsc@fstrunc@a\jsc@tmpa.****\@nil}
  \def\jsc@fstrunc@a#1.#2#3#4#5#6\@nil{%
    \if#5*\else
      \edef\jsc@tmpa{#1%
      \ifnum#2#3>\z@ .#2\ifnum#3>\z@ #3\fi\fi}%
    \fi}
  \def\jsc@preadjust@extract@font{%
    \let\jsc@req@size\f@size
    \dimen@\f@size\p@ \jsc@invscale\dimen@\jsc@magscale
    \advance\dimen@.005pt\relax \jsc@fstrunc\dimen@
    \let\jsc@ref@size\jsc@tmpa
    \let\f@size\jsc@ref@size}
  \def\execute@size@function#1{%
    \let\jsc@cref@size\f@size
    \let\f@size\jsc@req@size
    \csname s@fct@#1\endcsname}
  \let\jsc@DeclareErrorFont\DeclareErrorFont
  \def\DeclareErrorFont#1#2#3#4#5{%
    \@tempdimc#5\p@ \@tempdimc\jsc@magscale\@tempdimc
    \edef\jsc@tmpa{{#1}{#2}{#3}{#4}{\strip@pt\@tempdimc}}
    \expandafter\jsc@DeclareErrorFont\jsc@tmpa}
  \def\gen@sfcnt{%
    \edef\mandatory@arg{\mandatory@arg\jsc@cref@size}%
    \empty@sfcnt}
  \def\genb@sfcnt{%
    \edef\mandatory@arg{%
      \mandatory@arg\expandafter\genb@x\jsc@cref@size..\@@}%
    \empty@sfcnt}
  \DeclareErrorFont{OT1}{cmr}{m}{n}{10}
\fi
%    \end{macrocode}
%
% [2016-11-16] latex.ltx (ltspace.dtx)で定義されている |\smallskip| の,
% 単位 |pt| を |\jsc@mpt| に置き換えた |\jsc@smallskip| を定義します。
% これは |\maketitle| で用いられます。
% |\jsc@medskip| と |\jsc@bigskip| は必要ないのでコメントアウトしています。
%
% \begin{macro}{\jsc@smallskip}
% \begin{macro}{\jsc@medskip}
% \begin{macro}{\jsc@bigskip}
%    \begin{macrocode}
\def\jsc@smallskip{\vspace\jsc@smallskipamount}
%\def\jsc@medskip{\vspace\jsc@medskipamount}
%\def\jsc@bigskip{\vspace\jsc@bigskipamount}
%    \end{macrocode}
% \end{macro}
% \end{macro}
% \end{macro}
%
% \begin{macro}{\jsc@smallskipamount}
% \begin{macro}{\jsc@medskipamount}
% \begin{macro}{\jsc@bigskipamount}
%    \begin{macrocode}
\newskip\jsc@smallskipamount
\jsc@smallskipamount=3\jsc@mpt plus 1\jsc@mpt minus 1\jsc@mpt
%\newskip\jsc@medskipamount
%\jsc@medskipamount =6\jsc@mpt plus 2\jsc@mpt minus 2\jsc@mpt
%\newskip\jsc@bigskipamount
%\jsc@bigskipamoun =12\jsc@mpt plus 4\jsc@mpt minus 4\jsc@mpt
%    \end{macrocode}
% \end{macro}
% \end{macro}
% \end{macro}
%
% \paragraph{pagesizeスペシャルの出力}
% \mbox{}
% \begin{ZRnote}
% 削除。
% \end{ZRnote}
%
% \section{和文フォントの変更}
%
% \begin{ZRnote}
% 和文フォントの設定は和文ドライバの管轄。
% \end{ZRnote}
%
% \begin{macro}{\@}
%
% 欧文といえば,\LaTeX の |\def\@{\spacefactor\@m}| という定義(|\@m| は1000)
% では |I watch TV\@.| と書くと V とピリオドのペアカーニングが効かなくなります。
% そこで,次のような定義に直し,|I watch TV.\@| と書くことにします。
%
%    \begin{macrocode}
\chardef\bxjs@periodchar=`\.
\bxjs@protected\def\bxjs@SE{\spacefactor\sfcode\bxjs@periodchar}
\def\@{\bxjs@SE{}}
%    \end{macrocode}
% \end{macro}
%
% \section{フォントサイズ}
%
% フォントサイズを変える命令(|\normalsize|,|\small| など)
% の実際の挙動の設定は,三つの引数をとる命令 |\@setfontsize| を使って,
% たとえば
% \begin{quote}
%   |\@setfontsize{\normalsize}{10}{16}|
% \end{quote}
% のようにして行います。これは
% \begin{quote}
%   |\normalsize| は10ポイントのフォントを使い,行送りは16ポイントである
% \end{quote}
% という意味です。
% ただし,処理を速くするため,
% 以下では10と同義の\LaTeX の内部命令 |\@xpt| を使っています。
% この |\@xpt| の類は次のものがあり,\LaTeX 本体で定義されています。
%\begin{verbatim}
%   \@vpt      5         \@vipt    6      \@viipt   7
%   \@viiipt   8         \@ixpt    9      \@xpt    10
%   \@xipt    10.95      \@xiipt  12      \@xivpt  14.4
%\end{verbatim}
%
%^^A\begin{macro}{\@setfontsize}
%
% ここでは |\@setfontsize| の定義を少々変更して,
% 段落の字下げ |\parindent|,
% 和文文字間のスペース |\kanjiskip|,
% 和文・欧文間のスペース |\xkanjiskip| を変更しています。
%
% |\kanjiskip| は\pLaTeXe で |0pt plus .4pt minus .5pt| に設定していますが,
% これはそもそも文字サイズの変更に応じて変わるべきものです。
% それに,プラスになったりマイナスになったりするのは,
% 追い出しと追い込みの混在が生じ,統一性を欠きます。
% なるべく追い出しになるようにプラスの値だけにしたいところですが,
% ごくわずかなマイナスは許すことにしました。
%
% |\xkanjiskip| については,四分つまり全角の1/4を標準として,
% 追い出すために三分あるいは二分まで延ばすのが一般的ですが,
% ここではTimesやPalatinoのスペースがほぼ四分であることに着目して,
% これに一致させています。これなら書くときにスペースを空けても
% 空けなくても同じ出力になります。
%
% |\parindent| については,0(以下)でなければ全角幅(1zw)に直します。
%
% [2008-02-18] |english| オプションで |\parindent| を 1em にしました。
%
% \begin{ZRnote}
% \begin{macro}{\set@fontsize}
% |\fontsize| 命令(|\large| 等でなく)でフォントサイズ変更した場合
% にもフックが実行されるように、|\@setfontsize| では
% なく |\set@fontsize| に対してパッチを当てるように変更。
%    \begin{macrocode}
\def\bxjs@tmpa{\def\set@fontsize##1##2##3}
\expandafter\bxjs@tmpa\expandafter{%
  \set@fontsize{#1}{#2}{#3}%
% 末尾にコードを追加
  \expandafter\def\expandafter\size@update\expandafter{%
    \size@update
    \jsFontSizeChanged}%
}
%    \end{macrocode}
% \end{macro}
%
% \begin{macro}{\jsFontSizeChanged}
% フォントサイズ変更時に呼ばれるフック。
% |\jsZw| を再設定している。
% その後でユーザ定義用のフック |\jsResetDimen| を実行する。
%    \begin{macrocode}
\newcommand*\jsFontSizeChanged{%
  \jsZw=\f@size\p@
  \jsZw=\jsScale \jsZw
  \ifdim\parindent>\z@
    \if@english \parindent=1em
    \else       \parindent=1\jsZw
    \fi
  \fi\relax
  \jsResetDimen}
%    \end{macrocode}
% \end{macro}
%
% \begin{macro}{\jsResetDimen}
% ユーザ定義用のフック。
%    \begin{macrocode}
\newcommand*\jsResetDimen{}
%    \end{macrocode}
% \end{macro}
% \end{ZRnote}
%
% \begin{macro}{\jsc@setfontsize}
% クラスファイルの内部では,拡大率も考慮した |\jsc@setfontsize| を
% |\@setfontsize| の変わりに用いることにします。
%    \begin{macrocode}
\ifjsc@mag
  \let\jsc@setfontsize\@setfontsize
\else
  \def\jsc@setfontsize#1#2#3{%
    \@setfontsize#1{#2\jsc@mpt}{#3\jsc@mpt}}
\fi
%    \end{macrocode}
% \end{macro}
%
% これらのグルーをもってしても行分割ができない場合は,
% |\emergencystretch| に訴えます。
%
% \begin{ZRnote}
% これはフォントサイズ非依存なので |\Cwd| で書くのが適当だが、
% |\Cwd| はまだ定義されていない。
% \end{ZRnote}
%    \begin{macrocode}
\emergencystretch 3\jsZw
%    \end{macrocode}
%
% \begin{macro}{\ifnarrowbaselines}
% \begin{macro}{\narrowbaselines}
% \begin{macro}{\widebaselines}
%
% 欧文用に行間を狭くする論理変数と,それを真・偽にするためのコマンドです。
%
% [2003-06-30] 数式に入るところで |\narrowbaselines|
% を実行しているので |\abovedisplayskip| 等が初期化
% されてしまうというshintokさんのご指摘に対して,
% しっぽ愛好家さんが次の修正を教えてくださいました。
%
% [2008-02-18] |english| オプションで最初の段落のインデントをしないようにしました。
%
% TODO: Hasumiさん [qa:54539] のご指摘は考慮中です。
%
% \begin{ZRnote}
% 別行立て数式に入るときに |\narrowbaselines| が呼ばれるが、
% このコードでは「数式中で |\normalsize| などのサイズ命令
% (|\@currsize| の実体)が呼ばれた」ことになり警告が出る。
% JSクラスでは、|\@setfontsize| 中の |\@nomath| 実行を消して
% 「そもそもサイズ命令で警告が出ない」ようにしている。
% 警告が常に出ないのも望ましくないので、BXJSクラスの実装では、
% |\narrowbaselines| の時だけ警告が出ないようにする。
% \end{ZRnote}
%
%    \begin{macrocode}
\newif\ifnarrowbaselines
\if@english
  \narrowbaselinestrue
\fi
\def\narrowbaselines{%
  \narrowbaselinestrue
  \skip0=\abovedisplayskip
  \skip2=\abovedisplayshortskip
  \skip4=\belowdisplayskip
  \skip6=\belowdisplayshortskip
% 一時的に警告を無効化する
  \let\bxjs@ltx@nomath\@nomath
  \let\@nomath\@gobble
  \@currsize\selectfont
  \let\@nomath\bxjs@ltx@nomath
  \abovedisplayskip=\skip0
  \abovedisplayshortskip=\skip2
  \belowdisplayskip=\skip4
  \belowdisplayshortskip=\skip6\relax}
\def\widebaselines{\narrowbaselinesfalse\@currsize\selectfont}
%    \end{macrocode}
% \end{macro}
% \end{macro}
% \end{macro}
%
% \begin{ZRnote}
% |microtype| パッケージを読み込んだ場合、|\normalsize| 等の
% フォントサイズ変更命令の定義の中にif文が使われていると、
% 不可解なエラーが発生する。
% これは |microtype| が邪悪なトリックを使用しているせいなのだが、
% 一応こちら側で対策をとることにする。
%    \begin{macrocode}
\def\bxjs@if@narrowbaselines{%
  \ifnarrowbaselines\expandafter\@firstoftwo
  \else \expandafter\@secondoftwo
  \fi
}
%    \end{macrocode}
% \end{ZRnote}
%
% \begin{macro}{\normalsize}
%
% 標準のフォントサイズと行送りを選ぶコマンドです。
%
% 本文10ポイントのときの行送りは,
% 欧文の標準クラスファイルでは12ポイント,
% アスキーの和文クラスファイルでは15ポイントになっていますが,
% ここでは16ポイントにしました。
% ただし |\narrowbaselines| で欧文用の12ポイントになります。
%
% 公称10ポイントの和文フォントが約9.25ポイント
% (アスキーのものの0.961倍)であることもあり,
% 行送りがかなりゆったりとしたと思います。
% 実際,$16/9.25 \approx 1.73$ であり,
% 和文の推奨値の一つ「二分四分」(1.75)
% に近づきました。
%
% \begin{ZRnote}
% |microtype| 対策のためif文を避ける。
% \end{ZRnote}
%    \begin{macrocode}
\renewcommand{\normalsize}{%
  \bxjs@if@narrowbaselines{%
    \jsc@setfontsize\normalsize\@xpt\@xiipt
  }{%else
    \jsc@setfontsize\normalsize\@xpt{\n@baseline}%
  }%
%    \end{macrocode}
%
% 数式の上のアキ(|\abovedisplayskip|),
% 短い数式の上のアキ(|\abovedisplayshortskip|),
% 数式の下のアキ(|\belowdisplayshortskip|)の設定です。
%
% [2003-02-16] ちょっと変えました。
%
% [2009-08-26] \TeX\ Q\,\&\,A 52569から始まる議論について逡巡して
% いましたが,結局,微調節してみることにしました。
%
%    \begin{macrocode}
  \abovedisplayskip 11\p@? \@plus3\p@? \@minus4\p@?
  \abovedisplayshortskip \z@ \@plus3\p@?
  \belowdisplayskip 9\p@? \@plus3\p@? \@minus4\p@?
  \belowdisplayshortskip \belowdisplayskip
%    \end{macrocode}
%
% 最後に,リスト環境のトップレベルのパラメータ |\@listI| を,
% |\@listi| にコピーしておきます。|\@listI| の設定は後で出てきます。
%
%    \begin{macrocode}
  \let\@listi\@listI}
%    \end{macrocode}
%
% ここで実際に標準フォントサイズで初期化します。
%
%    \begin{macrocode}
\normalsize
%    \end{macrocode}
%
% \end{macro}
%
% \begin{macro}{\Cht}
% \begin{macro}{\Cdp}
% \begin{macro}{\Cwd}
% \begin{macro}{\Cvs}
% \begin{macro}{\Chs}
%
% 基準となる長さの設定をします。
% \pLaTeXe カーネル(\texttt{plfonts.dtx})で宣言されている
% パラメータに実際の値を設定します。
% たとえば |\Cwd| は |\normalfont| の全角幅(1zw)です。
%
% \begin{ZRnote}
% まず |\Cwd| 等の変数を定義する。
%    \begin{macrocode}
\ifx\Cht\@undefined \newdimen\Cht \fi
\ifx\Cdp\@undefined \newdimen\Cdp \fi
\ifx\Cwd\@undefined \newdimen\Cwd \fi
\ifx\Cvs\@undefined \newdimen\Cvs \fi
\ifx\Chs\@undefined \newdimen\Chs \fi
%    \end{macrocode}
%
% 規約上、現在の |\jsZw| の値が |\Cwd| である。
% |\Cht| と |\Cdp| は単純に |\Cwd| の88\%と12\%の値とする。
% \end{ZRnote}
%    \begin{macrocode}
\setlength\Cht{0.88\jsZw}
\setlength\Cdp{0.12\jsZw}
\setlength\Cwd{1\jsZw}
\setlength\Cvs{\baselineskip}
\setlength\Chs{1\jsZw}
%    \end{macrocode}
% \end{macro}
% \end{macro}
% \end{macro}
% \end{macro}
% \end{macro}
%
% \begin{macro}{\small}
%
% |\small| も |\normalsize| と同様に設定します。
% 行送りは,|\normalsize| が16ポイントなら,
% 割合からすれば $16 \times 0.9 = 14.4$ ポイントになりますが,
% |\small| の使われ方を考えて,ここでは和文13ポイント,
% 欧文11ポイントとします。
% また,|\topsep| と |\parsep| は,元はそれぞれ $4 \pm 2$,$2 \pm 1$
% ポイントでしたが,ここではゼロ(|\z@|)にしました。
%
% \begin{ZRnote}
% |microtype| 対策のためif文を避ける。
% 後の |\footnotesize| も同様。
% \end{ZRnote}
%    \begin{macrocode}
\newcommand{\small}{%
  \bxjs@if@narrowbaselines{%
%<!kiyou>    \jsc@setfontsize\small\@ixpt{11}%
%<kiyou>    \jsc@setfontsize\small{8.8888}{11}%
  }{%else
%<!kiyou>    \jsc@setfontsize\small\@ixpt{13}%
%<kiyou>    \jsc@setfontsize\small{8.8888}{13.2418}%
  }%
  \abovedisplayskip 9\p@? \@plus3\p@? \@minus4\p@?
  \abovedisplayshortskip  \z@ \@plus3\p@?
  \belowdisplayskip \abovedisplayskip
  \belowdisplayshortskip \belowdisplayskip
  \def\@listi{\leftmargin\leftmargini
              \topsep \z@
              \parsep \z@
              \itemsep \parsep}}
%    \end{macrocode}
% \end{macro}
%
% \begin{macro}{\footnotesize}
%
% |\footnotesize| も同様です。
% |\topsep| と |\parsep| は,元はそれぞれ $3 \pm 1$,$2 \pm 1$
% ポイントでしたが,ここではゼロ(|\z@|)にしました。
%
%    \begin{macrocode}
\newcommand{\footnotesize}{%
  \bxjs@if@narrowbaselines{%
%<!kiyou>    \jsc@setfontsize\footnotesize\@viiipt{9.5}%
%<kiyou>    \jsc@setfontsize\footnotesize{8.8888}{11}%
  }{%else
%<!kiyou>    \jsc@setfontsize\footnotesize\@viiipt{11}%
%<kiyou>    \jsc@setfontsize\footnotesize{8.8888}{13.2418}%
  }%
  \abovedisplayskip 6\p@? \@plus2\p@? \@minus3\p@?
  \abovedisplayshortskip  \z@ \@plus2\p@?
  \belowdisplayskip \abovedisplayskip
  \belowdisplayshortskip \belowdisplayskip
  \def\@listi{\leftmargin\leftmargini
              \topsep \z@
              \parsep \z@
              \itemsep \parsep}}
%    \end{macrocode}
% \end{macro}
%
% \begin{macro}{\scriptsize}
% \begin{macro}{\tiny}
% \begin{macro}{\large}
% \begin{macro}{\Large}
% \begin{macro}{\LARGE}
% \begin{macro}{\huge}
% \begin{macro}{\Huge}
% \begin{macro}{\HUGE}
%
% それ以外のサイズは,本文に使うことがないので,
% 単にフォントサイズと行送りだけ変更します。
% 特に注意すべきは |\large| で,
% これは二段組のときに節見出しのフォントとして使い,
% 行送りを |\normalsize| と同じにすることによって,
% 節見出しが複数行にわたっても段間で行が揃うようにします。
%
% [2004-11-03] |\HUGE| を追加。
%
%    \begin{macrocode}
\newcommand{\scriptsize}{\jsc@setfontsize\scriptsize\@viipt\@viiipt}
\newcommand{\tiny}{\jsc@setfontsize\tiny\@vpt\@vipt}
\if@twocolumn
%<!kiyou>  \newcommand{\large}{\jsc@setfontsize\large\@xiipt{\n@baseline}}
%<kiyou>  \newcommand{\large}{\jsc@setfontsize\large{11.111}{\n@baseline}}
\else
%<!kiyou>  \newcommand{\large}{\jsc@setfontsize\large\@xiipt{17}}
%<kiyou>  \newcommand{\large}{\jsc@setfontsize\large{11.111}{17}}
\fi
%<!kiyou>\newcommand{\Large}{\jsc@setfontsize\Large\@xivpt{21}}
%<kiyou>\newcommand{\Large}{\jsc@setfontsize\Large{12.222}{21}}
\newcommand{\LARGE}{\jsc@setfontsize\LARGE\@xviipt{25}}
\newcommand{\huge}{\jsc@setfontsize\huge\@xxpt{28}}
\newcommand{\Huge}{\jsc@setfontsize\Huge\@xxvpt{33}}
\newcommand{\HUGE}{\jsc@setfontsize\HUGE{30}{40}}
%    \end{macrocode}
% \end{macro}
% \end{macro}
% \end{macro}
% \end{macro}
% \end{macro}
% \end{macro}
% \end{macro}
% \end{macro}
%
% 別行立て数式の中では |\narrowbaselines| にします。
% 和文の行送りのままでは,行列や場合分けの行送り,
% 連分数の高さなどが不釣合いに大きくなるためです。
%
% 本文中の数式の中では |\narrowbaselines| にしていません。
% 本文中ではなるべく行送りが変わるような大きいものを使わず,
% 行列は |amsmath| の |smallmatrix| 環境を使うのがいいでしょう。
%
%    \begin{macrocode}
\everydisplay=\expandafter{\the\everydisplay \narrowbaselines}
%    \end{macrocode}
%
% しかし,このおかげで別行数式の上下のスペースが少し違ってしまいました。
% とりあえず |amsmath| の |equation| 関係は |okumacro| のほうで逃げていますが,
% もっとうまい逃げ道があればお教えください。
%
% 見出し用のフォントは |\bfseries| 固定ではなく,|\headfont|
% という命令で定めることにします。
% これは太ゴシックが使えるときは |\sffamily| |\bfseries|
% でいいと思いますが,通常の中ゴシックでは単に |\sffamily|
% だけのほうがよさそうです。
% 『p\LaTeXe 美文書作成入門』(1997年)では |\sffamily|
% |\fontseries{sbc}| として新ゴMと合わせましたが,
% |\fontseries{sbc}| はちょっと幅が狭いように感じました。
%
%    \begin{macrocode}
% \newcommand{\headfont}{\bfseries}
\newcommand{\headfont}{\sffamily}
% \newcommand{\headfont}{\sffamily\fontseries{sbc}\selectfont}
%    \end{macrocode}
%
% \section{レイアウト}
%
% \paragraph{二段組}
%
% \begin{macro}{\columnsep}
% \begin{macro}{\columnseprule}
%
% |\columnsep| は二段組のときの左右の段間の幅です。
% 元は10ptでしたが,2zwにしました。
% このスペースの中央に |\columnseprule| の幅の罫線が引かれます。
%
%    \begin{macrocode}
%<!kiyou>\setlength\columnsep{2\Cwd}
%<kiyou>\setlength\columnsep{28truebp}
\setlength\columnseprule{0\p@}
%    \end{macrocode}
% \end{macro}
% \end{macro}
%
% \paragraph{段落}
%
% \begin{macro}{\lineskip}
% \begin{macro}{\normallineskip}
% \begin{macro}{\lineskiplimit}
% \begin{macro}{\normallineskiplimit}
%
% 上下の行の文字が |\lineskiplimit| より接近したら,
% |\lineskip| より近づかないようにします。
% 元は0ptでしたが1ptに変更しました。
% \texttt{normal...} の付いた方は保存用です。
%
%    \begin{macrocode}
\setlength\lineskip{1\p@?}
\setlength\normallineskip{1\p@?}
\setlength\lineskiplimit{1\p@?}
\setlength\normallineskiplimit{1\p@?}
%    \end{macrocode}
% \end{macro}
% \end{macro}
% \end{macro}
% \end{macro}
%
% \begin{macro}{\baselinestretch}
%
% 実際の行送りが |\baselineskip| の何倍かを表すマクロです。たとえば
% \begin{quote}
%   |\renewcommand{\baselinestretch}{2}|
% \end{quote}
% とすると,行送りが通常の2倍になります。ただし,
% これを設定すると,たとえ |\baselineskip| が伸縮するように
% 設定しても,行送りの伸縮ができなくなります。
% 行送りの伸縮はしないのが一般的です。
%
%    \begin{macrocode}
\renewcommand{\baselinestretch}{}
%    \end{macrocode}
% \end{macro}
%
% \begin{macro}{\parskip}
% \begin{macro}{\parindent}
%
% |\parskip| は段落間の追加スペースです。
% 元は 0pt plus 1pt になっていましたが,ここではゼロにしました。
% |\parindent| は段落の先頭の字下げ幅です。
%
%    \begin{macrocode}
\setlength\parskip{0\p@}
\if@slide
  \setlength\parindent{0\p@}
\else
  \setlength\parindent{1\Cwd}
\fi
%    \end{macrocode}
% \end{macro}
% \end{macro}
%
% \begin{macro}{\@lowpenalty}
% \begin{macro}{\@medpenalty}
% \begin{macro}{\@highpenalty}
%
% |\nopagebreak|,|\nolinebreak| は引数に応じて次のペナルティ値
% のうちどれかを選ぶようになっています。
% ここはオリジナル通りです。
%
%    \begin{macrocode}
\@lowpenalty   51
\@medpenalty  151
\@highpenalty 301
%    \end{macrocode}
% \end{macro}
% \end{macro}
% \end{macro}
%
% \begin{macro}{\interlinepenalty}
%
% 段落中の改ページのペナルティです。デフォルトは 0 です。
%
%    \begin{macrocode}
% \interlinepenalty 0
%    \end{macrocode}
% \end{macro}
%
% \begin{macro}{\brokenpenalty}
%
% ページの最後の行がハイフンで終わる際のペナルティです。
% デフォルトは 100 です。
%
%    \begin{macrocode}
% \brokenpenalty 100
%    \end{macrocode}
% \end{macro}
%
% \subsection{ページレイアウト}
%
% \begin{ZRnote}
% BXJSではページレイアウトの処理は |geometry| パッケージが担当している。
% \end{ZRnote}
%
% \paragraph{準備 \ZRX}
% \mbox{}
% \begin{ZRnote}
% 現状ではここで |\mag| を設定している。\par
% |\topskip| も指定する。
%    \begin{macrocode}
\ifjsc@mag
\mag=\bxjs@param@mag
\fi
\setlength{\topskip}{10\p@?}
%    \end{macrocode}
%
% |\bxjs@param@paper| が長さ指定(|{W}{H}|)の場合、
% |geometry| の形式(|papersize={W,H}|)に変換する。
%    \begin{macrocode}
\def\bxjs@read@a{\futurelet\bxjs@tmpa\bxjs@read@b}
\def\bxjs@read@b{%
  \ifx\bxjs@tmpa\bgroup \expandafter\bxjs@read@c
  \else \expandafter\bxjs@read@d \fi}
\def\bxjs@read@c#1#2#3\@nil{\def\bxjs@param@paper{papersize={#1,#2}}}
\def\bxjs@read@d#1\@nil{}
\expandafter\bxjs@read@a\bxjs@param@paper\@nil
%    \end{macrocode}
%
% \begin{macro}{\bxjs@layout@paper}
% |geometry| の用紙設定のオプション。
%    \begin{macrocode}
\edef\bxjs@layout@paper{%
  \ifjsc@mag truedimen,\fi
  \if@landscape landscape,\fi
  \bxjs@param@paper}
%    \end{macrocode}
% \end{macro}
%
% \begin{macro}{\bxjs@layout}
% |geometry| のページレイアウトのオプション列。
% 文書クラス毎に異なる。
%    \begin{macrocode}
%<*article|report>
\def\bxjs@layout{%
  headheight=\topskip,footskip=0.03367\paperheight,%
  headsep=\footskip-\topskip,includeheadfoot,%
  hscale=0.76,hmarginratio=1:1,%
  vscale=0.83,vmarginratio=1:1,%
}
%</article|report>
%<*book>
\ifbxjs@layout@buggyhmargin     %---
% アレ
\def\bxjs@layout{%
  headheight=\topskip,headsep=6\jsc@mmm,nofoot,includeheadfoot,%
  hmargin=36\jsc@mmm,hmarginratio=1:1,%
  vscale=0.83,vmarginratio=1:1,%
}
\else                           %---
% 非アレ
\def\bxjs@layout{%
  headheight=\topskip,headsep=6\jsc@mmm,nofoot,includeheadfoot,%
  hmargin=18\jsc@mmm,%
  vscale=0.83,vmarginratio=1:1,%
}
\fi                             %---
%</book>
%<*slide>
\def\bxjs@layout{%
  noheadfoot,%
  hscale=0.9,hmarginratio=1:1,%
  vscale=0.944,vmarginratio=1:1,%
}
%</slide>
%    \end{macrocode}
% \end{macro}
%
% \begin{macro}{\fullwidth}
% 〔寸法レジスタ〕
% ヘッダ・フッタ領域の横幅。
%    \begin{macrocode}
\newdimen\fullwidth
%    \end{macrocode}
% \end{macro}
% 
% \begin{macro}{\jsTextWidthLimit}
% 〔実数値マクロ〕
% |bxjsbook| における、|\textwidth| の上限の全角単位での値。
%    \begin{macrocode}
%<*book>
\newcommand\jsTextWidthLimit{40}
\ifx\bxjs@textwidth@limit\@undefined\else
  \let\c@bxjs@cnta\@tempcnta
  \setcounter{bxjs@cnta}{\bxjs@textwidth@limit}
  \long\edef\jsTextWidthLimit{\the\@tempcnta}
\fi
%</book>
%    \end{macrocode}
% \end{macro}
%
% \begin{macro}{\bxjs@postproc@layout}
% |geometry| の後処理。
%    \begin{macrocode}
\def\bxjs@postproc@layout{%
% ドライバ再設定
  \ifx\bxjs@geometry@driver\relax\else
    \let\Gm@driver\bxjs@geometry@driver
  \fi
% textwidth 調整
  \@tempdimb=\textwidth
  \if@twocolumn \@tempdima=2\Cwd \else \@tempdima=1\Cwd \fi
  \divide\textwidth\@tempdima \multiply\textwidth\@tempdima
  \advance\@tempdimb-\textwidth
  \advance\oddsidemargin 0.5\@tempdimb
  \advance\evensidemargin 0.5\@tempdimb
  \fullwidth=\textwidth
%<*book>
  \ifdim\textwidth>\jsTextWidthLimit\Cwd
    \textwidth=\jsTextWidthLimit\Cwd
    \addtolength\evensidemargin{\fullwidth-\textwidth}
  \fi
%</book>
% textheight 調整
  \@tempdimb=\textheight
  \advance\textheight-\topskip
  \divide\textheight\baselineskip \multiply\textheight\baselineskip
  \advance\textheight\topskip
  \advance\@tempdimb-\textheight
  \advance\topmargin0.5\@tempdimb
% headheight 調整
  \@tempdima=\topskip
  \advance\headheight\@tempdima
  \advance\topmargin-\@tempdima
% marginpar 設定
  \setlength\marginparsep{\columnsep}
  \setlength\marginparpush{\baselineskip}
  \setlength\marginparwidth{\paperwidth-\oddsidemargin-1truein%
      -\textwidth-10\jsc@mmm-\marginparsep}
  \divide\marginparwidth\Cwd \multiply\marginparwidth\Cwd
% 連動する変数
  \maxdepth=.5\topskip
  \stockwidth=\paperwidth
  \stockheight=\paperheight
}
%    \end{macrocode}
% \end{macro}
%
% \begin{macro}{\jsGeometryOptions}
% |geometry|パッケージに渡すオプションのリスト。
% \Note |geometry=user| 指定時にユーザが利用することを想定している。
%    \begin{macrocode}
\edef\jsGeometryOptions{%
  \bxjs@layout@paper,\bxjs@layout}
%    \end{macrocode}
% \end{macro}
% \end{ZRnote}
%
% \paragraph{geometry パッケージ読込 \ZRX}
% \mbox{}
% \begin{ZRnote}
% |geoemtry| オプションの値に応じて分岐する。
%
% まずは|geometry=class|の場合。
%    \begin{macrocode}
\ifx\bxjs@geometry\bxjs@geometry@class
%    \end{macrocode}
%
% |geometry| は |\topskip| が標準の行高(|\ht\strutbox|)より小さくならない
% ようにする自動調整を行うが、これをどうするかは未検討。
% 今のところ、単純に回避(無効化)している。
%    \begin{macrocode}
\@onlypreamble\bxjs@revert
\let\bxjs@revert\@empty
\edef\bxjs@tmpa{\the\ht\strutbox}
\ht\strutbox=10\p@?
\g@addto@macro\bxjs@revert{\ht\strutbox=\bxjs@tmpa\relax}
%    \end{macrocode}
%
% |geometry| のドライバオプション指定。
% |nopapersize| 指定時は、special命令出力を抑止するために
% ドライバを |none| にする。
% そうでない場合は、クラスで指定したドライバオプションが引き継がれる
% ので何もしなくてよいが、例外として、ドライバが |dvipdfmx| の
% 時は、現状の |geometry| は |dvipdfm| を指定する必要がある。
%    \begin{macrocode}
\ifbxjs@papersize
  \ifx\bxjs@driver@given\bxjs@driver@@dvipdfmx
    \PassOptionsToPackage{dvipdfm}{geometry}
  \fi
  \let\bxPapersizeSpecialDone=t
\else
  \PassOptionsToPackage{driver=none}{geometry}
\fi
%    \end{macrocode}
%
% ここで |geometry| を読み込む。
% \Note |geometry| のbegin-documentフックにおいて、Lua{\TeX}の
% 旧版互換を有効にする。
%    \begin{macrocode}
\edef\bxjs@nxt{%
  \noexpand\RequirePackage[\bxjs@layout@paper,\bxjs@layout]{geometry}}
\AtBeginDocument{\bxjs@pre@geometry@hook}
\AtBeginDocument{\ImposeOldLuaTeXBehavior}
\bxjs@nxt \bxjs@revert
\AtBeginDocument{\RevokeOldLuaTeXBehavior}
%    \end{macrocode}
%
% \begin{macro}{\bxjs@geometry@driver}
% |geometry| が用いるドライバの名前。
% \Note この値は一度決めた後は変わってほしくないので、
% |\bxjs@postproc@layout| において書き戻す処理を入れている。
%    \begin{macrocode}
\let\bxjs@geometry@driver\Gm@driver
\bxjs@postproc@layout
%    \end{macrocode}
% \end{macro}
%
% \begin{macro}{\bxjs@pre@geometry@hook}
% 1.2版より、|geometry| の4.x版の使用は非推奨とする。
% \Note 将来サポートを廃止する予定。
%    \begin{macrocode}
\@onlypreamble\bxjs@pre@geometry@hook
\def\bxjs@pre@geometry@hook{%
  \@ifpackageloaded{geometry}{%
    \@ifpackagelater{geometry}{2010/02/12}{}{%else
      \PackageWarningNoLine\bxjs@clsname
       {The 'geometry' package installed\MessageBreak
        is too old (< v5.0)}%
      \if x\jsEngine \ifnum\mag=\@m\else
        \def\bxjs@Gm@driver{pdftex}
        \ifx\pdfhorigin\@undefined \newdimen\pdfhorigin \fi
        \ifx\pdfvorigin\@undefined \newdimen\pdfvorigin \fi
      \fi\fi
    }%
    \ifjsWithpTeXng
      \ifx\Gm@driver\@empty
        \def\Gm@driver{pdftex}%
      \fi
    \fi
  }{}}
%    \end{macrocode}
% \end{macro}
%
% \begin{macro}{\setpagelayout}
% ページレイアウト設定のためのユーザ命令。
% \begin{itemize}
% \item |\setpagelayout{|\meta{text}|}| : 現在の geometry の設定の
% 一部を修正する。
% \item |\setpagelayout*{|\meta{text}|}| : 用紙以外の設定をリセットして、
% 改めて設定を行う。
% \end{itemize}
% どちらも設定の後で後処理 |\bxjs@postproc@layout| を実行する。
%    \begin{macrocode}
\def\setpagelayout{\@ifstar
  {\bxjs@reset@layout}{\bxjs@modify@layout}}
\def\bxjs@modify@layout#1{%
  \edef\bxjs@nxt{\noexpand\geometry{#1,truedimen}}%
  \bxjs@nxt\bxjs@postproc@layout}
\def\bxjs@reset@layout#1{%
  \edef\bxjs@nxt{\noexpand\geometry{reset,\bxjs@layout@paper,#1,truedimen}}%
  \bxjs@nxt\bxjs@postproc@layout}
%    \end{macrocode}
% \end{macro}
%
% |geometry=class|の場合の処理はここで終わり。
%
% 次に、|geometry=user|の場合の処理。
%    \begin{macrocode}
\else\ifx\bxjs@geometry\bxjs@geometry@user
%    \end{macrocode}
%
% この場合はユーザが何らかの方法(例えば |geometry| を読み込む)
% でページレイアウトを設定する必要がある。
% もし、本体開始時に |\textwidth| がカーネル設定の値(|.5\maxdimen|)
% のままになっている場合はエラーを出す。
% \Note |\jsUseMinimalPageLayout| は動作テスト用。
%    \begin{macrocode}
\AtBeginDocument{\bxjs@check@page@layout}
\@onlypreamble\bxjs@check@page@layout
\def\bxjs@check@page@layout{%
  \ifdim\textwidth=.5\maxdimen
    \ClassError\bxjs@clsname
     {Page layout is not properly set}%
     {\@ehd}
  \fi}
\def\jsUseMinimalPageLayout{%
  \setlength{\textwidth}{6.5in}%
  \setlength{\textheight}{8in}}
%    \end{macrocode}
%
% |\setpagelayout| はとりあえず無効にしておく。
%    \begin{macrocode}
\let\bxjs@geometry@driver\relax
\def\setpagelayout{\@ifstar
  {\bxjs@pagelayout@a}{\bxjs@pagelayout@a}}
\def\bxjs@pagelayout@a#1{%
  \ClassError\bxjs@clsname
   {Command '\string\setpagelayout' is not supported,\MessageBreak
    because 'geometry' value is not 'class'}\@eha}
%    \end{macrocode}
%
% |geometry=user|の場合の処理はここで終わり。
%    \begin{macrocode}
\fi\fi
%    \end{macrocode}
%
% ここからのコードは以下の点を除いて JS クラスのものを踏襲する。
% \begin{itemize}
% \item zw の代わりに |\jsZw| を用いる。
% \item article/report/book/slide の切り分けの処理が異なる。
% \end{itemize}
% \end{ZRnote}
%
% \paragraph{脚注}
%
% \begin{macro}{\footnotesep}
%
% 各脚注の頭に入る支柱(strut)の高さです。
% 脚注間に余分のアキが入らないように,
% |\footnotesize| の支柱の高さ(行送りの0.7倍)に等しくします。
%
% \begin{ZRnote}
% ここは元々は
%\begin{verbatim}
%{\footnotesize\global\setlength\footnotesep{\baselineskip}}
%\end{verbatim}
% としていたが、そもそも |\global||\setlength|~ は\Pkg{calc}使用時には
% 有意義な動作をしない。
% |\global||\footnotesep| だと所望の値が得られるが、
% 同時に |\footnotesize| のフォントを固定させてしまうという副作用をもつ。
% なので、実際の設定値を直接使うことにする。
% \end{ZRnote}
%    \begin{macrocode}
\footnotesep=11\p@? \footnotesep=0.7\footnotesep
%    \end{macrocode}
% \end{macro}
%
% \begin{macro}{\footins}
%
% |\skip\footins| は本文の最終行と最初の脚注との間の距離です。
% 標準の10ポイントクラスでは 9 plus 4 minus 2 ポイントになっていますが,
% 和文の行送りを考えてもうちょっと大きくします。
%
%    \begin{macrocode}
\setlength{\skip\footins}{16\p@? \@plus 5\p@? \@minus 2\p@?}
%    \end{macrocode}
% \end{macro}
%
% \paragraph{フロート関連}
%
% フロート(図,表)関連のパラメータは\LaTeXe 本体で定義されていますが,
% ここで設定変更します。本文ページ(本文とフロートが共存するページ)
% とフロートだけのページで設定が異なります。
% ちなみに,カウンタは内部では |\c@| を名前に冠したマクロになっています。
%
% \begin{macro}{\c@topnumber}
%
% |topnumber| カウンタは本文ページ上部のフロートの最大数です。
%
% [2003-08-23] ちょっと増やしました。
%
%    \begin{macrocode}
\setcounter{topnumber}{9}
%    \end{macrocode}
% \end{macro}
%
% \begin{macro}{\topfraction}
%
% 本文ページ上部のフロートが占有できる最大の割合です。
% フロートが入りやすいように,元の値 0.7 を 0.8 [2003-08-23: 0.85] に変えてあります。
%
%    \begin{macrocode}
\renewcommand{\topfraction}{.85}
%    \end{macrocode}
% \end{macro}
%
% \begin{macro}{\c@bottomnumber}
%
% |bottomnumber| カウンタは本文ページ下部のフロートの最大数です。
%
% [2003-08-23] ちょっと増やしました。
%
%    \begin{macrocode}
\setcounter{bottomnumber}{9}
%    \end{macrocode}
% \end{macro}
%
% \begin{macro}{\bottomfraction}
%
% 本文ページ下部のフロートが占有できる最大の割合です。元は 0.3 でした。
%
%    \begin{macrocode}
\renewcommand{\bottomfraction}{.8}
%    \end{macrocode}
% \end{macro}
%
% \begin{macro}{\c@totalnumber}
%
% |totalnumber| カウンタは本文ページに入りうるフロートの最大数です。
%
% [2003-08-23] ちょっと増やしました。
%
%    \begin{macrocode}
\setcounter{totalnumber}{20}
%    \end{macrocode}
% \end{macro}
%
% \begin{macro}{\textfraction}
%
% 本文ページに最低限入らなければならない本文の割合です。
% フロートが入りやすいように元の 0.2 を 0.1 に変えました。
%
%    \begin{macrocode}
\renewcommand{\textfraction}{.1}
%    \end{macrocode}
% \end{macro}
%
% \begin{macro}{\floatpagefraction}
%
% フロートだけのページでのフロートの最小割合です。
% これも 0.5 を 0.8 に変えてあります。
%
%    \begin{macrocode}
\renewcommand{\floatpagefraction}{.8}
%    \end{macrocode}
% \end{macro}
%
% \begin{macro}{\c@dbltopnumber}
%
% 二段組のとき本文ページ上部に出力できる
% 段抜きフロートの最大数です。
%
% [2003-08-23] ちょっと増やしました。
%
%    \begin{macrocode}
\setcounter{dbltopnumber}{9}
%    \end{macrocode}
% \end{macro}
%
% \begin{macro}{\dbltopfraction}
%
% 二段組のとき本文ページ上部に出力できる
% 段抜きフロートが占めうる最大の割合です。
% 0.7 を 0.8 に変えてあります。
%
%    \begin{macrocode}
\renewcommand{\dbltopfraction}{.8}
%    \end{macrocode}
% \end{macro}
%
% \begin{macro}{\dblfloatpagefraction}
%
% 二段組のときフロートだけのページに入るべき
% 段抜きフロートの最小割合です。
% 0.5 を 0.8 に変えてあります。
%
%    \begin{macrocode}
\renewcommand{\dblfloatpagefraction}{.8}
%    \end{macrocode}
% \end{macro}
%
% \begin{macro}{\floatsep}
% \begin{macro}{\textfloatsep}
% \begin{macro}{\intextsep}
%
% |\floatsep| はページ上部・下部のフロート間の距離です。
% |\textfloatsep| はページ上部・下部のフロートと本文との距離です。
% |\intextsep| は本文の途中に出力されるフロートと本文との距離です。
%
%    \begin{macrocode}
\setlength\floatsep    {12\p@? \@plus 2\p@? \@minus 2\p@?}
\setlength\textfloatsep{20\p@? \@plus 2\p@? \@minus 4\p@?}
\setlength\intextsep   {12\p@? \@plus 2\p@? \@minus 2\p@?}
%    \end{macrocode}
% \end{macro}
% \end{macro}
% \end{macro}
%
% \begin{macro}{\dblfloatsep}
% \begin{macro}{\dbltextfloatsep}
%
% 二段組のときの段抜きのフロートについての値です。
%
%    \begin{macrocode}
\setlength\dblfloatsep    {12\p@? \@plus 2\p@? \@minus 2\p@?}
\setlength\dbltextfloatsep{20\p@? \@plus 2\p@? \@minus 4\p@?}
%    \end{macrocode}
% \end{macro}
% \end{macro}
%
% \begin{macro}{\@fptop}
% \begin{macro}{\@fpsep}
% \begin{macro}{\@fpbot}
%
% フロートだけのページに入るグルーです。
% |\@fptop| はページ上部,
% |\@fpbot| はページ下部,
% |\@fpsep| はフロート間に入ります。
%
%    \begin{macrocode}
\setlength\@fptop{0\p@? \@plus 1fil}
\setlength\@fpsep{8\p@? \@plus 2fil}
\setlength\@fpbot{0\p@? \@plus 1fil}
%    \end{macrocode}
% \end{macro}
% \end{macro}
% \end{macro}
%
% \begin{macro}{\@dblfptop}
% \begin{macro}{\@dblfpsep}
% \begin{macro}{\@dblfpbot}
%
% 段抜きフロートについての値です。
%
%    \begin{macrocode}
\setlength\@dblfptop{0\p@? \@plus 1fil}
\setlength\@dblfpsep{8\p@? \@plus 2fil}
\setlength\@dblfpbot{0\p@? \@plus 1fil}
%    \end{macrocode}
% \end{macro}
% \end{macro}
% \end{macro}
%
% \section{ページスタイル}\label{sec:pagestyle}
%
% ページスタイルとして,\LaTeXe (欧文版)の標準クラス
% では |empty|,|plain|,|headings|,|myheadings| があります。
% このうち |empty|,|plain| スタイルは\LaTeXe 本体
% で定義されています。
%
% アスキーのクラスファイルでは |headnombre|,|footnombre|,
% |bothstyle|,|jpl@in| が追加されていますが,
% ここでは欧文標準のものだけにしました。
%
% ページスタイルは |\ps@...| の形のマクロで定義されています。
%
% \begin{macro}{\@evenhead}
% \begin{macro}{\@oddhead}
% \begin{macro}{\@evenfoot}
% \begin{macro}{\@oddfoot}
%
% |\@oddhead|,|\@oddfoot|,|\@evenhead|,|\@evenfoot| は
% 偶数・奇数ページの柱(ヘッダ,フッタ)を出力する命令です。
% これらは |\fullwidth| 幅の |\hbox| の中で呼び出されます。
% |\ps@...| の中で定義しておきます。
%
% \end{macro}
% \end{macro}
% \end{macro}
% \end{macro}
%
% 柱の内容は,|\chapter| が呼び出す |\chaptermark{何々}|,
% |\section| が呼び出す |\sectionmark{何々}| で設定します。
% 柱を扱う命令には次のものがあります。
%
% \begin{quote}
%   \begin{tabbing}
%     |\markboth{左}{右} | \= 両方の柱を設定します。\\
%     |\markright{右}|     \> 右の柱を設定します。\\
%     |\leftmark|          \> 左の柱を出力します。\\
%     |\rightmark|         \> 右の柱を出力します。
%   \end{tabbing}
% \end{quote}
%
% 柱を設定する命令は,右の柱が左の柱の下位にある場合は十分まともに
% 動作します。たとえば左マークを |\chapter|,右マークを |\section|
% で変更する場合がこれにあたります。
% しかし,同一ページに複数の |\markboth| があると,
% おかしな結果になることがあります。
%
% |\tableofcontents| のような命令で使われる |\@mkboth| は,
% |\ps@...| コマンド中で |\markboth| か |\@gobbletwo|(何もしない)
% に |\let| されます。
%
% \begin{macro}{\ps@empty}
%
% |empty| ページスタイルの定義です。
% \LaTeX 本体で定義されているものをコメントアウトした形で
% 載せておきます。
%
%    \begin{macrocode}
% \def\ps@empty{%
%   \let\@mkboth\@gobbletwo
%   \let\@oddhead\@empty
%   \let\@oddfoot\@empty
%   \let\@evenhead\@empty
%   \let\@evenfoot\@empty}
%    \end{macrocode}
% \end{macro}
%
% \begin{macro}{\ps@plainhead}
% \begin{macro}{\ps@plainfoot}
% \begin{macro}{\ps@plain}
%
% |plainhead| はシンプルなヘッダだけのページスタイルです。
%
% |plainfoot| はシンプルなフッタだけのページスタイルです。
%
% |plain| は |book| では |plainhead|,それ以外では |plainfoot| になります。
%
%    \begin{macrocode}
\def\ps@plainfoot{%
  \let\@mkboth\@gobbletwo
  \let\@oddhead\@empty
  \def\@oddfoot{\normalfont\hfil\thepage\hfil}%
  \let\@evenhead\@empty
  \let\@evenfoot\@oddfoot}
\def\ps@plainhead{%
  \let\@mkboth\@gobbletwo
  \let\@oddfoot\@empty
  \let\@evenfoot\@empty
  \def\@evenhead{%
    \if@mparswitch \hss \fi
    \hbox to \fullwidth{\textbf{\thepage}\hfil}%
    \if@mparswitch\else \hss \fi}%
  \def\@oddhead{%
    \hbox to \fullwidth{\hfil\textbf{\thepage}}\hss}}
%<book>\let\ps@plain\ps@plainhead
%<!book>\let\ps@plain\ps@plainfoot
%    \end{macrocode}
% \end{macro}
% \end{macro}
% \end{macro}
%
% \begin{macro}{\ps@headings}
%
% |headings| スタイルはヘッダに見出しとページ番号を出力します。
% ここではヘッダにアンダーラインを引くようにしてみました。
%
% まず article の場合です。
%
%    \begin{macrocode}
%<*article|slide>
\if@twoside
  \def\ps@headings{%
    \let\@oddfoot\@empty
    \let\@evenfoot\@empty
    \def\@evenhead{\if@mparswitch \hss \fi
      \underline{\hbox to \fullwidth{\textbf{\thepage}\hfil\leftmark}}%
      \if@mparswitch\else \hss \fi}%
    \def\@oddhead{%
      \underline{%
        \hbox to \fullwidth{{\rightmark}\hfil\textbf{\thepage}}}\hss}%
    \let\@mkboth\markboth
    \def\sectionmark##1{\markboth{%
       \ifnum \c@secnumdepth >\z@ \thesection \hskip1\jsZw\fi
       ##1}{}}%
    \def\subsectionmark##1{\markright{%
       \ifnum \c@secnumdepth >\@ne \thesubsection \hskip1\jsZw\fi
       ##1}}%
  }
\else % if not twoside
  \def\ps@headings{%
    \let\@oddfoot\@empty
    \def\@oddhead{%
      \underline{%
        \hbox to \fullwidth{{\rightmark}\hfil\textbf{\thepage}}}\hss}%
    \let\@mkboth\markboth
    \def\sectionmark##1{\markright{%
        \ifnum \c@secnumdepth >\z@ \thesection \hskip1\jsZw\fi
        ##1}}}
\fi
%</article|slide>
%    \end{macrocode}
%
% 次は book の場合です。
% [2011-05-10] しっぽ愛好家さん [qa:6370] のパッチを取り込ませていただきました
% (北見さん [qa:55896] のご指摘ありがとうございます)。
%
%    \begin{macrocode}
%<*book|report>
\newif\if@omit@number
\def\ps@headings{%
  \let\@oddfoot\@empty
  \let\@evenfoot\@empty
  \def\@evenhead{%
    \if@mparswitch \hss \fi
    \underline{\hbox to \fullwidth{\autoxspacing
        \textbf{\thepage}\hfil\leftmark}}%
    \if@mparswitch\else \hss \fi}%
  \def\@oddhead{\underline{\hbox to \fullwidth{\autoxspacing
        {\if@twoside\rightmark\else\leftmark\fi}\hfil\textbf{\thepage}}}\hss}%
  \let\@mkboth\markboth
  \def\chaptermark##1{\markboth{%
    \ifnum \c@secnumdepth >\m@ne
      \if@mainmatter
        \if@omit@number\else
          \@chapapp\thechapter\@chappos\hskip1\jsZw
        \fi
      \fi
    \fi
    ##1}{}}%
  \def\sectionmark##1{\markright{%
    \ifnum \c@secnumdepth >\z@ \thesection \hskip1\jsZw\fi
    ##1}}}%
%</book|report>
%    \end{macrocode}
%
% 最後は学会誌の場合です。
%
%    \begin{macrocode}
%<*jspf>
\def\ps@headings{%
  \def\@oddfoot{\normalfont\hfil\thepage\hfil}
  \def\@evenfoot{\normalfont\hfil\thepage\hfil}
  \def\@oddhead{\normalfont\hfil \@title \hfil}
  \def\@evenhead{\normalfont\hfil プラズマ・核融合学会誌\hfil}}
%</jspf>
%    \end{macrocode}
%
% \end{macro}
%
% \begin{macro}{\ps@myheadings}
%
% |myheadings| ページスタイルではユーザが |\markboth| や |\markright| で
% 柱を設定するため,ここでの定義は非常に簡単です。
%
% [2004-01-17] 渡辺徹さんのパッチを適用しました。
%
%    \begin{macrocode}
\def\ps@myheadings{%
  \let\@oddfoot\@empty\let\@evenfoot\@empty
  \def\@evenhead{%
    \if@mparswitch \hss \fi%
    \hbox to \fullwidth{\thepage\hfil\leftmark}%
    \if@mparswitch\else \hss \fi}%
  \def\@oddhead{%
    \hbox to \fullwidth{\rightmark\hfil\thepage}\hss}%
  \let\@mkboth\@gobbletwo
%<book|report>  \let\chaptermark\@gobble
  \let\sectionmark\@gobble
%<!book&!report>  \let\subsectionmark\@gobble
}
%    \end{macrocode}
% \end{macro}
%
% \section{文書のマークアップ}
%
% \subsection{表題}
%
% \begin{macro}{\title}
% \begin{macro}{\author}
% \begin{macro}{\date}
%
%    これらは\LaTeX 本体で次のように定義されています。
%    ここではコメントアウトした形で示します。
%
%    \begin{macrocode}
% \newcommand*{\title}[1]{\gdef\@title{#1}}
% \newcommand*{\author}[1]{\gdef\@author{#1}}
% \newcommand*{\date}[1]{\gdef\@date{#1}}
% \date{\today}
%    \end{macrocode}
% \end{macro}
% \end{macro}
% \end{macro}
%
% \begin{ZRnote}
% \begin{macro}{\subtitle}
% \begin{macro}{\jsSubtitle}
% 副題を設定する。
% \Note プレアンブルにおいて |\newcommand*{\subtitle}{...}| が
% 行われることへの対策として、
% |\subtitle| の定義を |\title| の実行まで遅延させることにする。
% もしどうしても主題より前に副題を設定したい場合は、
% |\jsSubtitle| 命令を直接用いればよい。
% 
% 本体を |\jsSubtitle| として定義する。
%    \begin{macrocode}
\newcommand*{\jsSubtitle}[1]{\gdef\bxjs@subtitle{#1}}
%\let\bxjs@subtitle\@undefined
%    \end{macrocode}
%
% |\title| にフックを入れる。
%    \begin{macrocode}
\renewcommand*{\title}[1]{\bxjs@decl@subtitle\gdef\@title{#1}}
\AtBeginDocument{\bxjs@decl@subtitle}
\def\bxjs@decl@subtitle{%
  \global\let\bxjs@decl@subtitle\relax
  \ifx\subtitle\@undefined
    \global\let\subtitle\jsSubtitle
  \fi}
%    \end{macrocode}
% \end{macro}
% \end{macro}
%
% \begin{macro}{\bxjs@annihilate@subtitle}
% |\subtitle| 命令を無効化する。
% \Note 独自の |\subtitle| が使われている場合は無効化しない。
%    \begin{macrocode}
\def\bxjs@annihilate@subtitle{%
  \ifx\subtitle\jsSubtitle \global\let\subtitle\relax \fi
  \global\let\jsSubtitle\relax}
%    \end{macrocode}
% \end{macro}
%
% \end{ZRnote}
%
% \begin{macro}{\etitle}
% \begin{macro}{\eauthor}
% \begin{macro}{\keywords}
%
% 某学会誌スタイルで使う英語のタイトル,英語の著者名,キーワード,メールアドレスです。
%
%    \begin{macrocode}
%<*jspf>
\newcommand*{\etitle}[1]{\gdef\@etitle{#1}}
\newcommand*{\eauthor}[1]{\gdef\@eauthor{#1}}
\newcommand*{\keywords}[1]{\gdef\@keywords{#1}}
\newcommand*{\email}[1]{\gdef\authors@mail{#1}}
\newcommand*{\AuthorsEmail}[1]{\gdef\authors@mail{author's e-mail:\ #1}}
%</jspf>
%    \end{macrocode}
% \end{macro}
% \end{macro}
% \end{macro}
%
% \begin{macro}{\plainifnotempty}
%
% 従来の標準クラスでは,文書全体のページスタイルを |empty| に
% しても表題のあるページだけ |plain| になってしまうことが
% ありました。これは |\maketitle| の定義中
% に |\thispagestyle|\hspace{0pt}|{plain}| が入っている
% ためです。この問題を解決するために,
% 「全体のページスタイルが |empty| でないなら
% このページのスタイルを |plain| にする」という次の
% 命令を作ることにします。
%
%    \begin{macrocode}
\def\plainifnotempty{%
  \ifx \@oddhead \@empty
    \ifx \@oddfoot \@empty
    \else
      \thispagestyle{plainfoot}%
    \fi
  \else
    \thispagestyle{plainhead}%
  \fi}
%    \end{macrocode}
% \end{macro}
%
% \begin{macro}{\maketitle}
%
% 表題を出力します。
% 著者名を出力する部分は,欧文の標準クラスファイルでは |\large|,
% 和文のものでは |\Large| になっていましたが,ここでは |\large|
% にしました。
%
% [2016-11-16] 新設された \texttt{nomag} および \texttt{nomag*} オプション
% の場合をデフォルト(\texttt{usemag} 相当)に合わせるため,|\smallskip| を
% |\jsc@smallskip| に置き換えました。|\smallskip| のままでは
% \texttt{nomag(*)} の場合にスケールしなくなり,レイアウトが変わってしまいます。
%
%    \begin{macrocode}
%<*article|book|report|slide>
\if@titlepage
  \newcommand{\maketitle}{%
    \begin{titlepage}%
      \let\footnotesize\small
      \let\footnoterule\relax
      \let\footnote\thanks
      \null\vfil
      \if@slide
        {\footnotesize \@date}%
        \begin{center}
          \mbox{} \\[1\jsZw]
          \large
          {\maybeblue\hrule height0\p@? depth2\p@?\relax}\par
          \jsc@smallskip
          \@title
          \ifx\bxjs@subtitle\@undefined\else
            \par\vskip\z@
            {\small \bxjs@subtitle\par}
          \fi
          \jsc@smallskip
          {\maybeblue\hrule height0\p@? depth2\p@?\relax}\par
          \vfill
          {\small \@author}%
        \end{center}
      \else
      \vskip 60\p@?
      \begin{center}%
        {\LARGE \@title \par}%
        \ifx\bxjs@subtitle\@undefined\else
          \vskip5\p@?
          {\normalsize \bxjs@subtitle\par}
        \fi
        \vskip 3em%
        {\large
          \lineskip .75em
          \begin{tabular}[t]{c}%
            \@author
          \end{tabular}\par}%
        \vskip 1.5em
        {\large \@date \par}%
      \end{center}%
      \fi
      \par
      \@thanks\vfil\null
    \end{titlepage}%
    \setcounter{footnote}{0}%
    \global\let\thanks\relax
    \global\let\maketitle\relax
    \global\let\@thanks\@empty
    \global\let\@author\@empty
    \global\let\@date\@empty
    \global\let\@title\@empty
    \global\let\title\relax
    \global\let\author\relax
    \global\let\date\relax
    \global\let\and\relax
    \bxjs@annihilate@subtitle
  }%
\else
  \newcommand{\maketitle}{\par
    \begingroup
      \renewcommand\thefootnote{\@fnsymbol\c@footnote}%
      \def\@makefnmark{\rlap{\@textsuperscript{\normalfont\@thefnmark}}}%
      \long\def\@makefntext##1{\advance\leftskip 3\jsZw
        \parindent 1\jsZw\noindent
        \llap{\@textsuperscript{\normalfont\@thefnmark}\hskip0.3\jsZw}##1}%
      \if@twocolumn
        \ifnum \col@number=\@ne
          \@maketitle
        \else
          \twocolumn[\@maketitle]%
        \fi
      \else
        \newpage
        \global\@topnum\z@  % Prevents figures from going at top of page.
        \@maketitle
      \fi
      \plainifnotempty
      \@thanks
    \endgroup
    \setcounter{footnote}{0}%
    \global\let\thanks\relax
    \global\let\maketitle\relax
    \global\let\@thanks\@empty
    \global\let\@author\@empty
    \global\let\@date\@empty
    \global\let\@title\@empty
    \global\let\title\relax
    \global\let\author\relax
    \global\let\date\relax
    \global\let\and\relax
    \bxjs@annihilate@subtitle
  }
%    \end{macrocode}
% \end{macro}
%
% \begin{macro}{\@maketitle}
%
% 独立した表題ページを作らない場合の表題の出力形式です。
%
%    \begin{macrocode}
  \def\@maketitle{%
    \newpage\null
    \vskip 2em
    \begin{center}%
      \let\footnote\thanks
      {\LARGE \@title \par}%
      \ifx\bxjs@subtitle\@undefined\else
        \vskip3\p@?
        {\normalsize \bxjs@subtitle\par}
      \fi
      \vskip 1.5em
      {\large
        \lineskip .5em
        \begin{tabular}[t]{c}%
          \@author
        \end{tabular}\par}%
      \vskip 1em
      {\large \@date}%
    \end{center}%
    \par\vskip 1.5em
%<article|slide>    \ifvoid\@abstractbox\else\centerline{\box\@abstractbox}\vskip1.5em\fi
  }
\fi
%</article|book|report|slide>
%<*jspf>
\newcommand{\maketitle}{\par
  \begingroup
    \renewcommand\thefootnote{\@fnsymbol\c@footnote}%
    \def\@makefnmark{\rlap{\@textsuperscript{\normalfont\@thefnmark}}}%
    \long\def\@makefntext##1{\advance\leftskip 3\jsZw
      \parindent 1\jsZw\noindent
      \llap{\@textsuperscript{\normalfont\@thefnmark}\hskip0.3\jsZw}##1}%
      \twocolumn[\@maketitle]%
    \plainifnotempty
    \@thanks
  \endgroup
  \setcounter{footnote}{0}%
  \global\let\thanks\relax
  \global\let\maketitle\relax
  \global\let\@thanks\@empty
  \global\let\@author\@empty
  \global\let\@date\@empty
% \global\let\@title\@empty % \@title は柱に使う
  \global\let\title\relax
  \global\let\author\relax
  \global\let\date\relax
  \global\let\and\relax
  \ifx\authors@mail\@undefined\else{%
    \def\@makefntext{\advance\leftskip 3\jsZw \parindent -3\jsZw}%
    \footnotetext[0]{\itshape\authors@mail}%
  }\fi
  \global\let\authors@mail\@undefined}
\def\@maketitle{%
  \newpage\null
  \vskip 6em % used to be 2em
  \begin{center}
    \let\footnote\thanks
    \ifx\@title\@undefined\else{\LARGE\headfont\@title\par}\fi
    \lineskip .5em
    \ifx\@author\@undefined\else
      \vskip 1em
      \begin{tabular}[t]{c}%
        \@author
      \end{tabular}\par
    \fi
    \ifx\@etitle\@undefined\else
      \vskip 1em
      {\large \@etitle \par}%
    \fi
    \ifx\@eauthor\@undefined\else
      \vskip 1em
      \begin{tabular}[t]{c}%
        \@eauthor
      \end{tabular}\par
    \fi
    \vskip 1em
    \@date
  \end{center}
  \vskip 1.5em
  \centerline{\box\@abstractbox}
  \ifx\@keywords\@undefined\else
    \vskip 1.5em
    \centerline{\parbox{157mm}{\textsf{Keywords:}\\ \small\@keywords}}
  \fi
  \vskip 1.5em}
%</jspf>
%    \end{macrocode}
% \end{macro}
%
% \subsection{章・節}
%
% \paragraph{構成要素}
%
% |\@startsection| マクロは6個の必須引数と,オプションとして |*| と
% 1個のオプション引数と1個の必須引数をとります。
%
% \begin{quote}
% |\@startsection{名}{レベル}{字下げ}{前アキ}{後アキ}{スタイル}| \\
% |              *[別見出し]{見出し}|
% \end{quote}
%
% それぞれの引数の意味は次の通りです。
%
% \begin{description}
% \item[名] ユーザレベルコマンドの名前です(例: section)。
% \item[レベル] 見出しの深さを示す数値です
%    (chapter=1, section=2, \ldots )。
%    この数値が |secnumdepth| 以下のとき見出し番号を出力します。
% \item[字下げ] 見出しの字下げ量です。
% \item[前アキ] この値の絶対値が見出し上側の空きです。
%    負の場合は,見出し直後の段落をインデントしません。
% \item[後アキ] 正の場合は,見出しの下の空きです。
%    負の場合は,絶対値が見出しの右の空きです
%    (見出しと同じ行から本文を始めます)。
% \item[スタイル] 見出しの文字スタイルの設定です。
% \item[\texttt{*}] この \texttt{*} 印がないと,見出し番号を付け,
%    見出し番号のカウンタに1を加算します。
% \item[別見出し] 目次や柱に出力する見出しです。
% \item[見出し] 見出しです。
% \end{description}
%
% 見出しの命令は通常 |\@startsection| とその最初の6個の引数として
% 定義されます。
%
% 次は |\@startsection| の定義です。
% 情報処理学会論文誌スタイルファイル(\texttt{ipsjcommon.sty})
% を参考にさせていただきましたが,完全に行送りが |\baselineskip|
% の整数倍にならなくてもいいから前の行と重ならないようにしました。
%
%    \begin{macrocode}
\def\@startsection#1#2#3#4#5#6{%
  \if@noskipsec \leavevmode \fi
  \par
% 見出し上の空きを \@tempskipa にセットする
  \@tempskipa #4\relax
% \@afterindent は見出し直後の段落を字下げするかどうかを表すスイッチ
  \if@english \@afterindentfalse \else \@afterindenttrue \fi
% 見出し上の空きが負なら見出し直後の段落を字下げしない
  \ifdim \@tempskipa <\z@
    \@tempskipa -\@tempskipa \@afterindentfalse
  \fi
  \if@nobreak
%   \everypar{\everyparhook}% これは間違い
    \everypar{}%
  \else
    \addpenalty\@secpenalty
% 次の行は削除
%   \addvspace\@tempskipa
% 次の \noindent まで追加
    \ifdim \@tempskipa >\z@
      \if@slide\else
        \null
        \vspace*{-\baselineskip}%
      \fi
      \vskip\@tempskipa
    \fi
  \fi
  \noindent
% 追加終わり
  \@ifstar
    {\@ssect{#3}{#4}{#5}{#6}}%
    {\@dblarg{\@sect{#1}{#2}{#3}{#4}{#5}{#6}}}}
%    \end{macrocode}
%
% |\@sect| と |\@xsect| は,
% 前のアキがちょうどゼロの場合にもうまくいくように,多少変えてあります。
% |\everyparhook| も挿入しています。
%
%    \begin{macrocode}
\def\@sect#1#2#3#4#5#6[#7]#8{%
  \ifnum #2>\c@secnumdepth
    \let\@svsec\@empty
  \else
    \refstepcounter{#1}%
    \protected@edef\@svsec{\@seccntformat{#1}\relax}%
  \fi
% 見出し後の空きを \@tempskipa にセット
  \@tempskipa #5\relax
% 条件判断の順序を入れ換えました
  \ifdim \@tempskipa<\z@
    \def\@svsechd{%
      #6{\hskip #3\relax
      \@svsec #8}%
      \csname #1mark\endcsname{#7}%
      \addcontentsline{toc}{#1}{%
        \ifnum #2>\c@secnumdepth \else
          \protect\numberline{\csname the#1\endcsname}%
        \fi
        #7}}% 目次にフルネームを載せるなら #8
  \else
    \begingroup
      \interlinepenalty \@M % 下から移動
      #6{%
        \@hangfrom{\hskip #3\relax\@svsec}%
%       \interlinepenalty \@M % 上に移動
        #8\@@par}%
    \endgroup
    \csname #1mark\endcsname{#7}%
    \addcontentsline{toc}{#1}{%
      \ifnum #2>\c@secnumdepth \else
        \protect\numberline{\csname the#1\endcsname}%
      \fi
      #7}% 目次にフルネームを載せるならここは #8
  \fi
  \@xsect{#5}}
%    \end{macrocode}
%
% 二つ挿入した |\everyparhook| のうち後者が |\paragraph| 類の後で2回実行され,
% それ以降は前者が実行されます。
%
% [2016-07-28] \texttt{slide}オプションと\texttt{twocolumn}オプションを
% 同時に指定した場合の罫線の位置を微調整しました。
%
%    \begin{macrocode}
\def\@xsect#1{%
% 見出しの後ろの空きを \@tempskipa にセット
  \@tempskipa #1\relax
% 条件判断の順序を変えました
  \ifdim \@tempskipa<\z@
    \@nobreakfalse
    \global\@noskipsectrue
    \everypar{%
      \if@noskipsec
        \global\@noskipsecfalse
       {\setbox\z@\lastbox}%
        \clubpenalty\@M
        \begingroup \@svsechd \endgroup
        \unskip
        \@tempskipa #1\relax
        \hskip -\@tempskipa
        \bxjs@ltj@inhibitglue
      \else
        \clubpenalty \@clubpenalty
        \everypar{\everyparhook}%
      \fi\everyparhook}%
  \else
    \par \nobreak
    \vskip \@tempskipa
    \@afterheading
  \fi
  \if@slide
    {\vskip\if@twocolumn-5\jsc@mpt\else-6\jsc@mpt\fi
     \maybeblue\hrule height0\jsc@mpt depth1\jsc@mpt
     \vskip\if@twocolumn 4\jsc@mpt\else 7\jsc@mpt\fi\relax}%
  \fi
  \par  % 2000-12-18
  \ignorespaces}
\def\@ssect#1#2#3#4#5{%
  \@tempskipa #3\relax
  \ifdim \@tempskipa<\z@
    \def\@svsechd{#4{\hskip #1\relax #5}}%
  \else
    \begingroup
      #4{%
        \@hangfrom{\hskip #1}%
          \interlinepenalty \@M #5\@@par}%
    \endgroup
  \fi
  \@xsect{#3}}
%    \end{macrocode}
%
% \begin{ZRnote}
% 上記の定義中の |\bxjs@ltj@inhibitglue| は{Lua\TeX-ja}で
% 用いられるフック。
%    \begin{macrocode}
\let\bxjs@ltj@inhibitglue\@empty
%    \end{macrocode}
% \end{ZRnote}
%
% \paragraph{柱関係の命令}
%
% \begin{macro}{\chaptermark}
% \begin{macro}{\sectionmark}
% \begin{macro}{\subsectionmark}
% \begin{macro}{\subsubsectionmark}
% \begin{macro}{\paragraphmark}
% \begin{macro}{\subparagraphmark}
%
% |\...mark| の形の命令を初期化します(第\ref{sec:pagestyle}節参照)。
% |\chaptermark| 以外は\LaTeX 本体で定義済みです。
%
%    \begin{macrocode}
\newcommand*\chaptermark[1]{}
% \newcommand*{\sectionmark}[1]{}
% \newcommand*{\subsectionmark}[1]{}
% \newcommand*{\subsubsectionmark}[1]{}
% \newcommand*{\paragraphmark}[1]{}
% \newcommand*{\subparagraphmark}[1]{}
%    \end{macrocode}
% \end{macro}
% \end{macro}
% \end{macro}
% \end{macro}
% \end{macro}
% \end{macro}
%
% \paragraph{カウンタの定義}
%
% \begin{macro}{\c@secnumdepth}
%
% |secnumdepth| は第何レベルの見出しまで
% 番号を付けるかを決めるカウンタです。
%
%    \begin{macrocode}
%<!book&!report>\setcounter{secnumdepth}{3}
%<book|report>\setcounter{secnumdepth}{2}
%    \end{macrocode}
% \end{macro}
%
% \begin{macro}{\c@chapter}
% \begin{macro}{\c@section}
% \begin{macro}{\c@subsection}
% \begin{macro}{\c@subsubsection}
% \begin{macro}{\c@paragraph}
% \begin{macro}{\c@subparagraph}
%
% 見出し番号のカウンタです。
% |\newcounter| の第1引数が新たに作るカウンタです。
% これは第2引数が増加するたびに 0 に戻されます。
% 第2引数は定義済みのカウンタです。
%
%    \begin{macrocode}
\newcounter{part}
%<book|report>\newcounter{chapter}
%<book|report>\newcounter{section}[chapter]
%<!book&!report>\newcounter{section}
\newcounter{subsection}[section]
\newcounter{subsubsection}[subsection]
\newcounter{paragraph}[subsubsection]
\newcounter{subparagraph}[paragraph]
%    \end{macrocode}
% \end{macro}
% \end{macro}
% \end{macro}
% \end{macro}
% \end{macro}
% \end{macro}
%
% \begin{macro}{\thepart}
% \begin{macro}{\thechapter}
% \begin{macro}{\thesection}
% \begin{macro}{\thesubsection}
% \begin{macro}{\thesubsubsection}
% \begin{macro}{\theparagraph}
% \begin{macro}{\thesubparagraph}
%
% カウンタの値を出力する命令 |\the何々| を定義します。
%
% カウンタを出力するコマンドには次のものがあります。
%
% \begin{quote}
%   |\arabic{COUNTER}   | 1, 2, 3, \ldots \\
%   |\roman{COUNTER}    | i, ii, iii, \ldots \\
%   |\Roman{COUNTER}    | I, II, III, \ldots \\
%   |\alph{COUNTER}     | a, b, c, \ldots \\
%   |\Alph{COUNTER}     | A, B, C, \ldots \\
%   |\kansuji{COUNTER}  | 一, 二, 三, \ldots
% \end{quote}
%
% 以下ではスペース節約のため |@| の付いた内部表現を多用しています。
%
%    \begin{macrocode}
\renewcommand{\thepart}{\@Roman\c@part}
%<!book&!report>% \renewcommand{\thesection}{\@arabic\c@section}
%<!book&!report>\renewcommand{\thesection}{\presectionname\@arabic\c@section\postsectionname}
%<!book&!report>\renewcommand{\thesubsection}{\@arabic\c@section.\@arabic\c@subsection}
%<*book|report>
\renewcommand{\thechapter}{\@arabic\c@chapter}
\renewcommand{\thesection}{\thechapter.\@arabic\c@section}
\renewcommand{\thesubsection}{\thesection.\@arabic\c@subsection}
%</book|report>
\renewcommand{\thesubsubsection}{%
   \thesubsection.\@arabic\c@subsubsection}
\renewcommand{\theparagraph}{%
   \thesubsubsection.\@arabic\c@paragraph}
\renewcommand{\thesubparagraph}{%
   \theparagraph.\@arabic\c@subparagraph}
%    \end{macrocode}
% \end{macro}
% \end{macro}
% \end{macro}
% \end{macro}
% \end{macro}
% \end{macro}
% \end{macro}
%
% \begin{macro}{\@chapapp}
% \begin{macro}{\@chappos}
%
% |\@chapapp| の初期値は |\prechaptername|(第)です。
%
% |\@chappos| の初期値は |\postchaptername|(章)です。
%
% |\appendix| は |\@chapapp| を |\appendixname| に,
% |\@chappos| を空に再定義します。
%
% [2003-03-02] |\@secapp| は外しました。
%
%    \begin{macrocode}
%<book|report>\newcommand{\@chapapp}{\prechaptername}
%<book|report>\newcommand{\@chappos}{\postchaptername}
%    \end{macrocode}
% \end{macro}
% \end{macro}
%
% \paragraph{前付,本文,後付}
%
% 本のうち章番号があるのが「本文」,
% それ以外が「前付」「後付」です。
%
% \begin{macro}{\frontmatter}
%
% ページ番号をローマ数字にし,章番号を付けないようにします。
%
%    \begin{macrocode}
%<*book|report>
\newcommand\frontmatter{%
  \if@openright
    \cleardoublepage
  \else
    \clearpage
  \fi
  \@mainmatterfalse
  \pagenumbering{roman}}
%    \end{macrocode}
% \end{macro}
%
% \begin{macro}{\mainmatter}
%
% ページ番号を算用数字にし,章番号を付けるようにします。
%
%    \begin{macrocode}
\newcommand\mainmatter{%
% \if@openright
    \cleardoublepage
% \else
%   \clearpage
% \fi
  \@mainmattertrue
  \pagenumbering{arabic}}
%    \end{macrocode}
% \end{macro}
%
% \begin{macro}{\backmatter}
%
% 章番号を付けないようにします。ページ番号の付け方は変わりません。
%
%    \begin{macrocode}
\newcommand\backmatter{%
  \if@openright
    \cleardoublepage
  \else
    \clearpage
  \fi
  \@mainmatterfalse}
%</book|report>
%    \end{macrocode}
% \end{macro}
%
% \paragraph{部}
%
% \begin{macro}{\part}
%
% 新しい部を始めます。
%
% |\secdef| を使って見出しを定義しています。
% このマクロは二つの引数をとります。
%
% \begin{quote}
% |\secdef{星なし}{星あり}|
% \end{quote}
%
% \begin{description}
% \item[星なし] \texttt{*} のない形の定義です。
% \item[星あり] \texttt{*} のある形の定義です。
% \end{description}
%
% |\secdef| は次のようにして使います。
%
%\begin{verbatim}
%   \def\chapter { ... \secdef \CMDA \CMDB }
%   \def\CMDA    [#1]#2{....} % \chapter[...]{...} の定義
%   \def\CMDB    #1{....}     % \chapter*{...} の定義
%\end{verbatim}
%
% まず |book| クラス以外です。
%
%    \begin{macrocode}
%<*!book&!report>
\newcommand\part{%
  \if@noskipsec \leavevmode \fi
  \par
  \addvspace{4ex}%
  \if@english \@afterindentfalse \else \@afterindenttrue \fi
  \secdef\@part\@spart}
%</!book&!report>
%    \end{macrocode}
%
% |book| スタイルの場合は,少し複雑です。
%
%    \begin{macrocode}
%<*book|report>
\newcommand\part{%
  \if@openright
    \cleardoublepage
  \else
    \clearpage
  \fi
  \thispagestyle{empty}% 欧文用標準スタイルでは plain
  \if@twocolumn
    \onecolumn
    \@restonecoltrue
  \else
    \@restonecolfalse
  \fi
  \null\vfil
  \secdef\@part\@spart}
%</book|report>
%    \end{macrocode}
% \end{macro}
%
% \begin{macro}{\@part}
%
% 部の見出しを出力します。
% |\bfseries| を |\headfont| に変えました。
%
% |book| クラス以外では |secnumdepth| が $-1$ より大きいとき
% 部番号を付けます。
%
%    \begin{macrocode}
%<*!book&!report>
\def\@part[#1]#2{%
  \ifnum \c@secnumdepth >\m@ne
    \refstepcounter{part}%
    \addcontentsline{toc}{part}{%
      \prepartname\thepart\postpartname\hspace{1\jsZw}#1}%
  \else
    \addcontentsline{toc}{part}{#1}%
  \fi
  \markboth{}{}%
  {\parindent\z@
    \raggedright
    \interlinepenalty \@M
    \normalfont
    \ifnum \c@secnumdepth >\m@ne
      \Large\headfont\prepartname\thepart\postpartname
      \par\nobreak
    \fi
    \huge \headfont #2%
    \markboth{}{}\par}%
  \nobreak
  \vskip 3ex
  \@afterheading}
%</!book&!report>
%    \end{macrocode}
%
% |book| クラスでは |secnumdepth| が $-2$ より大きいとき部番号を付けます。
%
%    \begin{macrocode}
%<*book|report>
\def\@part[#1]#2{%
  \ifnum \c@secnumdepth >-2\relax
    \refstepcounter{part}%
    \addcontentsline{toc}{part}{%
      \prepartname\thepart\postpartname\hspace{1\jsZw}#1}%
  \else
    \addcontentsline{toc}{part}{#1}%
  \fi
  \markboth{}{}%
  {\centering
    \interlinepenalty \@M
    \normalfont
    \ifnum \c@secnumdepth >-2\relax
      \huge\headfont \prepartname\thepart\postpartname
      \par\vskip20\p@?
    \fi
    \Huge \headfont #2\par}%
  \@endpart}
%</book|report>
%    \end{macrocode}
% \end{macro}
%
% \begin{macro}{\@spart}
%
% 番号を付けない部です。
%
%    \begin{macrocode}
%<*!book&!report>
\def\@spart#1{{%
    \parindent \z@ \raggedright
    \interlinepenalty \@M
    \normalfont
    \huge \headfont #1\par}%
  \nobreak
  \vskip 3ex
  \@afterheading}
%</!book&!report>
%<*book|report>
\def\@spart#1{{%
    \centering
    \interlinepenalty \@M
    \normalfont
    \Huge \headfont #1\par}%
  \@endpart}
%</book|report>
%    \end{macrocode}
% \end{macro}
%
% \begin{macro}{\@endpart}
%
% |\@part| と |\@spart| の最後で実行されるマクロです。
% 両面印刷のときは白ページを追加します。
% 二段組のときには,二段組に戻します。
%
% [2016-12-13] \texttt{openany} のときには白ページが追加されるのは変なので,
% その場合は追加しないようにしました。このバグは\LaTeX では
% classes.dtx v1.4b (2000/05/19)
% で修正されています。
%
%    \begin{macrocode}
%<*book|report>
\def\@endpart{\vfil\newpage
  \if@twoside
   \if@openright %% added (2016/12/13)
    \null
    \thispagestyle{empty}%
    \newpage
   \fi %% added (2016/12/13)
  \fi
  \if@restonecol
    \twocolumn
  \fi}
%</book|report>
%    \end{macrocode}
% \end{macro}
%
% \paragraph{章}
%
% \begin{macro}{\chapter}
%
%    章の最初のページスタイルは,全体が |empty| でなければ |plain| に
%    します。
%    また,|\@topnum| を 0 にして,
%    章見出しの上に図や表が来ないようにします。
%
%    \begin{macrocode}
%<*book|report>
\newcommand{\chapter}{%
  \if@openright\cleardoublepage\else\clearpage\fi
  \plainifnotempty % 元: \thispagestyle{plain}
  \global\@topnum\z@
  \if@english \@afterindentfalse \else \@afterindenttrue \fi
  \secdef
    {\@omit@numberfalse\@chapter}%
    {\@omit@numbertrue\@schapter}}
%    \end{macrocode}
% \end{macro}
%
% \begin{macro}{\@chapter}
%
% 章見出しを出力します。
% |secnumdepth| が0以上かつ |\@mainmatter| が真のとき章番号を出力します。
%
%    \begin{macrocode}
\def\@chapter[#1]#2{%
  \ifnum \c@secnumdepth >\m@ne
    \if@mainmatter
      \refstepcounter{chapter}%
      \typeout{\@chapapp\thechapter\@chappos}%
      \addcontentsline{toc}{chapter}%
        {\protect\numberline
%       %{\if@english\thechapter\else\@chapapp\thechapter\@chappos\fi}%
        {\@chapapp\thechapter\@chappos}%
        #1}%
    \else\addcontentsline{toc}{chapter}{#1}\fi
  \else
    \addcontentsline{toc}{chapter}{#1}%
  \fi
  \chaptermark{#1}%
  \addtocontents{lof}{\protect\addvspace{10\jsc@mpt}}%
  \addtocontents{lot}{\protect\addvspace{10\jsc@mpt}}%
  \if@twocolumn
    \@topnewpage[\@makechapterhead{#2}]%
  \else
    \@makechapterhead{#2}%
    \@afterheading
  \fi}
%    \end{macrocode}
% \end{macro}
%
% \begin{macro}{\@makechapterhead}
%
%    実際に章見出しを組み立てます。
%    |\bfseries| を |\headfont| に変えました。
%
%    \begin{macrocode}
\def\@makechapterhead#1{%
  \vspace*{2\Cvs}% 欧文は50pt
  {\parindent \z@ \raggedright \normalfont
    \ifnum \c@secnumdepth >\m@ne
      \if@mainmatter
        \huge\headfont \@chapapp\thechapter\@chappos
        \par\nobreak
        \vskip \Cvs % 欧文は20pt
      \fi
    \fi
    \interlinepenalty\@M
    \Huge \headfont #1\par\nobreak
    \vskip 3\Cvs}} % 欧文は40pt
%    \end{macrocode}
% \end{macro}
%
% \begin{macro}{\@schapter}
%
% |\chapter*{...}| コマンドの本体です。
% |\chaptermark| を補いました。
%
%    \begin{macrocode}
\def\@schapter#1{%
  \chaptermark{#1}%
  \if@twocolumn
    \@topnewpage[\@makeschapterhead{#1}]%
  \else
    \@makeschapterhead{#1}\@afterheading
  \fi}
%    \end{macrocode}
% \end{macro}
%
% \begin{macro}{\@makeschapterhead}
%
% 番号なしの章見出しです。
%
%    \begin{macrocode}
\def\@makeschapterhead#1{%
  \vspace*{2\Cvs}% 欧文は50pt
  {\parindent \z@ \raggedright
    \normalfont
    \interlinepenalty\@M
    \Huge \headfont #1\par\nobreak
    \vskip 3\Cvs}} % 欧文は40pt
%</book|report>
%    \end{macrocode}
% \end{macro}
%
% \paragraph{下位レベルの見出し}
%
% \begin{macro}{\section}
%
% 欧文版では |\@startsection| の第4引数を負にして最初の段落の
% 字下げを禁止していますが,
% 和文版では正にして字下げするようにしています。
%
% 段組のときはなるべく左右の段が狂わないように工夫しています。
%
%    \begin{macrocode}
\if@twocolumn
  \newcommand{\section}{%
%<jspf>\ifx\maketitle\relax\else\maketitle\fi
    \@startsection{section}{1}{\z@}%
%<!kiyou>    {0.6\Cvs}{0.4\Cvs}%
%<kiyou>    {\Cvs}{0.5\Cvs}%
%   {\normalfont\large\headfont\@secapp}}
    {\normalfont\large\headfont\raggedright}}
\else
  \newcommand{\section}{%
    \if@slide\clearpage\fi
    \@startsection{section}{1}{\z@}%
    {\Cvs \@plus.5\Cdp \@minus.2\Cdp}% 前アキ
    {.5\Cvs \@plus.3\Cdp}% 後アキ
%   {\normalfont\Large\headfont\@secapp}}
    {\normalfont\Large\headfont\raggedright}}
\fi
%    \end{macrocode}
%
% \end{macro}
%
% \begin{macro}{\subsection}
%
% 同上です。
%
%    \begin{macrocode}
\if@twocolumn
  \newcommand{\subsection}{\@startsection{subsection}{2}{\z@}%
    {\z@}{\if@slide .4\Cvs \else \z@ \fi}%
    {\normalfont\normalsize\headfont}}
\else
  \newcommand{\subsection}{\@startsection{subsection}{2}{\z@}%
    {\Cvs \@plus.5\Cdp \@minus.2\Cdp}% 前アキ
    {.5\Cvs \@plus.3\Cdp}% 後アキ
    {\normalfont\large\headfont}}
\fi
%    \end{macrocode}
%
% \end{macro}
%
% \begin{macro}{\subsubsection}
%
% [2016-07-22] \texttt{slide}オプション指定時に |\subsubsection| の文字列
% と罫線が重なる問題に対処しました(forum:1982)。
%
%    \begin{macrocode}
\if@twocolumn
  \newcommand{\subsubsection}{\@startsection{subsubsection}{3}{\z@}%
    {\z@}{\if@slide .4\Cvs \else \z@ \fi}%
    {\normalfont\normalsize\headfont}}
\else
  \newcommand{\subsubsection}{\@startsection{subsubsection}{3}{\z@}%
    {\Cvs \@plus.5\Cdp \@minus.2\Cdp}%
    {\if@slide .5\Cvs \@plus.3\Cdp \else \z@ \fi}%
    {\normalfont\normalsize\headfont}}
\fi
%    \end{macrocode}
% \end{macro}
%
% \begin{macro}{\paragraph}
%
%    見出しの後ろで改行されません。
%
% [2016-11-16] 従来は |\paragraph| の最初に出るマークを「■」に固定して
% いましたが,このマークを変更可能にするため |\jsParagraphMark| というマクロ
% に切り出しました。これで,たとえば
%\begin{verbatim}
%  \renewcommand{\jsParagraphMark}{★}
%\end{verbatim}
% とすれば「★」に変更できますし,マークを空にすることも容易です。
% なお,某学会クラスでは従来どおりマークは付きません。
%
% \begin{ZRnote}
%
% \Note BXJSクラスでは、1.1版[2016-02-14]から |\jsParagraphMark| を
% サポートしている。
%
% 段落のマーク(■)が必ず和文フォントで出力されるようにする。
%
% |\jsJaChar| はstandard和文ドライバが読み込まれた場合は\
% |\jachar| と同義になるが、それ以外は何もしない。
% \end{ZRnote}
%    \begin{macrocode}
\newcommand\jsParagraphMark{\jsJaChar{■}}
\ifx\bxjs@paragraph@mark\@undefined\else
  \long\edef\jsParagraphMark{\noexpand\jsJaChar{\bxjs@paragraph@mark}}
\fi
\let\jsJaChar\@empty
\if@twocolumn
  \newcommand{\paragraph}{\@startsection{paragraph}{4}{\z@}%
    {\z@}{\if@slide .4\Cvs \else -1\jsZw\fi}% 改行せず 1\jsZw のアキ
%<jspf>    {\normalfont\normalsize\headfont}}
%<!jspf>    {\normalfont\normalsize\headfont\jsParagraphMark}}
\else
  \newcommand{\paragraph}{\@startsection{paragraph}{4}{\z@}%
    {0.5\Cvs \@plus.5\Cdp \@minus.2\Cdp}%
    {\if@slide .5\Cvs \@plus.3\Cdp \else -1\jsZw\fi}% 改行せず 1\jsZw のアキ
%<jspf>    {\normalfont\normalsize\headfont}}
%<!jspf>    {\normalfont\normalsize\headfont\jsParagraphMark}}
\fi
%    \end{macrocode}
% \end{macro}
%
% \begin{macro}{\subparagraph}
%
%    見出しの後ろで改行されません。
%
%    \begin{macrocode}
\if@twocolumn
  \newcommand{\subparagraph}{\@startsection{subparagraph}{5}{\z@}%
    {\z@}{\if@slide .4\Cvs \@plus.3\Cdp \else -1\jsZw\fi}%
    {\normalfont\normalsize\headfont}}
\else
  \newcommand{\subparagraph}{\@startsection{subparagraph}{5}{\z@}%
    {\z@}{\if@slide .5\Cvs \@plus.3\Cdp \else -1\jsZw\fi}%
    {\normalfont\normalsize\headfont}}
\fi
%    \end{macrocode}
% \end{macro}
%
% \subsection{リスト環境}
%
% 第 $k$ レベルのリストの初期化をするのが |\@list|$k$ です
% ($k = \mathtt{i}, \mathtt{ii}, \mathtt{iii}, \mathtt{iv}$)。
% |\@list|$k$ は |\leftmargin| を |\leftmargin|$k$ に設定します。
%
% \begin{macro}{\leftmargini}
%
% 二段組であるかないかに応じてそれぞれ 2em,2.5em でしたが,
% ここでは全角幅の2倍にしました。
%
% [2002-05-11] 3zw に変更しました。
%
% [2005-03-19] 二段組は 2zw に戻しました。
%
%    \begin{macrocode}
\if@slide
  \setlength\leftmargini{1\jsZw}
\else
  \if@twocolumn
    \setlength\leftmargini{2\jsZw}
  \else
    \setlength\leftmargini{3\jsZw}
  \fi
\fi
%    \end{macrocode}
% \end{macro}
%
% \begin{macro}{\leftmarginii}
% \begin{macro}{\leftmarginiii}
% \begin{macro}{\leftmarginiv}
% \begin{macro}{\leftmarginv}
% \begin{macro}{\leftmarginvi}
%
%    |ii|,|iii|,|iv| は |\labelsep| と
%    それぞれ `(m)',`vii.',`M.' の幅との和より大きくする
%    ことになっています。ここでは全角幅の整数倍に丸めました。
%
%    \begin{macrocode}
\if@slide
  \setlength\leftmarginii {1\jsZw}
  \setlength\leftmarginiii{1\jsZw}
  \setlength\leftmarginiv {1\jsZw}
  \setlength\leftmarginv  {1\jsZw}
  \setlength\leftmarginvi {1\jsZw}
\else
  \setlength\leftmarginii {2\jsZw}
  \setlength\leftmarginiii{2\jsZw}
  \setlength\leftmarginiv {2\jsZw}
  \setlength\leftmarginv  {1\jsZw}
  \setlength\leftmarginvi {1\jsZw}
\fi
%    \end{macrocode}
% \end{macro}
% \end{macro}
% \end{macro}
% \end{macro}
% \end{macro}
%
% \begin{macro}{\labelsep}
% \begin{macro}{\labelwidth}
%
%    |\labelsep| はラベルと本文の間の距離です。
%    |\labelwidth| はラベルの幅です。
%    これは二分に変えました。
%
%    \begin{macrocode}
\setlength  \labelsep  {0.5\jsZw} % .5em
\setlength  \labelwidth{\leftmargini}
\addtolength\labelwidth{-\labelsep}
%    \end{macrocode}
% \end{macro}
% \end{macro}
%
% \begin{macro}{\partopsep}
%
%    リスト環境の前に空行がある場合,
%    |\parskip| と |\topsep| に |\partopsep| を
%    加えた値だけ縦方向の空白ができます。
%    0 に改変しました。
%
%    \begin{macrocode}
\setlength\partopsep{\z@} % {2\p@ \@plus 1\p@ \@minus 1\p@}
%    \end{macrocode}
% \end{macro}
%
% \begin{macro}{\@beginparpenalty}
% \begin{macro}{\@endparpenalty}
% \begin{macro}{\@itempenalty}
%
%    リストや段落環境の前後,リスト項目間に挿入されるペナルティです。
%
%    \begin{macrocode}
\@beginparpenalty -\@lowpenalty
\@endparpenalty   -\@lowpenalty
\@itempenalty     -\@lowpenalty
%    \end{macrocode}
% \end{macro}
% \end{macro}
% \end{macro}
%
% \begin{macro}{\@listi}
% \begin{macro}{\@listI}
%
%    |\@listi| は |\leftmargin|,|\parsep|,|\topsep|,|\itemsep| などの
%    トップレベルの定義をします。
%    この定義は,フォントサイズコマンドによって変更されます
%    (たとえば |\small| の中では小さい値に設定されます)。
%    このため,|\normalsize| がすべてのパラメータを戻せるように,
%    |\@listI| で |\@listi| のコピーを保存します。
%    元の値はかなり複雑ですが,ここでは簡素化してしまいました。
%    特に最初と最後に行送りの半分の空きが入るようにしてあります。
%    アスキーの標準スタイルでは
%    トップレベルの |itemize|,|enumerate| 環境でだけ
%    最初と最後に行送りの半分の空きが入るようになっていました。
%
%    [2004-09-27] |\topsep| のグルー $_{-0.1}^{+0.2}$ |\baselineskip|
%    を思い切って外しました。
%
%    \begin{macrocode}
\def\@listi{\leftmargin\leftmargini
  \parsep \z@
  \topsep 0.5\baselineskip
  \itemsep \z@ \relax}
\let\@listI\@listi
%    \end{macrocode}
%
%    念のためパラメータを初期化します(実際には不要のようです)。
%
%    \begin{macrocode}
\@listi
%    \end{macrocode}
% \end{macro}
% \end{macro}
%
% \begin{macro}{\@listii}
% \begin{macro}{\@listiii}
% \begin{macro}{\@listiv}
% \begin{macro}{\@listv}
% \begin{macro}{\@listvi}
%
%    第2\zrWDash6レベルのリスト環境のパラメータの設定です。
%
%    \begin{macrocode}
\def\@listii{\leftmargin\leftmarginii
  \labelwidth\leftmarginii \advance\labelwidth-\labelsep
  \topsep \z@
  \parsep \z@
  \itemsep\parsep}
\def\@listiii{\leftmargin\leftmarginiii
  \labelwidth\leftmarginiii \advance\labelwidth-\labelsep
  \topsep \z@
  \parsep \z@
  \itemsep\parsep}
\def\@listiv {\leftmargin\leftmarginiv
              \labelwidth\leftmarginiv
              \advance\labelwidth-\labelsep}
\def\@listv  {\leftmargin\leftmarginv
              \labelwidth\leftmarginv
              \advance\labelwidth-\labelsep}
\def\@listvi {\leftmargin\leftmarginvi
              \labelwidth\leftmarginvi
              \advance\labelwidth-\labelsep}
%    \end{macrocode}
% \end{macro}
% \end{macro}
% \end{macro}
% \end{macro}
% \end{macro}
%
% \paragraph{enumerate環境}
%
% |enumerate| 環境はカウンタ |enumi|,|enumii|,|enumiii|,
% |enumiv| を使います。|enum|$n$ は第 $n$ レベルの番号です。
%
% \begin{macro}{\theenumi}
% \begin{macro}{\theenumii}
% \begin{macro}{\theenumiii}
% \begin{macro}{\theenumiv}
%
%    出力する番号の書式を設定します。
%    これらは\LaTeX 本体(\texttt{ltlists.dtx} 参照)で定義済みですが,
%    ここでは表し方を変えています。
%    |\@arabic|,|\@alph|,|\@roman|,|\@Alph| はそれぞれ
%    算用数字,小文字アルファベット,小文字ローマ数字,大文字アルファベット
%    で番号を出力する命令です。
%
%    \begin{macrocode}
\renewcommand{\theenumi}{\@arabic\c@enumi}
\renewcommand{\theenumii}{\@alph\c@enumii}
\renewcommand{\theenumiii}{\@roman\c@enumiii}
\renewcommand{\theenumiv}{\@Alph\c@enumiv}
%    \end{macrocode}
% \end{macro}
% \end{macro}
% \end{macro}
% \end{macro}
%
% \begin{macro}{\labelenumi}
% \begin{macro}{\labelenumii}
% \begin{macro}{\labelenumiii}
% \begin{macro}{\labelenumiv}
%
%    |enumerate| 環境の番号を出力する命令です。
%    第2レベル以外は最後に欧文のピリオドが付きますが,
%    これは好みに応じて取り払ってください。
%    第2レベルの番号のかっこは和文用に換え,
%    その両側に入る余分なグルーを |\inhibitglue| で
%    取り除いています。
% \begin{ZRnote}
% 和文の括弧で囲むための補助命令 |\jsInJaParen| を
% 定義して |\labelenumii| でそれを用いている。
% \Note 現状の |zxjatype| の |\inhibitglue| の実装には
% 「前後のグルーを消してしまう」という不備があって、
% そのため |enumii| の出力が異常になるという不具合があった。
% |zxjatype| を修正するまでの回避策として、サイズがゼロの
% 罫(|\bxjs@dust|)でガードしておく。
% \end{ZRnote}
%
%    \begin{macrocode}
\def\bxjs@dust{\vrule\@width\z@\@height\z@\@depth\z@}
\newcommand*{\jsInJaParen}[1]{%
  \bxjs@dust\jsInhibitGlue(\theenumii)\jsInhibitGlue\bxjs@dust}
\newcommand{\labelenumi}{\theenumi.}
\newcommand{\labelenumii}{\jsInJaParen{(\theenumii)}}
\newcommand{\labelenumiii}{\theenumiii.}
\newcommand{\labelenumiv}{\theenumiv.}
%    \end{macrocode}
% \end{macro}
% \end{macro}
% \end{macro}
% \end{macro}
%
% \begin{macro}{\p@enumii}
% \begin{macro}{\p@enumiii}
% \begin{macro}{\p@enumiv}
%
%    |\p@enum|$n$ は |\ref| コマンドで |enumerate| 環境の第 $n$ レベルの
%    項目が参照されるときの書式です。
%    これも第2レベルは和文用かっこにしました。
%
%    \begin{macrocode}
\renewcommand{\p@enumii}{\theenumi}
\renewcommand{\p@enumiii}{\theenumi\jsInhibitGlue(\theenumii)}
\renewcommand{\p@enumiv}{\p@enumiii\theenumiii}
%    \end{macrocode}
% \end{macro}
% \end{macro}
% \end{macro}
%
% \paragraph{itemize環境}
%
% \begin{macro}{\labelitemi}
% \begin{macro}{\labelitemii}
% \begin{macro}{\labelitemiii}
% \begin{macro}{\labelitemiv}
%    |itemize| 環境の第 $n$ レベルのラベルを作るコマンドです。
%    \begin{macrocode}
\newcommand\labelitemi{\textbullet}
\newcommand\labelitemii{\normalfont\bfseries \textendash}
\newcommand\labelitemiii{\textasteriskcentered}
\newcommand\labelitemiv{\textperiodcentered}
%    \end{macrocode}
% \end{macro}
% \end{macro}
% \end{macro}
% \end{macro}
%
% \paragraph{description環境}
%
% \begin{environment}{description}
%
% 本来の |description| 環境では,項目名が短いと,説明部分の頭が
% それに引きずられて左に出てしまいます。
% これを解決した新しい |description| の実装です。
%
%    \begin{macrocode}
\newenvironment{description}{%
  \list{}{%
    \labelwidth=\leftmargin
    \labelsep=1\jsZw
    \advance \labelwidth by -\labelsep
    \let \makelabel=\descriptionlabel}}{\endlist}
%    \end{macrocode}
% \end{environment}
%
% \begin{macro}{\descriptionlabel}
%
%    |description| 環境のラベルを出力するコマンドです。
%    好みに応じて |#1| の前に適当な空き
%    (たとえば |\hspace{1\jsZw}|)を入れるのもいいと思います。
%
%    \begin{macrocode}
\newcommand*\descriptionlabel[1]{\normalfont\headfont #1\hfil}
%    \end{macrocode}
% \end{macro}
%
% \paragraph{概要}
%
% \begin{environment}{abstract}
%
% 概要(要旨,梗概)を出力する環境です。
% |book| クラスでは各章の初めにちょっとしたことを書くのに使います。
% |titlepage| オプション付きの |article| クラスでは,
% 独立したページに出力されます。
% |abstract| 環境は元は |quotation| 環境で作られていましたが,
% |quotation| 環境の右マージンをゼロにしたので,
% |list| 環境で作り直しました。
%
% JSPFスタイルでは実際の出力は |\maketitle| で行われます。
%
%    \begin{macrocode}
%<*book|report>
\newenvironment{abstract}{%
  \begin{list}{}{%
    \listparindent=1\jsZw
    \itemindent=\listparindent
    \rightmargin=\z@
    \leftmargin=5\jsZw}\item[]}{\end{list}\vspace{\baselineskip}}
%</book|report>
%<*article|slide>
\newbox\@abstractbox
\if@titlepage
  \newenvironment{abstract}{%
    \titlepage
    \null\vfil
    \@beginparpenalty\@lowpenalty
    \begin{center}%
      \headfont \abstractname
      \@endparpenalty\@M
    \end{center}}%
  {\par\vfil\null\endtitlepage}
\else
  \newenvironment{abstract}{%
    \if@twocolumn
      \ifx\maketitle\relax
        \section*{\abstractname}%
      \else
        \global\setbox\@abstractbox\hbox\bgroup
        \begin{minipage}[b]{\textwidth}
          \small\parindent1\jsZw
          \begin{center}%
            {\headfont \abstractname\vspace{-.5em}\vspace{\z@}}%
          \end{center}%
          \list{}{%
            \listparindent\parindent
            \itemindent \listparindent
            \rightmargin \leftmargin}%
          \item\relax
      \fi
    \else
      \small
      \begin{center}%
        {\headfont \abstractname\vspace{-.5em}\vspace{\z@}}%
      \end{center}%
      \list{}{%
        \listparindent\parindent
        \itemindent \listparindent
        \rightmargin \leftmargin}%
      \item\relax
    \fi}{\if@twocolumn
      \ifx\maketitle\relax
      \else
        \endlist\end{minipage}\egroup
      \fi
    \else
      \endlist
    \fi}
\fi
%</article|slide>
%<*jspf>
\newbox\@abstractbox
\newenvironment{abstract}{%
  \global\setbox\@abstractbox\hbox\bgroup
  \begin{minipage}[b]{157mm}{\sffamily Abstract}\par
    \small
    \if@english \parindent6mm \else \parindent1\jsZw \fi}%
  {\end{minipage}\egroup}
%</jspf>
%    \end{macrocode}
% \end{environment}
%
% \paragraph{キーワード}
%
% \begin{environment}{keywords}
%
% キーワードを準備する環境です。
% 実際の出力は |\maketitle| で行われます。
%
%    \begin{macrocode}
%<*jspf>
%\newbox\@keywordsbox
%\newenvironment{keywords}{%
%  \global\setbox\@keywordsbox\hbox\bgroup
%  \begin{minipage}[b]{157mm}{\sffamily Keywords:}\par
%    \small\parindent0\jsZw}%
%  {\end{minipage}\egroup}
%</jspf>
%    \end{macrocode}
% \end{environment}
%
% \paragraph{verse環境}
%
% \begin{environment}{verse}
%
% 詩のための |verse| 環境です。
%
%    \begin{macrocode}
\newenvironment{verse}{%
  \let \\=\@centercr
  \list{}{%
    \itemsep \z@
    \itemindent -2\jsZw % 元: -1.5em
    \listparindent\itemindent
    \rightmargin \z@
    \advance\leftmargin 2\jsZw}% 元: 1.5em
  \item\relax}{\endlist}
%    \end{macrocode}
% \end{environment}
%
% \paragraph{quotation環境}
%
% \begin{environment}{quotation}
%
% 段落の頭の字下げ量を1.5emから |\parindent| に変えました。
% また,右マージンを 0 にしました。
%
%    \begin{macrocode}
\newenvironment{quotation}{%
  \list{}{%
    \listparindent\parindent
    \itemindent\listparindent
    \rightmargin \z@}%
  \item\relax}{\endlist}
%    \end{macrocode}
% \end{environment}
%
% \paragraph{quote環境}
%
% \begin{environment}{quote}
%
% |quote| 環境は,段落がインデントされないことを除き,
% |quotation| 環境と同じです。
%
%    \begin{macrocode}
\newenvironment{quote}%
  {\list{}{\rightmargin\z@}\item\relax}{\endlist}
%    \end{macrocode}
% \end{environment}
%
% \paragraph{定理など}
%
% |ltthm.dtx| 参照。たとえば次のように定義します。
%\begin{verbatim}
%  \newtheorem{definition}{定義}
%  \newtheorem{axiom}{公理}
%  \newtheorem{theorem}{定理}
%\end{verbatim}
%
% [2001-04-26] 定理の中はイタリック体になりましたが,
% これでは和文がゴシック体になってしまうので,
% |\itshape| を削除しました。
%
% [2009-08-23] |\bfseries| を |\headfont| に直し,
% |\labelsep| を 1\,zw にし,括弧を全角にしました。
%
%    \begin{macrocode}
\def\@begintheorem#1#2{\trivlist\labelsep=1\jsZw
   \item[\hskip \labelsep{\headfont #1\ #2}]}
\def\@opargbegintheorem#1#2#3{\trivlist\labelsep=1\jsZw
      \item[\hskip \labelsep{\headfont #1\ #2(#3)}]}
%    \end{macrocode}
%
% \begin{environment}{titlepage}
%
% タイトルを独立のページに出力するのに使われます。
%
%    \begin{macrocode}
\newenvironment{titlepage}{%
%<book|report>    \cleardoublepage
    \if@twocolumn
      \@restonecoltrue\onecolumn
    \else
      \@restonecolfalse\newpage
    \fi
    \thispagestyle{empty}%
    \setcounter{page}\@ne
  }%
  {\if@restonecol\twocolumn \else \newpage \fi
    \if@twoside\else
      \setcounter{page}\@ne
    \fi}
%    \end{macrocode}
% \end{environment}
%
% \paragraph{付録}
%
% \begin{macro}{\appendix}
%
% 本文と付録を分離するコマンドです。
%
%    \begin{macrocode}
%<*!book&!report>
\newcommand{\appendix}{\par
  \setcounter{section}{0}%
  \setcounter{subsection}{0}%
  \gdef\presectionname{\appendixname}%
  \gdef\postsectionname{}%
% \gdef\thesection{\@Alph\c@section}% [2003-03-02]
  \gdef\thesection{\presectionname\@Alph\c@section\postsectionname}%
  \gdef\thesubsection{\@Alph\c@section.\@arabic\c@subsection}}
%</!book&!report>
%<*book|report>
\newcommand{\appendix}{\par
  \setcounter{chapter}{0}%
  \setcounter{section}{0}%
  \gdef\@chapapp{\appendixname}%
  \gdef\@chappos{}%
  \gdef\thechapter{\@Alph\c@chapter}}
%</book|report>
%    \end{macrocode}
% \end{macro}
%
% \subsection{パラメータの設定}
%
% \paragraph{arrayとtabular環境}
%
% \begin{macro}{\arraycolsep}
%
% |array| 環境の列間には |\arraycolsep| の2倍の幅の空きが入ります。
%
%    \begin{macrocode}
\setlength\arraycolsep{5\p@?}
%    \end{macrocode}
% \end{macro}
%
% \begin{macro}{\tabcolsep}
%
% |tabular| 環境の列間には |\tabcolsep| の2倍の幅の空きが入ります。
%
%    \begin{macrocode}
\setlength\tabcolsep{6\p@?}
%    \end{macrocode}
% \end{macro}
%
% \begin{macro}{\arrayrulewidth}
%
% |array|,|tabular| 環境内の罫線の幅です。
%
%    \begin{macrocode}
\setlength\arrayrulewidth{.4\p@}
%    \end{macrocode}
% \end{macro}
%
% \begin{macro}{\doublerulesep}
%
% |array|,|tabular| 環境での二重罫線間のアキです。
%
%    \begin{macrocode}
\setlength\doublerulesep{2\p@}
%    \end{macrocode}
% \end{macro}
%
% \paragraph{tabbing環境}
%
% \begin{macro}{\tabbingsep}
%
% |\'| コマンドで入るアキです。
%
%    \begin{macrocode}
\setlength\tabbingsep{\labelsep}
%    \end{macrocode}
% \end{macro}
%
% \paragraph{minipage環境}
%
% \begin{macro}{\@mpfootins}
%
% |minipage| 環境の脚注の |\skip|\hspace{0pt}|\@mpfootins|
% は通常のページの |\skip|\hspace{0pt}|\footins|
% と同じ働きをします。
%
%    \begin{macrocode}
\skip\@mpfootins = \skip\footins
%    \end{macrocode}
% \end{macro}
%
% \paragraph{framebox環境}
%
% \begin{macro}{\fboxsep}
%
% |\fbox|,|\framebox| で内側のテキストと枠との間の空きです。
%
% \begin{macro}{\fboxrule}
%
% |\fbox|,|\framebox| の罫線の幅です。
%
%    \begin{macrocode}
\setlength\fboxsep{3\p@?}
\setlength\fboxrule{.4\p@}
%    \end{macrocode}
% \end{macro}
% \end{macro}
%
% \paragraph{equationとeqnarray環境}
%
% \begin{macro}{\theequation}
%
% 数式番号を出力するコマンドです。
%
%    \begin{macrocode}
%<!book&!report>\renewcommand \theequation {\@arabic\c@equation}
%<*book|report>
\@addtoreset{equation}{chapter}
\renewcommand\theequation
  {\ifnum \c@chapter>\z@ \thechapter.\fi \@arabic\c@equation}
%</book|report>
%    \end{macrocode}
% \end{macro}
%
% \begin{macro}{\jot}
%
%    |eqnarray| の行間に余分に入るアキです。
%    デフォルトの値をコメントアウトして示しておきます。
%
%    \begin{macrocode}
% \setlength\jot{3pt}
%    \end{macrocode}
% \end{macro}
%
% \begin{macro}{\@eqnnum}
%
%    数式番号の形式です。
%    デフォルトの値をコメントアウトして示しておきます。
%
%    |\jsInhibitGlue(\theequation)\jsInhibitGlue| のように和文かっこ
%    を使うことも可能です。
%
%    \begin{macrocode}
% \def\@eqnnum{(\theequation)}
%    \end{macrocode}
% \end{macro}
%
%    |amsmath| パッケージを使う場合は |\tagform@| を次のように修正します。
%
%    \begin{macrocode}
% \def\tagform@#1{\maketag@@@{(\ignorespaces#1\unskip\@@italiccorr)}}
%    \end{macrocode}
%
% \subsection{フロート}
%
% タイプ \texttt{TYPE} のフロートオブジェクトを
% 扱うには,次のマクロを定義します。
% \begin{description}
% \item[\texttt{\bslash fps@TYPE}]
%   フロートを置く位置(float placement specifier)です。
% \item[\texttt{\bslash ftype@TYPE}]
%   フロートの番号です。2の累乗(1,2,4,\ldots )でなければなりません。
% \item[\texttt{\bslash ext@TYPE}]
%   フロートの目次を出力するファイルの拡張子です。
% \item[\texttt{\bslash fnum@TYPE}]
%   キャプション用の番号を生成するマクロです。
% \item[\texttt{\bslash @makecaption}{\meta{num}}{\meta{text}}]
%   キャプションを出力するマクロです。
%   \meta{num} は |\fnum@...| の生成する番号,
%   \meta{text} はキャプションのテキストです。
%   テキストは適当な幅の |\parbox| に入ります。
% \end{description}
%
% \paragraph{figure環境}
%
% \begin{macro}{\c@figure}
%
% 図番号のカウンタです。
%
% \begin{macro}{\thefigure}
%
% 図番号を出力するコマンドです。
%
%    \begin{macrocode}
%<*!book&!report>
\newcounter{figure}
\renewcommand \thefigure {\@arabic\c@figure}
%</!book&!report>
%<*book|report>
\newcounter{figure}[chapter]
\renewcommand \thefigure
     {\ifnum \c@chapter>\z@ \thechapter.\fi \@arabic\c@figure}
%</book|report>
%    \end{macrocode}
% \end{macro}
% \end{macro}
%
% \begin{macro}{\fps@figure}
% \begin{macro}{\ftype@figure}
% \begin{macro}{\ext@figure}
% \begin{macro}{\fnum@figure}
%
%    |figure| のパラメータです。
%    |\figurename| の直後に |~| が入っていましたが,
%    ここでは外しました。
%
%    \begin{macrocode}
\def\fps@figure{tbp}
\def\ftype@figure{1}
\def\ext@figure{lof}
\def\fnum@figure{\figurename\nobreak\thefigure}
%    \end{macrocode}
% \end{macro}
% \end{macro}
% \end{macro}
% \end{macro}
%
% \begin{environment}{figure}
% \begin{environment}{figure*}
%
% |*| 形式は段抜きのフロートです。
%
%    \begin{macrocode}
\newenvironment{figure}%
               {\@float{figure}}%
               {\end@float}
\newenvironment{figure*}%
               {\@dblfloat{figure}}%
               {\end@dblfloat}
%    \end{macrocode}
% \end{environment}
% \end{environment}
%
% \paragraph{table環境}
%
% \begin{macro}{\c@table}
% \begin{macro}{\thetable}
%
% 表番号カウンタと表番号を出力するコマンドです。
% アスキー版では |\thechapter.| が |\thechapter{}・| になっていますが,
% ここではオリジナルのままにしています。
%
%    \begin{macrocode}
%<*!book&!report>
\newcounter{table}
\renewcommand\thetable{\@arabic\c@table}
%</!book&!report>
%<*book|report>
\newcounter{table}[chapter]
\renewcommand \thetable
     {\ifnum \c@chapter>\z@ \thechapter.\fi \@arabic\c@table}
%</book|report>
%    \end{macrocode}
% \end{macro}
% \end{macro}
%
% \begin{macro}{\fps@table}
% \begin{macro}{\ftype@table}
% \begin{macro}{\ext@table}
% \begin{macro}{\fnum@table}
%
% |table| のパラメータです。
% |\tablename| の直後に |~| が入っていましたが,
% ここでは外しました。
%
%    \begin{macrocode}
\def\fps@table{tbp}
\def\ftype@table{2}
\def\ext@table{lot}
\def\fnum@table{\tablename\nobreak\thetable}
%    \end{macrocode}
% \end{macro}
% \end{macro}
% \end{macro}
% \end{macro}
%
% \begin{environment}{table}
% \begin{environment}{table*}
%
% |*| は段抜きのフロートです。
%
%    \begin{macrocode}
\newenvironment{table}%
               {\@float{table}}%
               {\end@float}
\newenvironment{table*}%
               {\@dblfloat{table}}%
               {\end@dblfloat}
%    \end{macrocode}
% \end{environment}
% \end{environment}
%
% \subsection{キャプション}
%
% \begin{macro}{\@makecaption}
%
% |\caption| コマンドにより呼び出され,
% 実際にキャプションを出力するコマンドです。
% 第1引数はフロートの番号,
% 第2引数はテキストです。
%
% \begin{macro}{\abovecaptionskip}
% \begin{macro}{\belowcaptionskip}
%
% それぞれキャプションの前後に挿入されるスペースです。
% |\belowcaptionskip| が0になっていましたので,
% キャプションを表の上につけた場合にキャプションと表が
% くっついてしまうのを直しました。
%
%    \begin{macrocode}
\newlength\abovecaptionskip
\newlength\belowcaptionskip
\setlength\abovecaptionskip{5\p@?} % 元: 10\p@
\setlength\belowcaptionskip{5\p@?} % 元: 0\p@
%    \end{macrocode}
% \end{macro}
% \end{macro}
%
% 実際のキャプションを出力します。
% オリジナルと異なり,文字サイズを |\small| にし,
% キャプションの幅を2cm狭くしました。
%
% [2003-11-05] ロジックを少し変えてみました。
%
%    \begin{macrocode}
%<*!jspf>
% \long\def\@makecaption#1#2{{\small
%   \advance\leftskip1cm
%   \advance\rightskip1cm
%   \vskip\abovecaptionskip
%   \sbox\@tempboxa{#1\hskip1\jsZw\relax #2}%
%   \ifdim \wd\@tempboxa >\hsize
%     #1\hskip1\jsZw\relax #2\par
%   \else
%     \global \@minipagefalse
%     \hb@xt@\hsize{\hfil\box\@tempboxa\hfil}%
%   \fi
%   \vskip\belowcaptionskip}}
\long\def\@makecaption#1#2{{\small
  \advance\leftskip .0628\linewidth
  \advance\rightskip .0628\linewidth
  \vskip\abovecaptionskip
  \sbox\@tempboxa{#1\hskip1\jsZw\relax #2}%
  \ifdim \wd\@tempboxa <\hsize \centering \fi
  #1\hskip1\jsZw\relax #2\par
  \vskip\belowcaptionskip}}
%</!jspf>
%<*jspf>
\long\def\@makecaption#1#2{%
  \vskip\abovecaptionskip
  \sbox\@tempboxa{\small\sffamily #1\quad #2}%
  \ifdim \wd\@tempboxa >\hsize
    {\small\sffamily
      \list{#1}{%
        \renewcommand{\makelabel}[1]{##1\hfil}
        \itemsep    \z@
        \itemindent \z@
        \labelsep   \z@
        \labelwidth 11mm
        \listparindent\z@
        \leftmargin 11mm}\item\relax #2\endlist}
  \else
    \global \@minipagefalse
    \hb@xt@\hsize{\hfil\box\@tempboxa\hfil}%
  \fi
  \vskip\belowcaptionskip}
%</jspf>
%    \end{macrocode}
% \end{macro}
%
% \section{フォントコマンド}
%
% ここでは\LaTeX~2.09で使われていたコマンドを定義します。
% これらはテキストモードと数式モードのどちらでも動作します。
% これらは互換性のためのもので,
% できるだけ |\text...| と |\math...| を使ってください。
%
% [2016-07-15] KOMA-Script中の |\scr@DeclareOldFontCommand| に倣い、
% これらの命令を使うときには警告を発することにしました。
%
% [2016-07-16] 警告を最初の一回だけ発することにしました。また,
% 例外的に警告を出さないようにするスイッチも付けます。
%
% \begin{macro}{\if@jsc@warnoldfontcmd}
% \begin{macro}{\if@jsc@warnoldfontcmdexception}
% \begin{ZRnote}
% |\if@jsc@warnoldfontcmd| はBXJSクラスでは不使用。\par
% |\if@jsc@warnoldfontcmdexception| は |\allow/disallowoldfontcommands|
% の状態を表す。
% \end{ZRnote}
%    \begin{macrocode}
\newif\if@jsc@warnoldfontcmd
\@jsc@warnoldfontcmdtrue
\newif\if@jsc@warnoldfontcmdexception
\@jsc@warnoldfontcmdexceptionfalse
%    \end{macrocode}
% \end{macro}
% \end{macro}
%
% \begin{macro}{\jsc@DeclareOldFontCommand}
%    \begin{macrocode}
\newcommand*{\jsc@DeclareOldFontCommand}[3]{%
  \g@addto@macro\bxjs@oldfontcmd@list{\do#1}%
  \DeclareOldFontCommand{#1}{%
    \bxjs@oldfontcmd{#1}#2%
  }{%
    \bxjs@oldfontcmd{#1}#3%
  }%
}
\DeclareRobustCommand*{\jsc@warnoldfontcmd}[1]{%
  \ClassInfo\bxjs@clsname
   {Old font command '\string#1' is used!!\MessageBreak
    The first occurrence is}%
}
%    \end{macrocode}
% \end{macro}
%
% \begin{ZRnote}
%
% \begin{macro}{\allowoldfontcommands}
% “二文字フォント命令”の使用を許可する(警告しない)。
% \begin{macro}{\disallowoldfontcommands}
% “二文字フォント命令”の使用に対して警告を出す。
%    \begin{macrocode}
\newcommand*{\allowoldfontcommands}{%
  \@jsc@warnoldfontcmdexceptiontrue}
\newcommand*{\disallowoldfontcommands}{%
  \@jsc@warnoldfontcmdexceptionfalse}
%    \end{macrocode}
% \end{macro}
% \end{macro}
%
% \Note 1.x版ではWarningではなくInfoに留めておく。
%    \begin{macrocode}
\let\bxjs@oldfontcmd@list\@empty
\def\bxjs@oldfontcmd#1{%
  \expandafter\bxjs@oldfontcmd@a\csname bxjs@ofc/\string#1\endcsname#1}
\def\bxjs@oldfontcmd@a#1#2{%
  \if@jsc@warnoldfontcmdexception\else
    \global\@jsc@warnoldfontcmdfalse
    \ifx#1\relax
      \global\let#1=t%
      \jsc@warnoldfontcmd{#2}%
    \fi
  \fi}
\def\bxjs@warnoldfontcmd@final{%
  \par
  \let\@tempa\@empty
  \def\do##1{%
    \expandafter\ifx\csname bxjs@ofc/\string##1\endcsname\relax\else
      \edef\@tempa{\@tempa \space\string##1}\fi}
  \bxjs@oldfontcmd@list
  \ifx\@tempa\@empty\else
    \ClassWarningNoLine\bxjs@clsname
     {Some old font commands were used in text\MessageBreak
      (see the log file for detail)}%
    \ClassInfo\bxjs@clsname
     {Some old font commands were used in text:\MessageBreak
      \space\@tempa\MessageBreak
      You should note, that since 1994 LaTeX2e provides a\MessageBreak
      new font selection scheme called NFSS2 with several\MessageBreak
      new, combinable font commands. The
      class provides\MessageBreak
      the old font commands
      only for compatibility%
      \@gobble}%
  \fi}
\AtEndDocument{\bxjs@warnoldfontcmd@final}
%    \end{macrocode}
%
% \end{ZRnote}
%
% \begin{macro}{\mc}
% \begin{macro}{\gt}
% \begin{macro}{\rm}
% \begin{macro}{\sf}
% \begin{macro}{\tt}
%
% フォントファミリを変更します。
%
%    \begin{macrocode}
\jsc@DeclareOldFontCommand{\mc}{\normalfont\mcfamily}{\mathmc}
\jsc@DeclareOldFontCommand{\gt}{\normalfont\gtfamily}{\mathgt}
\jsc@DeclareOldFontCommand{\rm}{\normalfont\rmfamily}{\mathrm}
\jsc@DeclareOldFontCommand{\sf}{\normalfont\sffamily}{\mathsf}
\jsc@DeclareOldFontCommand{\tt}{\normalfont\ttfamily}{\mathtt}
%    \end{macrocode}
% \end{macro}
% \end{macro}
% \end{macro}
% \end{macro}
% \end{macro}
%
% \begin{macro}{\bf}
%
% ボールドシリーズにします。通常のミーディアムシリーズに戻す
% コマンドは |\mdseries| です。
%
%    \begin{macrocode}
\jsc@DeclareOldFontCommand{\bf}{\normalfont\bfseries}{\mathbf}
%    \end{macrocode}
% \end{macro}
%
% \begin{macro}{\it}
% \begin{macro}{\sl}
% \begin{macro}{\sc}
%
% フォントシェイプを変えるコマンドです。
% 斜体とスモールキャップスは数式中では何もしません
% (警告メッセージを出力します)。
% 通常のアップライト体に戻すコマンドは |\upshape| です。
%
%    \begin{macrocode}
\jsc@DeclareOldFontCommand{\it}{\normalfont\itshape}{\mathit}
\jsc@DeclareOldFontCommand{\sl}{\normalfont\slshape}{\@nomath\sl}
\jsc@DeclareOldFontCommand{\sc}{\normalfont\scshape}{\@nomath\sc}
%    \end{macrocode}
% \end{macro}
% \end{macro}
% \end{macro}
%
% \begin{macro}{\cal}
% \begin{macro}{\mit}
%
% 数式モード以外では何もしません(警告を出します)。
%
%    \begin{macrocode}
\DeclareRobustCommand*{\cal}{\@fontswitch\relax\mathcal}
\DeclareRobustCommand*{\mit}{\@fontswitch\relax\mathnormal}
%    \end{macrocode}
% \end{macro}
% \end{macro}
%
% \section{相互参照}
%
% \subsection{目次の類}
%
% |\section| コマンドは |.toc| ファイルに次のような行を出力します。
% \begin{quote}
%   |\contentsline{section}{タイトル}{ページ}|
% \end{quote}
% たとえば |\section| に見出し番号が付く場合,上の「タイトル」は
% \begin{quote}
%   |\numberline{番号}{見出し}|
% \end{quote}
% となります。
% この「番号」は |\thesection| コマンドで生成された見出し番号です。
%
% |figure| 環境の |\caption| コマンドは |.lof| ファイルに
% 次のような行を出力します。
% \begin{quote}
% |\contentsline{figure}{\numberline{番号}{キャプション}{ページ}|
% \end{quote}
% この「番号」は |\thefigure| コマンドで生成された図番号です。
%
% |table| 環境も同様です。
%
% |\contentsline{...}| は |\l@...| というコマンドを実行するので,
% あらかじめ |\l@chapter|,|\l@section|,|\l@figure| などを
% 定義しておかなければなりません。
% これらの多くは |\@dottedtocline| コマンドを使って定義します。
% これは
% \begin{quote}
%   |\@dottedtocline{レベル}{インデント}{幅}{タイトル}{ページ}|
% \end{quote}
% という書式です。
% \begin{description}
% \item[レベル] この値が |tocdepth| 以下のときだけ出力されます。
%   |\chapter| はレベル0,|\section| はレベル1,等々です。
% \item[インデント] 左側の字下げ量です。
% \item[幅] 「タイトル」に |\numberline| コマンドが含まれる場合,
%   節番号が入る箱の幅です。
% \end{description}
%
% \begin{macro}{\@pnumwidth}
%
% ページ番号の入る箱の幅です。
%
% \begin{macro}{\@tocrmarg}
%
% 右マージンです。
% |\@tocrmarg| $\ge$ |\@pnumwidth| とします。
%
% \begin{macro}{\@dotsep}
%
% 点の間隔です(単位 mu)。
%
% \begin{macro}{\c@tocdepth}
%
% 目次ページに出力する見出しレベルです。
% 元は \texttt{article} で3,その他で2でしたが,
% ここでは一つずつ減らしています。
%
%    \begin{macrocode}
\newcommand\@pnumwidth{1.55em}
\newcommand\@tocrmarg{2.55em}
\newcommand\@dotsep{4.5}
%<!book&!report>\setcounter{tocdepth}{2}
%<book|report>\setcounter{tocdepth}{1}
%    \end{macrocode}
% \end{macro}
% \end{macro}
% \end{macro}
% \end{macro}
%
% \paragraph{目次}
%
% \begin{macro}{\tableofcontents}
%
% 目次を生成します。
%
% \begin{macro}{\jsc@tocl@width}
%
% [2013-12-30] |\prechaptername| などから見積もった目次のラベルの長さです。(by ts)
%
%    \begin{macrocode}
\newdimen\jsc@tocl@width
\newcommand{\tableofcontents}{%
%<*book|report>
  \settowidth\jsc@tocl@width{\headfont\prechaptername\postchaptername}%
  \settowidth\@tempdima{\headfont\appendixname}%
  \ifdim\jsc@tocl@width<\@tempdima \setlength\jsc@tocl@width{\@tempdima}\fi
  \ifdim\jsc@tocl@width<2\jsZw \divide\jsc@tocl@width by 2 \advance\jsc@tocl@width 1\jsZw\fi
  \if@twocolumn
    \@restonecoltrue\onecolumn
  \else
    \@restonecolfalse
  \fi
  \chapter*{\contentsname}%
  \@mkboth{\contentsname}{}%
%</book|report>
%<*!book&!report>
  \settowidth\jsc@tocl@width{\headfont\presectionname\postsectionname}%
  \settowidth\@tempdima{\headfont\appendixname}%
  \ifdim\jsc@tocl@width<\@tempdima\relax\setlength\jsc@tocl@width{\@tempdima}\fi
  \ifdim\jsc@tocl@width<2\jsZw \divide\jsc@tocl@width by 2 \advance\jsc@tocl@width 1\jsZw\fi
  \section*{\contentsname}%
  \@mkboth{\contentsname}{\contentsname}%
%</!book&!report>
  \@starttoc{toc}%
%<book|report>  \if@restonecol\twocolumn\fi
}
%    \end{macrocode}
% \end{macro}\end{macro}
%
% \begin{macro}{\l@part}
%
% 部の目次です。
%
%    \begin{macrocode}
\newcommand*{\l@part}[2]{%
  \ifnum \c@tocdepth >-2\relax
%<!book&!report>    \addpenalty\@secpenalty
%<book|report>    \addpenalty{-\@highpenalty}%
    \addvspace{2.25em \@plus\p@?}%
    \begingroup
      \parindent \z@
%     \@pnumwidth should be \@tocrmarg
%     \rightskip \@pnumwidth
      \rightskip \@tocrmarg
      \parfillskip -\rightskip
      {\leavevmode
        \large \headfont
        \setlength\@lnumwidth{4\jsZw}%
        #1\hfil \hb@xt@\@pnumwidth{\hss #2}}\par
      \nobreak
%<book|report>    \global\@nobreaktrue
%<book|report>    \everypar{\global\@nobreakfalse\everypar{}}%
    \endgroup
  \fi}
%    \end{macrocode}
% \end{macro}
%
% \begin{macro}{\l@chapter}
%
% 章の目次です。|\@lnumwidth| を4.683zwに増やしました。
%
% [2013-12-30] |\@lnumwidth| を |\jsc@tocl@width| から
% 決めるようにしてみました。(by ts)
%
%    \begin{macrocode}
%<*book|report>
\newcommand*{\l@chapter}[2]{%
  \ifnum \c@tocdepth >\m@ne
    \addpenalty{-\@highpenalty}%
    \addvspace{1.0em \@plus\p@?}
%   \vskip 1.0em \@plus\p@   % book.clsでは↑がこうなっている
    \begingroup
      \parindent\z@
%     \rightskip\@pnumwidth
      \rightskip\@tocrmarg
      \parfillskip-\rightskip
      \leavevmode\headfont
%     % \if@english\setlength\@lnumwidth{5.5em}\else\setlength\@lnumwidth{4.683\jsZw}\fi
      \setlength\@lnumwidth{\jsc@tocl@width}\advance\@lnumwidth 2.683\jsZw
      \advance\leftskip\@lnumwidth \hskip-\leftskip
      #1\nobreak\hfil\nobreak\hbox to\@pnumwidth{\hss#2}\par
      \penalty\@highpenalty
    \endgroup
  \fi}
%</book|report>
%    \end{macrocode}
% \end{macro}
%
% \begin{macro}{\l@section}
%
% 節の目次です。
%
%    \begin{macrocode}
%<*!book&!report>
\newcommand*{\l@section}[2]{%
  \ifnum \c@tocdepth >\z@
    \addpenalty{\@secpenalty}%
    \addvspace{1.0em \@plus\p@?}%
    \begingroup
      \parindent\z@
%     \rightskip\@pnumwidth
      \rightskip\@tocrmarg
      \parfillskip-\rightskip
      \leavevmode\headfont
%     % \setlength\@lnumwidth{4\jsZw}% 元1.5em [2003-03-02]
      \setlength\@lnumwidth{\jsc@tocl@width}\advance\@lnumwidth 2\jsZw
      \advance\leftskip\@lnumwidth \hskip-\leftskip
      #1\nobreak\hfil\nobreak\hbox to\@pnumwidth{\hss#2}\par
    \endgroup
  \fi}
%</!book&!report>
%    \end{macrocode}
%
% インデントと幅はそれぞれ1.5em,2.3emでしたが,
% 1zw,3.683zwに変えました。
%    \begin{macrocode}
%<book|report> % \newcommand*{\l@section}{\@dottedtocline{1}{1\jsZw}{3.683\jsZw}}
%    \end{macrocode}
%
% [2013-12-30] 上のインデントは |\jsc@tocl@width| から決めるように
% しました。(by ts)
%
% \end{macro}
%
% \begin{macro}{\l@subsection}
% \begin{macro}{\l@subsubsection}
% \begin{macro}{\l@paragraph}
% \begin{macro}{\l@subparagraph}
%
% さらに下位レベルの目次項目の体裁です。
% あまり使ったことがありませんので,要修正かもしれません。
%
% [2013-12-30] ここも  |\jsc@tocl@width| から決めるように
% してみました。(by ts)
%
%    \begin{macrocode}
%<*!book&!report>
% \newcommand*{\l@subsection}   {\@dottedtocline{2}{1.5em}{2.3em}}
% \newcommand*{\l@subsubsection}{\@dottedtocline{3}{3.8em}{3.2em}}
% \newcommand*{\l@paragraph}    {\@dottedtocline{4}{7.0em}{4.1em}}
% \newcommand*{\l@subparagraph} {\@dottedtocline{5}{10em}{5em}}
%
% \newcommand*{\l@subsubsection}{\@dottedtocline{3}{2\jsZw}{3\jsZw}}
% \newcommand*{\l@paragraph}    {\@dottedtocline{4}{3\jsZw}{3\jsZw}}
% \newcommand*{\l@subparagraph} {\@dottedtocline{5}{4\jsZw}{3\jsZw}}
%
\newcommand*{\l@subsection}{%
          \@tempdima\jsc@tocl@width \advance\@tempdima -1\jsZw
          \@dottedtocline{2}{\@tempdima}{3\jsZw}}
\newcommand*{\l@subsubsection}{%
          \@tempdima\jsc@tocl@width \advance\@tempdima 0\jsZw
          \@dottedtocline{3}{\@tempdima}{4\jsZw}}
\newcommand*{\l@paragraph}{%
          \@tempdima\jsc@tocl@width \advance\@tempdima 1\jsZw
          \@dottedtocline{4}{\@tempdima}{5\jsZw}}
\newcommand*{\l@subparagraph}{%
          \@tempdima\jsc@tocl@width \advance\@tempdima 2\jsZw
          \@dottedtocline{5}{\@tempdima}{6\jsZw}}
%</!book&!report>
%<*book|report>
% \newcommand*{\l@subsection}   {\@dottedtocline{2}{3.8em}{3.2em}}
% \newcommand*{\l@subsubsection}{\@dottedtocline{3}{7.0em}{4.1em}}
% \newcommand*{\l@paragraph}    {\@dottedtocline{4}{10em}{5em}}
% \newcommand*{\l@subparagraph} {\@dottedtocline{5}{12em}{6em}}
\newcommand*{\l@section}{%
          \@tempdima\jsc@tocl@width \advance\@tempdima -1\jsZw
          \@dottedtocline{1}{\@tempdima}{3.683\jsZw}}
\newcommand*{\l@subsection}{%
          \@tempdima\jsc@tocl@width \advance\@tempdima 2.683\jsZw
          \@dottedtocline{2}{\@tempdima}{3.5\jsZw}}
\newcommand*{\l@subsubsection}{%
          \@tempdima\jsc@tocl@width \advance\@tempdima 6.183\jsZw
          \@dottedtocline{3}{\@tempdima}{4.5\jsZw}}
\newcommand*{\l@paragraph}{%
          \@tempdima\jsc@tocl@width \advance\@tempdima 10.683\jsZw
          \@dottedtocline{4}{\@tempdima}{5.5\jsZw}}
\newcommand*{\l@subparagraph}{%
          \@tempdima\jsc@tocl@width \advance\@tempdima 16.183\jsZw
          \@dottedtocline{5}{\@tempdima}{6.5\jsZw}}
%</book|report>
%    \end{macrocode}
% \end{macro}
% \end{macro}
% \end{macro}
% \end{macro}
%
% \begin{macro}{\numberline}
% \begin{macro}{\@lnumwidth}
%
% 欧文版\LaTeX では |\numberline{...}| は幅 |\@tempdima| の箱に左詰め
% で出力する命令ですが,
% アスキー版では |\@tempdima| の代わりに |\@lnumwidth| という変数
% で幅を決めるように再定義しています。
% 後続文字が全角か半角かでスペースが変わらないように |\hspace|
% を入れておきました。
%
%    \begin{macrocode}
\newdimen\@lnumwidth
\def\numberline#1{\hb@xt@\@lnumwidth{#1\hfil}\hspace{0pt}}
%    \end{macrocode}
% \end{macro}
% \end{macro}
%
% \begin{macro}{\@dottedtocline}
%
% \LaTeX 本体(\texttt{ltsect.dtx} 参照)での定義と同じですが,
% |\@tempdima| を |\@lnumwidth| に変えています。
%
%    \begin{macrocode}
\def\@dottedtocline#1#2#3#4#5{\ifnum #1>\c@tocdepth \else
  \vskip \z@ \@plus.2\p@?
  {\leftskip #2\relax \rightskip \@tocrmarg \parfillskip -\rightskip
    \parindent #2\relax\@afterindenttrue
   \interlinepenalty\@M
   \leavevmode
   \@lnumwidth #3\relax
   \advance\leftskip \@lnumwidth \null\nobreak\hskip -\leftskip
    {#4}\nobreak
    \leaders\hbox{$\m@th \mkern \@dotsep mu\hbox{.}\mkern \@dotsep
       mu$}\hfill \nobreak\hb@xt@\@pnumwidth{%
         \hfil\normalfont \normalcolor #5}\par}\fi}
%    \end{macrocode}
% \end{macro}
%
% \paragraph{図目次と表目次}
%
% \begin{macro}{\listoffigures}
%
% 図目次を出力します。
%
%    \begin{macrocode}
\newcommand{\listoffigures}{%
%<*book|report>
  \if@twocolumn\@restonecoltrue\onecolumn
  \else\@restonecolfalse\fi
  \chapter*{\listfigurename}%
  \@mkboth{\listfigurename}{}%
%</book|report>
%<*!book&!report>
  \section*{\listfigurename}%
  \@mkboth{\listfigurename}{\listfigurename}%
%</!book&!report>
  \@starttoc{lof}%
%<book|report>  \if@restonecol\twocolumn\fi
}
%    \end{macrocode}
% \end{macro}
%
% \begin{macro}{\l@figure}
%
% 図目次の項目を出力します。
%
%    \begin{macrocode}
\newcommand*{\l@figure}{\@dottedtocline{1}{1\jsZw}{3.683\jsZw}}
%    \end{macrocode}
% \end{macro}
%
% \begin{macro}{\listoftables}
%
% 表目次を出力します。
%
%    \begin{macrocode}
\newcommand{\listoftables}{%
%<*book|report>
  \if@twocolumn\@restonecoltrue\onecolumn
  \else\@restonecolfalse\fi
  \chapter*{\listtablename}%
  \@mkboth{\listtablename}{}%
%</book|report>
%<*!book&!report>
  \section*{\listtablename}%
  \@mkboth{\listtablename}{\listtablename}%
%</!book&!report>
  \@starttoc{lot}%
%<book|report>  \if@restonecol\twocolumn\fi
}
%    \end{macrocode}
% \end{macro}
%
% \begin{macro}{\l@table}
%
% 表目次は図目次と同じです。
%
%    \begin{macrocode}
\let\l@table\l@figure
%    \end{macrocode}
% \end{macro}
%
% \subsection{参考文献}
%
% \begin{macro}{\bibindent}
%
% オープンスタイルの参考文献で使うインデント幅です。
% 元は 1.5em でした。
%
%    \begin{macrocode}
\newdimen\bibindent
\setlength\bibindent{2\jsZw}
%    \end{macrocode}
% \end{macro}
%
% \begin{environment}{thebibliography}
%
% 参考文献リストを出力します。
%
% [2016-07-16] \LaTeX~2.09で使われていたフォントコマンドの警告を,
% 文献スタイル(.bst)ではよく |\bf| がいまだに用いられることが多いため,
% |thebibliography| 環境内では例外的に出さないようにしました。
%
%    \begin{macrocode}
\newenvironment{thebibliography}[1]{%
  \@jsc@warnoldfontcmdexceptiontrue
  \global\let\presectionname\relax
  \global\let\postsectionname\relax
%<article|slide>  \section*{\refname}\@mkboth{\refname}{\refname}%
%<*kiyou>
  \vspace{1.5\baselineskip}
  \subsubsection*{\refname}\@mkboth{\refname}{\refname}%
  \vspace{0.5\baselineskip}
%</kiyou>
%<book|report>  \chapter*{\bibname}\@mkboth{\bibname}{}%
%<book|report>  \addcontentsline{toc}{chapter}{\bibname}%
   \list{\@biblabel{\@arabic\c@enumiv}}%
        {\settowidth\labelwidth{\@biblabel{#1}}%
         \leftmargin\labelwidth
         \advance\leftmargin\labelsep
         \@openbib@code
         \usecounter{enumiv}%
         \let\p@enumiv\@empty
         \renewcommand\theenumiv{\@arabic\c@enumiv}}%
%<kiyou>   \small
   \sloppy
   \clubpenalty4000
   \@clubpenalty\clubpenalty
   \widowpenalty4000%
   \sfcode`\.\@m}
  {\def\@noitemerr
    {\@latex@warning{Empty `thebibliography' environment}}%
   \endlist}
%    \end{macrocode}
% \end{environment}
%
% \begin{macro}{\newblock}
%
% |\newblock| はデフォルトでは小さなスペースを生成します。
%
%    \begin{macrocode}
\newcommand{\newblock}{\hskip .11em\@plus.33em\@minus.07em}
%    \end{macrocode}
% \end{macro}
%
% \begin{macro}{\@openbib@code}
%
% |\@openbib@code| はデフォルトでは何もしません。
% この定義は |openbib| オプションによって変更されます。
%
%    \begin{macrocode}
\let\@openbib@code\@empty
%    \end{macrocode}
% \end{macro}
%
% \begin{macro}{\@biblabel}
%
% |\bibitem[...]| のラベルを作ります。
% \texttt{ltbibl.dtx} の定義の半角 [] を全角[]に変え,
% 余分なスペースが入らないように |\jsInhibitGlue| ではさみました。
% とりあえずコメントアウトしておきますので,必要に応じて生かしてください。
%
%    \begin{macrocode}
% \def\@biblabel#1{\jsInhibitGlue[#1]\jsInhibitGlue}
%    \end{macrocode}
% \end{macro}
%
% \begin{macro}{\cite}
% \begin{macro}{\@cite}
% \begin{macro}{\@citex}
%
% 文献の番号を出力する部分は \texttt{ltbibl.dtx} で定義されていますが,
% コンマとかっこを和文フォントにするには次のようにします。
% とりあえずコメントアウトしておきましたので,必要に応じて生かしてください。
% かっこの前後に入るグルーを |\jsInhibitGlue| で取っていますので,
% オリジナル同様,\verb*+Knuth~\cite{knu} + のように半角空白
% で囲んでください。
%
%    \begin{macrocode}
% \def\@citex[#1]#2{\leavevmode
%   \let\@citea\@empty
%   \@cite{\@for\@citeb:=#2\do
%     {\@citea\def\@citea{,\inhibitglue\penalty\@m\ }%
%      \edef\@citeb{\expandafter\@firstofone\@citeb\@empty}%
%      \if@filesw\immediate\write\@auxout{\string\citation{\@citeb}}\fi
%      \@ifundefined{b@\@citeb}{\mbox{\normalfont\bfseries ?}%
%        \G@refundefinedtrue
%        \@latex@warning
%          {Citation `\@citeb' on page \thepage \space undefined}}%
%        {\@cite@ofmt{\csname b@\@citeb\endcsname}}}}{#1}}
% \def\@cite#1#2{\jsInhibitGlue[{#1\if@tempswa ,#2\fi}]\jsInhibitGlue}
%    \end{macrocode}
%
% 引用番号を上ツキの 1)のようなスタイルにするには次のようにします。
% |\cite| の先頭に |\unskip| を付けて先行のスペース(\verb|~| も)
% を帳消しにしています。
%
%    \begin{macrocode}
% \DeclareRobustCommand\cite{\unskip
%   \@ifnextchar [{\@tempswatrue\@citex}{\@tempswafalse\@citex[]}}
% \def\@cite#1#2{$^{\hbox{\scriptsize{#1\if@tempswa
%   ,\jsInhibitGlue\ #2\fi})}}$}
%    \end{macrocode}
% \end{macro}
% \end{macro}
% \end{macro}
%
% \subsection{索引}
%
% \begin{environment}{theindex}
%
% 2\zrWDash3段組の索引を作成します。
% 最後が偶数ページのときにマージンがずれる現象を直しました(Thanks: 藤村さん)。
%
%    \begin{macrocode}
\newenvironment{theindex}{% 索引を3段組で出力する環境
    \if@twocolumn
      \onecolumn\@restonecolfalse
    \else
      \clearpage\@restonecoltrue
    \fi
    \columnseprule.4pt \columnsep 2\jsZw
    \ifx\multicols\@undefined
%<book|report>      \twocolumn[\@makeschapterhead{\indexname}%
%<book|report>      \addcontentsline{toc}{chapter}{\indexname}]%
%<!book&!report>      \def\presectionname{}\def\postsectionname{}%
%<!book&!report>      \twocolumn[\section*{\indexname}]%
    \else
      \ifdim\textwidth<\fullwidth
        \setlength{\evensidemargin}{\oddsidemargin}
        \setlength{\textwidth}{\fullwidth}
        \setlength{\linewidth}{\fullwidth}
%<book|report>        \begin{multicols}{3}[\chapter*{\indexname}%
%<book|report>        \addcontentsline{toc}{chapter}{\indexname}]%
%<!book&!report>        \def\presectionname{}\def\postsectionname{}%
%<!book&!report>        \begin{multicols}{3}[\section*{\indexname}]%
      \else
%<book|report>        \begin{multicols}{2}[\chapter*{\indexname}%
%<book|report>        \addcontentsline{toc}{chapter}{\indexname}]%
%<!book&!report>        \def\presectionname{}\def\postsectionname{}%
%<!book&!report>        \begin{multicols}{2}[\section*{\indexname}]%
      \fi
    \fi
%<book|report>    \@mkboth{\indexname}{}%
%<!book&!report>    \@mkboth{\indexname}{\indexname}%
    \plainifnotempty % \thispagestyle{plain}
    \parindent\z@
    \parskip\z@ \@plus .3\p@?\relax
    \let\item\@idxitem
    \raggedright
    \footnotesize\narrowbaselines
  }{
    \ifx\multicols\@undefined
      \if@restonecol\onecolumn\fi
    \else
      \end{multicols}
    \fi
    \clearpage
  }
%    \end{macrocode}
% \end{environment}
%
% \begin{macro}{\@idxitem}
% \begin{macro}{\subitem}
% \begin{macro}{\subsubitem}
%
% 索引項目の字下げ幅です。|\@idxitem| は |\item| の項目の字下げ幅です。
%
%    \begin{macrocode}
\newcommand{\@idxitem}{\par\hangindent 4\jsZw} % 元 40pt
\newcommand{\subitem}{\@idxitem \hspace*{2\jsZw}} % 元 20pt
\newcommand{\subsubitem}{\@idxitem \hspace*{3\jsZw}} % 元 30pt
%    \end{macrocode}
% \end{macro}
% \end{macro}
% \end{macro}
%
% \begin{macro}{\indexspace}
%
% 索引で先頭文字ごとのブロックの間に入るスペースです。
%
%    \begin{macrocode}
\newcommand{\indexspace}{\par \vskip 10\p@? \@plus5\p@? \@minus3\p@?\relax}
%    \end{macrocode}
% \end{macro}
%
% \begin{macro}{\seename}
% \begin{macro}{\alsoname}
%
% 索引の |\see|,|\seealso| コマンドで出力されるものです。
% デフォルトはそれぞれ \emph{see},\emph{see also} という英語ですが,
% ここではとりあえず両方とも「→」に変えました。
% $\Rightarrow$(|$\Rightarrow$|)などでもいいでしょう。
%
%    \begin{macrocode}
\newcommand\seename{\if@english see\else →\fi}
\newcommand\alsoname{\if@english see also\else →\fi}
%    \end{macrocode}
% \end{macro}
% \end{macro}
%
% \subsection{脚注}
%
% \begin{macro}{\footnote}
% \begin{macro}{\footnotemark}
%
% 和文の句読点・閉じかっこ類の直後で用いた際に余分なアキが入るのを防ぐため,
% |\inhibitglue| を入れることにします。
% p\LaTeX の日付が2016/09/03より新しい場合は,このパッチが不要なのであてません。
%
% \begin{ZRnote}
% パッチの必要性は「|\pltx@foot@penalty| が未定義か」で行う。
% |\inhibitglue| の代わりに |\jsInhibitGlue| を使う。
% \end{ZRnote}
%    \begin{macrocode}
\ifx\pltx@foot@penalty\@undefined
  \let\footnotes@ve=\footnote
  \def\footnote{\jsInhibitGlue\footnotes@ve}
  \let\footnotemarks@ve=\footnotemark
  \def\footnotemark{\jsInhibitGlue\footnotemarks@ve}
\fi
%    \end{macrocode}
% \end{macro}
% \end{macro}
%
% \begin{macro}{\@makefnmark}
%
% 脚注番号を付ける命令です。
% ここでは脚注番号の前に記号 $*$ を付けています。
% 「注1」の形式にするには |\textasteriskcentered|
%  を |注\kern0.1em| にしてください。
% |\@xfootnotenext| と合わせて,
% もし脚注番号が空なら記号も出力しないようにしてあります。
%
% [2002-04-09] インプリメントの仕方を変えたため消しました。
%
% [2013-04-23] 新しい\pTeX では脚注番号のまわりにスペースが入りすぎることを防ぐため,
% 北川さんのパッチ [qa:57090] を取り込みました。
%
% [2013-05-14] plcore.ltx に倣った形に書き直しました(Thanks: 北川さん)。
%
% [2016-07-11] コミュニティ版p\LaTeX の変更に追随しました(Thanks: 角藤さん)。
% p\LaTeX の日付が2016/04/17より新しい場合は,このパッチが不要なのであてません。
%
% \begin{ZRnote}
% {\pTeX}依存のコードなので、minimal和文ドライバ実装に移動。
% \end{ZRnote}
% \end{macro}
%
% \begin{macro}{\thefootnote}
%
% 脚注番号に * 印が付くようにしました。
% ただし,番号がゼロのときは * 印も脚注番号も付きません。
%
% [2003-08-15] |\textasteriskcentered| ではフォントによって
% 下がりすぎるので変更しました。
%
% [2016-10-08] TODO: 脚注番号が |newtxtext| や |newpxtext| の使用時に
% おかしくなってしまいます。これらのパッケージは内部で |\thefootnote| を
% 再定義していますので,気になる場合はパッケージを読み込むときに
% \texttt{defaultsups} オプションを付けてください(qa:57284, qa:57287)。
%
%    \begin{macrocode}
\def\thefootnote{\ifnum\c@footnote>\z@\leavevmode\lower.5ex\hbox{*}\@arabic\c@footnote\fi}
%    \end{macrocode}
%
% 「注1」の形式にするには次のようにしてください。
%
%    \begin{macrocode}
% \def\thefootnote{\ifnum\c@footnote>\z@注\kern0.1\jsZw\@arabic\c@footnote\fi}
%    \end{macrocode}
%
% \end{macro}
%
% \begin{macro}{\footnoterule}
%
% 本文と脚注の間の罫線です。
%
%    \begin{macrocode}
\renewcommand{\footnoterule}{%
  \kern-2.6\p@? \kern-.4\p@
  \hrule width .4\columnwidth
  \kern 2.6\p@?}
%    \end{macrocode}
% \end{macro}
%
% \begin{macro}{\c@footnote}
%
% 脚注番号は章ごとにリセットされます。
%
%    \begin{macrocode}
%<book|report>\@addtoreset{footnote}{chapter}
%    \end{macrocode}
% \end{macro}
%
% \begin{macro}{\@footnotetext}
%
% 脚注で |\verb| が使えるように改変してあります。
% Jeremy Gibbons, \textit{\TeX\ and TUG NEWS},
%  Vol.~2, No.~4 (1993), p.~9)
%
% [2016-08-25] コミュニティ版\pLaTeX の「閉じ括弧類の直後に
% |\footnotetext| が続く場合に改行が起きることがある問題に対処」
% と同等のコードを追加しました。
%
% [2016-09-08] コミュニティ版\pLaTeX のバグ修正に追随しました。
%
% [2016-11-29] 古い\pLaTeX で使用された場合を考慮してコードを改良。
% ^^A 脚注直後に改行を可能にするために|\null|を入れる場合,
% ^^A 同時にペナルティも考慮しなければ誤った改行が起きる可能性がある。
% ^^A このため,|\ifhmode\null\fi|は
% ^^A   |\ifx\pltx@foot@penalty\@undefined\else ... \fi|
% ^^A 条件の内側に置いておくのが安全。
%
%    \begin{macrocode}
\long\def\@footnotetext{%
  \insert\footins\bgroup
    \normalfont\footnotesize
    \interlinepenalty\interfootnotelinepenalty
    \splittopskip\footnotesep
    \splitmaxdepth \dp\strutbox \floatingpenalty \@MM
    \hsize\columnwidth \@parboxrestore
    \protected@edef\@currentlabel{%
       \csname p@footnote\endcsname\@thefnmark
    }%
    \color@begingroup
      \@makefntext{%
        \rule\z@\footnotesep\ignorespaces}%
      \futurelet\next\fo@t}
\def\fo@t{\ifcat\bgroup\noexpand\next \let\next\f@@t
                                \else \let\next\f@t\fi \next}
\def\f@@t{\bgroup\aftergroup\@foot\let\next}
\def\f@t#1{#1\@foot}
\def\@foot{\@finalstrut\strutbox\color@endgroup\egroup
  \ifx\pltx@foot@penalty\@undefined\else
    \ifhmode\null\fi
    \ifnum\pltx@foot@penalty=\z@\else
      \penalty\pltx@foot@penalty
      \pltx@foot@penalty\z@
    \fi
  \fi}
%    \end{macrocode}
% \end{macro}
%
% \begin{macro}{\@makefntext}
%
% 実際に脚注を出力する命令です。
% |\@makefnmark| は脚注の番号を出力する命令です。
% ここでは脚注が左端から一定距離に来るようにしてあります。
%
%    \begin{macrocode}
\newcommand\@makefntext[1]{%
  \advance\leftskip 3\jsZw
  \parindent 1\jsZw
  \noindent
  \llap{\@makefnmark\hskip0.3\jsZw}#1}
%    \end{macrocode}
% \end{macro}
%
% \begin{macro}{\@xfootnotenext}
%
% 最初の |\footnotetext{...}| は番号が付きません。
% 著者の所属などを脚注の欄に書くときに便利です。
%
% すでに |\footnote| を使った後なら |\footnotetext[0]{...}|
% とすれば番号を付けない脚注になります。
% ただし,この場合は脚注番号がリセットされてしまうので,
% 工夫が必要です。
%
% [2002-04-09] インプリメントの仕方を変えたため消しました。
%
%    \begin{macrocode}
% \def\@xfootnotenext[#1]{%
%   \begingroup
%      \ifnum#1>\z@
%        \csname c@\@mpfn\endcsname #1\relax
%        \unrestored@protected@xdef\@thefnmark{\thempfn}%
%      \else
%        \unrestored@protected@xdef\@thefnmark{}%
%      \fi
%   \endgroup
%   \@footnotetext}
%    \end{macrocode}
% \end{macro}
%
% \begin{ZRnote}
% ここまでのコードは JS クラスを踏襲する。
% \end{ZRnote}
%
% \section{段落の頭へのグルー挿入禁止}
%
% 段落頭のかぎかっこなどを見かけ1字半下げから全角1字下げに直します。
%
% \begin{ZRnote}
% \begin{macro}{\jsInhibitGlueAtParTop}
% 「段落頭の括弧の空き補正」の処理を |\jsInhibitGlueAtParTop|
% という命令にして、これを再定義可能にした。
%    \begin{macrocode}
\let\jsInhibitGlueAtParTop\@empty
%    \end{macrocode}
% \end{macro}
%
% \begin{macro}{\everyparhook}
% 全ての段落の冒頭で実行されるフック。
% これの初期値を先述の |\jsInhibitGlueAtParTop| とする。
%    \begin{macrocode}
\def\everyparhook{\jsInhibitGlueAtParTop}
\AtBeginDocument{\everypar{\everyparhook}}
%    \end{macrocode}
% \end{macro}
% \end{ZRnote}
%
% [2016-07-18] |\inhibitglue| の発行対象を |\inhibitxspcode| が2に
% 設定されているものすべてに拡大しました。
%
% [2016-12-01] すぐ上の変更で |\@tempa| を使っていたのがよくなかった
% ので,プレフィックスを付けて |\jsc@tempa| にしました(forum:2085)。
%
% \begin{ZRnote}
% \begin{macro}{\@inhibitglue}
% JSクラスでの |\jsInhibitGlueAtParTop| の実装。
% (これは{(u)\pTeX}専用である。)
%
% \Note |\jsc@tempa| は実はテンポラリではなく「この処理専用の
% ユニーク制御綴」である(でないと不正である)が、間違って別の箇所で
% 使う危険性が高いので |\bxjs@ig@temp| に置き換えた。
%    \begin{macrocode}
\def\@inhibitglue{%
  \futurelet\@let@token\@@inhibitglue}
\begingroup
\let\GDEF=\gdef
\let\CATCODE=\catcode
\let\ENDGROUP=\endgroup
\CATCODE`k=12
\CATCODE`a=12
\CATCODE`n=12
\CATCODE`j=12
\CATCODE`i=12
\CATCODE`c=12
\CATCODE`h=12
\CATCODE`r=12
\CATCODE`t=12
\CATCODE`e=12
\GDEF\KANJI@CHARACTER{kanji character }
\ENDGROUP
\def\@@inhibitglue{%
  \expandafter\expandafter\expandafter\jsc@inhibitglue\expandafter\meaning\expandafter\@let@token\KANJI@CHARACTER\relax\jsc@end}
\expandafter\def\expandafter\jsc@inhibitglue\expandafter#\expandafter1\KANJI@CHARACTER#2#3\jsc@end{%
  \def\bxjs@ig@temp{#1}%
  \ifx\bxjs@ig@temp\@empty
    \ifnum\the\inhibitxspcode`#2=2\relax
      \inhibitglue
    \fi
  \fi}
%    \end{macrocode}
% \end{macro}
%
% \end{ZRnote}
%
% これだけではいけないようです。あちこちに |\everypar| を初期化するコマンドが
% 隠されていました。
%
% まず,環境の直後の段落です。
%
% [2016-11-19] ltlists.dtx 2015/05/10 v1.0tの変更に追随して |\clubpenalty| の
% リセットを追加しました。
%
%    \begin{macrocode}
\def\@doendpe{%
  \@endpetrue
  \def\par{%
    \@restorepar\clubpenalty\@clubpenalty\everypar{\everyparhook}\par\@endpefalse}%
  \everypar{{\setbox\z@\lastbox}\everypar{\everyparhook}\@endpefalse\everyparhook}}
%    \end{macrocode}
%
% |\item| 命令の直後です。
%
%    \begin{macrocode}
\def\@item[#1]{%
  \if@noparitem
    \@donoparitem
  \else
    \if@inlabel
      \indent \par
    \fi
    \ifhmode
      \unskip\unskip \par
    \fi
    \if@newlist
      \if@nobreak
        \@nbitem
      \else
        \addpenalty\@beginparpenalty
        \addvspace\@topsep
        \addvspace{-\parskip}%
      \fi
    \else
      \addpenalty\@itempenalty
      \addvspace\itemsep
    \fi
    \global\@inlabeltrue
  \fi
  \everypar{%
    \@minipagefalse
    \global\@newlistfalse
    \if@inlabel
      \global\@inlabelfalse
      {\setbox\z@\lastbox
       \ifvoid\z@
         \kern-\itemindent
       \fi}%
      \box\@labels
      \penalty\z@
    \fi
    \if@nobreak
      \@nobreakfalse
      \clubpenalty \@M
    \else
      \clubpenalty \@clubpenalty
      \everypar{\everyparhook}%
    \fi
    \bxjs@ltj@inhibitglue
    \everyparhook}%
  \if@noitemarg
    \@noitemargfalse
    \if@nmbrlist
      \refstepcounter\@listctr
    \fi
  \fi
  \sbox\@tempboxa{\makelabel{#1}}%
  \global\setbox\@labels\hbox{%
    \unhbox\@labels
    \hskip \itemindent
    \hskip -\labelwidth
    \hskip -\labelsep
    \ifdim \wd\@tempboxa >\labelwidth
      \box\@tempboxa
    \else
      \hbox to\labelwidth {\unhbox\@tempboxa}%
    \fi
    \hskip \labelsep}%
  \ignorespaces}
%    \end{macrocode}
%
% 二つ挿入した |\everyparhook| のうち後者が |\section| 類の直後に2回,
% 前者が3回目以降に実行されます。
%
%    \begin{macrocode}
\def\@afterheading{%
  \@nobreaktrue
  \everypar{%
    \if@nobreak
      \@nobreakfalse
      \clubpenalty \@M
      \if@afterindent \else
        {\setbox\z@\lastbox}%
      \fi
    \else
      \clubpenalty \@clubpenalty
      \everypar{\everyparhook}%
    \fi\everyparhook}}
%    \end{macrocode}
%
% |\@gnewline| についてはちょっと複雑な心境です。
% もともとのp\LaTeXe は段落の頭にグルーが入る方で統一されていました。
% しかし |\\| の直後にはグルーが入らず,不統一でした。
% そこで |\\| の直後にもグルーを入れるように直していただいた経緯があります。
% しかし,ここでは逆にグルーを入れない方で統一したいので,
% また元に戻してしまいました。
%
% しかし単に戻すだけでも駄目みたいなので,ここでも最後にグルーを消しておきます。
%
%    \begin{macrocode}
\def\@gnewline #1{%
  \ifvmode
    \@nolnerr
  \else
    \unskip \reserved@e {\reserved@f#1}\nobreak \hfil \break \null
    \jsInhibitGlue \ignorespaces
  \fi}
%    \end{macrocode}
%
% \section{いろいろなロゴ}
%
% \LaTeX 関連のロゴを作り直します。
%
% [2016-07-14] ロゴの定義は\texttt{jslogo}パッケージに移転しました。
% 後方互換のため,\texttt{jsclasses}ではデフォルトでこれを読み込みます。
% \texttt{nojslogo}オプションが指定されている場合は読み込みません。
%
% \begin{ZRnote}
% BXJSクラスでも |jslogo| オプション指定の場合に jslogo パッケージを
% 読み込むようにした。
% ただしJSクラスと異なり、既定では読み込まない。
% \Note |\小|、|\上小| の制御綴は定義しない。
% \end{ZRnote}
%    \begin{macrocode}
\if@jslogo
  \IfFileExists{jslogo.sty}{%
    \RequirePackage{jslogo}%
  }{%
    \ClassWarningNoLine\bxjs@clsname
     {The package 'jslogo' is not installed.\MessageBreak
      It is included in the recent release of\MessageBreak
      the 'jsclasses' bundle}
  }
\fi
%    \end{macrocode}
%
% \section{\texttt{amsmath} との衝突の回避}
%
% \begin{ZRnote}
% 最近の |\LaTeX| では該当の問題は対処されているので削除。
% \end{ZRnote}
%
% \section{初期設定}
%
% \paragraph{いろいろな語}
%
% \begin{macro}{\prepartname}
% \begin{macro}{\postpartname}
% \begin{macro}{\prechaptername}
% \begin{macro}{\postchaptername}
% \begin{macro}{\presectionname}
% \begin{macro}{\postsectionname}
%    \begin{macrocode}
\newcommand{\prepartname}{\if@english Part~\else 第\fi}
\newcommand{\postpartname}{\if@english\else 部\fi}
%<book|report>\newcommand{\prechaptername}{\if@english Chapter~\else 第\fi}
%<book|report>\newcommand{\postchaptername}{\if@english\else 章\fi}
\newcommand{\presectionname}{}%  第
\newcommand{\postsectionname}{}% 節
%    \end{macrocode}
% \end{macro}
% \end{macro}
% \end{macro}
% \end{macro}
% \end{macro}
% \end{macro}
%
% \begin{macro}{\contentsname}
% \begin{macro}{\listfigurename}
% \begin{macro}{\listtablename}
%    \begin{macrocode}
\newcommand{\contentsname}{\if@english Contents\else 目次\fi}
\newcommand{\listfigurename}{\if@english List of Figures\else 図目次\fi}
\newcommand{\listtablename}{\if@english List of Tables\else 表目次\fi}
%    \end{macrocode}
% \end{macro}
% \end{macro}
% \end{macro}
%
% \begin{macro}{\refname}
% \begin{macro}{\bibname}
% \begin{macro}{\indexname}
%    \begin{macrocode}
\newcommand{\refname}{\if@english References\else 参考文献\fi}
\newcommand{\bibname}{\if@english Bibliography\else 参考文献\fi}
\newcommand{\indexname}{\if@english Index\else 索引\fi}
%    \end{macrocode}
% \end{macro}
% \end{macro}
% \end{macro}
%
% \begin{macro}{\figurename}
% \begin{macro}{\tablename}
%    \begin{macrocode}
\newcommand{\figurename}{\if@english Fig.~\else 図\fi}
\newcommand{\tablename}{\if@english Table~\else 表\fi}
%    \end{macrocode}
% \end{macro}
% \end{macro}
%
% \begin{macro}{\appendixname}
% \begin{macro}{\abstractname}
%    \begin{macrocode}
% \newcommand{\appendixname}{\if@english Appendix~\else 付録\fi}
\newcommand{\appendixname}{\if@english \else 付録\fi}
%<!book&!report>\newcommand{\abstractname}{\if@english Abstract\else 概要\fi}
%    \end{macrocode}
% \end{macro}
% \end{macro}
%
% \paragraph{今日の日付}
%
% \LaTeX で処理した日付を出力します。
% |jarticle| などと違って,標準を西暦にし,余分な空白が入らないように改良しました。
% 和暦にするには |\和暦| と書いてください。
%
% \begin{ZRnote}
% 環境変数 |SOURCE_DATE_EPOCH_TEX_PRIMITIVES| が設定されている場合は
% “今日”が過去の日付になる可能性があるが、その場合、
% 和暦表記は平成2年(1990年)以降でのみサポートする。
% \Note “新元号”への対応?
% \end{ZRnote}
%
% \begin{macro}{\today}
%    \begin{macrocode}
\@tempswafalse
\if p\jsEngine \@tempswatrue \fi
\if n\jsEngine \@tempswatrue \fi
\if@tempswa \expandafter\@firstoftwo
\else       \expandafter\@secondoftwo
\fi
{%
% 欧文8bitTeXの場合
\newif\ifjsSeireki \jsSeirekitrue
\def\bxjs@decl@Seireki@cmds{%
  \def\西暦{\jsSeirekitrue}%
  \def\和暦{\jsSeirekifalse}}
\def\Seireki{\jsSeirekitrue}
\def\Wareki{\jsSeirekifalse}
\def\bxjs@if@use@seireki{%
  \ifjsSeireki \expandafter\@firstoftwo
  \else \expandafter\@secondoftwo \fi}
}{%
\newif\if西暦 \西暦true
\def\bxjs@decl@Seireki@cmds{%
  \def\西暦{\西暦true}%
  \def\和暦{\西暦false}}
\def\Seireki{\西暦true}
\def\Wareki{\西暦false}
\def\bxjs@if@use@seireki{%
  \if西暦 \expandafter\@firstoftwo
  \else \expandafter\@secondoftwo \fi}
}
\bxjs@decl@Seireki@cmds
% \bxjs@unxp
\let\bxjs@unxp\@firstofone
\bxjs@test@engine\unexpanded{\let\bxjs@unxp\unexpanded}
% \bxjs@iai
\if \if p\jsEngine T\else\if n\jsEngine T\else F\fi\fi T
  \def\bxjs@iai{\noexpand~}
\else \def\bxjs@iai{}
\fi
% \heisei
\newcount\heisei \heisei\year \advance\heisei-1988\relax
% \today
\edef\bxjs@today{%
  \if@english
    \ifcase\month\or
      January\or February\or March\or April\or May\or June\or
      July\or August\or September\or October\or November\or December\fi
      \space\number\day, \number\year
  \else
    \ifnum\heisei>\@ne
      \expandafter\noexpand\expandafter\bxjs@if@use@seireki
    \else \expandafter\@firstoftwo
    \fi {%
      \number\year\bxjs@iai\bxjs@unxp{年}%
      \bxjs@iai\number\month\bxjs@iai\bxjs@unxp{月}%
      \bxjs@iai\number\day\bxjs@iai\bxjs@unxp{日}%
    }{%
      \bxjs@unxp{平成}\bxjs@iai\number\heisei\bxjs@iai\bxjs@unxp{年}%
      \bxjs@iai\number\month\bxjs@iai\bxjs@unxp{月}%
      \bxjs@iai\number\day\bxjs@iai\bxjs@unxp{日}%
    }%
  \fi}
\let\today\bxjs@today
%    \end{macrocode}
% \end{macro}
%
% \begin{ZRnote}
% texjporg版の日本語用Babel定義ファイル(|japanese.ldf|)が読み込まれた
% 場合に影響を受けないようにする。
%    \begin{macrocode}
\AtBeginDocument{%
  \ifx\bbl@jpn@Seirekitrue\@undefined\else
    \bxjs@decl@Seireki@cmds
    \g@addto@macro\datejapanese{%
      \let\today\bxjs@today}%
  \fi}
%    \end{macrocode}
% \end{ZRnote}
%
% \paragraph{ハイフネーション例外}
%
% \TeX のハイフネーションルールの補足です(ペンディング:eng-lish)
%
%    \begin{macrocode}
\hyphenation{ado-be post-script ghost-script phe-nom-e-no-log-i-cal man-u-script}
%    \end{macrocode}
%
% \paragraph{ページ設定}
%
% ページ設定の初期化です。
%
%    \begin{macrocode}
%<slide>\pagestyle{empty}%
%<article|report>\pagestyle{plain}%
%<book>\pagestyle{headings}%
\pagenumbering{arabic}
\if@twocolumn
  \twocolumn
  \sloppy
  \flushbottom
\else
  \onecolumn
  \raggedbottom
\fi
%<*slide>
  \renewcommand\familydefault{\sfdefault}
  \raggedright
%</slide>
%    \end{macrocode}
%
% \paragraph{BXJS独自の追加処理 \ZRX}
% \mbox{}
% \begin{ZRnote}
%
% 和文ドライバのファイルを読み込む。
%    \begin{macrocode}
\catcode`\?=12
\ifx\bxjs@jadriver\relax\else
\input{bxjsja-\bxjs@jadriver.def}
\fi
%    \end{macrocode}
%
% 最後に日本語文字のカテゴリコードを元に戻す。
%    \begin{macrocode}
\bxjs@restore@jltrcc
%</cls>
%    \end{macrocode}
%
% \end{ZRnote}
%
% 以上です。
%
%^^A////////////////////////////////////////////////////////
% \clearpage
% \appendix
%^^A========================================================
% \section{和文ドライバの仕様 \ZRX}
%
% 次の命令がBXJSクラス本体と和文ドライバの連携のために
% 用意されている。
% このうち、★印を付けたものは“書込”が許されるものである。
%
% \begin{itemize}
% \item |\jsDocClass| \zrNote{文字トークンの let}\quad
%   文書クラスの種類を示し、次のいずれかと一致する
%   (|\if| で判定可能)。
%   \begin{quote}\begin{tabular}{l@{\qquad}l}
%   |\jsArticle|   & |bxjsarticle| クラス\\
%   |\jsBook|      & |bxjsbook| クラス\\
%   |\jsReport|    & |bxjsreport| クラス\\
%   |\jsSlide|     & |bxjsslide| クラス
%   \end{tabular}\end{quote}
% \item |\jsEngine| \zrNote{文字トークンの let}\quad
%   使用されているエンジンの種別。
%   (|\if| で判定可能)。
%   \begin{quote}\begin{tabular}{l@{\qquad}l}
%   |p|   & pdf{\TeX}(DVIモードも含む)\\
%   |l|   & Lua{\TeX}(〃)\\
%   |x|   & {\XeTeX}\\
%   |j|   & {\pTeX}または{\upTeX}\\
%   |n|   & 以上の何れでもない
%   \end{tabular}\end{quote}
% \item |\ifjsWithupTeX| \zrNote{スイッチ}\quad
%   使用されているエンジンが{\upTeX}であるか。
% \item |\ifjsWitheTeX| \zrNote{スイッチ}\quad
%   使用されているエンジンが{\eTeX}拡張であるか。
% \item |\ifjsInPdfMode| \zrNote{スイッチ}\quad
%   使用されているエンジンが(pdf{\TeX}・Lua{\TeX}の)
%   PDFモードであるか。
% \item |\jsUnusualPtSize| \zrNote{整数定数を表す文字列のマクロ}\quad
%   基底フォントサイズが |10pt|、|11pt|、|12pt| のいずれでもない
%   場合の |\@ptsize| の値。
%   (|\@ptsize| 自体があまり有用でないと思われる。)
% \item |\jsScale| \zrNote{実数を表す文字列のマクロ}\quad
%   和文フォントサイズの要求サイズに対するスケール。
%   クラスオプション |scale| で指定される。
%   (既定値は0.924715。)
% \item |\jsJaFont| \zrNote{マクロ}\quad
%   和文フォント設定を表す文字列。
%   クラスオプション |jafont| で指定された値。
% \item |\jsJaParam| \zrNote{マクロ}\quad
%   和文モジュールに渡すパラメタを表す文字列。
%   この値が何を表すかは決まってなくて、各々の和文モジュールが
%   独自に解釈する。
%   クラスオプション |japaram| で指定された値。
% \item |\jsInhibitGlue| \zrNote{マクロ}\quad
%   |\inhibitglue| という命令が定義されていればそれを実行し、
%   そうでなければ何もしない。
%   JSクラスで |\inhibitglue| を用いている箇所は
%   全て |\jsInhibitGlue| に置き換えられている。
%   従って、|\inhibitglue| は未定義でも動作するが、その実装が
%   ある場合はBXJSクラスはそれを活用する。
% \item |\jsInhibitGlueAtParTop| \zrNote{マクロ}★\quad
%   段落先頭におけるカギ括弧の位置調整を行うマクロ。
%   全ての段落先頭で呼び出される。
% \item |\jsZw| \zrNote{内部寸法値}\quad
%   「現在の全角幅」を表す変数。
%   JSクラスでzw単位で設定されている長さパラメタはこの変数を単位と
%   して設定されている。
%   この変数の値は実際に用いられる「和文フォント」のメトリックに
%   基づくのではなく、機械的に\
%   |\jsScale| $\times$(フォントサイズ)
%   であると定められている
%   (フォントサイズ変更の度に再設定される)。
%   従って、「和文コンポーネント」はこの設定と辻褄が合うように
%   和文フォントサイズを調整する必要がある。
%   ほとんどの場合、和文フォントをNFSSで規定する際に |\jsScale|
%   の値をスケール値として与えれば上手くいく。
% \item |\jsFontSizeChanged| \zrNote{マクロ}\quad
%   フォントサイズが変更された時に必ず呼び出されるマクロ。
% \item |\jsResetDimen| \zrNote{マクロ}★\quad
%   上記 |\jsFontSizeChanged| の中で呼び出される、
%   ユーザ(和文モジュール)用のフック。
%   フォントサイズに依存するパラメタをここで設定することができる。
%   既定の定義は空。
% \end{itemize}
%
% 以下で標準で用意されている和文ドライバの実装を示す。
%    \begin{macrocode}
%<*drv>
%    \end{macrocode}
%
%^^A========================================================
% \section{和文ドライバ:minimal \ZRX}
%
% |jadriver| の指定が無い場合に適用されるドライバ。
% また、standard ドライバはまずこのドライバファイルを
% 読み込んでいる。
%
% このドライバでは、各エンジンについての必要最低限の処理だけを
% 行っている。
% 日本語処理のためのパッケージ(xeCJK や Lua{\TeX}-ja 等)
% を自分で読み込んで適切な設定を行うという使用状況を想定している。
%
% ただし、(u){\pTeX}エンジンについては例外で、和文処理機構の
% 選択の余地がないため、このドライバにおいて、
% 「JSクラスと同等の指定」を完成させるためのコードを記述する。
%
%^^A----------------
% \subsection{補助マクロ}
%
%    \begin{macrocode}
%<*minimal>
%% このファイルは日本語文字を含みます
%    \end{macrocode}
%
% \begin{macro}{\DeclareJaTextFontCommand}
% 和文書体のための、「余計なこと」をしない |\DeclareTextFontCommand|。
%    \begin{macrocode}
\def\DeclareJaTextFontCommand#1#2{%
  \DeclareRobustCommand#1[1]{%
    \relax
    \ifmmode \expandafter\nfss@text \fi
    {#2##1}}%
}
%    \end{macrocode}
% \end{macro}
%
% \begin{macro}{\bxjs@if@sf@default}
% |\familydefault| の定義が“|\sfdefault|”である場合に
% 引数のコードを実行する。
%    \begin{macrocode}
\long\def\bxjs@@CSsfdefault{\sfdefault}%
\@onlypreamble\bxjs@if@sf@default
\def\bxjs@if@sf@default#1{%
  \ifx\familydefault\bxjs@@CSsfdefault#1\fi
  \AtBeginDocument{%
    \ifx\familydefault\bxjs@@CSsfdefault#1\fi}%
}
%    \end{macrocode}
% \end{macro}
%
% \begin{macro}{\jsLetHeadChar}
% |\jsLetHeadChar\CS{|\Meta{トークン列}|}|\Means
% トークン列の先頭の文字を抽出し、|\CS| をその文字トークン
% (に展開されるマクロ)として定義する。
% \Note 先頭にあるのが制御綴やグループである場合は |\CS| は |\relax|
% に等置される。
% \Note 文字トークンは“|\the|-文字列”のカテゴリコードをもつ。
% \Note 非Unicodeエンジンの場合は文字列がUTF-8で符号化されていると
% 見なし、先頭が高位バイトの場合は1文字分のバイト列(のトークン列)
% を抽出する。
% この場合は元のカテゴリコードが保持される。
%    \begin{macrocode}
\def\jsLetHeadChar#1#2{%
  \begingroup
    \escapechar=`\\ %
    \let\bxjs@tmpa={% brace-match-hack
    \bxjs@let@hchar@exp#2}%
  \endgroup
  \let#1\bxjs@g@tmpa}
\def\bxjs@let@hchar@exp{%
  \futurelet\@let@token\bxjs@let@hchar@exp@a}
\def\bxjs@let@hchar@exp@a{%
  \bxjs@cond\ifcat\noexpand\@let@token\bgroup\fi{% 波括弧
    \bxjs@let@hchar@out\let\relax
  }{\bxjs@cond\ifcat\noexpand\@let@token\@sptoken\fi{% 空白
    \bxjs@let@hchar@out\let\space%
  }{\bxjs@cond\if\noexpand\@let@token\@backslashchar\fi{% バックスラッシュ
    \bxjs@let@hchar@out\let\@backslashchar
  }{\bxjs@let@hchar@exp@b}}}}
\def\bxjs@let@hchar@exp@b#1{%
  \expandafter\bxjs@let@hchar@exp@c\string#1?\@nil#1}
\def\bxjs@let@hchar@exp@c#1#2\@nil{%
%\message{<#1#2>}%
  \bxjs@cond\if#1\@backslashchar\fi{% 制御綴
    \bxjs@cond\expandafter\ifx\noexpand\@let@token\@let@token\fi{%
      \bxjs@let@hchar@out\let\relax
    }{%else
      \expandafter\bxjs@let@hchar@exp
    }%
  }{%else
    \bxjs@let@hchar@chr#1%
  }}
\def\bxjs@let@hchar@chr#1{%
  \bxjs@let@hchar@out\def{{#1}}}
\def\bxjs@let@hchar@out#1#2{%
  \global#1\bxjs@g@tmpa#2\relax
  \toks@\bgroup}% skip to right brace
%    \end{macrocode}
% UTF-8のバイト列を扱うコード。
%    \begin{macrocode}
\chardef\bxjs@let@hchar@csta=128
\chardef\bxjs@let@hchar@cstb=192
\chardef\bxjs@let@hchar@cstc=224
\chardef\bxjs@let@hchar@cstd=240
\chardef\bxjs@let@hchar@cste=248
\let\bxjs@let@hchar@chr@ue@a\bxjs@let@hchar@chr
\def\bxjs@let@hchar@chr@ue#1{%
  \@tempcnta=`#1\relax
%\message{\the\@tempcnta}%
  \bxjs@cond\ifnum\@tempcnta<\bxjs@let@hchar@csta\fi{%
    \bxjs@let@hchar@chr@ue@a#1%
  }{\bxjs@cond\ifnum\@tempcnta<\bxjs@let@hchar@cstb\fi{%
    \bxjs@let@hchar@out\let\relax
  }{\bxjs@cond\ifnum\@tempcnta<\bxjs@let@hchar@cstc\fi{%
    \bxjs@let@hchar@chr@ue@b
  }{\bxjs@cond\ifnum\@tempcnta<\bxjs@let@hchar@cstd\fi{%
    \bxjs@let@hchar@chr@ue@c
  }{\bxjs@cond\ifnum\@tempcnta<\bxjs@let@hchar@cste\fi{%
    \bxjs@let@hchar@chr@ue@d
  }{%else
    \bxjs@let@hchar@out\let\relax
  }}}}}}
\def\bxjs@let@hchar@chr@ue@a#1{%
  \bxjs@let@hchar@out\def{{#1}}}
\def\bxjs@let@hchar@chr@ue@b#1#2{%
  \bxjs@let@hchar@out\def{{#1#2}}}
\def\bxjs@let@hchar@chr@ue@c#1#2#3{%
  \bxjs@let@hchar@out\def{{#1#2#3}}}
\def\bxjs@let@hchar@chr@ue@d#1#2#3#4{%
  \bxjs@let@hchar@out\def{{#1#2#3#4}}}
%    \end{macrocode}
% \end{macro}
%
%^^A----------------
% \subsection{(u){\pTeX}用の設定}
%
%    \begin{macrocode}
\ifx j\jsEngine
%    \end{macrocode}
%
% 基本的に、JSクラスのコードの中で、「和文コンポーネントの管轄」
% としてBXJSクラスで除外されている部分に相当するが、
% 若干の変更が加えられている。
%
% \paragraph{補助マクロ}
%
% |\jsLetHeadChar| を和文文字トークンに対応させる。
%    \begin{macrocode}
\def\bxjs@let@hchar@chr@pp#1{%
  \expandafter\bxjs@let@hchar@chr@pp@a\meaning#1\relax#1}
\def\bxjs@let@hchar@chr@pp@a#1#2\relax#3{%
%\message{(#1)}%
  \bxjs@cond\if#1t\fi{%
    \bxjs@let@hchar@chr@ue#3%
  }{%else
    \bxjs@let@hchar@out\def{{#3}}%
  }}
\let\bxjs@let@hchar@chr\bxjs@let@hchar@chr@pp
%    \end{macrocode}
%
% \paragraph{エンジン依存の定義}
%
% 最初にエンジン({\pTeX}かu{\pTeX}か)に依存する定義を行う。
% |\ifjsWithupTeX| はBXJSにおいて定義されているスイッチで、
% エンジンがu{\pTeX}であるかを表す。
%
% |\jsc@JYn| および |\jsc@JTn| は標準の和文横書きおよび縦書き用
% エンコーディングを表す。
%    \begin{macrocode}
\edef\jsc@JYn{\ifjsWithupTeX JY2\else JY1\fi}
\edef\jsc@JTn{\ifjsWithupTeX JT2\else JT1\fi}
\edef\jsc@pfx@{\ifjsWithupTeX u\fi}
%    \end{macrocode}
%
% |\bxjs@declarefontshape| は標準の和文フォント宣言である。
% 後で |\bxjs@scale| を求めるため一旦マクロにしておく。
% |\bxjs@sizereference| は全角幅を測定する時に参照するフォント。
%
% まずu{\pTeX}の場合の定義を示す。
% JSクラスの |uplatex| オプション指定時の定義と同じである。
%    \begin{macrocode}
\@onlypreamble\bxjs@declarefontshape
\ifjsWithupTeX
\def\bxjs@declarefontshape{%
\DeclareFontShape{JY2}{mc}{m}{n}{<->s*[\bxjs@scale]upjpnrm-h}{}%
\DeclareFontShape{JY2}{gt}{m}{n}{<->s*[\bxjs@scale]upjpngt-h}{}%
\DeclareFontShape{JT2}{mc}{m}{n}{<->s*[\bxjs@scale]upjpnrm-v}{}%
\DeclareFontShape{JT2}{gt}{m}{n}{<->s*[\bxjs@scale]upjpngt-v}{}%
}
\def\bxjs@sizereference{upjisr-h}
%    \end{macrocode}
%
% {\pTeX}の場合の定義を示す。
% JSクラスのフォント種別オプション非指定時の定義と同じである。
%    \begin{macrocode}
\else
\def\bxjs@declarefontshape{%
\DeclareFontShape{JY1}{mc}{m}{n}{<->s*[\bxjs@scale]jis}{}%
\DeclareFontShape{JY1}{gt}{m}{n}{<->s*[\bxjs@scale]jisg}{}%
\DeclareFontShape{JT1}{mc}{m}{n}{<->s*[\bxjs@scale]tmin10}{}%
\DeclareFontShape{JT1}{gt}{m}{n}{<->s*[\bxjs@scale]tgoth10}{}%
}
\def\bxjs@sizereference{jis}
\fi
%    \end{macrocode}
%
% 既に使用されている標準和文フォント定義がもしあれば取り消す。
%    \begin{macrocode}
\def\bxjs@tmpa#1/#2/#3/#4/#5\relax{%
  \def\bxjs@y{#5}}
\ifjsWithpTeXng \def\bxjs@y{10}%
\else
\expandafter\expandafter\expandafter\bxjs@tmpa
 \expandafter\string\the\jfont\relax
\fi
\@for\bxjs@x:={\jsc@JYn/mc/m/n,\jsc@JYn/gt/m/n,%
               \jsc@JTn/mc/m/n,\jsc@JTn/gt/m/n}\do
  {\expandafter\let\csname\bxjs@x/10\endcsname=\@undefined
   \expandafter\let\csname\bxjs@x/\bxjs@y\endcsname=\@undefined}
%    \end{macrocode}
%
% \paragraph{和文フォントスケールの補正}
%
% 実は、{\pTeX}の標準的な和文フォント(JFMのこと、例えば |jis|)
% では、指定された |\jsScale|(この値を $s$ とする)をそのまま
% 使って定義すると期待通りの大きさにならない。
% これらのJFMでは1\,zwの大きさが指定されたサイズではなく
% 既にスケール(この値を $f$ とする;|jis| では0.962216倍)
% が掛けられた値になっているからである。
% そのため、ここでは $s/f$ を求めてその値をマクロ |\bxjs@scale|
% に保存する。
%    \begin{macrocode}
\begingroup
% 参照用フォント(\bxjs@sizereference)の全角空白の幅を取得
  \font\bxjs@tmpa=\bxjs@sizereference\space at 10pt
  \setbox\z@\hbox{\bxjs@tmpa\char\jis"2121\relax}
% 幅が丁度10ptなら補正は不要
  \ifdim\wd\z@=10pt
    \global\let\bxjs@scale\jsScale
  \else
% (10*s)/(10*f)として計算、\bxjs@invscaleはBXJSで定義
    \edef\bxjs@tmpa{\strip@pt\wd\z@}
    \@tempdima=10pt \@tempdima=\jsScale\@tempdima
    \bxjs@invscale\@tempdima\bxjs@tmpa
    \xdef\bxjs@scale{\strip@pt\@tempdima}
  \fi
\endgroup
%\typeout{\string\bxjs@scale : \bxjs@scale}
%    \end{macrocode}
%
% \paragraph{和文フォント関連定義}
%
% |\bxjs@scale| が決まったので先に保存した標準和文フォント
% 宣言を実行する。
%    \begin{macrocode}
\bxjs@declarefontshape
%    \end{macrocode}
%
% フォント代替の明示的定義。
%    \begin{macrocode}
\DeclareFontShape{\jsc@JYn}{mc}{m}{it}{<->ssub*mc/m/n}{}
\DeclareFontShape{\jsc@JYn}{mc}{m}{sl}{<->ssub*mc/m/n}{}
\DeclareFontShape{\jsc@JYn}{mc}{m}{sc}{<->ssub*mc/m/n}{}
\DeclareFontShape{\jsc@JYn}{gt}{m}{it}{<->ssub*gt/m/n}{}
\DeclareFontShape{\jsc@JYn}{gt}{m}{sl}{<->ssub*gt/m/n}{}
\DeclareFontShape{\jsc@JYn}{mc}{bx}{it}{<->ssub*gt/m/n}{}
\DeclareFontShape{\jsc@JYn}{mc}{bx}{sl}{<->ssub*gt/m/n}{}
\DeclareFontShape{\jsc@JTn}{mc}{m}{it}{<->ssub*mc/m/n}{}
\DeclareFontShape{\jsc@JTn}{mc}{m}{sl}{<->ssub*mc/m/n}{}
\DeclareFontShape{\jsc@JTn}{mc}{m}{sc}{<->ssub*mc/m/n}{}
\DeclareFontShape{\jsc@JTn}{gt}{m}{it}{<->ssub*gt/m/n}{}
\DeclareFontShape{\jsc@JTn}{gt}{m}{sl}{<->ssub*gt/m/n}{}
\DeclareFontShape{\jsc@JTn}{mc}{bx}{it}{<->ssub*gt/m/n}{}
\DeclareFontShape{\jsc@JTn}{mc}{bx}{sl}{<->ssub*gt/m/n}{}
%    \end{macrocode}
%
% 欧文総称フォント命令で和文フォントが連動するように修正する。
% その他の和文フォント関係の定義を行う。
%    \begin{macrocode}
\DeclareRobustCommand\rmfamily
  {\not@math@alphabet\rmfamily\mathrm
   \romanfamily\rmdefault\kanjifamily\mcdefault\selectfont}
\DeclareRobustCommand\sffamily
  {\not@math@alphabet\sffamily\mathsf
   \romanfamily\sfdefault\kanjifamily\gtdefault\selectfont}
\DeclareRobustCommand\ttfamily
  {\not@math@alphabet\ttfamily\mathtt
   \romanfamily\ttdefault\kanjifamily\gtdefault\selectfont}
\DeclareJaTextFontCommand{\textmc}{\mcfamily}
\DeclareJaTextFontCommand{\textgt}{\gtfamily}
\bxjs@if@sf@default{%
  \renewcommand\kanjifamilydefault{\gtdefault}}
%    \end{macrocode}
%
% 念のため。
%    \begin{macrocode}
\selectfont
%    \end{macrocode}
%
% \paragraph{パラメタの設定}
%
%    \begin{macrocode}
\prebreakpenalty\jis"2147=10000
\postbreakpenalty\jis"2148=10000
\prebreakpenalty\jis"2149=10000
\inhibitxspcode`!=1
\inhibitxspcode`〒=2
\xspcode`+=3
\xspcode`\%=3
%    \end{macrocode}
%
% |"80|\zrWDash|"FF| の範囲の |\spcode| を3に変更。
%    \begin{macrocode}
\@tempcnta="80 \@whilenum\@tempcnta<"100 \do{%
  \xspcode\@tempcnta=3\advance\@tempcnta\@ne}
%    \end{macrocode}
%
% |\jsInhibitGlueAtParTop| の定義。
% 「JSクラスでの定義」を利用する。
%    \begin{macrocode}
\let\jsInhibitGlueAtParTop\@inhibitglue
%    \end{macrocode}
%
% |\jsResetDimen| は空のままでよい。
%
% \paragraph{組方向依存の処理}
%
% 組方向判定のif-トークン(|\if?dir|)は{\pTeX}以外では未定義で
% あるため、そのままif文に入れることができない。
% これを回避するため部分的に|!|をエスケープ文字に使う。
%    \begin{macrocode}
\begingroup
\catcode`\!=0
%    \end{macrocode}
%
% \begin{macro}{\bxjs@ptex@dir}
% 現在の組方向: |t|=縦、|y|=横、|?|=その他。
%    \begin{macrocode}
\gdef\bxjs@ptex@dir{%
  !iftdir t%
  !else!ifydir y%
  !else ?%
  !fi!fi}
%    \end{macrocode}
% \end{macro}
%
% 新版の{\pTeX}で脚注番号の周囲の空きが過大になる現象への対処。
% \Note 現在の{p\LaTeX}カーネルでは対処が既に行われている。
% ここでは、|\@makefnmark| の定義が古いものであった場合に、
% 新しいものに置き換える。
%    \begin{macrocode}
% 古い \@makefnmark の定義
\long\def\bxjs@tmpa{\hbox{%
  !ifydir \@textsuperscript{\normalfont\@thefnmark}%
  !else\hbox{\yoko\@textsuperscript{\normalfont\@thefnmark}}!fi}}
\ifx\@makefnmark\bxjs@tmpa
\long\gdef\@makefnmark{%
  !ifydir \hbox{}\hbox{\@textsuperscript{\normalfont\@thefnmark}}\hbox{}%
  !else\hbox{\yoko\@textsuperscript{\normalfont\@thefnmark}}!fi}
\fi
%    \end{macrocode}
%
%    \begin{macrocode}
\endgroup
%    \end{macrocode}
%
%^^A----------------
% \subsection{pdf{\TeX}用の処理}
%
%    \begin{macrocode}
\else\ifx p\jsEngine
%    \end{macrocode}
%
%    \begin{macrocode}
\let\bxjs@let@hchar@chr\bxjs@let@hchar@chr@ue
%    \end{macrocode}
%
%    \begin{macrocode}
\@onlypreamble\bxjs@cjk@loaded
\def\bxjs@cjk@loaded{%
  \def\@footnotemark{%
    \leavevmode
    \ifhmode
      \edef\@x@sf{\the\spacefactor}%
      \ifdim\lastkern>\z@\ifdim\lastkern<5sp\relax
         \unkern\unkern
         \ifdim\lastskip>\z@ \unskip \fi
      \fi\fi
      \nobreak
    \fi
    \@makefnmark
    \ifhmode \spacefactor\@x@sf \fi
    \relax}%
  \let\bxjs@cjk@loaded\relax
}
\AtBeginDocument{%
  \@ifpackageloaded{CJK}{%
    \bxjs@cjk@loaded
  }{}%
}
%    \end{macrocode}
%
%^^A----------------
% \subsection{{\XeTeX}用の処理}
%
%    \begin{macrocode}
\else\ifx x\jsEngine
%    \end{macrocode}
%
% |\bxjs@let@hchar@chr| について、
% 「BMP外の文字の文字トークンに対して |\string| を適用すると
% サロゲートペアに分解される」という問題に対する応急措置を施す。
%    \begin{macrocode}
\def\bxjs@let@hchar@chr#1{%
  \@tempcnta`#1\relax \divide\@tempcnta"800\relax
  \bxjs@cond\ifnum\@tempcnta=27 \fi{%
    \bxjs@let@hchar@chr@xe
  }{\bxjs@let@hchar@out\def{{#1}}}}
\def\bxjs@let@hchar@chr@xe#1{%
  \lccode`0=`#1\relax
  \lowercase{\bxjs@let@hchar@out\def{{0}}}}
%    \end{macrocode}
%
% \begin{macro}{\bxjs@do@precisetext}
% |precisetext| オプションの処理。
%    \begin{macrocode}
\ifx\XeTeXgenerateactualtext\@undefined\else
  \def\bxjs@do@precisetext{%
    \XeTeXgenerateactualtext=\@ne}
\fi
%    \end{macrocode}
% \end{macro}
%
% \begin{macro}{\bxjs@do@simplejasetup}
% |simplejasetup| オプションの処理。
%    \begin{macrocode}
\@onlypreamble\bxjs@do@simplejasetup
\def\bxjs@do@simplejasetup{%
  \ifnum\XeTeXinterchartokenstate>\z@
  \else\ifnum\strcmp{\the\XeTeXlinebreakskip}{\the\z@}=\z@
    \jsSimpleJaSetup
    \ClassInfo\bxjs@clsname
     {'\string\jsSimpleJaSetup' is applied\@gobble}%
  \fi\fi}
%    \end{macrocode}
% \end{macro}
% \begin{macro}{\jsSimpleJaSetup}
% 日本語出力用の超簡易的な設定。
%    \begin{macrocode}
\newcommand*{\jsSimpleJaSetup}{%
  \XeTeXlinebreaklocale "ja"\relax
  \XeTeXlinebreakskip=0pt plus 1pt minus 0.1pt\relax
  \XeTeXlinebreakpenalty=0\relax}
%    \end{macrocode}
% \end{macro}
%
%^^A----------------
% \subsection{後処理(エンジン共通)}
%    \begin{macrocode}
\fi\fi\fi
%    \end{macrocode}
%
% |simplejasetup| オプションの処理。
%    \begin{macrocode}
\ifx\bxjs@do@simplejasetup\@undefined\else
  \AtBeginDocument{%
    \ifbxjs@simplejasetup
      \bxjs@do@simplejasetup
    \fi}
\fi
%    \end{macrocode}
%
% |precisetext| オプションの処理。
%    \begin{macrocode}
\ifbxjs@precisetext
  \ifx\bxjs@do@precisetext\@undefined
    \ClassWarning\bxjs@clsname
     {The current engine does not supprt the\MessageBreak
      'precisetext' option\@gobble}
  \else
    \bxjs@do@precisetext
  \fi
\fi
%    \end{macrocode}
%
% \paragraph{fancyhdr対策}
% |fancyhdr| オプションの値が |true| であり、
% かつ |fancyhdr| が使用された場合に以下の対策を行う。
% \begin{itemize}
% \item デフォルトの書式設定に含まれる“二文字フォント命令”を除去する。
% \item bxjsbookにおいて、ヘッダ・フッタの横幅を |\fullwidth| に変える。
% \end{itemize}
%
%    \begin{macrocode}
\ifbxjs@fancyhdr
%    \end{macrocode}
%
% \begin{macro}{\bxjs@adjust@fancyhdr}
% |fancyhdr| の初期設定に関する改変の処理。
% |fancyhdr| 読込完了と |\pagestyle{fancy}| 実行の間で実行されるべき。
%    \begin{macrocode}
\@onlypreamble\bxjs@adjust@fancyhdr
\def\bxjs@adjust@fancyhdr{%
%    \end{macrocode}
% ヘッダ・フッタの要素の書式について、それが既定のままであれば、
% “二文字フォント命令”を除去したものに置き換える。
% \Note 和文なので |\sl| は無い方がよいはず。
%    \begin{macrocode}
  \def\bxjs@tmpa{\fancyplain{}{\sl\rightmark}\strut}%
  \def\bxjs@tmpb{\fancyplain{}{\rightmark}\strut}%
  \ifx\f@ncyelh\bxjs@tmpa \global\let\f@ncyelh\bxjs@tmpb \fi
  \ifx\f@ncyerh\bxjs@tmpa \global\let\f@ncyerh\bxjs@tmpb \fi
  \ifx\f@ncyolh\bxjs@tmpa \global\let\f@ncyolh\bxjs@tmpb \fi
  \ifx\f@ncyorh\bxjs@tmpa \global\let\f@ncyorh\bxjs@tmpb \fi
  \def\bxjs@tmpa{\fancyplain{}{\sl\leftmark}\strut}%
  \def\bxjs@tmpb{\fancyplain{}{\leftmark}\strut}%
  \ifx\f@ncyelh\bxjs@tmpa \global\let\f@ncyelh\bxjs@tmpb \fi
  \ifx\f@ncyerh\bxjs@tmpa \global\let\f@ncyerh\bxjs@tmpb \fi
  \ifx\f@ncyolh\bxjs@tmpa \global\let\f@ncyolh\bxjs@tmpb \fi
  \ifx\f@ncyorh\bxjs@tmpa \global\let\f@ncyorh\bxjs@tmpb \fi
  \def\bxjs@tmpa{\rm\thepage\strut}%
  \def\bxjs@tmpb{\thepage\strut}%
  \ifx\f@ncyecf\bxjs@tmpa \global\let\f@ncyecf\bxjs@tmpb \fi
  \ifx\f@ncyocf\bxjs@tmpa \global\let\f@ncyocf\bxjs@tmpb \fi
%    \end{macrocode}
% |\fullwidth| が(定義済で)|\textwidth| よりも大きい場合、
% ヘッダ・フッタの横幅を |\fullwidth| に合わせる。
%    \begin{macrocode}
  \ifx\fullwidth\@undefined\else \ifdim\textwidth<\fullwidth
    \setlength{\@tempdima}{\fullwidth-\textwidth}%
    \edef\bxjs@tmpa{\noexpand\fancyhfoffset[EL,OR]{\the\@tempdima}%
    }\bxjs@tmpa
  \fi\fi
  \PackageInfo\bxjs@clsname
   {Patch to fancyhdr is applied\@gobble}}
%    \end{macrocode}
% \end{macro}
%
% \begin{macro}{\bxjs@pagestyle@hook}
% |\pagestyle| へのフックの本体。
%    \begin{macrocode}
\def\bxjs@pagestyle@hook{%
  \@ifpackageloaded{fancyhdr}{%
    \bxjs@adjust@fancyhdr
    \global\let\bxjs@adjust@fancyhdr\relax
  }{}}
%    \end{macrocode}
% \end{macro}
%
% |\pagestyle| にフックを入れ込む。
%    \begin{macrocode}
\let\bxjs@org@pagestyle\pagestyle
\def\pagestyle{%
  \bxjs@pagestyle@hook \bxjs@org@pagestyle}
%    \end{macrocode}
%
% begin-document フック。
% \Note これ以降に |fancyhdr| が読み込まれることはあり得ない。
%    \begin{macrocode}
\AtBeginDocument{%
  \bxjs@pagestyle@hook
  \global\let\bxjs@pagestyle@hook\relax}
%    \end{macrocode}
%
% 終わり。
%    \begin{macrocode}
\fi
%    \end{macrocode}
%
% 以上で終わり。
%    \begin{macrocode}
%</minimal>
%    \end{macrocode}
%
%^^A========================================================
% \section{和文ドライバ:standard \ZRX}
%
% 標準のドライバ。
%
% \begin{itemize}
% \item |\rmfamily|/|\sffamily|/|\ttfamily| での
%   和文ファミリ連動
% \item |\mcfamily|/|\gtfamily|
% \item |\textmc|/|\textgt|
% \item |\zw|
% \item |\jQ|/|\jH|
% \item |\trueQ|/|\trueH|/|\ascQ|
% \item |\setkanjiskip|/|\getkanjiskip|
% \item |\setxkanjiskip|/|\getxkanjiskip|
% \item |\autospacing|/|\noautospacing|
% \item |\autoxspacing|/|\noautoxspacing|
% \end{itemize}
%
% \paragraph{和文フォント指定の扱い}
%
% |standard| 和文ドライバでは |\jsJaFont| の値を和文フォントの
% “プリセット”の指定として用いる。
% プリセットの値は、{\TeX} Liveの |kanji-config-updmap| コマンドで
% 使う“ファミリ”と同じにすることを想定する。
% 特別な値として、|auto| は |kanji-config-updmap| で現在指定
% されているファミリを表す。
%
%^^A----------------
% \subsection{共通処理(1)}
%
% まず minimal ドライバを読み込む。
%    \begin{macrocode}
%<*standard>
%% このファイルは日本語文字を含みます
%%
%% This is file `bxjsja-minimal.def',
%% generated with the docstrip utility.
%%
%% The original source files were:
%%
%% bxjscls.dtx  (with options: `drv,minimal')
%% 
%% IMPORTANT NOTICE:
%% 
%% For the copyright see the source file.
%% 
%% Any modified versions of this file must be renamed
%% with new filenames distinct from bxjsja-minimal.def.
%% 
%% For distribution of the original source see the terms
%% for copying and modification in the file bxjscls.dtx.
%% 
%% This generated file may be distributed as long as the
%% original source files, as listed above, are part of the
%% same distribution. (The sources need not necessarily be
%% in the same archive or directory.)
%% \CharacterTable
%%  {Upper-case    \A\B\C\D\E\F\G\H\I\J\K\L\M\N\O\P\Q\R\S\T\U\V\W\X\Y\Z
%%   Lower-case    \a\b\c\d\e\f\g\h\i\j\k\l\m\n\o\p\q\r\s\t\u\v\w\x\y\z
%%   Digits        \0\1\2\3\4\5\6\7\8\9
%%   Exclamation   \!     Double quote  \"     Hash (number) \#
%%   Dollar        \$     Percent       \%     Ampersand     \&
%%   Acute accent  \'     Left paren    \(     Right paren   \)
%%   Asterisk      \*     Plus          \+     Comma         \,
%%   Minus         \-     Point         \.     Solidus       \/
%%   Colon         \:     Semicolon     \;     Less than     \<
%%   Equals        \=     Greater than  \>     Question mark \?
%%   Commercial at \@     Left bracket  \[     Backslash     \\
%%   Right bracket \]     Circumflex    \^     Underscore    \_
%%   Grave accent  \`     Left brace    \{     Vertical bar  \|
%%   Right brace   \}     Tilde         \~}
\ProvidesFile{bxjsja-minimal.def}
  [2016/08/31 v1.3-pre BXJS document classes]
%% このファイルは日本語文字を含みます
\def\DeclareJaTextFontCommand#1#2{%
  \DeclareRobustCommand#1[1]{%
    \relax
    \ifmmode \expandafter\nfss@text \fi
    {#2##1}}%
}
\long\def\bxjs@@CSsfdefault{\sfdefault}%
\@onlypreamble\bxjs@if@sf@default
\def\bxjs@if@sf@default#1{%
  \ifx\familydefault\bxjs@@CSsfdefault#1\fi
  \AtBeginDocument{%
    \ifx\familydefault\bxjs@@CSsfdefault#1\fi}%
}
\def\jsLetHeadChar#1#2{%
  \begingroup
    \escapechar=`\\ %
    \let\bxjs@tmpa={% brace-match-hack
    \bxjs@let@hchar@exp#2}%
  \endgroup
  \let#1\bxjs@g@tmpa}
\def\bxjs@let@hchar@exp{%
  \futurelet\@let@token\bxjs@let@hchar@exp@a}
\def\bxjs@let@hchar@exp@a{%
  \bxjs@cond\ifcat\noexpand\@let@token\bgroup\fi{% 波括弧
    \bxjs@let@hchar@out\let\relax
  }{\bxjs@cond\ifcat\noexpand\@let@token\@sptoken\fi{% 空白
    \bxjs@let@hchar@out\let\space%
  }{\bxjs@cond\if\noexpand\@let@token\@backslashchar\fi{% バックスラッシュ
    \bxjs@let@hchar@out\let\@backslashchar
  }{\bxjs@let@hchar@exp@b}}}}
\def\bxjs@let@hchar@exp@b#1{%
  \expandafter\bxjs@let@hchar@exp@c\string#1?\@nil#1}
\def\bxjs@let@hchar@exp@c#1#2\@nil{%
  \bxjs@cond\if#1\@backslashchar\fi{% 制御綴
    \bxjs@cond\expandafter\ifx\noexpand\@let@token\@let@token\fi{%
      \bxjs@let@hchar@out\let\relax
    }{%else
      \expandafter\bxjs@let@hchar@exp
    }%
  }{%else
    \bxjs@let@hchar@chr#1%
  }}
\def\bxjs@let@hchar@chr#1{%
  \bxjs@let@hchar@out\def{{#1}}}
\def\bxjs@let@hchar@out#1#2{%
  \global#1\bxjs@g@tmpa#2\relax
  \toks@\bgroup}% skip to right brace
\chardef\bxjs@let@hchar@csta=128
\chardef\bxjs@let@hchar@cstb=192
\chardef\bxjs@let@hchar@cstc=224
\chardef\bxjs@let@hchar@cstd=240
\chardef\bxjs@let@hchar@cste=248
\let\bxjs@let@hchar@chr@ue@a\bxjs@let@hchar@chr
\def\bxjs@let@hchar@chr@ue#1{%
  \@tempcnta=`#1\relax
  \bxjs@cond\ifnum\@tempcnta<\bxjs@let@hchar@csta\fi{%
    \bxjs@let@hchar@chr@ue@a#1%
  }{\bxjs@cond\ifnum\@tempcnta<\bxjs@let@hchar@cstb\fi{%
    \bxjs@let@hchar@out\let\relax
  }{\bxjs@cond\ifnum\@tempcnta<\bxjs@let@hchar@cstc\fi{%
    \bxjs@let@hchar@chr@ue@b
  }{\bxjs@cond\ifnum\@tempcnta<\bxjs@let@hchar@cstd\fi{%
    \bxjs@let@hchar@chr@ue@c
  }{\bxjs@cond\ifnum\@tempcnta<\bxjs@let@hchar@cste\fi{%
    \bxjs@let@hchar@chr@ue@d
  }{%else
    \bxjs@let@hchar@out\let\relax
  }}}}}}
\def\bxjs@let@hchar@chr@ue@a#1{%
  \bxjs@let@hchar@out\def{{#1}}}
\def\bxjs@let@hchar@chr@ue@b#1#2{%
  \bxjs@let@hchar@out\def{{#1#2}}}
\def\bxjs@let@hchar@chr@ue@c#1#2#3{%
  \bxjs@let@hchar@out\def{{#1#2#3}}}
\def\bxjs@let@hchar@chr@ue@d#1#2#3#4{%
  \bxjs@let@hchar@out\def{{#1#2#3#4}}}
\ifx j\jsEngine
\def\bxjs@let@hchar@chr@pp#1{%
  \expandafter\bxjs@let@hchar@chr@pp@a\meaning#1\relax#1}
\def\bxjs@let@hchar@chr@pp@a#1#2\relax#3{%
  \bxjs@cond\if#1t\fi{%
    \bxjs@let@hchar@chr@ue#3%
  }{%else
    \bxjs@let@hchar@out\def{{#3}}%
  }}
\let\bxjs@let@hchar@chr\bxjs@let@hchar@chr@pp
\edef\jsc@JYn{\ifjsWithupTeX JY2\else JY1\fi}
\edef\jsc@JTn{\ifjsWithupTeX JT2\else JT1\fi}
\edef\jsc@pfx@{\ifjsWithupTeX u\fi}
\@onlypreamble\bxjs@declarefontshape
\ifjsWithupTeX
\def\bxjs@declarefontshape{%
\DeclareFontShape{JY2}{mc}{m}{n}{<->s*[\bxjs@scale]upjpnrm-h}{}%
\DeclareFontShape{JY2}{gt}{m}{n}{<->s*[\bxjs@scale]upjpngt-h}{}%
\DeclareFontShape{JT2}{mc}{m}{n}{<->s*[\bxjs@scale]upjpnrm-v}{}%
\DeclareFontShape{JT2}{gt}{m}{n}{<->s*[\bxjs@scale]upjpngt-v}{}%
}
\def\bxjs@sizereference{upjisr-h}
\else
\def\bxjs@declarefontshape{%
\DeclareFontShape{JY1}{mc}{m}{n}{<->s*[\bxjs@scale]jis}{}%
\DeclareFontShape{JY1}{gt}{m}{n}{<->s*[\bxjs@scale]jisg}{}%
\DeclareFontShape{JT1}{mc}{m}{n}{<->s*[\bxjs@scale]tmin10}{}%
\DeclareFontShape{JT1}{gt}{m}{n}{<->s*[\bxjs@scale]tgoth10}{}%
}
\def\bxjs@sizereference{jis}
\fi
\def\bxjs@tmpa#1/#2/#3/#4/#5\relax{%
  \def\bxjs@y{#5}}
\ifjsWithpTeXng \def\bxjs@y{10}%
\else
\expandafter\expandafter\expandafter\bxjs@tmpa
 \expandafter\string\the\jfont\relax
\fi
\@for\bxjs@x:={\jsc@JYn/mc/m/n,\jsc@JYn/gt/m/n,%
               \jsc@JTn/mc/m/n,\jsc@JTn/gt/m/n}\do
  {\expandafter\let\csname\bxjs@x/10\endcsname=\@undefined
   \expandafter\let\csname\bxjs@x/\bxjs@y\endcsname=\@undefined}
\begingroup
  \font\bxjs@tmpa=\bxjs@sizereference\space at 10pt
  \setbox\z@\hbox{\bxjs@tmpa\char\jis"2121\relax}
  \ifdim\wd\z@=10pt
    \global\let\bxjs@scale\jsScale
  \else
    \edef\bxjs@tmpa{\strip@pt\wd\z@}
    \@tempdima=10pt \@tempdima=\jsScale\@tempdima
    \bxjs@invscale\@tempdima\bxjs@tmpa
    \xdef\bxjs@scale{\strip@pt\@tempdima}
  \fi
\endgroup
\bxjs@declarefontshape
\DeclareFontShape{\jsc@JYn}{mc}{m}{it}{<->ssub*mc/m/n}{}
\DeclareFontShape{\jsc@JYn}{mc}{m}{sl}{<->ssub*mc/m/n}{}
\DeclareFontShape{\jsc@JYn}{mc}{m}{sc}{<->ssub*mc/m/n}{}
\DeclareFontShape{\jsc@JYn}{gt}{m}{it}{<->ssub*gt/m/n}{}
\DeclareFontShape{\jsc@JYn}{gt}{m}{sl}{<->ssub*gt/m/n}{}
\DeclareFontShape{\jsc@JYn}{mc}{bx}{it}{<->ssub*gt/m/n}{}
\DeclareFontShape{\jsc@JYn}{mc}{bx}{sl}{<->ssub*gt/m/n}{}
\DeclareFontShape{\jsc@JTn}{mc}{m}{it}{<->ssub*mc/m/n}{}
\DeclareFontShape{\jsc@JTn}{mc}{m}{sl}{<->ssub*mc/m/n}{}
\DeclareFontShape{\jsc@JTn}{mc}{m}{sc}{<->ssub*mc/m/n}{}
\DeclareFontShape{\jsc@JTn}{gt}{m}{it}{<->ssub*gt/m/n}{}
\DeclareFontShape{\jsc@JTn}{gt}{m}{sl}{<->ssub*gt/m/n}{}
\DeclareFontShape{\jsc@JTn}{mc}{bx}{it}{<->ssub*gt/m/n}{}
\DeclareFontShape{\jsc@JTn}{mc}{bx}{sl}{<->ssub*gt/m/n}{}
\DeclareRobustCommand\rmfamily
  {\not@math@alphabet\rmfamily\mathrm
   \romanfamily\rmdefault\kanjifamily\mcdefault\selectfont}
\DeclareRobustCommand\sffamily
  {\not@math@alphabet\sffamily\mathsf
   \romanfamily\sfdefault\kanjifamily\gtdefault\selectfont}
\DeclareRobustCommand\ttfamily
  {\not@math@alphabet\ttfamily\mathtt
   \romanfamily\ttdefault\kanjifamily\gtdefault\selectfont}
\DeclareJaTextFontCommand{\textmc}{\mcfamily}
\DeclareJaTextFontCommand{\textgt}{\gtfamily}
\bxjs@if@sf@default{%
  \renewcommand\kanjifamilydefault{\gtdefault}}
\selectfont
\prebreakpenalty\jis"2147=10000
\postbreakpenalty\jis"2148=10000
\prebreakpenalty\jis"2149=10000
\inhibitxspcode`!=1
\inhibitxspcode`〒=2
\xspcode`+=3
\xspcode`\%=3
\@tempcnta="80 \@whilenum\@tempcnta<"100 \do{%
  \xspcode\@tempcnta=3\advance\@tempcnta\@ne}
\let\jsInhibitGlueAtParTop\@inhibitglue
\begingroup
\catcode`\!=0
\gdef\bxjs@ptex@dir{%
  !iftdir t%
  !else!ifydir y%
  !else ?%
  !fi!fi}
\long\def\bxjs@tmpa{\hbox{%
  !ifydir \@textsuperscript{\normalfont\@thefnmark}%
  !else\hbox{\yoko\@textsuperscript{\normalfont\@thefnmark}}!fi}}
\ifx\@makefnmark\bxjs@tmpa
\long\gdef\@makefnmark{%
  !ifydir \hbox{}\hbox{\@textsuperscript{\normalfont\@thefnmark}}\hbox{}%
  !else\hbox{\yoko\@textsuperscript{\normalfont\@thefnmark}}!fi}
\fi
\endgroup
\else\ifx p\jsEngine
\let\bxjs@let@hchar@chr\bxjs@let@hchar@chr@ue
\@onlypreamble\bxjs@cjk@loaded
\def\bxjs@cjk@loaded{%
  \def\@footnotemark{%
    \leavevmode
    \ifhmode
      \edef\@x@sf{\the\spacefactor}%
      \ifdim\lastkern>\z@\ifdim\lastkern<5sp\relax
         \unkern\unkern
         \ifdim\lastskip>\z@ \unskip \fi
      \fi\fi
      \nobreak
    \fi
    \@makefnmark
    \ifhmode \spacefactor\@x@sf \fi
    \relax}%
  \let\bxjs@cjk@loaded\relax
}
\AtBeginDocument{%
  \@ifpackageloaded{CJK}{%
    \bxjs@cjk@loaded
  }{}%
}
\else\ifx x\jsEngine
\def\bxjs@let@hchar@chr#1{%
  \@tempcnta`#1\relax \divide\@tempcnta"800\relax
  \bxjs@cond\ifnum\@tempcnta=27 \fi{%
    \bxjs@let@hchar@chr@xe
  }{\bxjs@let@hchar@out\def{{#1}}}}
\def\bxjs@let@hchar@chr@xe#1{%
  \lccode`0=`#1\relax
  \lowercase{\bxjs@let@hchar@out\def{{0}}}}
\ifx\XeTeXgenerateactualtext\@undefined\else
  \def\bxjs@do@precisetext{%
    \XeTeXgenerateactualtext=\@ne}
\fi
\@onlypreamble\bxjs@do@simplejasetup
\def\bxjs@do@simplejasetup{%
  \ifnum\XeTeXinterchartokenstate>\z@
  \else\ifnum\strcmp{\the\XeTeXlinebreakskip}{\the\z@}=\z@
    \jsSimpleJaSetup
    \ClassInfo\bxjs@clsname
     {'\string\jsSimpleJaSetup' is applied\@gobble}%
  \fi\fi}
\newcommand*{\jsSimpleJaSetup}{%
  \XeTeXlinebreaklocale "ja"\relax
  \XeTeXlinebreakskip=0pt plus 1pt minus 0.1pt\relax
  \XeTeXlinebreakpenalty=0\relax}
\fi\fi\fi
\ifx\bxjs@do@simplejasetup\@undefined\else
  \AtBeginDocument{%
    \ifbxjs@simplejasetup
      \bxjs@do@simplejasetup
    \fi}
\fi
\ifbxjs@precisetext
  \ifx\bxjs@do@precisetext\@undefined
    \ClassWarning\bxjs@clsname
     {The current engine does not supprt the\MessageBreak
      'precisetext' option\@gobble}
  \else
    \bxjs@do@precisetext
  \fi
\fi
\ifbxjs@fancyhdr
\@onlypreamble\bxjs@adjust@fancyhdr
\def\bxjs@adjust@fancyhdr{%
  \def\bxjs@tmpa{\fancyplain{}{\sl\rightmark}\strut}%
  \def\bxjs@tmpb{\fancyplain{}{\rightmark}\strut}%
  \ifx\f@ncyelh\bxjs@tmpa \global\let\f@ncyelh\bxjs@tmpb \fi
  \ifx\f@ncyerh\bxjs@tmpa \global\let\f@ncyerh\bxjs@tmpb \fi
  \ifx\f@ncyolh\bxjs@tmpa \global\let\f@ncyolh\bxjs@tmpb \fi
  \ifx\f@ncyorh\bxjs@tmpa \global\let\f@ncyorh\bxjs@tmpb \fi
  \def\bxjs@tmpa{\fancyplain{}{\sl\leftmark}\strut}%
  \def\bxjs@tmpb{\fancyplain{}{\leftmark}\strut}%
  \ifx\f@ncyelh\bxjs@tmpa \global\let\f@ncyelh\bxjs@tmpb \fi
  \ifx\f@ncyerh\bxjs@tmpa \global\let\f@ncyerh\bxjs@tmpb \fi
  \ifx\f@ncyolh\bxjs@tmpa \global\let\f@ncyolh\bxjs@tmpb \fi
  \ifx\f@ncyorh\bxjs@tmpa \global\let\f@ncyorh\bxjs@tmpb \fi
  \def\bxjs@tmpa{\rm\thepage\strut}%
  \def\bxjs@tmpb{\thepage\strut}%
  \ifx\f@ncyecf\bxjs@tmpa \global\let\f@ncyecf\bxjs@tmpb \fi
  \ifx\f@ncyocf\bxjs@tmpa \global\let\f@ncyocf\bxjs@tmpb \fi
  \ifx\fullwidth\@undefined\else \ifdim\textwidth<\fullwidth
    \setlength{\@tempdima}{\fullwidth-\textwidth}%
    \edef\bxjs@tmpa{\noexpand\fancyhfoffset[EL,OR]{\the\@tempdima}%
    }\bxjs@tmpa
  \fi\fi
  \PackageInfo\bxjs@clsname
   {Patch to fancyhdr is applied\@gobble}}
\def\bxjs@pagestyle@hook{%
  \@ifpackageloaded{fancyhdr}{%
    \bxjs@adjust@fancyhdr
    \global\let\bxjs@adjust@fancyhdr\relax
  }{}}
\let\bxjs@org@pagestyle\pagestyle
\def\pagestyle{%
  \bxjs@pagestyle@hook \bxjs@org@pagestyle}
\AtBeginDocument{%
  \bxjs@pagestyle@hook
  \global\let\bxjs@pagestyle@hook\relax}
\fi
\endinput
%%
%% End of file `bxjsja-minimal.def'.

%    \end{macrocode}
%
% |simplejasetup| はstandardでは無効になる。
%    \begin{macrocode}
\bxjs@simplejasetupfalse
%    \end{macrocode}
%
% \paragraph{共通命令の実装}
%
% |\jQ| 等の「単位」系の共通命令を実装する。
% まず{$\varepsilon$-\TeX}拡張が使えるか検査する。
%    \begin{macrocode}
\ifjsWitheTeX
%    \end{macrocode}
% 使える場合は、「|\dimexpr|外部寸法表記|\relax|」の形式
% (これは内部値なので単位として使える)で各命令定義する。
%
% \begin{macro}{\jQ}
% \begin{macro}{\jH}
% |\jQ| と |\jH| はともに0.25\,mmに等しい。
%    \begin{macrocode}
  \@tempdima=0.25mm
  \protected\edef\jQ{\dimexpr\the\@tempdima\relax}
  \let\jH\jQ
%    \end{macrocode}
% \end{macro}
% \end{macro}
%
% \begin{macro}{\trueQ}
% \begin{macro}{\trueH}
% |\trueQ| と |\trueH| はともに0.25\,true\,mmに等しい。
%    \begin{macrocode}
  \ifjsc@mag
    \@tempdimb=\jsBaseFontSize\relax
    \edef\bxjs@tmpa{\strip@pt\@tempdimb}%
    \@tempdima=2.5mm
    \bxjs@invscale\@tempdima\bxjs@tmpa
    \protected\edef\trueQ{\dimexpr\the\@tempdima\relax}
    \@tempdima=10pt
    \bxjs@invscale\@tempdima\bxjs@tmpa
    \protected\edef\bxjs@truept{\dimexpr\the\@tempdima\relax}
  \else \let\trueQ\jQ \let\bxjs@truept\p@
  \fi
  \let\trueH\trueQ
%    \end{macrocode}
% \end{macro}
% \end{macro}
%
% \begin{macro}{\ascQ}
% \begin{macro}{\ascpt}
% |\ascQ| は |\trueQ| を和文スケール値で割った値。
% 例えば、|\fontsize{12\ascQ}{16\trueH}| とすると、
% 和文が12Qになる。
%
% 同様に、|\ascpt| は |truept| を和文スケールで割った値。
%    \begin{macrocode}
  \@tempdima\trueQ \bxjs@invscale\@tempdima\jsScale
  \protected\edef\ascQ{\dimexpr\the\@tempdima\relax}
  \@tempdima\bxjs@truept \bxjs@invscale\@tempdima\jsScale
  \protected\edef\ascpt{\dimexpr\the\@tempdima\relax}
\fi
%    \end{macrocode}
% \end{macro}
% \end{macro}
%
% 続いて、和文間空白・和欧文間空白関連の命令を実装する。
% (エンジン依存のコード。)
%
% \begin{macro}{\bxjs@kanjiskip}
% 和文間空白の量を表すテキスト。
%    \begin{macrocode}
\def\bxjs@kanjiskip{0pt}
%    \end{macrocode}
% \end{macro}
%
% \begin{macro}{\setkanjiskip}
% 和文間空白の量を設定する。
%    \begin{macrocode}
\newcommand*\setkanjiskip[1]{%
  \edef\bxjs@kanjiskip{#1}%
  \bxjs@reset@kanjiskip}
%    \end{macrocode}
% \end{macro}
%
% \begin{macro}{\getkanjiskip}
% 和文間空白の量を表すテキストに展開する。
%    \begin{macrocode}
\newcommand*\getkanjiskip{%
  \bxjs@kanjiskip}
%    \end{macrocode}
% \end{macro}
%
% \begin{macro}{\ifbxjs@kanjiskip@enabled}
% 和文間空白の挿入が有効か。
% ただし{\pTeX}では自身の |\(no)autospacing| での制御を
% 用いるのでこの変数は常に真とする。
%    \begin{macrocode}
\newif\ifbxjs@kanjiskip@enabled \bxjs@kanjiskip@enabledtrue
%    \end{macrocode}
% \end{macro}
%
% \begin{macro}{\bxjs@enable@kanjiskip}
% \begin{macro}{\bxjs@disable@kanjiskip}
% 和文間空白の挿入を有効/無効にする。({\pTeX}以外)
%    \begin{macrocode}
\bxjs@robust@def\bxjs@enable@kanjiskip{%
  \bxjs@kanjiskip@enabledtrue
  \bxjs@reset@kanjiskip}
\bxjs@robust@def\bxjs@disable@kanjiskip{%
  \bxjs@kanjiskip@enabledfalse
  \bxjs@reset@kanjiskip}
%    \end{macrocode}
% \end{macro}
% \end{macro}
%
% \begin{macro}{\bxjs@reset@kanjiskip}
% 現在の和文間空白の設定を実際にエンジンに反映させる。
%    \begin{macrocode}
\bxjs@robust@def\bxjs@reset@kanjiskip{%
  \ifbxjs@kanjiskip@enabled
    \setlength{\@tempskipa}{\bxjs@kanjiskip}%
  \else \@tempskipa\z@
  \fi
  \bxjs@apply@kanjiskip}
%    \end{macrocode}
% \end{macro}
%
% \begin{macro}{\bxjs@xkanjiskip}
% \begin{macro}{\setxkanjiskip}
% \begin{macro}{\getxkanjiskip}
% \begin{macro}{\ifbxjs@xkanjiskip@enabled}
% \begin{macro}{\bxjs@enable@xkanjiskip}
% \begin{macro}{\bxjs@disable@xkanjiskip}
% \begin{macro}{\bxjs@reset@xkanjiskip}
% 和欧文間空白について同様のものを用意する。
%    \begin{macrocode}
\def\bxjs@xkanjiskip{0pt}
\newcommand*\setxkanjiskip[1]{%
  \edef\bxjs@xkanjiskip{#1}%
  \bxjs@reset@xkanjiskip}
\newcommand*\getxkanjiskip{%
  \bxjs@xkanjiskip}
\newif\ifbxjs@xkanjiskip@enabled \bxjs@xkanjiskip@enabledtrue
\bxjs@robust@def\bxjs@enable@xkanjiskip{%
  \bxjs@xkanjiskip@enabledtrue
  \bxjs@reset@xkanjiskip}
\bxjs@robust@def\bxjs@disable@xkanjiskip{%
  \bxjs@xkanjiskip@enabledfalse
  \bxjs@reset@xkanjiskip}
\bxjs@robust@def\bxjs@reset@xkanjiskip{%
  \ifbxjs@xkanjiskip@enabled
    \setlength{\@tempskipa}{\bxjs@xkanjiskip}%
  \else \@tempskipa\z@
  \fi
  \bxjs@apply@xkanjiskip}
%    \end{macrocode}
% \end{macro}
% \end{macro}
% \end{macro}
% \end{macro}
% \end{macro}
% \end{macro}
% \end{macro}
%
% |\jsResetDimen| を用いて、フォントサイズが変更された時に
% 空白の量が追随するようにする。
%    \begin{macrocode}
\g@addto@macro\jsResetDimen{%
  \bxjs@reset@kanjiskip
  \bxjs@reset@xkanjiskip}
\let\bxjs@apply@kanjiskip\relax
\let\bxjs@apply@xkanjiskip\relax
%    \end{macrocode}
%
% \paragraph{和文フォント指定の扱い}
%
% \begin{macro}{\bxjs@adjust@jafont}
% ムニャムニャ…。
%    \begin{macrocode}
\@onlypreamble\bxjs@adjust@jafont
\def\bxjs@adjust@jafont#1{%
  \ifx\jsJaFont\bxjs@@auto
    \bxjs@get@kanjiEmbed
    \ifx\bxjs@kanjiEmbed\relax
      \let\bxjs@tmpa\@empty
    \else
      \let\bxjs@tmpa\bxjs@kanjiEmbed
    \fi
  \else
    \let\bxjs@tmpa\jsJaFont
  \fi
  \if f#1\ifx\bxjs@tmpa\bxjs@@noEmbed
    \ClassWarningNoLine\bxjs@clsname
     {Option 'jafont=noEmbed' is ignored, because it is\MessageBreak
      not available on the current situation}%
    \let\bxjs@tmpa\@empty
  \fi\fi
}
\def\bxjs@@auto{auto}
\def\bxjs@@noEmbed{noEmbed}
%    \end{macrocode}
% \end{macro}
%
% \begin{macro}{\bxjs@kanjiEmbed}
% 現在の |updmap| の |kanjiEmbed| パラメタの値。
% |\bxjs@get@kanjiEmbed| により設定される。
%    \begin{macrocode}
\let\bxjs@kanjiEmbed\relax
%    \end{macrocode}
% \end{macro}
%
% \begin{macro}{\bxjs@get@kanjiEmbed}
% 現在の |updmap| の |kanjiEmbed| パラメタの値を取得する。
%    \begin{macrocode}
\@onlypreamble\bxjs@get@kanjiEmbed
\def\bxjs@get@kanjiEmbed{%
  \begingroup\setbox\z@=\hbox{%
    \global\let\bxjs@g@tmpa\relax
    \endlinechar\m@ne
    \let\do\@makeother\dospecials
    \catcode32=10 \catcode12=10 %form-feed
    \let\bxjs@tmpa\@empty
    \openin\@inputcheck="|kpsewhich updmap.cfg"\relax
    \ifeof\@inputcheck\else
      \read\@inputcheck to\bxjs@tmpa
      \closein\@inputcheck
    \fi
    \ifx\bxjs@tmpa\@empty\else
      \openin\@inputcheck="\bxjs@tmpa"\relax
      \@tempswatrue
      \loop\if@tempswa
        \read\@inputcheck to\bxjs@tmpa
        \expandafter\bxjs@get@ke@a\bxjs@tmpa\@nil kanjiEmbed \@nil\@nnil
        \ifx\bxjs@tmpa\relax\else
          \global\let\bxjs@g@tmpa\bxjs@tmpa
          \@tempswafalse
        \fi
        \ifeof\@inputcheck \@tempswafalse \fi
      \repeat
    \fi
  }\endgroup
  \let\bxjs@kanjiEmbed\bxjs@g@tmpa
}
\@onlypreamble\bxjs@get@ke@a
\def\bxjs@get@ke@a#1kanjiEmbed #2\@nil#3\@nnil{%
  \ifx$#1$\def\bxjs@tmpa{#2}%
  \else \let\bxjs@tmpa\relax
  \fi}
%    \end{macrocode}
% \end{macro}
%
% \begin{macro}{\jachar}
% |\jachar{|\meta{文字}|}|\Means
% 和文文字として出力する。
%    \begin{macrocode}
\newcommand*\jachar[1]{%
  \begingroup
%    \end{macrocode}
% |\jsLetHeadChar| で先頭の“文字”を拾って
% それを |\bxjs@jachar| に渡す。
%    \begin{macrocode}
    \jsLetHeadChar\bxjs@tmpa{#1}%
    \ifx\bxjs@tmpa\relax
      \ClassWarningNoLine\bxjs@clsname
        {Illegal argument given to \string\jachar}%
    \else
      \expandafter\bxjs@jachar\expandafter{\bxjs@tmpa}%
    \fi
  \endgroup}
%    \end{macrocode}
% |\jsJaChar| を |\jachar| と等価にする。
%    \begin{macrocode}
\let\jsJaChar\jachar
%    \end{macrocode}
% 下請けの |\bxjs@jachar| の実装はエンジンにより異なる。
%    \begin{macrocode}
\let\bxjs@jachar\@firstofone
%    \end{macrocode}
% \end{macro}
%
% \paragraph{hyperref対策}
%
% 出力ページサイズに館する処理は |geometry| パッケージが行うので、
% |hyperref| 側の処理は無効にしておく。
%    \begin{macrocode}
\PassOptionsToPackage{setpagesize=false}{hyperref}
%    \end{macrocode}
%
% \begin{macro}{\bxjs@fix@hyperref@unicode}
% |hyperref| の |unicode| オプションの値を固定する。
%    \begin{macrocode}
\@onlypreamble\bxjs@fix@hyperref@unicode
\def\bxjs@fix@hyperref@unicode#1{%
  \PassOptionsToPackage{bxjs/hook=#1}{hyperref}%
  \@namedef{KV@Hyp@bxjs/hook}##1{%
    \KV@Hyp@unicode{##1}%
    \def\KV@Hyp@unicode####1{%
      \expandafter\ifx\csname if##1\expandafter\endcsname
         \csname if####1\endcsname\else
        \ClassWarningNoLine\bxjs@clsname
        {Blcoked hyperref option 'unicode=####1'}%
      \fi
    }%
  }%
}
%    \end{macrocode}
% \end{macro}
%
% \begin{macro}{\bxjs@urgent@special}
% DVIのなるべく早い位置にspecialを出力する。
%    \begin{macrocode}
\@onlypreamble\bxjs@urgent@special
\def\bxjs@urgent@special#1{%
  \AtBeginDvi{\special{#1}}%
  \AtBeginDocument{%
    \@ifpackageloaded{atbegshi}{%
      \begingroup
        \toks\z@{\special{#1}}%
        \toks\tw@\expandafter{\AtBegShi@HookFirst}%
        \xdef\AtBegShi@HookFirst{\the\toks@\the\toks\tw@}%
      \endgroup
    }{}%
  }%
}
%    \end{macrocode}
% \end{macro}
%
%^^A----------------
% \subsection{{\pTeX}用設定}
%
%    \begin{macrocode}
\if j\jsEngine
%    \end{macrocode}
%
% \paragraph{共通命令の実装}
%
%    \begin{macrocode}
\def\bxjs@apply@kanjiskip{%
  \kanjiskip\@tempskipa}
\def\bxjs@apply@xkanjiskip{%
  \xkanjiskip\@tempskipa}
%    \end{macrocode}
%
% |\jaJaChar| のサブマクロ。
%    \begin{macrocode}
\def\bxjs@jachar#1{%
  \bxjs@jachar@a#1....\@nil}
\def\bxjs@jachar@a#1#2#3#4#5\@nil{%
%    \end{macrocode}
% 引数が単一トークンなら和文文字トークンが得られたと
% 見なしてそれをそのまま出力する。
%    \begin{macrocode}
  \ifx.#2#1%
%    \end{macrocode}
% 引数が複数トークンの場合は、UTF-8のバイト列であると見なし、
% そのスカラー値を |\@tempcnta| に代入する。
%    \begin{macrocode}
  \else\ifx.#3%
    \@tempcnta`#1 \multiply\@tempcnta64
    \advance\@tempcnta`#2 \advance\@tempcnta-"3080
    \bxjs@jachar@b
  \else\ifx.#4%
    \@tempcnta`#1 \multiply\@tempcnta64
    \advance\@tempcnta`#2 \multiply\@tempcnta64
    \advance\@tempcnta`#3 \advance\@tempcnta-"E2080
    \bxjs@jachar@b
  \else
    \@tempcnta`#1 \multiply\@tempcnta64
    \advance\@tempcnta`#2 \multiply\@tempcnta64
    \advance\@tempcnta`#3 \multiply\@tempcnta64
    \advance\@tempcnta`#4 \advance\@tempcnta-"3C82080
    \bxjs@jachar@b
  \fi\fi\fi}
%    \end{macrocode}
% 符号値が |\@tempcnta| の和文文字を出力する処理。
%    \begin{macrocode}
\ifjsWithupTeX
  \def\bxjs@jachar@b{\kchar\@tempcnta}
\else
  \def\bxjs@jachar@b{%
    \ifx\bxUInt\@undefined\else
      \bxUInt{\@tempcnta}%
    \fi}
\fi
%    \end{macrocode}
%
% \paragraph{和文フォント指定の扱い}
%
% {\pTeX}は既定で |kanji-config-updmap| の設定に従うため、
% |\jsJaFont| が |auto| の場合は何もする必要がない。
% 無指定でも |auto| でもない場合は、|\jsJaFont| をオプションに
% して |pxchfon| パッケージを読み込む。
%    \begin{macrocode}
\let\bxjs@tmpa\jsJaFont
\ifx\bxjs@tmpa\bxjs@@auto
  \let\bxjs@tmpa\@empty
\else\ifx\bxjs@tmpa\bxjs@@noEmbed
  \def\bxjs@tmpa{noembed}
\fi\fi
\ifx\jsJaFont\@empty\else
  \edef\bxjs@nxt{%
    \noexpand\RequirePackage[\jsJaFont]
        {pxchfon}[2010/05/12]}% v0.5
  \bxjs@nxt
\fi
%    \end{macrocode}
%
% \paragraph{otfパッケージ対策}
%
% インストールされている |otf| パッケージが |scale| オプションに
% 対応している場合は |scale=(\jsScaleの値)| を事前に |otf| に渡す。
% \Note otf.sty の中に「|\RequirePackage{keyval}|」の行が存在する
% かにより判定している。
% (もっといい方法はないのか……。)
%
%    \begin{macrocode}
\begingroup
  \global\let\@gtempa\relax
  \catcode`\|=0 \catcode`\\=12
  |def|bxjs@check#1|@nil{%
    |bxjs@check@a#1|@nil\RequirePackage|@nnil}%
  |def|bxjs@check@a#1\RequirePackage#2|@nnil{%
    |ifx$#1$|bxjs@check@b#2|@nil keyval|@nnil |fi}%
  |catcode`|\=0 \catcode`\|=12
  \def\bxjs@check@b#1keyval#2\@nnil{%
    \ifx$#2$\else
      \xdef\@gtempa{%
        \noexpand\PassOptionsToPackage{scale=\jsScale}{otf}}%
    \fi}
\@firstofone{%
  \catcode10=12 \endlinechar\m@ne
  \let\do\@makeother \dospecials \catcode32=10
  \openin\@inputcheck=otf.sty\relax
  \@tempswatrue
  \loop\if@tempswa
    \ifeof\@inputcheck \@tempswafalse \fi
    \if@tempswa
      \read\@inputcheck to\bxjs@line
      \expandafter\bxjs@check\bxjs@line\@nil
    \fi
  \repeat
  \closein\@inputcheck
\endgroup}
\@gtempa
%    \end{macrocode}
%
% \paragraph{hyperref対策}
%
% |unicode| にしてはいけない。
%    \begin{macrocode}
\bxjs@fix@hyperref@unicode{false}
%    \end{macrocode}
%
% |tounicode| special命令を出力する。
%    \begin{macrocode}
\if \ifx\bxjs@driver@given\bxjs@driver@@dvipdfmx T%
    \else\ifjsWithpTeXng T\else F\fi\fi T%
  \ifnum\jis"2121="A1A1 %euc
    \bxjs@urgent@special{pdf:tounicode EUC-UCS2}
  \else\ifnum\jis"2121="8140 %sjis
    \bxjs@urgent@special{pdf:tounicode 90ms-RKSJ-UCS2}
  \else\ifnum\jis"2121="3000 %uptex
    \ifbxjs@bigcode
      \bxjs@urgent@special{pdf:tounicode UTF8-UTF16}
      \PassOptionsToPackage{bigcode}{pxjahyper}
    \else
      \bxjs@urgent@special{pdf:tounicode UTF8-UCS2}
    \fi
  \fi\fi\fi
  \let\bxToUnicodeSpecialDone=t
\fi
%    \end{macrocode}
%
% \paragraph{microtype対策}
%
%    \begin{macrocode}
\@namedef{ver@microtype.sty}{2000/01/01}
\newcommand*\UseMicrotypeSet[2][]{}
%    \end{macrocode}
%
%^^A----------------
% \subsection{pdf{\TeX}用設定: CJK + bxcjkjatype}
%
%    \begin{macrocode}
\else\if p\jsEngine
%    \end{macrocode}
%
% \paragraph{bxcjkjatypeパッケージの読込}
%
% |\jsJaFont| が指定されている場合は、その値を |bxcjkjatype| の
% オプション(プリセット指定)に渡す。
% (|auto| ならば |\bxjs@get@kanjiEmbed| を実行する。)
% スケール値(|\jsScale|)の反映は bxcjkjatype の側で行われる。
%    \begin{macrocode}
\bxjs@adjust@jafont{f}
\edef\bxjs@nxt{%
  \noexpand\RequirePackage[%
      \ifx\bxjs@tmpa\@empty\else \bxjs@tmpa,\fi
      whole,autotilde]{bxcjkjatype}[2013/10/15]}% v0.2c
\bxjs@nxt
\bxjs@cjk@loaded
%    \end{macrocode}
%
% \paragraph{hyperref対策}
%
% |bxcjkjatype| 使用時は |unicode| にするべき。
% \Note 取りあえず固定はしない。
%    \begin{macrocode}
\PassOptionsToPackage{unicode}{hyperref}
%    \end{macrocode}
%
% |\hypersetup| 命令で(|CJK*| 環境に入れなくても)日本語文字を
% 含む文書情報を設定できるようにするための細工。
% \Note bxcjkjatype を |whole| 付きで使っていることが前提。
% \Note パッケージオプションでの指定に対応するのは、
% 「アクティブな高位バイトトークンがその場で展開されてしまう」
% ため困難である。
%    \begin{macrocode}
\ifx\bxcjkjatypeHyperrefPatchDone\@undefined
\begingroup
  \CJK@input{UTF8.bdg}
\endgroup
\g@addto@macro\pdfstringdefPreHook{%
  \@nameuse{CJK@UTF8Binding}%
}
\fi
%    \end{macrocode}
%
% |~| が和欧文間空白である場合はPDF文字列中で空白文字で
% なく空に展開させる。
%    \begin{macrocode}
\ifx\bxcjkjatypeHyperrefPatchDone\@undefined
\g@addto@macro\pdfstringdefPreHook{%
  \ifx~\bxjs@@CJKtilde
    \let\bxjs@org@LetUnexpandableSpace\HyPsd@LetUnexpandableSpace
    \let\HyPsd@LetUnexpandableSpace\bxjs@LetUnexpandableSpace
    \let~\@empty
  \fi
}
\def\bxjs@@CJKtilde{\CJKecglue\ignorespaces}
\def\bxjs@@tildecmd{~}
\def\bxjs@LetUnexpandableSpace#1{%
  \def\bxjs@tmpa{#1}\ifx\bxjs@tmpa\bxjs@@tildecmd\else
    \bxjs@org@LetUnexpandableSpace#1%
  \fi}
\fi
%    \end{macrocode}
%
% \paragraph{共通命令の実装}
%
%    \begin{macrocode}
\newskip\jsKanjiSkip
\newskip\jsXKanjiSkip
\ifx\CJKecglue\@undefined
  \def\CJKtilde{\CJK@global\def~{\CJKecglue\ignorespaces}}
\fi
\let\autospacing\bxjs@enable@kanjiskip
\let\noautospacing\bxjs@disable@kanjiskip
\protected\def\bxjs@CJKglue{\hskip\jsKanjiSkip}
\def\bxjs@apply@kanjiskip{%
  \jsKanjiSkip\@tempskipa
  \let\CJKglue\bxjs@CJKglue}
\let\autoxspacing\bxjs@enable@xkanjiskip
\let\noautoxspacing\bxjs@disable@xkanjiskip
\protected\def\bxjs@CJKecglue{\hskip\jsXKanjiSkip}
\def\bxjs@apply@xkanjiskip{%
  \jsXKanjiSkip\@tempskipa
  \let\CJKecglue\bxjs@CJKecglue}
%    \end{macrocode}
%
% |\jachar| のサブマクロの実装。
%    \begin{macrocode}
\def\bxjs@jachar#1{%
  \CJKforced{#1}}
%    \end{macrocode}
%
%^^A----------------
% \subsection{{\XeTeX}用設定: xeCJK + zxjatype}
%
%    \begin{macrocode}
\else\if x\jsEngine
%    \end{macrocode}
%
% \paragraph{zxjatypeパッケージの読込}
%
% スケール値(|\jsScale|)の反映は zxjatype の側で行われる。
%    \begin{macrocode}
\RequirePackage{zxjatype}
\PassOptionsToPackage{no-math}{fontspec}%!
\PassOptionsToPackage{xetex}{graphicx}%!
\PassOptionsToPackage{xetex}{graphics}%!
\ifx\zxJaFamilyName\@undefined
  \ClassError\bxjs@clsname
  {xeCJK or zxjatype is too old}\@ehc
\fi
%    \end{macrocode}
%
% \paragraph{和文フォント定義}
%
% |\jsJaFont| が指定された場合は、その値をオプションと
% して |zxjafont| を読み込む。
% 非指定の場合はIPAexフォントを使用する。
%    \begin{macrocode}
\bxjs@adjust@jafont{f}
\ifx\bxjs@tmpa\@empty
  \setCJKmainfont[BoldFont=IPAexGothic]{IPAexMincho}
  \setCJKsansfont[BoldFont=IPAexGothic]{IPAexGothic}
\else
  \edef\bxjs@nxt{%
    \noexpand\RequirePackage[\bxjs@tmpa]%
        {zxjafont}[2013/01/28]}% v0.2a
  \bxjs@nxt
\fi
%    \end{macrocode}
%
% \paragraph{hyperref対策}
%
% |unicode| オプションの指定に関する話。
%
% {\XeTeX}の場合は、xdvipdfmxがUTF-8→UTF-16の変換を行う機能を
% 持っているため、本来はspecial命令の文字列の文字コード変換は不要である。
% ところが、|hyperref| での方針としては、{\XeTeX}の場合にも
% パッケージ側で文字コード変換を行う方が望ましいと考えている。
% 実際、|unicode| を無効にしていると警告が出て強制的に有効化される。
% 一方で、過去(r35125まで)^^A2014/09/20
% のxdvipdfmxでは、文字列をUTF-16に変換した状態で与えるのは不正と
% 見なしていて警告が発生する。
%
% これを踏まえて、ここでは、
% 「{\XeTeX}のバージョンが0.99992以上の場合に |unicode| を既定で
% 有効にする」
% ことにする。
% \Note 取りあえず固定はしない。
%    \begin{macrocode}
\ifnum\strcmp{\the\XeTeXversion\XeTeXrevision}{0.99992}>\m@ne
  \PassOptionsToPackage{unicode}{hyperref}
\fi
%    \end{macrocode}
%
% \paragraph{段落頭でのグルー挿入禁止}
%
% どうやら、\Pkg{zxjatype}の |\inhibitglue| の実装が極めて杜撰なため、
% 1.0版での実装では全く期待通りの動作をしていないし、
% そもそも(少なくとも現状の)\Pkg{xeCJK}では、
% 段落頭での |\inhibitglue| は実行しないほうがJSクラスの出力に
% 近いものが得られるらしい。
%
% 従って、|\jsInhibitGlueAtParTop| は結局何もしないことにする。
%    \begin{macrocode}
\let\jsInhibitGlueAtParTop\@empty
%    \end{macrocode}
%
% \paragraph{共通命令の実装}
%
%    \begin{macrocode}
\newskip\jsKanjiSkip
\newskip\jsXKanjiSkip
\ifx\CJKecglue\@undefined
  \def\CJKtilde{\CJK@global\def~{\CJKecglue\ignorespaces}}
\fi
\let\autospacing\bxjs@enable@kanjiskip
\let\noautospacing\bxjs@disable@kanjiskip
\protected\def\bxjs@CJKglue{\hskip\jsKanjiSkip}
\def\bxjs@apply@kanjiskip{%
  \jsKanjiSkip\@tempskipa
  \xeCJKsetup{CJKglue={\bxjs@CJKglue}}}
\let\autoxspacing\bxjs@enable@xkanjiskip
\let\noautoxspacing\bxjs@disable@xkanjiskip
\protected\def\bxjs@CJKecglue{\hskip\jsXKanjiSkip}
\def\bxjs@apply@xkanjiskip{%
  \jsXKanjiSkip\@tempskipa
  \xeCJKsetup{CJKecglue={\bxjs@CJKecglue}}}
%    \end{macrocode}
%
% |\mcfamily|、|\gtfamily| は本来は zxjatype の方で定義すべき
% であろうが、現状は暫定的にここで定義する。
%    \begin{macrocode}
\ifx\mcfamily\@undefined
  \protected\def\mcfamily{\CJKfamily{\CJKrmdefault}}
  \protected\def\gtfamily{\CJKfamily{\CJKsfdefault}}
\fi
%    \end{macrocode}
%
% |\jachar| のサブマクロの実装。
%    \begin{macrocode}
\def\bxjs@jachar#1{%
  \xeCJKDeclareCharClass{CJK}{`#1}\relax
  #1}
%    \end{macrocode}
%
%^^A----------------
% \subsection{Lua{\TeX}用設定: Lua{\TeX}-ja}
%
%    \begin{macrocode}
\else\if l\jsEngine
%    \end{macrocode}
%
% \paragraph{Lua{\TeX}-jaパッケージの読込}
%
% |luatexja| とともに |luatexja-fontspec| パッケージを読み込む。
%
% |luatexja| は自前の |\zw|(これは実際の現在和文フォントに
% 基づく値を返す)を定義するので、|\zw| の定義を消しておく。
% なお、レイアウト定義の「全角幅」は「規定」に基づく |\jsZw| で
% あることに注意が必要。
%
%    \begin{macrocode}
\let\zw\@undefined
\RequirePackage{luatexja}
\RequirePackage{luatexja-fontspec}
\PassOptionsToPackage{pdftex}{graphicx}%!
\PassOptionsToPackage{pdftex}{graphics}%!
%    \end{macrocode}
%
% \paragraph{和文フォント定義}
%
% |luatexja-fontspec| で使用する和文スケール値を |\jsScale| と合致
% させたいのだが……もっと良い方法はないのか?
%    \begin{macrocode}
\ExplSyntaxOn
\fp_gset:Nn \g_ltj_fontspec_scale_fp { \jsScale }
\ExplSyntaxOff
%    \end{macrocode}
%
% |\jsJaFont| が指定された場合は、その値をオプションと
% して |luatexja-preset| を読み込む。
% 非指定の場合は、|luatexja-preset| パッケージの |ipaex|
% オプション(IPAexフォント使用)と等価な設定を用いる
% (|luatexja-preset| は読み込まない)。
%    \begin{macrocode}
\bxjs@adjust@jafont{t}
\ifx\bxjs@tmpa\bxjs@@noEmbed
  \def\bxjs@tmpa{noembed}
\fi
\ifx\bxjs@tmpa\@empty
  \defaultjfontfeatures{ Kerning=Off }
  \setmainjfont[BoldFont=IPAexGothic,JFM=ujis]{IPAexMincho}
  \setsansjfont[BoldFont=IPAexGothic,JFM=ujis]{IPAexGothic}
\else
  \edef\bxjs@nxt{%
    \noexpand\RequirePackage[\bxjs@tmpa]
        {luatexja-preset}}%
  \bxjs@nxt
\fi
%    \end{macrocode}
%
% 欧文総称フォント命令で和文フォントが連動するように修正する。
% その他の和文フォント関係の定義を行う。
%    \begin{macrocode}
\DeclareRobustCommand\rmfamily
  {\not@math@alphabet\rmfamily\mathrm
   \romanfamily\rmdefault\kanjifamily\mcdefault\selectfont}
\DeclareRobustCommand\sffamily
  {\not@math@alphabet\sffamily\mathsf
   \romanfamily\sfdefault\kanjifamily\gtdefault\selectfont}
\DeclareRobustCommand\ttfamily
  {\not@math@alphabet\ttfamily\mathtt
   \romanfamily\ttdefault\kanjifamily\gtdefault\selectfont}
\AtBeginDocument{%
  \reDeclareMathAlphabet{\mathrm}{\mathrm}{\mathmc}
  \reDeclareMathAlphabet{\mathbf}{\mathbf}{\mathgt}}%
\bxjs@if@sf@default{%
  \renewcommand\kanjifamilydefault{\gtdefault}}
%    \end{macrocode}
%
% \paragraph{和文パラメタの設定}
%
%    \begin{macrocode}
% 次の3つは既定値の通り
%\ltjsetparameter{prebreakpenalty={`’,10000}}
%\ltjsetparameter{postbreakpenalty={`“,10000}}
%\ltjsetparameter{prebreakpenalty={`”,10000}}
\ltjsetparameter{jaxspmode={`!,1}}
\ltjsetparameter{jaxspmode={`〒,2}}
\ltjsetparameter{alxspmode={`+,3}}
\ltjsetparameter{alxspmode={`\%,3}}
%    \end{macrocode}
%
% \paragraph{段落頭でのグルー挿入禁止}
%
%    \begin{macrocode}
\protected\def\@inhibitglue{%
  \directlua{%
    luatexja.jfmglue.create_beginpar_node()}}
\let\bxjs@ltj@inhibitglue\@inhibitglue
\let\@@inhibitglue\@undefined
%    \end{macrocode}
%
% \paragraph{hyperref対策}
%
% |unicode| にするべき。
%    \begin{macrocode}
\bxjs@fix@hyperref@unicode{true}
%    \end{macrocode}
%
% \paragraph{共通命令の実装}
%
%    \begin{macrocode}
\protected\def\autospacing{%
  \ltjsetparameter{autospacing=true}}
\protected\def\noautospacing{%
  \ltjsetparameter{autospacing=false}}
\protected\def\autoxspacing{%
  \ltjsetparameter{autoxspacing=true}}
\protected\def\noautoxspacing{%
  \ltjsetparameter{autoxspacing=false}}
\def\bxjs@apply@kanjiskip{%
  \ltjsetparameter{kanjiskip={\@tempskipa}}}
\def\bxjs@apply@xkanjiskip{%
  \ltjsetparameter{xkanjiskip={\@tempskipa}}}
%    \end{macrocode}
%
% |\jachar| のサブマクロの実装。
%    \begin{macrocode}
\def\bxjs@jachar#1{%
  \ltjjachar`#1\relax}
%    \end{macrocode}
%
%^^A----------------
% \subsection{共通処理(2)}
%
%    \begin{macrocode}
\fi\fi\fi\fi
%    \end{macrocode}
%
% \paragraph{共通命令の実装}
% \begin{macro}{\textmc}
% \begin{macro}{\textgt}
% minimal ドライバ実装中で定義した |\DeclareJaTextFontCommand|
% を利用する。
%    \begin{macrocode}
\DeclareJaTextFontCommand{\textmc}{\mcfamily}
\DeclareJaTextFontCommand{\textgt}{\gtfamily}
%    \end{macrocode}
% \end{macro}
% \end{macro}
%
% \paragraph{和文・和欧文間空白の初期値}
%
%    \begin{macrocode}
\setkanjiskip{0pt plus.1\jsZw minus.01\jsZw}
\ifx\jsDocClass\jsSlide \setxkanjiskip{0.1em}
\else \setxkanjiskip{0.25em plus 0.15em minus 0.06em}
\fi
%    \end{macrocode}
%
% 以上で終わり。
%
%    \begin{macrocode}
%</standard>
%    \end{macrocode}
%
%^^A========================================================
% \section{和文ドライバ:modern \ZRX}
%
% モダーンな設定。
%
% standardドライバの設定を引き継ぐ。
%    \begin{macrocode}
%<*modern>
%%
%% This is file `bxjsja-standard.def',
%% generated with the docstrip utility.
%%
%% The original source files were:
%%
%% bxjscls.dtx  (with options: `drv,standard')
%% 
%% IMPORTANT NOTICE:
%% 
%% For the copyright see the source file.
%% 
%% Any modified versions of this file must be renamed
%% with new filenames distinct from bxjsja-standard.def.
%% 
%% For distribution of the original source see the terms
%% for copying and modification in the file bxjscls.dtx.
%% 
%% This generated file may be distributed as long as the
%% original source files, as listed above, are part of the
%% same distribution. (The sources need not necessarily be
%% in the same archive or directory.)
\ProvidesFile{bxjsja-standard.def}
  [2017/03/25 v1.6-pre BXJS document classes]
%% このファイルは日本語文字を含みます
%%
%% This is file `bxjsja-minimal.def',
%% generated with the docstrip utility.
%%
%% The original source files were:
%%
%% bxjscls.dtx  (with options: `drv,minimal')
%% 
%% IMPORTANT NOTICE:
%% 
%% For the copyright see the source file.
%% 
%% Any modified versions of this file must be renamed
%% with new filenames distinct from bxjsja-minimal.def.
%% 
%% For distribution of the original source see the terms
%% for copying and modification in the file bxjscls.dtx.
%% 
%% This generated file may be distributed as long as the
%% original source files, as listed above, are part of the
%% same distribution. (The sources need not necessarily be
%% in the same archive or directory.)
%% \CharacterTable
%%  {Upper-case    \A\B\C\D\E\F\G\H\I\J\K\L\M\N\O\P\Q\R\S\T\U\V\W\X\Y\Z
%%   Lower-case    \a\b\c\d\e\f\g\h\i\j\k\l\m\n\o\p\q\r\s\t\u\v\w\x\y\z
%%   Digits        \0\1\2\3\4\5\6\7\8\9
%%   Exclamation   \!     Double quote  \"     Hash (number) \#
%%   Dollar        \$     Percent       \%     Ampersand     \&
%%   Acute accent  \'     Left paren    \(     Right paren   \)
%%   Asterisk      \*     Plus          \+     Comma         \,
%%   Minus         \-     Point         \.     Solidus       \/
%%   Colon         \:     Semicolon     \;     Less than     \<
%%   Equals        \=     Greater than  \>     Question mark \?
%%   Commercial at \@     Left bracket  \[     Backslash     \\
%%   Right bracket \]     Circumflex    \^     Underscore    \_
%%   Grave accent  \`     Left brace    \{     Vertical bar  \|
%%   Right brace   \}     Tilde         \~}
\ProvidesFile{bxjsja-minimal.def}
  [2016/08/31 v1.3-pre BXJS document classes]
%% このファイルは日本語文字を含みます
\def\DeclareJaTextFontCommand#1#2{%
  \DeclareRobustCommand#1[1]{%
    \relax
    \ifmmode \expandafter\nfss@text \fi
    {#2##1}}%
}
\long\def\bxjs@@CSsfdefault{\sfdefault}%
\@onlypreamble\bxjs@if@sf@default
\def\bxjs@if@sf@default#1{%
  \ifx\familydefault\bxjs@@CSsfdefault#1\fi
  \AtBeginDocument{%
    \ifx\familydefault\bxjs@@CSsfdefault#1\fi}%
}
\def\jsLetHeadChar#1#2{%
  \begingroup
    \escapechar=`\\ %
    \let\bxjs@tmpa={% brace-match-hack
    \bxjs@let@hchar@exp#2}%
  \endgroup
  \let#1\bxjs@g@tmpa}
\def\bxjs@let@hchar@exp{%
  \futurelet\@let@token\bxjs@let@hchar@exp@a}
\def\bxjs@let@hchar@exp@a{%
  \bxjs@cond\ifcat\noexpand\@let@token\bgroup\fi{% 波括弧
    \bxjs@let@hchar@out\let\relax
  }{\bxjs@cond\ifcat\noexpand\@let@token\@sptoken\fi{% 空白
    \bxjs@let@hchar@out\let\space%
  }{\bxjs@cond\if\noexpand\@let@token\@backslashchar\fi{% バックスラッシュ
    \bxjs@let@hchar@out\let\@backslashchar
  }{\bxjs@let@hchar@exp@b}}}}
\def\bxjs@let@hchar@exp@b#1{%
  \expandafter\bxjs@let@hchar@exp@c\string#1?\@nil#1}
\def\bxjs@let@hchar@exp@c#1#2\@nil{%
  \bxjs@cond\if#1\@backslashchar\fi{% 制御綴
    \bxjs@cond\expandafter\ifx\noexpand\@let@token\@let@token\fi{%
      \bxjs@let@hchar@out\let\relax
    }{%else
      \expandafter\bxjs@let@hchar@exp
    }%
  }{%else
    \bxjs@let@hchar@chr#1%
  }}
\def\bxjs@let@hchar@chr#1{%
  \bxjs@let@hchar@out\def{{#1}}}
\def\bxjs@let@hchar@out#1#2{%
  \global#1\bxjs@g@tmpa#2\relax
  \toks@\bgroup}% skip to right brace
\chardef\bxjs@let@hchar@csta=128
\chardef\bxjs@let@hchar@cstb=192
\chardef\bxjs@let@hchar@cstc=224
\chardef\bxjs@let@hchar@cstd=240
\chardef\bxjs@let@hchar@cste=248
\let\bxjs@let@hchar@chr@ue@a\bxjs@let@hchar@chr
\def\bxjs@let@hchar@chr@ue#1{%
  \@tempcnta=`#1\relax
  \bxjs@cond\ifnum\@tempcnta<\bxjs@let@hchar@csta\fi{%
    \bxjs@let@hchar@chr@ue@a#1%
  }{\bxjs@cond\ifnum\@tempcnta<\bxjs@let@hchar@cstb\fi{%
    \bxjs@let@hchar@out\let\relax
  }{\bxjs@cond\ifnum\@tempcnta<\bxjs@let@hchar@cstc\fi{%
    \bxjs@let@hchar@chr@ue@b
  }{\bxjs@cond\ifnum\@tempcnta<\bxjs@let@hchar@cstd\fi{%
    \bxjs@let@hchar@chr@ue@c
  }{\bxjs@cond\ifnum\@tempcnta<\bxjs@let@hchar@cste\fi{%
    \bxjs@let@hchar@chr@ue@d
  }{%else
    \bxjs@let@hchar@out\let\relax
  }}}}}}
\def\bxjs@let@hchar@chr@ue@a#1{%
  \bxjs@let@hchar@out\def{{#1}}}
\def\bxjs@let@hchar@chr@ue@b#1#2{%
  \bxjs@let@hchar@out\def{{#1#2}}}
\def\bxjs@let@hchar@chr@ue@c#1#2#3{%
  \bxjs@let@hchar@out\def{{#1#2#3}}}
\def\bxjs@let@hchar@chr@ue@d#1#2#3#4{%
  \bxjs@let@hchar@out\def{{#1#2#3#4}}}
\ifx j\jsEngine
\def\bxjs@let@hchar@chr@pp#1{%
  \expandafter\bxjs@let@hchar@chr@pp@a\meaning#1\relax#1}
\def\bxjs@let@hchar@chr@pp@a#1#2\relax#3{%
  \bxjs@cond\if#1t\fi{%
    \bxjs@let@hchar@chr@ue#3%
  }{%else
    \bxjs@let@hchar@out\def{{#3}}%
  }}
\let\bxjs@let@hchar@chr\bxjs@let@hchar@chr@pp
\edef\jsc@JYn{\ifjsWithupTeX JY2\else JY1\fi}
\edef\jsc@JTn{\ifjsWithupTeX JT2\else JT1\fi}
\edef\jsc@pfx@{\ifjsWithupTeX u\fi}
\@onlypreamble\bxjs@declarefontshape
\ifjsWithupTeX
\def\bxjs@declarefontshape{%
\DeclareFontShape{JY2}{mc}{m}{n}{<->s*[\bxjs@scale]upjpnrm-h}{}%
\DeclareFontShape{JY2}{gt}{m}{n}{<->s*[\bxjs@scale]upjpngt-h}{}%
\DeclareFontShape{JT2}{mc}{m}{n}{<->s*[\bxjs@scale]upjpnrm-v}{}%
\DeclareFontShape{JT2}{gt}{m}{n}{<->s*[\bxjs@scale]upjpngt-v}{}%
}
\def\bxjs@sizereference{upjisr-h}
\else
\def\bxjs@declarefontshape{%
\DeclareFontShape{JY1}{mc}{m}{n}{<->s*[\bxjs@scale]jis}{}%
\DeclareFontShape{JY1}{gt}{m}{n}{<->s*[\bxjs@scale]jisg}{}%
\DeclareFontShape{JT1}{mc}{m}{n}{<->s*[\bxjs@scale]tmin10}{}%
\DeclareFontShape{JT1}{gt}{m}{n}{<->s*[\bxjs@scale]tgoth10}{}%
}
\def\bxjs@sizereference{jis}
\fi
\def\bxjs@tmpa#1/#2/#3/#4/#5\relax{%
  \def\bxjs@y{#5}}
\ifjsWithpTeXng \def\bxjs@y{10}%
\else
\expandafter\expandafter\expandafter\bxjs@tmpa
 \expandafter\string\the\jfont\relax
\fi
\@for\bxjs@x:={\jsc@JYn/mc/m/n,\jsc@JYn/gt/m/n,%
               \jsc@JTn/mc/m/n,\jsc@JTn/gt/m/n}\do
  {\expandafter\let\csname\bxjs@x/10\endcsname=\@undefined
   \expandafter\let\csname\bxjs@x/\bxjs@y\endcsname=\@undefined}
\begingroup
  \font\bxjs@tmpa=\bxjs@sizereference\space at 10pt
  \setbox\z@\hbox{\bxjs@tmpa\char\jis"2121\relax}
  \ifdim\wd\z@=10pt
    \global\let\bxjs@scale\jsScale
  \else
    \edef\bxjs@tmpa{\strip@pt\wd\z@}
    \@tempdima=10pt \@tempdima=\jsScale\@tempdima
    \bxjs@invscale\@tempdima\bxjs@tmpa
    \xdef\bxjs@scale{\strip@pt\@tempdima}
  \fi
\endgroup
\bxjs@declarefontshape
\DeclareFontShape{\jsc@JYn}{mc}{m}{it}{<->ssub*mc/m/n}{}
\DeclareFontShape{\jsc@JYn}{mc}{m}{sl}{<->ssub*mc/m/n}{}
\DeclareFontShape{\jsc@JYn}{mc}{m}{sc}{<->ssub*mc/m/n}{}
\DeclareFontShape{\jsc@JYn}{gt}{m}{it}{<->ssub*gt/m/n}{}
\DeclareFontShape{\jsc@JYn}{gt}{m}{sl}{<->ssub*gt/m/n}{}
\DeclareFontShape{\jsc@JYn}{mc}{bx}{it}{<->ssub*gt/m/n}{}
\DeclareFontShape{\jsc@JYn}{mc}{bx}{sl}{<->ssub*gt/m/n}{}
\DeclareFontShape{\jsc@JTn}{mc}{m}{it}{<->ssub*mc/m/n}{}
\DeclareFontShape{\jsc@JTn}{mc}{m}{sl}{<->ssub*mc/m/n}{}
\DeclareFontShape{\jsc@JTn}{mc}{m}{sc}{<->ssub*mc/m/n}{}
\DeclareFontShape{\jsc@JTn}{gt}{m}{it}{<->ssub*gt/m/n}{}
\DeclareFontShape{\jsc@JTn}{gt}{m}{sl}{<->ssub*gt/m/n}{}
\DeclareFontShape{\jsc@JTn}{mc}{bx}{it}{<->ssub*gt/m/n}{}
\DeclareFontShape{\jsc@JTn}{mc}{bx}{sl}{<->ssub*gt/m/n}{}
\DeclareRobustCommand\rmfamily
  {\not@math@alphabet\rmfamily\mathrm
   \romanfamily\rmdefault\kanjifamily\mcdefault\selectfont}
\DeclareRobustCommand\sffamily
  {\not@math@alphabet\sffamily\mathsf
   \romanfamily\sfdefault\kanjifamily\gtdefault\selectfont}
\DeclareRobustCommand\ttfamily
  {\not@math@alphabet\ttfamily\mathtt
   \romanfamily\ttdefault\kanjifamily\gtdefault\selectfont}
\DeclareJaTextFontCommand{\textmc}{\mcfamily}
\DeclareJaTextFontCommand{\textgt}{\gtfamily}
\bxjs@if@sf@default{%
  \renewcommand\kanjifamilydefault{\gtdefault}}
\selectfont
\prebreakpenalty\jis"2147=10000
\postbreakpenalty\jis"2148=10000
\prebreakpenalty\jis"2149=10000
\inhibitxspcode`!=1
\inhibitxspcode`〒=2
\xspcode`+=3
\xspcode`\%=3
\@tempcnta="80 \@whilenum\@tempcnta<"100 \do{%
  \xspcode\@tempcnta=3\advance\@tempcnta\@ne}
\let\jsInhibitGlueAtParTop\@inhibitglue
\begingroup
\catcode`\!=0
\gdef\bxjs@ptex@dir{%
  !iftdir t%
  !else!ifydir y%
  !else ?%
  !fi!fi}
\long\def\bxjs@tmpa{\hbox{%
  !ifydir \@textsuperscript{\normalfont\@thefnmark}%
  !else\hbox{\yoko\@textsuperscript{\normalfont\@thefnmark}}!fi}}
\ifx\@makefnmark\bxjs@tmpa
\long\gdef\@makefnmark{%
  !ifydir \hbox{}\hbox{\@textsuperscript{\normalfont\@thefnmark}}\hbox{}%
  !else\hbox{\yoko\@textsuperscript{\normalfont\@thefnmark}}!fi}
\fi
\endgroup
\else\ifx p\jsEngine
\let\bxjs@let@hchar@chr\bxjs@let@hchar@chr@ue
\@onlypreamble\bxjs@cjk@loaded
\def\bxjs@cjk@loaded{%
  \def\@footnotemark{%
    \leavevmode
    \ifhmode
      \edef\@x@sf{\the\spacefactor}%
      \ifdim\lastkern>\z@\ifdim\lastkern<5sp\relax
         \unkern\unkern
         \ifdim\lastskip>\z@ \unskip \fi
      \fi\fi
      \nobreak
    \fi
    \@makefnmark
    \ifhmode \spacefactor\@x@sf \fi
    \relax}%
  \let\bxjs@cjk@loaded\relax
}
\AtBeginDocument{%
  \@ifpackageloaded{CJK}{%
    \bxjs@cjk@loaded
  }{}%
}
\else\ifx x\jsEngine
\def\bxjs@let@hchar@chr#1{%
  \@tempcnta`#1\relax \divide\@tempcnta"800\relax
  \bxjs@cond\ifnum\@tempcnta=27 \fi{%
    \bxjs@let@hchar@chr@xe
  }{\bxjs@let@hchar@out\def{{#1}}}}
\def\bxjs@let@hchar@chr@xe#1{%
  \lccode`0=`#1\relax
  \lowercase{\bxjs@let@hchar@out\def{{0}}}}
\ifx\XeTeXgenerateactualtext\@undefined\else
  \def\bxjs@do@precisetext{%
    \XeTeXgenerateactualtext=\@ne}
\fi
\@onlypreamble\bxjs@do@simplejasetup
\def\bxjs@do@simplejasetup{%
  \ifnum\XeTeXinterchartokenstate>\z@
  \else\ifnum\strcmp{\the\XeTeXlinebreakskip}{\the\z@}=\z@
    \jsSimpleJaSetup
    \ClassInfo\bxjs@clsname
     {'\string\jsSimpleJaSetup' is applied\@gobble}%
  \fi\fi}
\newcommand*{\jsSimpleJaSetup}{%
  \XeTeXlinebreaklocale "ja"\relax
  \XeTeXlinebreakskip=0pt plus 1pt minus 0.1pt\relax
  \XeTeXlinebreakpenalty=0\relax}
\fi\fi\fi
\ifx\bxjs@do@simplejasetup\@undefined\else
  \AtBeginDocument{%
    \ifbxjs@simplejasetup
      \bxjs@do@simplejasetup
    \fi}
\fi
\ifbxjs@precisetext
  \ifx\bxjs@do@precisetext\@undefined
    \ClassWarning\bxjs@clsname
     {The current engine does not supprt the\MessageBreak
      'precisetext' option\@gobble}
  \else
    \bxjs@do@precisetext
  \fi
\fi
\ifbxjs@fancyhdr
\@onlypreamble\bxjs@adjust@fancyhdr
\def\bxjs@adjust@fancyhdr{%
  \def\bxjs@tmpa{\fancyplain{}{\sl\rightmark}\strut}%
  \def\bxjs@tmpb{\fancyplain{}{\rightmark}\strut}%
  \ifx\f@ncyelh\bxjs@tmpa \global\let\f@ncyelh\bxjs@tmpb \fi
  \ifx\f@ncyerh\bxjs@tmpa \global\let\f@ncyerh\bxjs@tmpb \fi
  \ifx\f@ncyolh\bxjs@tmpa \global\let\f@ncyolh\bxjs@tmpb \fi
  \ifx\f@ncyorh\bxjs@tmpa \global\let\f@ncyorh\bxjs@tmpb \fi
  \def\bxjs@tmpa{\fancyplain{}{\sl\leftmark}\strut}%
  \def\bxjs@tmpb{\fancyplain{}{\leftmark}\strut}%
  \ifx\f@ncyelh\bxjs@tmpa \global\let\f@ncyelh\bxjs@tmpb \fi
  \ifx\f@ncyerh\bxjs@tmpa \global\let\f@ncyerh\bxjs@tmpb \fi
  \ifx\f@ncyolh\bxjs@tmpa \global\let\f@ncyolh\bxjs@tmpb \fi
  \ifx\f@ncyorh\bxjs@tmpa \global\let\f@ncyorh\bxjs@tmpb \fi
  \def\bxjs@tmpa{\rm\thepage\strut}%
  \def\bxjs@tmpb{\thepage\strut}%
  \ifx\f@ncyecf\bxjs@tmpa \global\let\f@ncyecf\bxjs@tmpb \fi
  \ifx\f@ncyocf\bxjs@tmpa \global\let\f@ncyocf\bxjs@tmpb \fi
  \ifx\fullwidth\@undefined\else \ifdim\textwidth<\fullwidth
    \setlength{\@tempdima}{\fullwidth-\textwidth}%
    \edef\bxjs@tmpa{\noexpand\fancyhfoffset[EL,OR]{\the\@tempdima}%
    }\bxjs@tmpa
  \fi\fi
  \PackageInfo\bxjs@clsname
   {Patch to fancyhdr is applied\@gobble}}
\def\bxjs@pagestyle@hook{%
  \@ifpackageloaded{fancyhdr}{%
    \bxjs@adjust@fancyhdr
    \global\let\bxjs@adjust@fancyhdr\relax
  }{}}
\let\bxjs@org@pagestyle\pagestyle
\def\pagestyle{%
  \bxjs@pagestyle@hook \bxjs@org@pagestyle}
\AtBeginDocument{%
  \bxjs@pagestyle@hook
  \global\let\bxjs@pagestyle@hook\relax}
\fi
\endinput
%%
%% End of file `bxjsja-minimal.def'.

\bxjs@simplejasetupfalse
\ifjsWitheTeX
  \@tempdima=0.25mm
  \protected\edef\jQ{\dimexpr\the\@tempdima\relax}
  \let\jH\jQ
  \ifjsc@mag
    \@tempdimb=\jsBaseFontSize\relax
    \edef\bxjs@tmpa{\strip@pt\@tempdimb}%
    \@tempdima=2.5mm
    \bxjs@invscale\@tempdima\bxjs@tmpa
    \protected\edef\trueQ{\dimexpr\the\@tempdima\relax}
    \@tempdima=10pt
    \bxjs@invscale\@tempdima\bxjs@tmpa
    \protected\edef\bxjs@truept{\dimexpr\the\@tempdima\relax}
  \else \let\trueQ\jQ \let\bxjs@truept\p@
  \fi
  \let\trueH\trueQ
  \@tempdima\trueQ \bxjs@invscale\@tempdima\jsScale
  \protected\edef\ascQ{\dimexpr\the\@tempdima\relax}
  \@tempdima\bxjs@truept \bxjs@invscale\@tempdima\jsScale
  \protected\edef\ascpt{\dimexpr\the\@tempdima\relax}
\fi
\def\bxjs@kanjiskip{0pt}
\newcommand*\setkanjiskip[1]{%
  \edef\bxjs@kanjiskip{#1}%
  \bxjs@reset@kanjiskip}
\newcommand*\getkanjiskip{%
  \bxjs@kanjiskip}
\newif\ifbxjs@kanjiskip@enabled \bxjs@kanjiskip@enabledtrue
\bxjs@robust@def\bxjs@enable@kanjiskip{%
  \bxjs@kanjiskip@enabledtrue
  \bxjs@reset@kanjiskip}
\bxjs@robust@def\bxjs@disable@kanjiskip{%
  \bxjs@kanjiskip@enabledfalse
  \bxjs@reset@kanjiskip}
\bxjs@robust@def\bxjs@reset@kanjiskip{%
  \ifbxjs@kanjiskip@enabled
    \setlength{\@tempskipa}{\bxjs@kanjiskip}%
  \else \@tempskipa\z@
  \fi
  \bxjs@apply@kanjiskip}
\def\bxjs@xkanjiskip{0pt}
\newcommand*\setxkanjiskip[1]{%
  \edef\bxjs@xkanjiskip{#1}%
  \bxjs@reset@xkanjiskip}
\newcommand*\getxkanjiskip{%
  \bxjs@xkanjiskip}
\newif\ifbxjs@xkanjiskip@enabled \bxjs@xkanjiskip@enabledtrue
\bxjs@robust@def\bxjs@enable@xkanjiskip{%
  \bxjs@xkanjiskip@enabledtrue
  \bxjs@reset@xkanjiskip}
\bxjs@robust@def\bxjs@disable@xkanjiskip{%
  \bxjs@xkanjiskip@enabledfalse
  \bxjs@reset@xkanjiskip}
\bxjs@robust@def\bxjs@reset@xkanjiskip{%
  \ifbxjs@xkanjiskip@enabled
    \setlength{\@tempskipa}{\bxjs@xkanjiskip}%
  \else \@tempskipa\z@
  \fi
  \bxjs@apply@xkanjiskip}
\g@addto@macro\jsResetDimen{%
  \bxjs@reset@kanjiskip
  \bxjs@reset@xkanjiskip}
\let\bxjs@apply@kanjiskip\relax
\let\bxjs@apply@xkanjiskip\relax
\@onlypreamble\bxjs@adjust@jafont
\def\bxjs@adjust@jafont#1{%
  \ifx\jsJaFont\bxjs@@auto
    \bxjs@get@kanjiEmbed
    \ifx\bxjs@kanjiEmbed\relax
      \let\bxjs@tmpa\@empty
    \else
      \let\bxjs@tmpa\bxjs@kanjiEmbed
    \fi
  \else
    \let\bxjs@tmpa\jsJaFont
  \fi
  \if f#1\ifx\bxjs@tmpa\bxjs@@noEmbed
    \ClassWarningNoLine\bxjs@clsname
     {Option 'jafont=noEmbed' is ignored, because it is\MessageBreak
      not available on the current situation}%
    \let\bxjs@tmpa\@empty
  \fi\fi
}
\def\bxjs@@auto{auto}
\def\bxjs@@noEmbed{noEmbed}
\let\bxjs@kanjiEmbed\relax
\@onlypreamble\bxjs@get@kanjiEmbed
\def\bxjs@get@kanjiEmbed{%
  \begingroup\setbox\z@=\hbox{%
    \global\let\bxjs@g@tmpa\relax
    \endlinechar\m@ne
    \let\do\@makeother\dospecials
    \catcode32=10 \catcode12=10 %form-feed
    \let\bxjs@tmpa\@empty
    \openin\@inputcheck="|kpsewhich updmap.cfg"\relax
    \ifeof\@inputcheck\else
      \read\@inputcheck to\bxjs@tmpa
      \closein\@inputcheck
    \fi
    \ifx\bxjs@tmpa\@empty\else
      \openin\@inputcheck="\bxjs@tmpa"\relax
      \@tempswatrue
      \loop\if@tempswa
        \read\@inputcheck to\bxjs@tmpa
        \expandafter\bxjs@get@ke@a\bxjs@tmpa\@nil kanjiEmbed \@nil\@nnil
        \ifx\bxjs@tmpa\relax\else
          \global\let\bxjs@g@tmpa\bxjs@tmpa
          \@tempswafalse
        \fi
        \ifeof\@inputcheck \@tempswafalse \fi
      \repeat
    \fi
  }\endgroup
  \let\bxjs@kanjiEmbed\bxjs@g@tmpa
}
\@onlypreamble\bxjs@get@ke@a
\def\bxjs@get@ke@a#1kanjiEmbed #2\@nil#3\@nnil{%
  \ifx$#1$\def\bxjs@tmpa{#2}%
  \else \let\bxjs@tmpa\relax
  \fi}
\newcommand*\jachar[1]{%
  \begingroup
    \jsLetHeadChar\bxjs@tmpa{#1}%
    \ifx\bxjs@tmpa\relax
      \ClassWarningNoLine\bxjs@clsname
        {Illegal argument given to \string\jachar}%
    \else
      \expandafter\bxjs@jachar\expandafter{\bxjs@tmpa}%
    \fi
  \endgroup}
\let\jsJaChar\jachar
\let\bxjs@jachar\@firstofone
\PassOptionsToPackage{setpagesize=false}{hyperref}
\@onlypreamble\bxjs@fix@hyperref@unicode
\def\bxjs@fix@hyperref@unicode#1{%
  \PassOptionsToPackage{bxjs/hook=#1}{hyperref}%
  \@namedef{KV@Hyp@bxjs/hook}##1{%
    \KV@Hyp@unicode{##1}%
    \def\KV@Hyp@unicode####1{%
      \expandafter\ifx\csname if##1\expandafter\endcsname
         \csname if####1\endcsname\else
        \ClassWarningNoLine\bxjs@clsname
        {Blcoked hyperref option 'unicode=####1'}%
      \fi
    }%
  }%
}
\@onlypreamble\bxjs@urgent@special
\def\bxjs@urgent@special#1{%
  \AtBeginDvi{\special{#1}}%
  \AtBeginDocument{%
    \@ifpackageloaded{atbegshi}{%
      \begingroup
        \toks\z@{\special{#1}}%
        \toks\tw@\expandafter{\AtBegShi@HookFirst}%
        \xdef\AtBegShi@HookFirst{\the\toks@\the\toks\tw@}%
      \endgroup
    }{}%
  }%
}
\if j\jsEngine
\def\bxjs@apply@kanjiskip{%
  \kanjiskip\@tempskipa}
\def\bxjs@apply@xkanjiskip{%
  \xkanjiskip\@tempskipa}
\def\bxjs@jachar#1{%
  \bxjs@jachar@a#1....\@nil}
\def\bxjs@jachar@a#1#2#3#4#5\@nil{%
  \ifx.#2#1%
  \else\ifx.#3%
    \@tempcnta`#1 \multiply\@tempcnta64
    \advance\@tempcnta`#2 \advance\@tempcnta-"3080
    \bxjs@jachar@b
  \else\ifx.#4%
    \@tempcnta`#1 \multiply\@tempcnta64
    \advance\@tempcnta`#2 \multiply\@tempcnta64
    \advance\@tempcnta`#3 \advance\@tempcnta-"E2080
    \bxjs@jachar@b
  \else
    \@tempcnta`#1 \multiply\@tempcnta64
    \advance\@tempcnta`#2 \multiply\@tempcnta64
    \advance\@tempcnta`#3 \multiply\@tempcnta64
    \advance\@tempcnta`#4 \advance\@tempcnta-"3C82080
    \bxjs@jachar@b
  \fi\fi\fi}
\ifjsWithupTeX
  \def\bxjs@jachar@b{\kchar\@tempcnta}
\else
  \def\bxjs@jachar@b{%
    \ifx\bxUInt\@undefined\else
      \bxUInt{\@tempcnta}%
    \fi}
\fi
\let\bxjs@tmpa\jsJaFont
\ifx\bxjs@tmpa\bxjs@@auto
  \let\bxjs@tmpa\@empty
\else\ifx\bxjs@tmpa\bxjs@@noEmbed
  \def\bxjs@tmpa{noembed}
\fi\fi
\ifx\jsJaFont\@empty\else
  \edef\bxjs@nxt{%
    \noexpand\RequirePackage[\jsJaFont]
        {pxchfon}[2010/05/12]}% v0.5
  \bxjs@nxt
\fi
\begingroup
  \global\let\@gtempa\relax
  \catcode`\|=0 \catcode`\\=12
  |def|bxjs@check#1|@nil{%
    |bxjs@check@a#1|@nil\RequirePackage|@nnil}%
  |def|bxjs@check@a#1\RequirePackage#2|@nnil{%
    |ifx$#1$|bxjs@check@b#2|@nil keyval|@nnil |fi}%
  |catcode`|\=0 \catcode`\|=12
  \def\bxjs@check@b#1keyval#2\@nnil{%
    \ifx$#2$\else
      \xdef\@gtempa{%
        \noexpand\PassOptionsToPackage{scale=\jsScale}{otf}}%
    \fi}
\@firstofone{%
  \catcode10=12 \endlinechar\m@ne
  \let\do\@makeother \dospecials \catcode32=10
  \openin\@inputcheck=otf.sty\relax
  \@tempswatrue
  \loop\if@tempswa
    \ifeof\@inputcheck \@tempswafalse \fi
    \if@tempswa
      \read\@inputcheck to\bxjs@line
      \expandafter\bxjs@check\bxjs@line\@nil
    \fi
  \repeat
  \closein\@inputcheck
\endgroup}
\@gtempa
\ifbxjs@hyperref@enc
  \bxjs@fix@hyperref@unicode{false}
\fi
\if \ifx\bxjs@driver@given\bxjs@driver@@dvipdfmx T%
    \else\ifjsWithpTeXng T\else F\fi\fi T%
  \ifnum\jis"2121="A1A1 %euc
    \bxjs@urgent@special{pdf:tounicode EUC-UCS2}
  \else\ifnum\jis"2121="8140 %sjis
    \bxjs@urgent@special{pdf:tounicode 90ms-RKSJ-UCS2}
  \else\ifnum\jis"2121="3000 %uptex
    \ifbxjs@bigcode
      \bxjs@urgent@special{pdf:tounicode UTF8-UTF16}
      \PassOptionsToPackage{bigcode}{pxjahyper}
    \else
      \bxjs@urgent@special{pdf:tounicode UTF8-UCS2}
    \fi
  \fi\fi\fi
  \let\bxToUnicodeSpecialDone=t
\fi
\ifx f\bxjs@enablejfam\else
  \@enablejfamtrue
\fi
\if@enablejfam
  \DeclareSymbolFont{mincho}{\jsc@JYn}{mc}{m}{n}
  \DeclareSymbolFontAlphabet{\mathmc}{mincho}
  \SetSymbolFont{mincho}{bold}{\jsc@JYn}{gt}{m}{n}
  \jfam\symmincho
  \DeclareMathAlphabet{\mathgt}{\jsc@JYn}{gt}{m}{n}
  \AtBeginDocument{%
    \ifx\reDeclareMathAlphabet\@undefined\else
      \reDeclareMathAlphabet{\mathrm}{\@mathrm}{\@mathmc}%
      \reDeclareMathAlphabet{\mathbf}{\@mathbf}{\@mathgt}%
      \reDeclareMathAlphabet{\mathsf}{\@mathsf}{\@mathgt}%
    \fi}
\fi
\else\if p\jsEngine
\bxjs@adjust@jafont{f}
\edef\bxjs@nxt{%
  \noexpand\RequirePackage[%
      \ifx\bxjs@tmpa\@empty\else \bxjs@tmpa,\fi
      whole,autotilde]{bxcjkjatype}[2013/10/15]}% v0.2c
\bxjs@nxt
\bxjs@cjk@loaded
\ifbxjs@hyperref@enc
  \PassOptionsToPackage{unicode}{hyperref}
\fi
\ifx\bxcjkjatypeHyperrefPatchDone\@undefined
\begingroup
  \CJK@input{UTF8.bdg}
\endgroup
\g@addto@macro\pdfstringdefPreHook{%
  \@nameuse{CJK@UTF8Binding}%
}
\fi
\ifx\bxcjkjatypeHyperrefPatchDone\@undefined
\g@addto@macro\pdfstringdefPreHook{%
  \ifx~\bxjs@@CJKtilde
    \let\bxjs@org@LetUnexpandableSpace\HyPsd@LetUnexpandableSpace
    \let\HyPsd@LetUnexpandableSpace\bxjs@LetUnexpandableSpace
    \let~\@empty
  \fi
}
\def\bxjs@@CJKtilde{\CJKecglue\ignorespaces}
\def\bxjs@@tildecmd{~}
\def\bxjs@LetUnexpandableSpace#1{%
  \def\bxjs@tmpa{#1}\ifx\bxjs@tmpa\bxjs@@tildecmd\else
    \bxjs@org@LetUnexpandableSpace#1%
  \fi}
\fi
\newskip\jsKanjiSkip
\newskip\jsXKanjiSkip
\ifx\CJKecglue\@undefined
  \def\CJKtilde{\CJK@global\def~{\CJKecglue\ignorespaces}}
\fi
\let\autospacing\bxjs@enable@kanjiskip
\let\noautospacing\bxjs@disable@kanjiskip
\protected\def\bxjs@CJKglue{\hskip\jsKanjiSkip}
\def\bxjs@apply@kanjiskip{%
  \jsKanjiSkip\@tempskipa
  \let\CJKglue\bxjs@CJKglue}
\let\autoxspacing\bxjs@enable@xkanjiskip
\let\noautoxspacing\bxjs@disable@xkanjiskip
\protected\def\bxjs@CJKecglue{\hskip\jsXKanjiSkip}
\def\bxjs@apply@xkanjiskip{%
  \jsXKanjiSkip\@tempskipa
  \let\CJKecglue\bxjs@CJKecglue}
\def\bxjs@jachar#1{%
  \CJKforced{#1}}
\ifx t\bxjs@enablejfam
  \ClassWarningNoLine\bxjs@clsname
   {You cannot use 'enablejfam=true', since the\MessageBreak
    CJK package does not support Japanese math}
\fi
\else\if x\jsEngine
\RequirePackage{zxjatype}
\PassOptionsToPackage{no-math}{fontspec}%!
\PassOptionsToPackage{xetex}{graphicx}%!
\PassOptionsToPackage{xetex}{graphics}%!
\ifx\zxJaFamilyName\@undefined
  \ClassError\bxjs@clsname
  {xeCJK or zxjatype is too old}\@ehc
\fi
\bxjs@adjust@jafont{f}
\ifx\bxjs@tmpa\@empty
  \setCJKmainfont[BoldFont=IPAexGothic]{IPAexMincho}
  \setCJKsansfont[BoldFont=IPAexGothic]{IPAexGothic}
\else
  \edef\bxjs@nxt{%
    \noexpand\RequirePackage[\bxjs@tmpa]%
        {zxjafont}[2013/01/28]}% v0.2a
  \bxjs@nxt
\fi
\ifnum\strcmp{\the\XeTeXversion\XeTeXrevision}{0.99992}>\m@ne
  \ifbxjs@hyperref@enc
    \PassOptionsToPackage{unicode}{hyperref}
  \fi
\fi
\let\jsInhibitGlueAtParTop\@empty
\newskip\jsKanjiSkip
\newskip\jsXKanjiSkip
\ifx\CJKecglue\@undefined
  \def\CJKtilde{\CJK@global\def~{\CJKecglue\ignorespaces}}
\fi
\let\autospacing\bxjs@enable@kanjiskip
\let\noautospacing\bxjs@disable@kanjiskip
\protected\def\bxjs@CJKglue{\hskip\jsKanjiSkip}
\def\bxjs@apply@kanjiskip{%
  \jsKanjiSkip\@tempskipa
  \xeCJKsetup{CJKglue={\bxjs@CJKglue}}}
\let\autoxspacing\bxjs@enable@xkanjiskip
\let\noautoxspacing\bxjs@disable@xkanjiskip
\protected\def\bxjs@CJKecglue{\hskip\jsXKanjiSkip}
\def\bxjs@apply@xkanjiskip{%
  \jsXKanjiSkip\@tempskipa
  \xeCJKsetup{CJKecglue={\bxjs@CJKecglue}}}
\ifx\mcfamily\@undefined
  \protected\def\mcfamily{\CJKfamily{\CJKrmdefault}}
  \protected\def\gtfamily{\CJKfamily{\CJKsfdefault}}
\fi
\def\bxjs@jachar#1{%
  \xeCJKDeclareCharClass{CJK}{`#1}\relax
  #1}
\ifx t\bxjs@enablejfam
  \@enablejfamtrue
\fi
\if@enablejfam
  \xeCJKsetup{CJKmath=true}
\fi
\else\if l\jsEngine
\let\zw\@undefined
\RequirePackage{luatexja}
\RequirePackage{luatexja-fontspec}
\ExplSyntaxOn
\fp_gset:Nn \g_ltj_fontspec_scale_fp { \jsScale }
\ExplSyntaxOff
\bxjs@adjust@jafont{t}
\ifx\bxjs@tmpa\bxjs@@noEmbed
  \def\bxjs@tmpa{noembed}
\fi
\ifx\bxjs@tmpa\@empty
  \defaultjfontfeatures{ Kerning=Off }
  \setmainjfont[BoldFont=IPAexGothic,JFM=ujis]{IPAexMincho}
  \setsansjfont[BoldFont=IPAexGothic,JFM=ujis]{IPAexGothic}
\else
  \edef\bxjs@nxt{%
    \noexpand\RequirePackage[\bxjs@tmpa]
        {luatexja-preset}}%
  \bxjs@nxt
\fi
\DeclareRobustCommand\rmfamily
  {\not@math@alphabet\rmfamily\mathrm
   \romanfamily\rmdefault\kanjifamily\mcdefault\selectfont}
\DeclareRobustCommand\sffamily
  {\not@math@alphabet\sffamily\mathsf
   \romanfamily\sfdefault\kanjifamily\gtdefault\selectfont}
\DeclareRobustCommand\ttfamily
  {\not@math@alphabet\ttfamily\mathtt
   \romanfamily\ttdefault\kanjifamily\gtdefault\selectfont}
\AtBeginDocument{%
  \reDeclareMathAlphabet{\mathrm}{\mathrm}{\mathmc}
  \reDeclareMathAlphabet{\mathbf}{\mathbf}{\mathgt}%
  \reDeclareMathAlphabet{\mathsf}{\mathsf}{\mathgt}}%
\bxjs@if@sf@default{%
  \renewcommand\kanjifamilydefault{\gtdefault}}
\ltjsetparameter{jaxspmode={`!,1}}
\ltjsetparameter{jaxspmode={`〒,2}}
\ltjsetparameter{alxspmode={`+,3}}
\ltjsetparameter{alxspmode={`\%,3}}
\protected\def\@inhibitglue{%
  \directlua{%
    luatexja.jfmglue.create_beginpar_node()}}
\let\bxjs@ltj@inhibitglue\@inhibitglue
\let\@@inhibitglue\@undefined
\ifbxjs@hyperref@enc
  \bxjs@fix@hyperref@unicode{true}
\fi
\protected\def\autospacing{%
  \ltjsetparameter{autospacing=true}}
\protected\def\noautospacing{%
  \ltjsetparameter{autospacing=false}}
\protected\def\autoxspacing{%
  \ltjsetparameter{autoxspacing=true}}
\protected\def\noautoxspacing{%
  \ltjsetparameter{autoxspacing=false}}
\def\bxjs@apply@kanjiskip{%
  \ltjsetparameter{kanjiskip={\@tempskipa}}}
\def\bxjs@apply@xkanjiskip{%
  \ltjsetparameter{xkanjiskip={\@tempskipa}}}
\def\bxjs@jachar#1{%
  \ltjjachar`#1\relax}
\ifx f\bxjs@enablejfam
  \ClassWarningNoLine\bxjs@clsname
   {You cannot use 'enablejfam=false', since the\MessageBreak
    LuaTeX-ja always provides Japanese math families}
\fi
\fi\fi\fi\fi
\DeclareJaTextFontCommand{\textmc}{\mcfamily}
\DeclareJaTextFontCommand{\textgt}{\gtfamily}
\ifx\mathmc\@undefined
  \DeclareJaMathFontCommand{\mathmc}{\mcfamily}
  \DeclareJaMathFontCommand{\mathgt}{\gtfamily}
\fi
\setkanjiskip{0pt plus.1\jsZw minus.01\jsZw}
\ifx\jsDocClass\jsSlide \setxkanjiskip{0.1em}
\else \setxkanjiskip{0.25em plus 0.15em minus 0.06em}
\fi
\endinput
%%
%% End of file `bxjsja-standard.def'.

%    \end{macrocode}
%
%^^A----------------
%\subsection{フォント設定}
% 
% T1エンコーディングに変更する。
% \Note 以下のコードは |\usepackage[T1]{fontenc}| と同等。
%    \begin{macrocode}
\ifnum0\if x\jsEngine1\fi\if l\jsEngine1\fi=\z@
\def\encodingdefault{T1}%
\input{t1enc.def}%
\fontencoding\encodingdefault\selectfont
\fi
%    \end{macrocode}
%
% 基本フォントをLatin Modernフォントファミリに変更する。
% \Note 以下は |\usepackage[noamth]{lmodern}| と同じ。
% ユーザは後で |lmodern| を好きなオプションを付けて読み込む
% ことができる。
%    \begin{macrocode}
\ifnum0\if x\jsEngine1\fi\if l\jsEngine1\fi=\z@
\renewcommand{\rmdefault}{lmr}
\renewcommand{\sfdefault}{lmss}
\renewcommand{\ttdefault}{lmtt}
\fi
%    \end{macrocode}
%
% 大型演算子用の数式フォントの設定。
% \Note |amsfonts| パッケージと同等にする。
%    \begin{macrocode}
\DeclareFontShape{OMX}{cmex}{m}{n}{%
  <-7.5>cmex7<7.5-8.5>cmex8%
  <8.5-9.5>cmex9<9.5->cmex10}{}%
\expandafter\let\csname OMX/cmex/m/n/10\endcsname\relax
%    \end{macrocode}
% |amsmath| 読込時に上書きされるのを防ぐ。
%    \begin{macrocode}
\def\cmex@opt{10}
%    \end{macrocode}
%
%^^A----------------
% \subsection{fixltx2e読込}
%
% \Note |fixltx2e| 廃止前の{\LaTeX}カーネルの場合。
%    \begin{macrocode}
\ifx\@IncludeInRelease\@undefined
\RequirePackage{fixltx2e}
\fi
%    \end{macrocode}
%
%^^A----------------
% \subsection{和文カテゴリコード}
%
% 和文カテゴリコード設定のための補助パッケージを読みこむ。
%    \begin{macrocode}
\RequirePackage{bxjscjkcat}
%    \end{macrocode}
%
%^^A----------------
% \subsection{完了}
% おしまい。
%    \begin{macrocode}
%</modern>
%    \end{macrocode}
%
%^^A========================================================
% \section{和文ドライバ:pandoc \ZRX}
%
% Pandoc用の何か。
%
% standardドライバの設定を引き継ぐ。
%    \begin{macrocode}
%<*pandoc>
%%
%% This is file `bxjsja-standard.def',
%% generated with the docstrip utility.
%%
%% The original source files were:
%%
%% bxjscls.dtx  (with options: `drv,standard')
%% 
%% IMPORTANT NOTICE:
%% 
%% For the copyright see the source file.
%% 
%% Any modified versions of this file must be renamed
%% with new filenames distinct from bxjsja-standard.def.
%% 
%% For distribution of the original source see the terms
%% for copying and modification in the file bxjscls.dtx.
%% 
%% This generated file may be distributed as long as the
%% original source files, as listed above, are part of the
%% same distribution. (The sources need not necessarily be
%% in the same archive or directory.)
\ProvidesFile{bxjsja-standard.def}
  [2017/03/25 v1.6-pre BXJS document classes]
%% このファイルは日本語文字を含みます
%%
%% This is file `bxjsja-minimal.def',
%% generated with the docstrip utility.
%%
%% The original source files were:
%%
%% bxjscls.dtx  (with options: `drv,minimal')
%% 
%% IMPORTANT NOTICE:
%% 
%% For the copyright see the source file.
%% 
%% Any modified versions of this file must be renamed
%% with new filenames distinct from bxjsja-minimal.def.
%% 
%% For distribution of the original source see the terms
%% for copying and modification in the file bxjscls.dtx.
%% 
%% This generated file may be distributed as long as the
%% original source files, as listed above, are part of the
%% same distribution. (The sources need not necessarily be
%% in the same archive or directory.)
%% \CharacterTable
%%  {Upper-case    \A\B\C\D\E\F\G\H\I\J\K\L\M\N\O\P\Q\R\S\T\U\V\W\X\Y\Z
%%   Lower-case    \a\b\c\d\e\f\g\h\i\j\k\l\m\n\o\p\q\r\s\t\u\v\w\x\y\z
%%   Digits        \0\1\2\3\4\5\6\7\8\9
%%   Exclamation   \!     Double quote  \"     Hash (number) \#
%%   Dollar        \$     Percent       \%     Ampersand     \&
%%   Acute accent  \'     Left paren    \(     Right paren   \)
%%   Asterisk      \*     Plus          \+     Comma         \,
%%   Minus         \-     Point         \.     Solidus       \/
%%   Colon         \:     Semicolon     \;     Less than     \<
%%   Equals        \=     Greater than  \>     Question mark \?
%%   Commercial at \@     Left bracket  \[     Backslash     \\
%%   Right bracket \]     Circumflex    \^     Underscore    \_
%%   Grave accent  \`     Left brace    \{     Vertical bar  \|
%%   Right brace   \}     Tilde         \~}
\ProvidesFile{bxjsja-minimal.def}
  [2016/08/31 v1.3-pre BXJS document classes]
%% このファイルは日本語文字を含みます
\def\DeclareJaTextFontCommand#1#2{%
  \DeclareRobustCommand#1[1]{%
    \relax
    \ifmmode \expandafter\nfss@text \fi
    {#2##1}}%
}
\long\def\bxjs@@CSsfdefault{\sfdefault}%
\@onlypreamble\bxjs@if@sf@default
\def\bxjs@if@sf@default#1{%
  \ifx\familydefault\bxjs@@CSsfdefault#1\fi
  \AtBeginDocument{%
    \ifx\familydefault\bxjs@@CSsfdefault#1\fi}%
}
\def\jsLetHeadChar#1#2{%
  \begingroup
    \escapechar=`\\ %
    \let\bxjs@tmpa={% brace-match-hack
    \bxjs@let@hchar@exp#2}%
  \endgroup
  \let#1\bxjs@g@tmpa}
\def\bxjs@let@hchar@exp{%
  \futurelet\@let@token\bxjs@let@hchar@exp@a}
\def\bxjs@let@hchar@exp@a{%
  \bxjs@cond\ifcat\noexpand\@let@token\bgroup\fi{% 波括弧
    \bxjs@let@hchar@out\let\relax
  }{\bxjs@cond\ifcat\noexpand\@let@token\@sptoken\fi{% 空白
    \bxjs@let@hchar@out\let\space%
  }{\bxjs@cond\if\noexpand\@let@token\@backslashchar\fi{% バックスラッシュ
    \bxjs@let@hchar@out\let\@backslashchar
  }{\bxjs@let@hchar@exp@b}}}}
\def\bxjs@let@hchar@exp@b#1{%
  \expandafter\bxjs@let@hchar@exp@c\string#1?\@nil#1}
\def\bxjs@let@hchar@exp@c#1#2\@nil{%
  \bxjs@cond\if#1\@backslashchar\fi{% 制御綴
    \bxjs@cond\expandafter\ifx\noexpand\@let@token\@let@token\fi{%
      \bxjs@let@hchar@out\let\relax
    }{%else
      \expandafter\bxjs@let@hchar@exp
    }%
  }{%else
    \bxjs@let@hchar@chr#1%
  }}
\def\bxjs@let@hchar@chr#1{%
  \bxjs@let@hchar@out\def{{#1}}}
\def\bxjs@let@hchar@out#1#2{%
  \global#1\bxjs@g@tmpa#2\relax
  \toks@\bgroup}% skip to right brace
\chardef\bxjs@let@hchar@csta=128
\chardef\bxjs@let@hchar@cstb=192
\chardef\bxjs@let@hchar@cstc=224
\chardef\bxjs@let@hchar@cstd=240
\chardef\bxjs@let@hchar@cste=248
\let\bxjs@let@hchar@chr@ue@a\bxjs@let@hchar@chr
\def\bxjs@let@hchar@chr@ue#1{%
  \@tempcnta=`#1\relax
  \bxjs@cond\ifnum\@tempcnta<\bxjs@let@hchar@csta\fi{%
    \bxjs@let@hchar@chr@ue@a#1%
  }{\bxjs@cond\ifnum\@tempcnta<\bxjs@let@hchar@cstb\fi{%
    \bxjs@let@hchar@out\let\relax
  }{\bxjs@cond\ifnum\@tempcnta<\bxjs@let@hchar@cstc\fi{%
    \bxjs@let@hchar@chr@ue@b
  }{\bxjs@cond\ifnum\@tempcnta<\bxjs@let@hchar@cstd\fi{%
    \bxjs@let@hchar@chr@ue@c
  }{\bxjs@cond\ifnum\@tempcnta<\bxjs@let@hchar@cste\fi{%
    \bxjs@let@hchar@chr@ue@d
  }{%else
    \bxjs@let@hchar@out\let\relax
  }}}}}}
\def\bxjs@let@hchar@chr@ue@a#1{%
  \bxjs@let@hchar@out\def{{#1}}}
\def\bxjs@let@hchar@chr@ue@b#1#2{%
  \bxjs@let@hchar@out\def{{#1#2}}}
\def\bxjs@let@hchar@chr@ue@c#1#2#3{%
  \bxjs@let@hchar@out\def{{#1#2#3}}}
\def\bxjs@let@hchar@chr@ue@d#1#2#3#4{%
  \bxjs@let@hchar@out\def{{#1#2#3#4}}}
\ifx j\jsEngine
\def\bxjs@let@hchar@chr@pp#1{%
  \expandafter\bxjs@let@hchar@chr@pp@a\meaning#1\relax#1}
\def\bxjs@let@hchar@chr@pp@a#1#2\relax#3{%
  \bxjs@cond\if#1t\fi{%
    \bxjs@let@hchar@chr@ue#3%
  }{%else
    \bxjs@let@hchar@out\def{{#3}}%
  }}
\let\bxjs@let@hchar@chr\bxjs@let@hchar@chr@pp
\edef\jsc@JYn{\ifjsWithupTeX JY2\else JY1\fi}
\edef\jsc@JTn{\ifjsWithupTeX JT2\else JT1\fi}
\edef\jsc@pfx@{\ifjsWithupTeX u\fi}
\@onlypreamble\bxjs@declarefontshape
\ifjsWithupTeX
\def\bxjs@declarefontshape{%
\DeclareFontShape{JY2}{mc}{m}{n}{<->s*[\bxjs@scale]upjpnrm-h}{}%
\DeclareFontShape{JY2}{gt}{m}{n}{<->s*[\bxjs@scale]upjpngt-h}{}%
\DeclareFontShape{JT2}{mc}{m}{n}{<->s*[\bxjs@scale]upjpnrm-v}{}%
\DeclareFontShape{JT2}{gt}{m}{n}{<->s*[\bxjs@scale]upjpngt-v}{}%
}
\def\bxjs@sizereference{upjisr-h}
\else
\def\bxjs@declarefontshape{%
\DeclareFontShape{JY1}{mc}{m}{n}{<->s*[\bxjs@scale]jis}{}%
\DeclareFontShape{JY1}{gt}{m}{n}{<->s*[\bxjs@scale]jisg}{}%
\DeclareFontShape{JT1}{mc}{m}{n}{<->s*[\bxjs@scale]tmin10}{}%
\DeclareFontShape{JT1}{gt}{m}{n}{<->s*[\bxjs@scale]tgoth10}{}%
}
\def\bxjs@sizereference{jis}
\fi
\def\bxjs@tmpa#1/#2/#3/#4/#5\relax{%
  \def\bxjs@y{#5}}
\ifjsWithpTeXng \def\bxjs@y{10}%
\else
\expandafter\expandafter\expandafter\bxjs@tmpa
 \expandafter\string\the\jfont\relax
\fi
\@for\bxjs@x:={\jsc@JYn/mc/m/n,\jsc@JYn/gt/m/n,%
               \jsc@JTn/mc/m/n,\jsc@JTn/gt/m/n}\do
  {\expandafter\let\csname\bxjs@x/10\endcsname=\@undefined
   \expandafter\let\csname\bxjs@x/\bxjs@y\endcsname=\@undefined}
\begingroup
  \font\bxjs@tmpa=\bxjs@sizereference\space at 10pt
  \setbox\z@\hbox{\bxjs@tmpa\char\jis"2121\relax}
  \ifdim\wd\z@=10pt
    \global\let\bxjs@scale\jsScale
  \else
    \edef\bxjs@tmpa{\strip@pt\wd\z@}
    \@tempdima=10pt \@tempdima=\jsScale\@tempdima
    \bxjs@invscale\@tempdima\bxjs@tmpa
    \xdef\bxjs@scale{\strip@pt\@tempdima}
  \fi
\endgroup
\bxjs@declarefontshape
\DeclareFontShape{\jsc@JYn}{mc}{m}{it}{<->ssub*mc/m/n}{}
\DeclareFontShape{\jsc@JYn}{mc}{m}{sl}{<->ssub*mc/m/n}{}
\DeclareFontShape{\jsc@JYn}{mc}{m}{sc}{<->ssub*mc/m/n}{}
\DeclareFontShape{\jsc@JYn}{gt}{m}{it}{<->ssub*gt/m/n}{}
\DeclareFontShape{\jsc@JYn}{gt}{m}{sl}{<->ssub*gt/m/n}{}
\DeclareFontShape{\jsc@JYn}{mc}{bx}{it}{<->ssub*gt/m/n}{}
\DeclareFontShape{\jsc@JYn}{mc}{bx}{sl}{<->ssub*gt/m/n}{}
\DeclareFontShape{\jsc@JTn}{mc}{m}{it}{<->ssub*mc/m/n}{}
\DeclareFontShape{\jsc@JTn}{mc}{m}{sl}{<->ssub*mc/m/n}{}
\DeclareFontShape{\jsc@JTn}{mc}{m}{sc}{<->ssub*mc/m/n}{}
\DeclareFontShape{\jsc@JTn}{gt}{m}{it}{<->ssub*gt/m/n}{}
\DeclareFontShape{\jsc@JTn}{gt}{m}{sl}{<->ssub*gt/m/n}{}
\DeclareFontShape{\jsc@JTn}{mc}{bx}{it}{<->ssub*gt/m/n}{}
\DeclareFontShape{\jsc@JTn}{mc}{bx}{sl}{<->ssub*gt/m/n}{}
\DeclareRobustCommand\rmfamily
  {\not@math@alphabet\rmfamily\mathrm
   \romanfamily\rmdefault\kanjifamily\mcdefault\selectfont}
\DeclareRobustCommand\sffamily
  {\not@math@alphabet\sffamily\mathsf
   \romanfamily\sfdefault\kanjifamily\gtdefault\selectfont}
\DeclareRobustCommand\ttfamily
  {\not@math@alphabet\ttfamily\mathtt
   \romanfamily\ttdefault\kanjifamily\gtdefault\selectfont}
\DeclareJaTextFontCommand{\textmc}{\mcfamily}
\DeclareJaTextFontCommand{\textgt}{\gtfamily}
\bxjs@if@sf@default{%
  \renewcommand\kanjifamilydefault{\gtdefault}}
\selectfont
\prebreakpenalty\jis"2147=10000
\postbreakpenalty\jis"2148=10000
\prebreakpenalty\jis"2149=10000
\inhibitxspcode`!=1
\inhibitxspcode`〒=2
\xspcode`+=3
\xspcode`\%=3
\@tempcnta="80 \@whilenum\@tempcnta<"100 \do{%
  \xspcode\@tempcnta=3\advance\@tempcnta\@ne}
\let\jsInhibitGlueAtParTop\@inhibitglue
\begingroup
\catcode`\!=0
\gdef\bxjs@ptex@dir{%
  !iftdir t%
  !else!ifydir y%
  !else ?%
  !fi!fi}
\long\def\bxjs@tmpa{\hbox{%
  !ifydir \@textsuperscript{\normalfont\@thefnmark}%
  !else\hbox{\yoko\@textsuperscript{\normalfont\@thefnmark}}!fi}}
\ifx\@makefnmark\bxjs@tmpa
\long\gdef\@makefnmark{%
  !ifydir \hbox{}\hbox{\@textsuperscript{\normalfont\@thefnmark}}\hbox{}%
  !else\hbox{\yoko\@textsuperscript{\normalfont\@thefnmark}}!fi}
\fi
\endgroup
\else\ifx p\jsEngine
\let\bxjs@let@hchar@chr\bxjs@let@hchar@chr@ue
\@onlypreamble\bxjs@cjk@loaded
\def\bxjs@cjk@loaded{%
  \def\@footnotemark{%
    \leavevmode
    \ifhmode
      \edef\@x@sf{\the\spacefactor}%
      \ifdim\lastkern>\z@\ifdim\lastkern<5sp\relax
         \unkern\unkern
         \ifdim\lastskip>\z@ \unskip \fi
      \fi\fi
      \nobreak
    \fi
    \@makefnmark
    \ifhmode \spacefactor\@x@sf \fi
    \relax}%
  \let\bxjs@cjk@loaded\relax
}
\AtBeginDocument{%
  \@ifpackageloaded{CJK}{%
    \bxjs@cjk@loaded
  }{}%
}
\else\ifx x\jsEngine
\def\bxjs@let@hchar@chr#1{%
  \@tempcnta`#1\relax \divide\@tempcnta"800\relax
  \bxjs@cond\ifnum\@tempcnta=27 \fi{%
    \bxjs@let@hchar@chr@xe
  }{\bxjs@let@hchar@out\def{{#1}}}}
\def\bxjs@let@hchar@chr@xe#1{%
  \lccode`0=`#1\relax
  \lowercase{\bxjs@let@hchar@out\def{{0}}}}
\ifx\XeTeXgenerateactualtext\@undefined\else
  \def\bxjs@do@precisetext{%
    \XeTeXgenerateactualtext=\@ne}
\fi
\@onlypreamble\bxjs@do@simplejasetup
\def\bxjs@do@simplejasetup{%
  \ifnum\XeTeXinterchartokenstate>\z@
  \else\ifnum\strcmp{\the\XeTeXlinebreakskip}{\the\z@}=\z@
    \jsSimpleJaSetup
    \ClassInfo\bxjs@clsname
     {'\string\jsSimpleJaSetup' is applied\@gobble}%
  \fi\fi}
\newcommand*{\jsSimpleJaSetup}{%
  \XeTeXlinebreaklocale "ja"\relax
  \XeTeXlinebreakskip=0pt plus 1pt minus 0.1pt\relax
  \XeTeXlinebreakpenalty=0\relax}
\fi\fi\fi
\ifx\bxjs@do@simplejasetup\@undefined\else
  \AtBeginDocument{%
    \ifbxjs@simplejasetup
      \bxjs@do@simplejasetup
    \fi}
\fi
\ifbxjs@precisetext
  \ifx\bxjs@do@precisetext\@undefined
    \ClassWarning\bxjs@clsname
     {The current engine does not supprt the\MessageBreak
      'precisetext' option\@gobble}
  \else
    \bxjs@do@precisetext
  \fi
\fi
\ifbxjs@fancyhdr
\@onlypreamble\bxjs@adjust@fancyhdr
\def\bxjs@adjust@fancyhdr{%
  \def\bxjs@tmpa{\fancyplain{}{\sl\rightmark}\strut}%
  \def\bxjs@tmpb{\fancyplain{}{\rightmark}\strut}%
  \ifx\f@ncyelh\bxjs@tmpa \global\let\f@ncyelh\bxjs@tmpb \fi
  \ifx\f@ncyerh\bxjs@tmpa \global\let\f@ncyerh\bxjs@tmpb \fi
  \ifx\f@ncyolh\bxjs@tmpa \global\let\f@ncyolh\bxjs@tmpb \fi
  \ifx\f@ncyorh\bxjs@tmpa \global\let\f@ncyorh\bxjs@tmpb \fi
  \def\bxjs@tmpa{\fancyplain{}{\sl\leftmark}\strut}%
  \def\bxjs@tmpb{\fancyplain{}{\leftmark}\strut}%
  \ifx\f@ncyelh\bxjs@tmpa \global\let\f@ncyelh\bxjs@tmpb \fi
  \ifx\f@ncyerh\bxjs@tmpa \global\let\f@ncyerh\bxjs@tmpb \fi
  \ifx\f@ncyolh\bxjs@tmpa \global\let\f@ncyolh\bxjs@tmpb \fi
  \ifx\f@ncyorh\bxjs@tmpa \global\let\f@ncyorh\bxjs@tmpb \fi
  \def\bxjs@tmpa{\rm\thepage\strut}%
  \def\bxjs@tmpb{\thepage\strut}%
  \ifx\f@ncyecf\bxjs@tmpa \global\let\f@ncyecf\bxjs@tmpb \fi
  \ifx\f@ncyocf\bxjs@tmpa \global\let\f@ncyocf\bxjs@tmpb \fi
  \ifx\fullwidth\@undefined\else \ifdim\textwidth<\fullwidth
    \setlength{\@tempdima}{\fullwidth-\textwidth}%
    \edef\bxjs@tmpa{\noexpand\fancyhfoffset[EL,OR]{\the\@tempdima}%
    }\bxjs@tmpa
  \fi\fi
  \PackageInfo\bxjs@clsname
   {Patch to fancyhdr is applied\@gobble}}
\def\bxjs@pagestyle@hook{%
  \@ifpackageloaded{fancyhdr}{%
    \bxjs@adjust@fancyhdr
    \global\let\bxjs@adjust@fancyhdr\relax
  }{}}
\let\bxjs@org@pagestyle\pagestyle
\def\pagestyle{%
  \bxjs@pagestyle@hook \bxjs@org@pagestyle}
\AtBeginDocument{%
  \bxjs@pagestyle@hook
  \global\let\bxjs@pagestyle@hook\relax}
\fi
\endinput
%%
%% End of file `bxjsja-minimal.def'.

\bxjs@simplejasetupfalse
\ifjsWitheTeX
  \@tempdima=0.25mm
  \protected\edef\jQ{\dimexpr\the\@tempdima\relax}
  \let\jH\jQ
  \ifjsc@mag
    \@tempdimb=\jsBaseFontSize\relax
    \edef\bxjs@tmpa{\strip@pt\@tempdimb}%
    \@tempdima=2.5mm
    \bxjs@invscale\@tempdima\bxjs@tmpa
    \protected\edef\trueQ{\dimexpr\the\@tempdima\relax}
    \@tempdima=10pt
    \bxjs@invscale\@tempdima\bxjs@tmpa
    \protected\edef\bxjs@truept{\dimexpr\the\@tempdima\relax}
  \else \let\trueQ\jQ \let\bxjs@truept\p@
  \fi
  \let\trueH\trueQ
  \@tempdima\trueQ \bxjs@invscale\@tempdima\jsScale
  \protected\edef\ascQ{\dimexpr\the\@tempdima\relax}
  \@tempdima\bxjs@truept \bxjs@invscale\@tempdima\jsScale
  \protected\edef\ascpt{\dimexpr\the\@tempdima\relax}
\fi
\def\bxjs@kanjiskip{0pt}
\newcommand*\setkanjiskip[1]{%
  \edef\bxjs@kanjiskip{#1}%
  \bxjs@reset@kanjiskip}
\newcommand*\getkanjiskip{%
  \bxjs@kanjiskip}
\newif\ifbxjs@kanjiskip@enabled \bxjs@kanjiskip@enabledtrue
\bxjs@robust@def\bxjs@enable@kanjiskip{%
  \bxjs@kanjiskip@enabledtrue
  \bxjs@reset@kanjiskip}
\bxjs@robust@def\bxjs@disable@kanjiskip{%
  \bxjs@kanjiskip@enabledfalse
  \bxjs@reset@kanjiskip}
\bxjs@robust@def\bxjs@reset@kanjiskip{%
  \ifbxjs@kanjiskip@enabled
    \setlength{\@tempskipa}{\bxjs@kanjiskip}%
  \else \@tempskipa\z@
  \fi
  \bxjs@apply@kanjiskip}
\def\bxjs@xkanjiskip{0pt}
\newcommand*\setxkanjiskip[1]{%
  \edef\bxjs@xkanjiskip{#1}%
  \bxjs@reset@xkanjiskip}
\newcommand*\getxkanjiskip{%
  \bxjs@xkanjiskip}
\newif\ifbxjs@xkanjiskip@enabled \bxjs@xkanjiskip@enabledtrue
\bxjs@robust@def\bxjs@enable@xkanjiskip{%
  \bxjs@xkanjiskip@enabledtrue
  \bxjs@reset@xkanjiskip}
\bxjs@robust@def\bxjs@disable@xkanjiskip{%
  \bxjs@xkanjiskip@enabledfalse
  \bxjs@reset@xkanjiskip}
\bxjs@robust@def\bxjs@reset@xkanjiskip{%
  \ifbxjs@xkanjiskip@enabled
    \setlength{\@tempskipa}{\bxjs@xkanjiskip}%
  \else \@tempskipa\z@
  \fi
  \bxjs@apply@xkanjiskip}
\g@addto@macro\jsResetDimen{%
  \bxjs@reset@kanjiskip
  \bxjs@reset@xkanjiskip}
\let\bxjs@apply@kanjiskip\relax
\let\bxjs@apply@xkanjiskip\relax
\@onlypreamble\bxjs@adjust@jafont
\def\bxjs@adjust@jafont#1{%
  \ifx\jsJaFont\bxjs@@auto
    \bxjs@get@kanjiEmbed
    \ifx\bxjs@kanjiEmbed\relax
      \let\bxjs@tmpa\@empty
    \else
      \let\bxjs@tmpa\bxjs@kanjiEmbed
    \fi
  \else
    \let\bxjs@tmpa\jsJaFont
  \fi
  \if f#1\ifx\bxjs@tmpa\bxjs@@noEmbed
    \ClassWarningNoLine\bxjs@clsname
     {Option 'jafont=noEmbed' is ignored, because it is\MessageBreak
      not available on the current situation}%
    \let\bxjs@tmpa\@empty
  \fi\fi
}
\def\bxjs@@auto{auto}
\def\bxjs@@noEmbed{noEmbed}
\let\bxjs@kanjiEmbed\relax
\@onlypreamble\bxjs@get@kanjiEmbed
\def\bxjs@get@kanjiEmbed{%
  \begingroup\setbox\z@=\hbox{%
    \global\let\bxjs@g@tmpa\relax
    \endlinechar\m@ne
    \let\do\@makeother\dospecials
    \catcode32=10 \catcode12=10 %form-feed
    \let\bxjs@tmpa\@empty
    \openin\@inputcheck="|kpsewhich updmap.cfg"\relax
    \ifeof\@inputcheck\else
      \read\@inputcheck to\bxjs@tmpa
      \closein\@inputcheck
    \fi
    \ifx\bxjs@tmpa\@empty\else
      \openin\@inputcheck="\bxjs@tmpa"\relax
      \@tempswatrue
      \loop\if@tempswa
        \read\@inputcheck to\bxjs@tmpa
        \expandafter\bxjs@get@ke@a\bxjs@tmpa\@nil kanjiEmbed \@nil\@nnil
        \ifx\bxjs@tmpa\relax\else
          \global\let\bxjs@g@tmpa\bxjs@tmpa
          \@tempswafalse
        \fi
        \ifeof\@inputcheck \@tempswafalse \fi
      \repeat
    \fi
  }\endgroup
  \let\bxjs@kanjiEmbed\bxjs@g@tmpa
}
\@onlypreamble\bxjs@get@ke@a
\def\bxjs@get@ke@a#1kanjiEmbed #2\@nil#3\@nnil{%
  \ifx$#1$\def\bxjs@tmpa{#2}%
  \else \let\bxjs@tmpa\relax
  \fi}
\newcommand*\jachar[1]{%
  \begingroup
    \jsLetHeadChar\bxjs@tmpa{#1}%
    \ifx\bxjs@tmpa\relax
      \ClassWarningNoLine\bxjs@clsname
        {Illegal argument given to \string\jachar}%
    \else
      \expandafter\bxjs@jachar\expandafter{\bxjs@tmpa}%
    \fi
  \endgroup}
\let\jsJaChar\jachar
\let\bxjs@jachar\@firstofone
\PassOptionsToPackage{setpagesize=false}{hyperref}
\@onlypreamble\bxjs@fix@hyperref@unicode
\def\bxjs@fix@hyperref@unicode#1{%
  \PassOptionsToPackage{bxjs/hook=#1}{hyperref}%
  \@namedef{KV@Hyp@bxjs/hook}##1{%
    \KV@Hyp@unicode{##1}%
    \def\KV@Hyp@unicode####1{%
      \expandafter\ifx\csname if##1\expandafter\endcsname
         \csname if####1\endcsname\else
        \ClassWarningNoLine\bxjs@clsname
        {Blcoked hyperref option 'unicode=####1'}%
      \fi
    }%
  }%
}
\@onlypreamble\bxjs@urgent@special
\def\bxjs@urgent@special#1{%
  \AtBeginDvi{\special{#1}}%
  \AtBeginDocument{%
    \@ifpackageloaded{atbegshi}{%
      \begingroup
        \toks\z@{\special{#1}}%
        \toks\tw@\expandafter{\AtBegShi@HookFirst}%
        \xdef\AtBegShi@HookFirst{\the\toks@\the\toks\tw@}%
      \endgroup
    }{}%
  }%
}
\if j\jsEngine
\def\bxjs@apply@kanjiskip{%
  \kanjiskip\@tempskipa}
\def\bxjs@apply@xkanjiskip{%
  \xkanjiskip\@tempskipa}
\def\bxjs@jachar#1{%
  \bxjs@jachar@a#1....\@nil}
\def\bxjs@jachar@a#1#2#3#4#5\@nil{%
  \ifx.#2#1%
  \else\ifx.#3%
    \@tempcnta`#1 \multiply\@tempcnta64
    \advance\@tempcnta`#2 \advance\@tempcnta-"3080
    \bxjs@jachar@b
  \else\ifx.#4%
    \@tempcnta`#1 \multiply\@tempcnta64
    \advance\@tempcnta`#2 \multiply\@tempcnta64
    \advance\@tempcnta`#3 \advance\@tempcnta-"E2080
    \bxjs@jachar@b
  \else
    \@tempcnta`#1 \multiply\@tempcnta64
    \advance\@tempcnta`#2 \multiply\@tempcnta64
    \advance\@tempcnta`#3 \multiply\@tempcnta64
    \advance\@tempcnta`#4 \advance\@tempcnta-"3C82080
    \bxjs@jachar@b
  \fi\fi\fi}
\ifjsWithupTeX
  \def\bxjs@jachar@b{\kchar\@tempcnta}
\else
  \def\bxjs@jachar@b{%
    \ifx\bxUInt\@undefined\else
      \bxUInt{\@tempcnta}%
    \fi}
\fi
\let\bxjs@tmpa\jsJaFont
\ifx\bxjs@tmpa\bxjs@@auto
  \let\bxjs@tmpa\@empty
\else\ifx\bxjs@tmpa\bxjs@@noEmbed
  \def\bxjs@tmpa{noembed}
\fi\fi
\ifx\jsJaFont\@empty\else
  \edef\bxjs@nxt{%
    \noexpand\RequirePackage[\jsJaFont]
        {pxchfon}[2010/05/12]}% v0.5
  \bxjs@nxt
\fi
\begingroup
  \global\let\@gtempa\relax
  \catcode`\|=0 \catcode`\\=12
  |def|bxjs@check#1|@nil{%
    |bxjs@check@a#1|@nil\RequirePackage|@nnil}%
  |def|bxjs@check@a#1\RequirePackage#2|@nnil{%
    |ifx$#1$|bxjs@check@b#2|@nil keyval|@nnil |fi}%
  |catcode`|\=0 \catcode`\|=12
  \def\bxjs@check@b#1keyval#2\@nnil{%
    \ifx$#2$\else
      \xdef\@gtempa{%
        \noexpand\PassOptionsToPackage{scale=\jsScale}{otf}}%
    \fi}
\@firstofone{%
  \catcode10=12 \endlinechar\m@ne
  \let\do\@makeother \dospecials \catcode32=10
  \openin\@inputcheck=otf.sty\relax
  \@tempswatrue
  \loop\if@tempswa
    \ifeof\@inputcheck \@tempswafalse \fi
    \if@tempswa
      \read\@inputcheck to\bxjs@line
      \expandafter\bxjs@check\bxjs@line\@nil
    \fi
  \repeat
  \closein\@inputcheck
\endgroup}
\@gtempa
\ifbxjs@hyperref@enc
  \bxjs@fix@hyperref@unicode{false}
\fi
\if \ifx\bxjs@driver@given\bxjs@driver@@dvipdfmx T%
    \else\ifjsWithpTeXng T\else F\fi\fi T%
  \ifnum\jis"2121="A1A1 %euc
    \bxjs@urgent@special{pdf:tounicode EUC-UCS2}
  \else\ifnum\jis"2121="8140 %sjis
    \bxjs@urgent@special{pdf:tounicode 90ms-RKSJ-UCS2}
  \else\ifnum\jis"2121="3000 %uptex
    \ifbxjs@bigcode
      \bxjs@urgent@special{pdf:tounicode UTF8-UTF16}
      \PassOptionsToPackage{bigcode}{pxjahyper}
    \else
      \bxjs@urgent@special{pdf:tounicode UTF8-UCS2}
    \fi
  \fi\fi\fi
  \let\bxToUnicodeSpecialDone=t
\fi
\ifx f\bxjs@enablejfam\else
  \@enablejfamtrue
\fi
\if@enablejfam
  \DeclareSymbolFont{mincho}{\jsc@JYn}{mc}{m}{n}
  \DeclareSymbolFontAlphabet{\mathmc}{mincho}
  \SetSymbolFont{mincho}{bold}{\jsc@JYn}{gt}{m}{n}
  \jfam\symmincho
  \DeclareMathAlphabet{\mathgt}{\jsc@JYn}{gt}{m}{n}
  \AtBeginDocument{%
    \ifx\reDeclareMathAlphabet\@undefined\else
      \reDeclareMathAlphabet{\mathrm}{\@mathrm}{\@mathmc}%
      \reDeclareMathAlphabet{\mathbf}{\@mathbf}{\@mathgt}%
      \reDeclareMathAlphabet{\mathsf}{\@mathsf}{\@mathgt}%
    \fi}
\fi
\else\if p\jsEngine
\bxjs@adjust@jafont{f}
\edef\bxjs@nxt{%
  \noexpand\RequirePackage[%
      \ifx\bxjs@tmpa\@empty\else \bxjs@tmpa,\fi
      whole,autotilde]{bxcjkjatype}[2013/10/15]}% v0.2c
\bxjs@nxt
\bxjs@cjk@loaded
\ifbxjs@hyperref@enc
  \PassOptionsToPackage{unicode}{hyperref}
\fi
\ifx\bxcjkjatypeHyperrefPatchDone\@undefined
\begingroup
  \CJK@input{UTF8.bdg}
\endgroup
\g@addto@macro\pdfstringdefPreHook{%
  \@nameuse{CJK@UTF8Binding}%
}
\fi
\ifx\bxcjkjatypeHyperrefPatchDone\@undefined
\g@addto@macro\pdfstringdefPreHook{%
  \ifx~\bxjs@@CJKtilde
    \let\bxjs@org@LetUnexpandableSpace\HyPsd@LetUnexpandableSpace
    \let\HyPsd@LetUnexpandableSpace\bxjs@LetUnexpandableSpace
    \let~\@empty
  \fi
}
\def\bxjs@@CJKtilde{\CJKecglue\ignorespaces}
\def\bxjs@@tildecmd{~}
\def\bxjs@LetUnexpandableSpace#1{%
  \def\bxjs@tmpa{#1}\ifx\bxjs@tmpa\bxjs@@tildecmd\else
    \bxjs@org@LetUnexpandableSpace#1%
  \fi}
\fi
\newskip\jsKanjiSkip
\newskip\jsXKanjiSkip
\ifx\CJKecglue\@undefined
  \def\CJKtilde{\CJK@global\def~{\CJKecglue\ignorespaces}}
\fi
\let\autospacing\bxjs@enable@kanjiskip
\let\noautospacing\bxjs@disable@kanjiskip
\protected\def\bxjs@CJKglue{\hskip\jsKanjiSkip}
\def\bxjs@apply@kanjiskip{%
  \jsKanjiSkip\@tempskipa
  \let\CJKglue\bxjs@CJKglue}
\let\autoxspacing\bxjs@enable@xkanjiskip
\let\noautoxspacing\bxjs@disable@xkanjiskip
\protected\def\bxjs@CJKecglue{\hskip\jsXKanjiSkip}
\def\bxjs@apply@xkanjiskip{%
  \jsXKanjiSkip\@tempskipa
  \let\CJKecglue\bxjs@CJKecglue}
\def\bxjs@jachar#1{%
  \CJKforced{#1}}
\ifx t\bxjs@enablejfam
  \ClassWarningNoLine\bxjs@clsname
   {You cannot use 'enablejfam=true', since the\MessageBreak
    CJK package does not support Japanese math}
\fi
\else\if x\jsEngine
\RequirePackage{zxjatype}
\PassOptionsToPackage{no-math}{fontspec}%!
\PassOptionsToPackage{xetex}{graphicx}%!
\PassOptionsToPackage{xetex}{graphics}%!
\ifx\zxJaFamilyName\@undefined
  \ClassError\bxjs@clsname
  {xeCJK or zxjatype is too old}\@ehc
\fi
\bxjs@adjust@jafont{f}
\ifx\bxjs@tmpa\@empty
  \setCJKmainfont[BoldFont=IPAexGothic]{IPAexMincho}
  \setCJKsansfont[BoldFont=IPAexGothic]{IPAexGothic}
\else
  \edef\bxjs@nxt{%
    \noexpand\RequirePackage[\bxjs@tmpa]%
        {zxjafont}[2013/01/28]}% v0.2a
  \bxjs@nxt
\fi
\ifnum\strcmp{\the\XeTeXversion\XeTeXrevision}{0.99992}>\m@ne
  \ifbxjs@hyperref@enc
    \PassOptionsToPackage{unicode}{hyperref}
  \fi
\fi
\let\jsInhibitGlueAtParTop\@empty
\newskip\jsKanjiSkip
\newskip\jsXKanjiSkip
\ifx\CJKecglue\@undefined
  \def\CJKtilde{\CJK@global\def~{\CJKecglue\ignorespaces}}
\fi
\let\autospacing\bxjs@enable@kanjiskip
\let\noautospacing\bxjs@disable@kanjiskip
\protected\def\bxjs@CJKglue{\hskip\jsKanjiSkip}
\def\bxjs@apply@kanjiskip{%
  \jsKanjiSkip\@tempskipa
  \xeCJKsetup{CJKglue={\bxjs@CJKglue}}}
\let\autoxspacing\bxjs@enable@xkanjiskip
\let\noautoxspacing\bxjs@disable@xkanjiskip
\protected\def\bxjs@CJKecglue{\hskip\jsXKanjiSkip}
\def\bxjs@apply@xkanjiskip{%
  \jsXKanjiSkip\@tempskipa
  \xeCJKsetup{CJKecglue={\bxjs@CJKecglue}}}
\ifx\mcfamily\@undefined
  \protected\def\mcfamily{\CJKfamily{\CJKrmdefault}}
  \protected\def\gtfamily{\CJKfamily{\CJKsfdefault}}
\fi
\def\bxjs@jachar#1{%
  \xeCJKDeclareCharClass{CJK}{`#1}\relax
  #1}
\ifx t\bxjs@enablejfam
  \@enablejfamtrue
\fi
\if@enablejfam
  \xeCJKsetup{CJKmath=true}
\fi
\else\if l\jsEngine
\let\zw\@undefined
\RequirePackage{luatexja}
\RequirePackage{luatexja-fontspec}
\ExplSyntaxOn
\fp_gset:Nn \g_ltj_fontspec_scale_fp { \jsScale }
\ExplSyntaxOff
\bxjs@adjust@jafont{t}
\ifx\bxjs@tmpa\bxjs@@noEmbed
  \def\bxjs@tmpa{noembed}
\fi
\ifx\bxjs@tmpa\@empty
  \defaultjfontfeatures{ Kerning=Off }
  \setmainjfont[BoldFont=IPAexGothic,JFM=ujis]{IPAexMincho}
  \setsansjfont[BoldFont=IPAexGothic,JFM=ujis]{IPAexGothic}
\else
  \edef\bxjs@nxt{%
    \noexpand\RequirePackage[\bxjs@tmpa]
        {luatexja-preset}}%
  \bxjs@nxt
\fi
\DeclareRobustCommand\rmfamily
  {\not@math@alphabet\rmfamily\mathrm
   \romanfamily\rmdefault\kanjifamily\mcdefault\selectfont}
\DeclareRobustCommand\sffamily
  {\not@math@alphabet\sffamily\mathsf
   \romanfamily\sfdefault\kanjifamily\gtdefault\selectfont}
\DeclareRobustCommand\ttfamily
  {\not@math@alphabet\ttfamily\mathtt
   \romanfamily\ttdefault\kanjifamily\gtdefault\selectfont}
\AtBeginDocument{%
  \reDeclareMathAlphabet{\mathrm}{\mathrm}{\mathmc}
  \reDeclareMathAlphabet{\mathbf}{\mathbf}{\mathgt}%
  \reDeclareMathAlphabet{\mathsf}{\mathsf}{\mathgt}}%
\bxjs@if@sf@default{%
  \renewcommand\kanjifamilydefault{\gtdefault}}
\ltjsetparameter{jaxspmode={`!,1}}
\ltjsetparameter{jaxspmode={`〒,2}}
\ltjsetparameter{alxspmode={`+,3}}
\ltjsetparameter{alxspmode={`\%,3}}
\protected\def\@inhibitglue{%
  \directlua{%
    luatexja.jfmglue.create_beginpar_node()}}
\let\bxjs@ltj@inhibitglue\@inhibitglue
\let\@@inhibitglue\@undefined
\ifbxjs@hyperref@enc
  \bxjs@fix@hyperref@unicode{true}
\fi
\protected\def\autospacing{%
  \ltjsetparameter{autospacing=true}}
\protected\def\noautospacing{%
  \ltjsetparameter{autospacing=false}}
\protected\def\autoxspacing{%
  \ltjsetparameter{autoxspacing=true}}
\protected\def\noautoxspacing{%
  \ltjsetparameter{autoxspacing=false}}
\def\bxjs@apply@kanjiskip{%
  \ltjsetparameter{kanjiskip={\@tempskipa}}}
\def\bxjs@apply@xkanjiskip{%
  \ltjsetparameter{xkanjiskip={\@tempskipa}}}
\def\bxjs@jachar#1{%
  \ltjjachar`#1\relax}
\ifx f\bxjs@enablejfam
  \ClassWarningNoLine\bxjs@clsname
   {You cannot use 'enablejfam=false', since the\MessageBreak
    LuaTeX-ja always provides Japanese math families}
\fi
\fi\fi\fi\fi
\DeclareJaTextFontCommand{\textmc}{\mcfamily}
\DeclareJaTextFontCommand{\textgt}{\gtfamily}
\ifx\mathmc\@undefined
  \DeclareJaMathFontCommand{\mathmc}{\mcfamily}
  \DeclareJaMathFontCommand{\mathgt}{\gtfamily}
\fi
\setkanjiskip{0pt plus.1\jsZw minus.01\jsZw}
\ifx\jsDocClass\jsSlide \setxkanjiskip{0.1em}
\else \setxkanjiskip{0.25em plus 0.15em minus 0.06em}
\fi
\endinput
%%
%% End of file `bxjsja-standard.def'.

%    \end{macrocode}
%
%^^A----------------
%\subsection{duploadシステム}
%
% パッケージが重複して読み込まれたときに“option clash”の
% 検査をスキップする。
% この時に何らかのコードを実行させることができる。
%
% \begin{macro}{\bxjs@set@dupload@proc}
% |\bxjs@set@dupload@proc{|\Meta{ファイル名}|}{|\Meta{定義本体}|}|
% 特定のファイルの読込が |\@filewithoptions| で指示されて、しかも
% そのファイルが読込済である場合に、オプション重複検査をスキップして、
% 代わりに\Meta{定義本体}のコードを実行する。
% このコード中で |#1| は渡されたオプション列のテキストに置換される。
%    \begin{macrocode}
\@onlypreamble\bxjs@set@dupload@proc
\def\bxjs@set@dupload@proc#1{%
  \expandafter\bxjs@set@dupload@proc@a\csname bxjs@dlp/#1\endcsname}
\@onlypreamble\bxjs@set@dupload@proc@a
\def\bxjs@set@dupload@proc@a#1{%
  \@onlypreamble#1\def#1##1}
%    \end{macrocode}
% \end{macro}
%
% \begin{macro}{\@if@ptions}
% |\@if@ptions| の再定義。
%    \begin{macrocode}
\@onlypreamble\bxjs@org@if@ptions
\let\bxjs@org@if@ptions\@if@ptions
\newif\ifbxjs@dlp
\def\@if@ptions#1#2#3{%
  \bxjs@dlpfalse
  \def\bxjs@tmpa{#1}\def\bxjs@tmpb{\@currext}%
  \ifx\bxjs@tmpa\bxjs@tmpb
    \expandafter\ifx\csname bxjs@dlp/#2.#1\endcsname\relax\else
      \bxjs@dlptrue \fi
  \fi
  \ifbxjs@dlp \expandafter\bxjs@do@dupload@proc
  \else \expandafter\bxjs@org@if@ptions
  \fi {#1}{#2}{#3}}
\AtBeginDocument{%
  \let\@if@ptions\bxjs@org@if@ptions}
\@onlypreamble\bxjs@do@dupload@proc
\def\bxjs@do@dupload@proc#1#2#3{%
  \csname bxjs@dlp/#2.#1\endcsname{#3}%
  \@firstoftwo}
%    \end{macrocode}
% \end{macro}
%
% \begin{macro}{\bxjs@mark@as@loaded}
% |\bxjs@mark@as@loaded{|\Meta{ファイル名}|}|\Means
% 特定のファイルに対して、(|\@filewithoptions| の処理に関して)
% 読込済であるとマークする。
%    \begin{macrocode}
\def\bxjs@mark@as@loaded#1{%
  \expandafter\bxjs@mal@a\csname ver@#1\endcsname{#1}}
\def\bxjs@mal@a#1#2{%
  \ifx#1\relax
    \def#1{2001/01/01}%
    \ClassInfo\bxjs@clsname
     {File '#2' marked as loaded\@gobble}%
  \fi}
%    \end{macrocode}
% \end{macro}
%
%^^A----------------
%\subsection{lang変数}
% |lang=ja| という言語指定が行われると、
% Pandocはこれに対応していないため
% 不完全なBabelやPolyglossiaの設定を出力してしまう。
% これを防ぐため、とりあえず両パッケージを無効化しておく。
%
%    \begin{macrocode}
\ifnum0\if x\jsEngine1\fi\if l\jsEngine1\fi>0
%    \end{macrocode}
% Polyglossiaについて。
%    \begin{macrocode}
\bxjs@mark@as@loaded{polyglossia.sty}
\bxjs@set@dupload@proc{polyglossia.sty}{%
  \ClassWarning\bxjs@clsname
   {Loading of polyglossia is blocked}}
\ifx\setmainlanguage\@undefined
\newcommand*\setmainlanguage[2][]{}
\newcommand*\setotherlanguage[2][]{%
  \ifcat_#2_\else
    \expandafter\let\csname #2\endcsname\@empty
    \expandafter\let\csname end#2\endcsname\@empty
    \expandafter\let\csname text#2\endcsname\@firstofone
  \fi}
\newcommand*\setotherlanguages[2][]{%
  \@for\bxjs@tmpa:={#2}\do{%
    \setotherlangauge{\bxjs@tmpa}}}
\fi
\else
%    \end{macrocode}
% Babelについて。
%    \begin{macrocode}
\bxjs@mark@as@loaded{babel.sty}
\bxjs@set@dupload@proc{babel.sty}{%
  \ClassWarning\bxjs@clsname
   {Loading of babel is blocked}}
\let\foreignlanguage\@secondoftwo
\let\otherlanguage\@gobble
\let\endotherlanguage\@empty
\fi
%    \end{macrocode}
%
%^^A----------------
%\subsection{geometry変数}
% |geometry| を“再度読み込んだ”場合に、
% そのパラメタで |\setpagelayout*| が呼ばれるようにする。
%
%    \begin{macrocode}
\bxjs@set@dupload@proc{geometry.sty}{%
  \setpagelayout*{#1}}
%    \end{macrocode}
%
%^^A----------------
%\subsection{CJKmainfont変数}
% Lua{\TeX}(+ Lua{TeX}-ja)の場合に CJKmainfont 変数が
% 指定された場合は |\setmainjfont| の指定にまわす。
%    \begin{macrocode}
\if l\jsEngine
  \bxjs@mark@as@loaded{xeCJK.sty}
  \providecommand*{\setCJKmainfont}{\setmainjfont}
\fi
%    \end{macrocode}
%
%^^A----------------
%\subsection{fixltx2eパッケージ}
% テンプレートでは |fixltx2e| パッケージを読み込むが、
% 最近(2015年版以降)の{\LaTeX}ではこれで警告が出る。
% これを抑止する。
%
% {\LaTeX}カーネルが新しい場合は |fixltx2e| を
% 読込済にする。
%    \begin{macrocode}
\ifx\@IncludeInRelease\@undefined\else
  \bxjs@mark@as@loaded{fixltx2e.sty}
\fi
%    \end{macrocode}
%
%^^A----------------
%\subsection{cmapパッケージ}
% エンジンが{(u)\pLaTeX}のときに |cmap| パッケージが
% 読み込まれるのを阻止する。
% (実際は警告が出るだけで無害であるが。)
%
%    \begin{macrocode}
\if j\jsEngine
  \bxjs@mark@as@loaded{cmap.sty}
\fi
%    \end{macrocode}
%
%^^A----------------
%\subsection{microtypeパッケージ}
% 警告が多すぎなので消す。
%
%    \begin{macrocode}
\PassOptionsToPackage{verbose=silent}{microtype}
%    \end{macrocode}
%
%^^A----------------
% \subsection{完了}
% おしまい。
%    \begin{macrocode}
%</pandoc>
%    \end{macrocode}
%
% 和文ドライバ実装はここまで。
%    \begin{macrocode}
%</drv>
%    \end{macrocode}
%
%^^A========================================================
% \section{補助パッケージ一覧 \ZRX}
%
% BXJSクラスの機能を実現するために用意されたものだが、
% 他のクラスの文書で読み込んで利用することもできる。
%
% \begin{itemize}
% \item bxjscjkcat: modernドライバ用の和文カテゴリを適用する。
% \end{itemize}
%
%    \begin{macrocode}
%<*anc>
%    \end{macrocode}
%
%^^A========================================================
% \section{補助パッケージ:bxjscompat \ZRX}
%
% ムニャムニャムニャ……。
%
%^^A----------------
% \subsection{準備}
%
%    \begin{macrocode}
%<*compat>
\def\bxac@pkgname{bxjscompat}
%    \end{macrocode}
%
% \begin{macro}{\bxjx@engine}
% エンジンの種別。
%    \begin{macrocode}
\let\bxac@engine=n
\def\bxac@do#1#2{%
  \edef\bxac@tmpa{\string#1}%
  \edef\bxac@tmpb{\meaning#1}%
  \ifx\bxac@tmpa\bxac@tmpb #2\fi}
\bxac@do\XeTeXversion{\let\bxac@engine=x}
\bxac@do\luatexversion{\let\bxac@engine=l}
%    \end{macrocode}
% \end{macro}
%
% \begin{macro}{\bxac@delayed@if@bxjs}
% もしBXJSクラスの読込中でこのパッケージが読み込まれているならば、
% BXJSのクラスの終わりまで実行を遅延する。
%    \begin{macrocode}
\ifx\jsAtEndOfClass\@undefined
  \let\bxac@delayed@if@bxjs\@firstofone
\else \let\bxac@delayed@if@bxjs\jsAtEndOfClass
\fi
%    \end{macrocode}
% \end{macro}
%
% \begin{macro}{\ImposeOldLuaTeXBehavior}
% \begin{macro}{\RevokeOldLuaTeXBehavior}
% ムニャムニャ。
%    \begin{macrocode}
\newif\ifbxac@in@old@behavior
\let\ImposeOldLuaTeXBehavior\relax
\let\RevokeOldLuaTeXBehavior\relax
%    \end{macrocode}
% \end{macro}
% \end{macro}
%
%^^A----------------
% \subsection{{\XeTeX}部分}
%    \begin{macrocode}
\ifx x\bxac@engine
%    \end{macrocode}
%
% {\XeTeX}文字クラスのムニャムニャ。
%    \begin{macrocode}
\@onlypreamble\bxac@adjust@charclass
\bxac@delayed@if@bxjs{%
  \@ifpackageloaded{xeCJK}{}{%else
    \ifx\xe@alloc@intercharclass\@undefined\else
        \ifnum\xe@alloc@intercharclass=\z@
      \PackageInfo\bxac@pkgname
        {Setting up interchar class for CJK...\@gobble}%
      \InputIfFileExists{load-unicode-xetex-classes.tex}{%
        \xe@alloc@intercharclass=3
      }{%else
        \PackageWarning\bxac@pkgname
          {Cannot find file 'load-unicode-xetex-classes.tex'%
           \@gobble}%
      }%
    \fi\fi
    \ifnum\XeTeXcharclass"3041=\z@
      \PackageInfo\bxac@pkgname
        {Adjusting interchar class for CJK...\@gobble}%
      \@for\bxac@x:={%
        3041,3043,3045,3047,3049,3063,3083,3085,3087,308E,%
        3095,3096,30A1,30A3,30A5,30A7,30A9,30C3,30E3,30E5,%
        30E7,30EE,30F5,30F6,30FC,31F0,31F1,31F2,31F3,31F4,%
        31F5,31F6,31F7,31F8,31F9,31FA,31FB,31FC,31FD,31FE,%
        31FF%
      }\do{\XeTeXcharclass"\bxac@x=\@ne}%
    \fi
  }%
}
%    \end{macrocode}
% 以上。
%    \begin{macrocode}
\fi
%    \end{macrocode}
%
%^^A----------------
% \subsection{Lua{\TeX}部分}
%    \begin{macrocode}
\ifx l\bxac@engine
%    \end{macrocode}
%
% ムニャムニャ。
%    \begin{macrocode}
\unless\ifnum\luatexversion<80 \ifnum\luatexversion<85
  \chardef\pdftexversion=200
  \def\pdftexrevision{0}
  \let\pdftexbanner\luatexbanner
\fi\fi
%    \end{macrocode}
%
% \begin{macro}{\ImposeOldLuaTeXBehavior}
% \begin{macro}{\RevokeOldLuaTeXBehavior}
% ムニャムニャ。
%    \begin{macrocode}
\begingroup\expandafter\expandafter\expandafter\endgroup
\expandafter\ifx\csname outputmode\endcsname\relax\else
\def\bxac@ob@list{%
  \do{\let}\pdfoutput{\outputmode}%
  \do{\let}\pdfpagewidth{\pagewidth}%
  \do{\let}\pdfpageheight{\pageheight}%
  \do{\protected\edef}\pdfhorigin{{\pdfvariable horigin}}%
  \do{\protected\edef}\pdfvorigin{{\pdfvariable vorigin}}}
\def\bxac@ob@do#1#2{\begingroup
  \expandafter\bxac@ob@do@a\csname bxac@\string#2\endcsname{#1}#2}
\def\bxac@ob@do@a#1#2#3#4{\endgroup
  \ifbxac@in@old@behavior \let#1#3\relax #2#3#4\relax
  \else \let#3#1\relax \let#1\@undefined
  \fi}
\protected\def\ImposeOldLuaTeXBehavior{%
  \unless\ifbxac@in@old@behavior
    \bxac@in@old@behaviortrue
    \let\do\bxac@ob@do \bxac@ob@list
  \fi}
\protected\def\RevokeOldLuaTeXBehavior{%
  \ifbxac@in@old@behavior
    \bxac@in@old@behaviorfalse
    \let\do\bxac@ob@do \bxac@ob@list
  \fi}
\fi
%    \end{macrocode}
% \end{macro}
% \end{macro}
%
% 漢字および完成形ハングルのカテゴリコードのムニャムニャ。
%    \begin{macrocode}
  \ifnum\luatexversion>64 \directlua{
    local function range(cs, ce, cc, ff)
      if ff or not tex.getcatcode(cs) == cc then
        local setcc = tex.setcatcode
        for c = cs, ce do setcc(c, cc) end
      end
    end
    range(0x3400, 0x4DB5, 11, false)
    range(0x4DB5, 0x4DBF, 11, true)
    range(0x4E00, 0x9FCC, 11, false)
    range(0x9FCD, 0x9FFF, 11, true)
    range(0xAC00, 0xD7A3, 11, false)
    range(0x20000, 0x2A6D6, 11, false)
    range(0x2A6D7, 0x2A6FF, 11, true)
    range(0x2A700, 0x2B734, 11, false)
    range(0x2B735, 0x2B73F, 11, true)
    range(0x2B740, 0x2B81D, 11, false)
    range(0x2B81E, 0x2B81F, 11, true)
    range(0x2B820, 0x2CEA1, 11, false)
    range(0x2CEA2, 0x2FFFD, 11, true)
  }\fi
%    \end{macrocode}
% 以上。
%    \begin{macrocode}
\fi
%    \end{macrocode}
%
%^^A----------------
% \subsection{完了}
% おしまい。
%    \begin{macrocode}
%</compat>
%    \end{macrocode}
%
%^^A========================================================
% \section{補助パッケージ:bxjscjkcat \ZRX}
%
% modernドライバ用の和文カテゴリを適用する。
%
%^^A----------------
% \subsection{準備}
%
%    \begin{macrocode}
%<*cjkcat>
\def\bxjx@pkgname{bxjscjkcat}
\newcount\bxjx@cnta
%    \end{macrocode}
%
% \begin{macro}{\bxjx@engine}
% エンジンの種別。
%    \begin{macrocode}
\let\bxjx@engine=n
\def\bxjx@do#1#2{%
  \edef\bxjx@tmpa{\string#1}%
  \edef\bxjx@tmpb{\meaning#1}%
  \ifx\bxjx@tmpa\bxjx@tmpb #2\fi}
\bxjx@do\kanjiskip{\let\bxjx@engine=j}
\bxjx@do\enablecjktoken{\let\bxjx@engine=u}
\bxjx@do\XeTeXversion{\let\bxjx@engine=x}
\bxjx@do\pdftexversion{\let\bxjx@engine=p}
\bxjx@do\luatexversion{\let\bxjx@engine=l}
%    \end{macrocode}
% \end{macro}
%
% それぞれのエンジンで、前提となる日本語処理パッケージが実際に
% 読み込まれているかを検査する。
%    \begin{macrocode}
\def\bxjx@do#1#2{%
  \if#1\bxjx@engine
    \@ifpackageloaded{#2}{}{%else
      \PackageError\bxjx@pkgname
       {Package '#2' must be loaded}%
       {Package loading is aborted.\MessageBreak\@ehc}%
      \endinput}
  \fi}
\bxjx@do{p}{bxcjkjatype}
\bxjx@do{x}{xeCJK}
\bxjx@do{l}{luatexja}
%    \end{macrocode}
%
% 古い{\LaTeX}の場合、|\TextOrMath| は |fixltx2e| パッケージで
% 提供される。
%    \begin{macrocode}
\ifx\TextOrMath\@undefined
  \RequirePackage{fixltx2e}
\fi
%    \end{macrocode}
%
%^^A----------------
% \subsection{和文カテゴリコードの設定}
%
% up{\LaTeX}の場合、和文カテゴリコードの設定を
% Lua{\TeX}-jaと(ほぼ)等価なものに変更する。
%
% \Note Lua{\TeX}-jaとの相違点:
% |A830|、|A960|、|1B000|。
%    \begin{macrocode}
\if u\bxjx@engine
\@for\bxjx@x:={%
0080,0100,0180,0250,02B0,0300,0500,0530,0590,0600,%
0700,0750,0780,07C0,0800,0840,08A0,0900,0980,0A00,%
0A80,0B00,0B80,0C00,0C80,0D00,0D80,0E00,0E80,0F00,%
1000,10A0,1200,1380,13A0,1400,1680,16A0,1700,1720,%
1740,1760,1780,1800,18B0,1900,1950,1980,19E0,1A00,%
1A20,1AB0,1B00,1B80,1BC0,1C00,1C50,1CC0,1CD0,1D00,%
1D80,1DC0,1E00,2440,27C0,27F0,2800,2A00,2C00,2C60,%
2C80,2D00,2D30,2D80,2DE0,2E00,4DC0,A4D0,A500,A640,%
A6A0,A700,A720,A800,A830,A840,A880,A8E0,A900,A930,%
A980,A9E0,AA00,AA60,AA80,AAE0,AB00,AB30,AB70,ABC0,%
D800,DB80,DC00,E000,FB00,FB50,FE00,FE70,%
10000,10080,10100,10140,10190,101D0,10280,102A0,%
102E0,10300,10330,10350,10380,103A0,10400,10450,%
10480,10500,10530,10600,10800,10840,10860,10880,%
108E0,10900,10920,10980,109A0,10A00,10A60,10A80,%
10AC0,10B00,10B40,10B60,10B80,10C00,10C80,10E60,%
11000,11080,110D0,11100,11150,11180,111E0,11200,%
11280,112B0,11300,11480,11580,11600,11680,11700,%
118A0,11AC0,12000,12400,12480,13000,14400,16800,%
16A40,16AD0,16B00,16F00,1BC00,1BCA0,1D000,1D100,%
1D200,1D300,1D360,1D400,1D800,1E800,1EE00,1F000,%
1F030,1F0A0,1F100,1F200,1F300,1F600,1F650,1F680,%
1F700,1F780,1F800,1F900,E0000,F0000,100000%
}\do{\kcatcode"\bxjx@x=15 }
\fi
%    \end{macrocode}
%
%^^A----------------
% \subsection{ギリシャ・キリル文字の扱い}
%
% \Note ここで「ギリシャ・キリル文字」はUnicodeとJIS X 0213に
% 共通して含まれるもののみを指すことにする。
%
% \begin{macro}{\bxjx@grkcyr@list}
% 対象のギリシャ・キリル文字に関するデータ。
%    \begin{macrocode}
\def\bxjx@grkcyr@list{%
\do{0391}{LGR}{\textAlpha}{A}%            % GR. C. L. ALPHA
\do{0392}{LGR}{\textBeta}{B}%             % GR. C. L. BETA
\do{0393}{LGR}{\textGamma}{\Gamma}%       % GR. C. L. GAMMA
\do{0394}{LGR}{\textDelta}{\Delta}%       % GR. C. L. DELTA
\do{0395}{LGR}{\textEpsilon}{E}%          % GR. C. L. EPSILON
\do{0396}{LGR}{\textZeta}{Z}%             % GR. C. L. ZETA
\do{0397}{LGR}{\textEta}{H}%              % GR. C. L. ETA
\do{0398}{LGR}{\textTheta}{\Theta}%       % GR. C. L. THETA
\do{0399}{LGR}{\textIota}{I}%             % GR. C. L. IOTA
\do{039A}{LGR}{\textKappa}{K}%            % GR. C. L. KAPPA
\do{039B}{LGR}{\textLambda}{\Lambda}%     % GR. C. L. LAMDA
\do{039C}{LGR}{\textMu}{M}%               % GR. C. L. MU
\do{039D}{LGR}{\textNu}{N}%               % GR. C. L. NU
\do{039E}{LGR}{\textXi}{\Xi}%             % GR. C. L. XI
\do{039F}{LGR}{\textOmicron}{O}%          % GR. C. L. OMICRON
\do{03A0}{LGR}{\textPi}{\Pi}%             % GR. C. L. PI
\do{03A1}{LGR}{\textRho}{P}%              % GR. C. L. RHO
\do{03A3}{LGR}{\textSigma}{\Sigma}%       % GR. C. L. SIGMA
\do{03A4}{LGR}{\textTau}{T}%              % GR. C. L. TAU
\do{03A5}{LGR}{\textUpsilon}{\Upsilon}%   % GR. C. L. UPSILON
\do{03A6}{LGR}{\textPhi}{\Phi}%           % GR. C. L. PHI
\do{03A7}{LGR}{\textChi}{X}%              % GR. C. L. CHI
\do{03A8}{LGR}{\textPsi}{\Psi}%           % GR. C. L. PSI
\do{03A9}{LGR}{\textOmega}{\Omega}%       % GR. C. L. OMEGA
\do{03B1}{LGR}{\textalpha}{\alpha}%       % GR. S. L. ALPHA
\do{03B2}{LGR}{\textbeta}{\beta}%         % GR. S. L. BETA
\do{03B3}{LGR}{\textgamma}{\gamma}%       % GR. S. L. GAMMA
\do{03B4}{LGR}{\textdelta}{\delta}%       % GR. S. L. DELTA
\do{03B5}{LGR}{\textepsilon}{\epsilon}%   % GR. S. L. EPSILON
\do{03B6}{LGR}{\textzeta}{\zeta}%         % GR. S. L. ZETA
\do{03B7}{LGR}{\texteta}{\eta}%           % GR. S. L. ETA
\do{03B8}{LGR}{\texttheta}{\theta}%       % GR. S. L. THETA
\do{03B9}{LGR}{\textiota}{\iota}%         % GR. S. L. IOTA
\do{03BA}{LGR}{\textkappa}{\kappa}%       % GR. S. L. KAPPA
\do{03BB}{LGR}{\textlambda}{\lambda}%     % GR. S. L. LAMDA
\do{03BC}{LGR}{\textmu}{\mu}%             % GR. S. L. MU
\do{03BD}{LGR}{\textnu}{\nu}%             % GR. S. L. NU
\do{03BE}{LGR}{\textxi}{\xi}%             % GR. S. L. XI
\do{03BF}{LGR}{\textomicron}{o}%          % GR. S. L. OMICRON
\do{03C0}{LGR}{\textpi}{\pi}%             % GR. S. L. PI
\do{03C1}{LGR}{\textrho}{\rho}%           % GR. S. L. RHO
\do{03C2}{LGR}{\textvarsigma}{\varsigma}% % GR. S. L. FINAL SIGMA
\do{03C3}{LGR}{\textsigma}{\sigma}%       % GR. S. L. SIGMA
\do{03C4}{LGR}{\texttau}{\tau}%           % GR. S. L. TAU
\do{03C5}{LGR}{\textupsilon}{\upsilon}%   % GR. S. L. UPSILON
\do{03C6}{LGR}{\textphi}{\phi}%           % GR. S. L. PHI
\do{03C7}{LGR}{\textchi}{\chi}%           % GR. S. L. CHI
\do{03C8}{LGR}{\textpsi}{\psi}%           % GR. S. L. PSI
\do{03C9}{LGR}{\textomega}{\omega}%       % GR. S. L. OMEGA
\do{0401}{T2A}{\CYRYO}{}%                 % CY. C. L. IO
\do{0410}{T2A}{\CYRA}{}%                  % CY. C. L. A
\do{0411}{T2A}{\CYRB}{}%                  % CY. C. L. BE
\do{0412}{T2A}{\CYRV}{}%                  % CY. C. L. VE
\do{0413}{T2A}{\CYRG}{}%                  % CY. C. L. GHE
\do{0414}{T2A}{\CYRD}{}%                  % CY. C. L. DE
\do{0415}{T2A}{\CYRE}{}%                  % CY. C. L. IE
\do{0416}{T2A}{\CYRZH}{}%                 % CY. C. L. ZHE
\do{0417}{T2A}{\CYRZ}{}%                  % CY. C. L. ZE
\do{0418}{T2A}{\CYRI}{}%                  % CY. C. L. I
\do{0419}{T2A}{\CYRISHRT}{}%              % CY. C. L. SHORT I
\do{041A}{T2A}{\CYRK}{}%                  % CY. C. L. KA
\do{041B}{T2A}{\CYRL}{}%                  % CY. C. L. EL
\do{041C}{T2A}{\CYRM}{}%                  % CY. C. L. EM
\do{041D}{T2A}{\CYRN}{}%                  % CY. C. L. EN
\do{041E}{T2A}{\CYRO}{}%                  % CY. C. L. O
\do{041F}{T2A}{\CYRP}{}%                  % CY. C. L. PE
\do{0420}{T2A}{\CYRR}{}%                  % CY. C. L. ER
\do{0421}{T2A}{\CYRS}{}%                  % CY. C. L. ES
\do{0422}{T2A}{\CYRT}{}%                  % CY. C. L. TE
\do{0423}{T2A}{\CYRU}{}%                  % CY. C. L. U
\do{0424}{T2A}{\CYRF}{}%                  % CY. C. L. EF
\do{0425}{T2A}{\CYRH}{}%                  % CY. C. L. HA
\do{0426}{T2A}{\CYRC}{}%                  % CY. C. L. TSE
\do{0427}{T2A}{\CYRCH}{}%                 % CY. C. L. CHE
\do{0428}{T2A}{\CYRSH}{}%                 % CY. C. L. SHA
\do{0429}{T2A}{\CYRSHCH}{}%               % CY. C. L. SHCHA
\do{042A}{T2A}{\CYRHRDSN}{}%              % CY. C. L. HARD SIGN
\do{042B}{T2A}{\CYRERY}{}%                % CY. C. L. YERU
\do{042C}{T2A}{\CYRSFTSN}{}%              % CY. C. L. SOFT SIGN
\do{042D}{T2A}{\CYREREV}{}%               % CY. C. L. E
\do{042E}{T2A}{\CYRYU}{}%                 % CY. C. L. YU
\do{042F}{T2A}{\CYRYA}{}%                 % CY. C. L. YA
\do{0430}{T2A}{\cyra}{}%                  % CY. S. L. A
\do{0431}{T2A}{\cyrb}{}%                  % CY. S. L. BE
\do{0432}{T2A}{\cyrv}{}%                  % CY. S. L. VE
\do{0433}{T2A}{\cyrg}{}%                  % CY. S. L. GHE
\do{0434}{T2A}{\cyrd}{}%                  % CY. S. L. DE
\do{0435}{T2A}{\cyre}{}%                  % CY. S. L. IE
\do{0436}{T2A}{\cyrzh}{}%                 % CY. S. L. ZHE
\do{0437}{T2A}{\cyrz}{}%                  % CY. S. L. ZE
\do{0438}{T2A}{\cyri}{}%                  % CY. S. L. I
\do{0439}{T2A}{\cyrishrt}{}%              % CY. S. L. SHORT I
\do{043A}{T2A}{\cyrk}{}%                  % CY. S. L. KA
\do{043B}{T2A}{\cyrl}{}%                  % CY. S. L. EL
\do{043C}{T2A}{\cyrm}{}%                  % CY. S. L. EM
\do{043D}{T2A}{\cyrn}{}%                  % CY. S. L. EN
\do{043E}{T2A}{\cyro}{}%                  % CY. S. L. O
\do{043F}{T2A}{\cyrp}{}%                  % CY. S. L. PE
\do{0440}{T2A}{\cyrr}{}%                  % CY. S. L. ER
\do{0441}{T2A}{\cyrs}{}%                  % CY. S. L. ES
\do{0442}{T2A}{\cyrt}{}%                  % CY. S. L. TE
\do{0443}{T2A}{\cyru}{}%                  % CY. S. L. U
\do{0444}{T2A}{\cyrf}{}%                  % CY. S. L. EF
\do{0445}{T2A}{\cyrh}{}%                  % CY. S. L. HA
\do{0446}{T2A}{\cyrc}{}%                  % CY. S. L. TSE
\do{0447}{T2A}{\cyrch}{}%                 % CY. S. L. CHE
\do{0448}{T2A}{\cyrsh}{}%                 % CY. S. L. SHA
\do{0449}{T2A}{\cyrshch}{}%               % CY. S. L. SHCHA
\do{044A}{T2A}{\cyrhrdsn}{}%              % CY. S. L. HARD SIGN
\do{044B}{T2A}{\cyrery}{}%                % CY. S. L. YERU
\do{044C}{T2A}{\cyrsftsn}{}%              % CY. S. L. SOFT SIGN
\do{044D}{T2A}{\cyrerev}{}%               % CY. S. L. E
\do{044E}{T2A}{\cyryu}{}%                 % CY. S. L. YU
\do{044F}{T2A}{\cyrya}{}%                 % CY. S. L. YA
\do{0451}{T2A}{\cyryo}{}%                 % CY. S. L. IO
\do{00A7}{TS1}{\textsection}{\mathsection}% SECTION SYMBOL
\do{00A8}{TS1}{\textasciidieresis}{}%      % DIAERESIS
\do{00B0}{TS1}{\textdegree}{\mathdegree}% % DEGREE SIGN
\do{00B1}{TS1}{\textpm}{\pm}%             % PLUS-MINUS SIGN
\do{00B4}{TS1}{\textasciiacute}{}%        % ACUTE ACCENT
\do{00B6}{TS1}{\textparagraph}{\mathparagraph}% PILCROW SIGN
\do{00D7}{TS1}{\texttimes}{\times}%       % MULTIPLICATION SIGN
\do{00F7}{TS1}{\textdiv}{\div}%           % DIVISION SIGN
}
%    \end{macrocode}
% \end{macro}
%    \begin{macrocode}
\providecommand*{\mathdegree}{{}^{\circ}}
%    \end{macrocode}
%
% \begin{macro}{\ifbxjx@gcc@cjk}
% 〔スイッチ〕
% ギリシャ・キリル文字を和文扱いにするか。
%    \begin{macrocode}
\newif\ifbxjx@gcc@cjk
%    \end{macrocode}
% \end{macro}
%
% \begin{macro}{\greekasCJK}
% ギリシャ・キリル文字を和文扱いにする。
% \begin{macro}{\nogreekasCJK}
% ギリシャ・キリル文字を欧文扱いにする。
%    \begin{macrocode}
\newcommand*\greekasCJK{%
  \bxjx@gcc@cjktrue}
\newcommand*\nogreekasCJK{%
  \bxjx@gcc@cjkfalse}
%    \end{macrocode}
% \end{macro}
% \end{macro}
%
% \begin{macro}{\bx@fake@grk}
% |\bx@fake@grk{|\Meta{出力文字}|}{|\Meta{基準文字}|}|\Means
%    \begin{macrocode}
\def\bxjx@do#1\relax{%
  \def\bxjx@fake@grk##1##2{%
    \expandafter\bxjx@fake@grk@a\meaning##2#1\@nil{##1}{##2}}%
  \def\bxjx@fake@grk@a##1#1##2\@nil##3##4{%
    \ifx\\##1\\%
      \bxjx@cnta##4\divide\bxjx@cnta\@cclvi
      \multiply\bxjx@cnta\@cclvi \advance\bxjx@cnta`##3\relax
      \mathchar\bxjx@cnta
    \else ##3\fi}
}\expandafter\bxjx@do\string\mathchar\relax
%    \end{macrocode}
% \end{macro}
%
% \paragraph{pdfLaTeX・upLaTeXの場合}
%    \begin{macrocode}
\ifnum0\if p\bxjx@engine1\fi\if u\bxjx@engine1\fi>0
%    \end{macrocode}
% まず |inputenc| を読み込んで入力エンコーディングを |utf8|
% に変更する。
%    \begin{macrocode}
\@ifpackageloaded{inputenc}{}{%else
  \RequirePackage[utf8]{inputenc}}
\def\bxjx@tmpa{utf8}
\ifx\bxjx@tmpa\inputencdoingname
  \PackageWarningNoLine\bxjx@pkgname
   {Input encoding changed to utf8}%
  \inputencoding{utf8}%
\fi
%    \end{macrocode}
%
% up{\LaTeX}の場合は当該の文字を含むブロックをの和文カテゴリコード
% を変更する。
%    \begin{macrocode}
\if u\bxjx@engine
\kcatcode"0370=15
\kcatcode"0400=15
\kcatcode"0500=15
\fi
%    \end{macrocode}
% 各文字について |\DeclareUnicodeCharacter| を実行する。
%    \begin{macrocode}
\def\do#1{%
  \@tempcnta="#1\relax
  \@tempcntb=\@tempcnta \divide\@tempcntb256
  \expandafter\let\csname bxjx@KCR/\the\@tempcntb\endcsname=t%
  \expandafter\bxjx@do@a\csname bxjx@KC/\the\@tempcnta\endcsname{#1}}
\def\bxjx@do@a#1#2#3#4#5{%
  \ifx\\#5\\%
    \def\bxjx@tmpa{\@inmathwarn#4}%
  \else\ifcat A\noexpand#5%
    \edef\bxjx@tmpa{\noexpand\bxjx@fake@grk{#5}%
      {\ifnum\uccode`#5=`#5\noexpand\Pi\else\noexpand\pi\fi}}%
  \else \def\bxjx@tmpa{#5}%
  \fi\fi
  \def\bxjx@tmpb{\bxjx@do@b{#1}{#2}{#3}{#4}}%
  \expandafter\bxjx@tmpb\expandafter{\bxjx@tmpa}}
\if u\bxjx@engine
% {\bxjx@KC/NN}{XXXX}{ENC}{\textCS}{\mathCS}
\def\bxjx@do@b#1#2#3#4#5{%
  \kchardef#1=\@tempcnta
  \DeclareTextCommandDefault{#4}{\bxjx@ja@or@not{#1}{#3}{#4}}%
  \DeclareUnicodeCharacter{#2}{\TextOrMath{#4}{#5}}}
\else\if p\bxjx@engine
\def\bxjx@do@b#1#2#3#4#5{%
  \mathchardef#1=\@tempcnta
  \DeclareTextCommandDefault{#4}{\bxjx@ja@or@not{\UTF{#2}}{#3}{#4}}%
  \DeclareUnicodeCharacter{#2}{\TextOrMath{#4}{#5}}}
\fi\fi
\bxjx@grkcyr@list
\let\bxjx@do@a\undefined
\let\bxjx@do@b\undefined
%    \end{macrocode}
%
% \begin{macro}{\bxjx@DeclareUnicodeCharacter}
% |\bxjx@DeclareUnicodeCharacter| を改変して、
% ギリシャ・キリル文字の場合に再定義を抑止したもの。
%    \begin{macrocode}
\@onlypreamble\bxjx@org@DeclareUnicodeCharacter
\let\bxjx@org@DeclareUnicodeCharacter\DeclareUnicodeCharacter
\@onlypreamble\bxjx@DeclareUnicodeCharacter
\def\bxjx@DeclareUnicodeCharacter#1#2{%
  \count@="#1\relax \bxjx@cnta\count@ \divide\bxjx@cnta256
  \expandafter\ifx\csname bxjx@KCR/\the\bxjx@cnta\endcsname\relax
    \bxjx@org@DeclareUnicodeCharacter{#1}{#2}%
  \else\expandafter\ifx\csname bxjx@KC/\the\count@\endcsname\relax
    \bxjx@org@DeclareUnicodeCharacter{#1}{#2}%
  \else
    \wlog{ \space\space skipped defining Unicode char U+#1}%
  \fi\fi}
%    \end{macrocode}
% \end{macro}
%
% \begin{macro}{\bxjx@ja@or@not}
% |\bxjx@ja@or@not|
%    \begin{macrocode}
\def\bxjx@ja@or@not#1#2#3{%
%    \end{macrocode}
% |\greekasCJK| の場合は、無条件に和文用コードを実行する。
%    \begin{macrocode}
  \ifbxjx@gcc@cjk #1%
%    \end{macrocode}
% |\nogreekasCJK| の場合は、エンコーディングを固定して欧文用の
% コードを実行するが、そのエンコーディングが未定義の場合は
% (フォールバックとして)和文用コードを使う。
%    \begin{macrocode}
  \else\expandafter\ifx\csname T@#2\endcsname\relax #1%
  \else \UseTextSymbol{#2}{#3}%
  \fi\fi}
%    \end{macrocode}
% \end{macro}
%
% \begin{macro}{\DeclareFontEncoding@}
% |\DeclareFontEncoding@| にパッチを当てて、
% |\DeclareFontEncoding| の実行中だけ
% 改変後の |\DeclareUnicodeCharacter| が使われるようにする。
%    \begin{macrocode}
\begingroup
\toks@\expandafter{\DeclareFontEncoding@{#1}{#2}{#3}}
\xdef\next{\def\noexpand\DeclareFontEncoding@##1##2##3{%
  \noexpand\bxjx@swap@DUC@cmd
  \the\toks@
  \noexpand\bxjx@swap@DUC@cmd}}
\endgroup\next
\def\bxjx@swap@DUC@cmd{%
  \let\bxjx@tmpa\DeclareUnicodeCharacter
  \let\DeclareUnicodeCharacter\bxjx@DeclareUnicodeCharacter
  \let\bxjx@DeclareUnicodeCharacter\bxjx@tmpa}
%    \end{macrocode}
% \end{macro}
%
% 以上。
%
% \paragraph{{\XeLaTeX}・Lua{\LaTeX}の場合}
%    \begin{macrocode}
\else\ifnum0\if x\bxjx@engine1\fi\if l\bxjx@engine1\fi>0
%    \end{macrocode}
%
% 各文字について、math activeを設定する。
%    \begin{macrocode}
\def\do#1{%
  \bxjx@cnta="#1\relax
  \begingroup
    \lccode`~=\bxjx@cnta
  \lowercase{\endgroup
    \bxjx@do@a{~}}{#1}}
\def\bxjx@do@a#1#2#3#4#5{%
  \ifx\\#5\\\let\bxjx@tmpa\relax
  \else\ifcat A\noexpand#5%
    \edef\bxjx@tmpa{\noexpand\bxjx@fake@grk{#5}%
      {\ifnum\uccode`#5=`#5\noexpand\Pi\else\noexpand\pi\fi}}%
  \else \def\bxjx@tmpa{#5}%
  \fi\fi
  \ifx\bxjx@tmpa\relax\else
    \mathcode\bxjx@cnta"8000 \let#1\bxjx@tmpa
  \fi}
%    \end{macrocode}
% 「Unicodeな数式」の設定が行われているかを(簡易的に)検査して、
% そうでない場合にのみ、こちらの設定を有効にする。
%    \begin{macrocode}
\mathchardef\bxjx@tmpa="119
\ifx\bxjx@tmpa\pi \bxjx@grkcyr@list \fi
\let\bxjx@do@a\undefined
%    \end{macrocode}
%
% Lua{\TeX}における |\(no)greekasCJK| の定義。
% |jacharrange| の設定を変更する。
%    \begin{macrocode}
\if l\bxjx@engine
  \protected\def\greekasCJK{%
    \bxjx@gcc@cjktrue
    \ltjsetparameter{jacharrange={+2, +8}}}
  \protected\def\nogreekasCJK{%
    \bxjx@gcc@cjkfalse
    \ltjsetparameter{jacharrange={-2, -8}}}
\fi
%    \end{macrocode}
%
% {\XeTeX}における |\(no)greekasCJK| の定義。
%    \begin{macrocode}
\if x\bxjx@engine
  \protected\def\greekasCJK{%
    \bxjx@gcc@cjktrue
    \def\do##1##2##3##4{\XeTeXcharclass"##1\@ne}%
    \bxjx@grkcyr@list}
  \protected\def\nogreekasCJK{%
    \bxjx@gcc@cjkfalse
    \def\do##1##2##3##4{\XeTeXcharclass"##1\z@}%
    \bxjx@grkcyr@list}
\fi
%    \end{macrocode}
%
% 以上。
%    \begin{macrocode}
\fi\fi
%    \end{macrocode}
%
%^^A----------------
% \subsection{初期設定}
% ギリシャ・キリル文字を欧文扱いにする。
%    \begin{macrocode}
\nogreekasCJK
%    \end{macrocode}
%
%^^A----------------
% \subsection{完了}
% おしまい。
%    \begin{macrocode}
%</cjkcat>
%    \end{macrocode}
%
% 補助パッケージ実装はここまで。
%    \begin{macrocode}
%</anc>
%    \end{macrocode}
%
% \Finale
%
\endinput
